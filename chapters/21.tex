\FChapter{Chapter Twenty-One}{21}

\Lettrine{P}{resentiments} \textsc{are strange things!} and so are sympathies; and so are
signs; and the three combined make one mystery to which humanity has not
yet found the key. I never laughed at presentiments in my life, because
I have had strange ones of my own. Sympathies, I believe, exist (for
instance, between far-distant, long-absent, wholly estranged relatives
asserting, notwithstanding their alienation, the unity of the source to
which each traces his origin) whose workings baffle mortal
comprehension. And signs, for aught we know, may be but the sympathies
of Nature with man.

When I was a little girl, only six years old, I one night heard Bessie
Leaven say to Martha Abbot that she had been dreaming about a little
child; and that to dream of children was a sure sign of trouble, either
to one's self or one's kin. The saying might have worn out of my
memory, had not a circumstance immediately followed which served
indelibly to fix it there. The next day Bessie was sent for home to the
deathbed of her little sister.

Of late I had often recalled this saying and this incident; for during
the past week scarcely a night had gone over my couch that had not
brought with it a dream of an infant, which I sometimes hushed in my
arms, sometimes dandled on my knee, sometimes watched playing with
daisies on a lawn, or again, dabbling its hands in running water. It
was a wailing child this night, and a laughing one the next: now it
nestled close to me, and now it ran from me; but whatever mood the
apparition evinced, whatever aspect it wore, it failed not for seven
successive nights to meet me the moment I entered the land of slumber.

I did not like this iteration of one idea---this strange recurrence of
one image, and I grew nervous as bedtime approached and the hour of the
vision drew near. It was from companionship with this baby-phantom I
had been roused on that moonlight night when I heard the cry; and it was
on the afternoon of the day following I was summoned downstairs by a
message that some one wanted me in \Mrs{} Fairfax's room. On repairing
thither, I found a man waiting for me, having the appearance of a
gentleman's servant: he was dressed in deep mourning, and the hat he
held in his hand was surrounded with a crape band.

\enquote{I daresay you hardly remember me, Miss,} he said, rising as I
entered; \enquote{but my name is Leaven: I lived coachman with \Mrs{} Reed
	when you were at Gateshead, eight or nine years since, and I live there
	still.}

\enquote{Oh, Robert! how do you do? I remember you very well: you used
	to give me a ride sometimes on Miss Georgiana's bay pony. And how is
	Bessie? You are married to Bessie?}

\enquote{Yes, Miss: my wife is very hearty, thank you; she brought me
	another little one about two months since---we have three now---and both
	mother and child are thriving.}

\enquote{And are the family well at the house, Robert?}

\enquote{I am sorry I can't give you better news of them, Miss: they are
	very badly at present---in great trouble.}

\enquote{I hope no one is dead,} I said, glancing at his black dress.
He too looked down at the crape round his hat and replied---

\enquote{\Mr{} John died yesterday was a week, at his chambers in London.}

\enquote{\Mr{} John?}

\enquote{Yes.}

\enquote{And how does his mother bear it?}

\enquote{Why, you see, Miss Eyre, it is not a common mishap: his life
	has been very wild: these last three years he gave himself up to strange
	ways, and his death was shocking.}

\enquote{I heard from Bessie he was not doing well.}

\enquote{Doing well! He could not do worse: he ruined his health and
	his estate amongst the worst men and the worst women. He got into debt
	and into jail: his mother helped him out twice, but as soon as he was
	free he returned to his old companions and habits. His head was not
	strong: the knaves he lived amongst fooled him beyond anything I ever
	heard. He came down to Gateshead about three weeks ago and wanted
	missis to give up all to him. Missis refused: her means have long been
	much reduced by his extravagance; so he went back again, and the next
	news was that he was dead. How he died, God knows!---they say he killed
	himself.}

I was silent: the things were frightful. Robert Leaven resumed---

\enquote{Missis had been out of health herself for some time: she had
	got very stout, but was not strong with it; and the loss of money and
	fear of poverty were quite breaking her down. The information about \Mr{}
	John's death and the manner of it came too suddenly: it brought on a
	stroke. She was three days without speaking; but last Tuesday she
	seemed rather better: she appeared as if she wanted to say something,
	and kept making signs to my wife and mumbling. It was only yesterday
	morning, however, that Bessie understood she was pronouncing your name;
	and at last she made out the words, \enquote{Bring Jane---fetch Jane
		Eyre: I want to speak to her.} Bessie is not sure whether she is in her
	right mind, or means anything by the words; but she told Miss Reed and
	Miss Georgiana, and advised them to send for you. The young ladies put
	it off at first; but their mother grew so restless, and said,
	\enquote{Jane, Jane,} so many times, that at last they consented. I
	left Gateshead yesterday: and if you can get ready, Miss, I should like
	to take you back with me early to-morrow morning.}

\enquote{Yes, Robert, I shall be ready: it seems to me that I ought to
	go.}

\enquote{I think so too, Miss. Bessie said she was sure you would not
	refuse: but I suppose you will have to ask leave before you can get
	off?}

\enquote{Yes; and I will do it now;} and having directed him to the
servants' hall, and recommended him to the care of John's wife, and the
attentions of John himself, I went in search of \Mr{} Rochester.

He was not in any of the lower rooms; he was not in the yard, the
stables, or the grounds. I asked \Mrs{} Fairfax if she had seen
him;---yes: she believed he was playing billiards with Miss Ingram. To
the billiard-room I hastened: the click of balls and the hum of voices
resounded thence; \Mr{} Rochester, Miss Ingram, the two Misses Eshton, and
their admirers, were all busied in the game. It required some courage
to disturb so interesting a party; my errand, however, was one I could
not defer, so I approached the master where he stood at Miss Ingram's
side. She turned as I drew near, and looked at me haughtily: her eyes
seemed to demand, \enquote{What can the creeping creature want now?} and
when I said, in a low voice, \enquote{\Mr{} Rochester,} she made a
movement as if tempted to order me away. I remember her appearance at
the moment---it was very graceful and very striking: she wore a morning
robe of sky-blue crape; a gauzy azure scarf was twisted in her hair.
She had been all animation with the game, and irritated pride did not
lower the expression of her haughty lineaments.

\enquote{Does that person want you?} she inquired of \Mr{} Rochester; and
\Mr{} Rochester turned to see who the \enquote{person} was. He made a
curious grimace---one of his strange and equivocal
demonstrations---threw down his cue and followed me from the room.

\enquote{Well, Jane?} he said, as he rested his back against the
schoolroom door, which he had shut.

\enquote{If you please, sir, I want leave of absence for a week or two.}

\enquote{What to do?---where to go?}

\enquote{To see a sick lady who has sent for me.}

\enquote{What sick lady?---where does she live?}

\enquote{At Gateshead; in ---shire.}

\enquote{-shire? That is a hundred miles off! Who may she be that
	sends for people to see her that distance?}

\enquote{Her name is Reed, sir---\Mrs{} Reed.}

\enquote{Reed of Gateshead? There was a Reed of Gateshead, a
	magistrate.}

\enquote{It is his widow, sir.}

\enquote{And what have you to do with her? How do you know her?}

\enquote{\Mr{} Reed was my uncle---my mother's brother.}

\enquote{The deuce he was! You never told me that before: you always
	said you had no relations.}

\enquote{None that would own me, sir. \Mr{} Reed is dead, and his wife
	cast me off.}

\enquote{Why?}

\enquote{Because I was poor, and burdensome, and she disliked me.}

\enquote{But Reed left children?---you must have cousins? Sir George
	Lynn was talking of a Reed of Gateshead yesterday, who, he said, was one
	of the veriest rascals on town; and Ingram was mentioning a Georgiana
	Reed of the same place, who was much admired for her beauty a season or
	two ago in London.}

\enquote{John Reed is dead, too, sir: he ruined himself and half-ruined
	his family, and is supposed to have committed suicide. The news so
	shocked his mother that it brought on an apoplectic attack.}

\enquote{And what good can you do her? Nonsense, Jane! I would never
	think of running a hundred miles to see an old lady who will, perhaps,
	be dead before you reach her: besides, you say she cast you off.}

\enquote{Yes, sir, but that is long ago; and when her circumstances were
	very different: I could not be easy to neglect her wishes now.}

\enquote{How long will you stay?}

\enquote{As short a time as possible, sir.}

\enquote{Promise me only to stay a week---}

\enquote{I had better not pass my word: I might be obliged to break it.}

\enquote{At all events you \emph{will} come back: you will not be induced under
	any pretext to take up a permanent residence with her?}

\enquote{Oh, no! I shall certainly return if all be well.}

\enquote{And who goes with you? You don't travel a hundred miles
	alone.}

\enquote{No, sir, she has sent her coachman.}

\enquote{A person to be trusted?}

\enquote{Yes, sir, he has lived ten years in the family.}

\Mr{} Rochester meditated. \enquote{When do you wish to go?}

\enquote{Early to-morrow morning, sir.}

\enquote{Well, you must have some money; you can't travel without money,
	and I daresay you have not much: I have given you no salary yet. How
	much have you in the world, Jane?} he asked, smiling.

I drew out my purse; a meagre thing it was. \enquote{Five shillings,
	sir.} He took the purse, poured the hoard into his palm, and chuckled
over it as if its scantiness amused him. Soon he produced his
pocket-book: \enquote{Here,} said he, offering me a note; it was fifty
pounds, and he owed me but fifteen. I told him I had no change.

\enquote{I don't want change; you know that. Take your wages.}

I declined accepting more than was my due. He scowled at first; then,
as if recollecting something, he said---

\enquote{Right, right! Better not give you all now: you would, perhaps,
	stay away three months if you had fifty pounds. There are ten; is it
	not plenty?}

\enquote{Yes, sir, but now you owe me five.}

\enquote{Come back for it, then; I am your banker for forty pounds.}

\enquote{\Mr{} Rochester, I may as well mention another matter of business
	to you while I have the opportunity.}

\enquote{Matter of business? I am curious to hear it.}

\enquote{You have as good as informed me, sir, that you are going
	shortly to be married?}

\enquote{Yes; what then?}

\enquote{In that case, sir, Adèle ought to go to school: I am sure you
	will perceive the necessity of it.}

\enquote{To get her out of my bride's way, who might otherwise walk over
	her rather too emphatically? There's sense in the suggestion; not a
	doubt of it. Adèle, as you say, must go to school; and you, of course,
	must march straight to---the devil?}

\enquote{I hope not, sir; but I must seek another situation somewhere.}

\enquote{In course!} he exclaimed, with a twang of voice and a
distortion of features equally fantastic and ludicrous. He looked at me
some minutes.

\enquote{And old Madam Reed, or the Misses, her daughters, will be
	solicited by you to seek a place, I suppose?}

\enquote{No, sir; I am not on such terms with my relatives as would
	justify me in asking favours of them---but I shall advertise.}

\enquote{You shall walk up the pyramids of Egypt!} he growled.
\enquote{At your peril you advertise! I wish I had only offered you a
	sovereign instead of ten pounds. Give me back nine pounds, Jane; I've a
	use for it.}

\enquote{And so have I, sir,} I returned, putting my hands and my purse
behind me. \enquote{I could not spare the money on any account.}

\enquote{Little niggard!} said he, \enquote{refusing me a pecuniary
	request! Give me five pounds, Jane.}

\enquote{Not five shillings, sir; nor five pence.}

\enquote{Just let me look at the cash.}

\enquote{No, sir; you are not to be trusted.}

\enquote{Jane!}

\enquote{Sir?}

\enquote{Promise me one thing.}

\enquote{I'll promise you anything, sir, that I think I am likely to
	perform.}

\enquote{Not to advertise: and to trust this quest of a situation to
	me. I'll find you one in time.}

\enquote{I shall be glad so to do, sir, if you, in your turn, will
	promise that I and Adèle shall be both safe out of the house before your
	bride enters it.}

\enquote{Very well! very well! I'll pledge my word on it. You go
	to-morrow, then?}

\enquote{Yes, sir; early.}

\enquote{Shall you come down to the drawing-room after dinner?}

\enquote{No, sir, I must prepare for the journey.}

\enquote{Then you and I must bid good-bye for a little while?}

\enquote{I suppose so, sir.}

\enquote{And how do people perform that ceremony of parting, Jane?
	Teach me; I'm not quite up to it.}

\enquote{They say, Farewell, or any other form they prefer.}

\enquote{Then say it.}

\enquote{Farewell, \Mr{} Rochester, for the present.}

\enquote{What must I say?}

\enquote{The same, if you like, sir.}

\enquote{Farewell, Miss Eyre, for the present; is that all?}

\enquote{Yes?}

\enquote{It seems stingy, to my notions, and dry, and unfriendly. I
	should like something else: a little addition to the rite. If one shook
	hands, for instance; but no---that would not content me either. So
	you'll do no more than say Farewell, Jane?}

\enquote{It is enough, sir: as much good-will may be conveyed in one
	hearty word as in many.}

\enquote{Very likely; but it is blank and cool---\enquote{Farewell}.}

\enquote{How long is he going to stand with his back against that door?}
I asked myself; \enquote{I want to commence my packing.} The
dinner-bell rang, and suddenly away he bolted, without another syllable:
I saw him no more during the day, and was off before he had risen in the
morning.

I reached the lodge at Gateshead about five o'clock in the afternoon of
the first of May: I stepped in there before going up to the hall. It
was very clean and neat: the ornamental windows were hung with little
white curtains; the floor was spotless; the grate and fire-irons were
burnished bright, and the fire burnt clear. Bessie sat on the hearth,
nursing her last-born, and Robert and his sister played quietly in a
corner.

\enquote{Bless you!---I knew you would come!} exclaimed \Mrs{} Leaven, as
I entered.

\enquote{Yes, Bessie,} said I, after I had kissed her; \enquote{and I
	trust I am not too late. How is \Mrs{} Reed?---Alive still, I hope.}

\enquote{Yes, she is alive; and more sensible and collected than she
	was. The doctor says she may linger a week or two yet; but he hardly
	thinks she will finally recover.}

\enquote{Has she mentioned me lately?}

\enquote{She was talking of you only this morning, and wishing you would
	come, but she is sleeping now, or was ten minutes ago, when I was up at
	the house. She generally lies in a kind of lethargy all the afternoon,
	and wakes up about six or seven. Will you rest yourself here an hour,
	Miss, and then I will go up with you?}

Robert here entered, and Bessie laid her sleeping child in the cradle
and went to welcome him: afterwards she insisted on my taking off my
bonnet and having some tea; for she said I looked pale and tired. I was
glad to accept her hospitality; and I submitted to be relieved of my
travelling garb just as passively as I used to let her undress me when a
child.

Old times crowded fast back on me as I watched her bustling
about---setting out the tea-tray with her best china, cutting bread and
butter, toasting a tea-cake, and, between whiles, giving little Robert
or Jane an occasional tap or push, just as she used to give me in former
days. Bessie had retained her quick temper as well as her light foot
and good looks.

Tea ready, I was going to approach the table; but she desired me to sit
still, quite in her old peremptory tones. I must be served at the
fireside, she said; and she placed before me a little round stand with
my cup and a plate of toast, absolutely as she used to accommodate me
with some privately purloined dainty on a nursery chair: and I smiled
and obeyed her as in bygone days.

She wanted to know if I was happy at Thornfield Hall, and what sort of a
person the mistress was; and when I told her there was only a master,
whether he was a nice gentleman, and if I liked him. I told her he was
rather an ugly man, but quite a gentleman; and that he treated me
kindly, and I was content. Then I went on to describe to her the gay
company that had lately been staying at the house; and to these details
Bessie listened with interest: they were precisely of the kind she
relished.

In such conversation an hour was soon gone: Bessie restored to me my
bonnet, \etc, and, accompanied by her, I quitted the lodge for the
hall. It was also accompanied by her that I had, nearly nine years ago,
walked down the path I was now ascending. On a dark, misty, raw morning
in January, I had left a hostile roof with a desperate and embittered
heart---a sense of outlawry and almost of reprobation---to seek the
chilly harbourage of Lowood: that bourne so far away and unexplored.
The same hostile roof now again rose before me: my prospects were
doubtful yet; and I had yet an aching heart. I still felt as a wanderer
on the face of the earth; but I experienced firmer trust in myself and
my own powers, and less withering dread of oppression. The gaping wound
of my wrongs, too, was now quite healed; and the flame of resentment
extinguished.

\enquote{You shall go into the breakfast-room first,} said Bessie, as
she preceded me through the hall; \enquote{the young ladies will be
	there.}

In another moment I was within that apartment. There was every article
of furniture looking just as it did on the morning I was first
introduced to \Mr{} Brocklehurst: the very rug he had stood upon still
covered the hearth. Glancing at the bookcases, I thought I could
distinguish the two volumes of Bewick's British Birds occupying their
old place on the third shelf, and Gulliver's Travels and the Arabian
Nights ranged just above. The inanimate objects were not changed; but
the living things had altered past recognition.

Two young ladies appeared before me; one very tall, almost as tall as
Miss Ingram---very thin too, with a sallow face and severe mien. There
was something ascetic in her look, which was augmented by the extreme
plainness of a straight-skirted, black, stuff dress, a starched linen
collar, hair combed away from the temples, and the nun-like ornament of
a string of ebony beads and a crucifix. This I felt sure was Eliza,
though I could trace little resemblance to her former self in that
elongated and colourless visage.

The other was as certainly Georgiana: but not the Georgiana I
remembered---the slim and fairy-like girl of eleven. This was a
full-blown, very plump damsel, fair as waxwork, with handsome and
regular features, languishing blue eyes, and ringleted yellow hair. The
hue of her dress was black too; but its fashion was so different from
her sister's---so much more flowing and becoming---it looked as stylish
as the other's looked puritanical.

In each of the sisters there was one trait of the mother---and only one;
the thin and pallid elder daughter had her parent's Cairngorm eye: the
blooming and luxuriant younger girl had her contour of jaw and
chin---perhaps a little softened, but still imparting an indescribable
hardness to the countenance otherwise so voluptuous and buxom.

Both ladies, as I advanced, rose to welcome me, and both addressed me by
the name of \enquote{Miss Eyre.} Eliza's greeting was delivered in a
short, abrupt voice, without a smile; and then she sat down again, fixed
her eyes on the fire, and seemed to forget me. Georgiana added to her
\enquote{How d'ye do?} several commonplaces about my journey, the
weather, and so on, uttered in rather a drawling tone: and accompanied
by sundry side-glances that measured me from head to foot---now
traversing the folds of my drab merino pelisse, and now lingering on the
plain trimming of my cottage bonnet. Young ladies have a remarkable way
of letting you know that they think you a \enquote{quiz} without
actually saying the words. A certain superciliousness of look, coolness
of manner, nonchalance of tone, express fully their sentiments on the
point, without committing them by any positive rudeness in word or deed.

A sneer, however, whether covert or open, had now no longer that power
over me it once possessed: as I sat between my cousins, I was surprised
to find how easy I felt under the total neglect of the one and the
semi-sarcastic attentions of the other---Eliza did not mortify, nor
Georgiana ruffle me. The fact was, I had other things to think about;
within the last few months feelings had been stirred in me so much more
potent than any they could raise---pains and pleasures so much more
acute and exquisite had been excited than any it was in their power to
inflict or bestow---that their airs gave me no concern either for good
or bad.

\enquote{How is \Mrs{} Reed?} I asked soon, looking calmly at Georgiana,
who thought fit to bridle at the direct address, as if it were an
unexpected liberty.

\enquote{\Mrs{} Reed? Ah! mama, you mean; she is extremely poorly: I
	doubt if you can see her to-night.}

\enquote{If,} said I, \enquote{you would just step upstairs and tell her
	I am come, I should be much obliged to you.}

Georgiana almost started, and she opened her blue eyes wild and wide.
\enquote{I know she had a particular wish to see me,} I added,
\enquote{and I would not defer attending to her desire longer than is
	absolutely necessary.}

\enquote{Mama dislikes being disturbed in an evening,} remarked Eliza.
I soon rose, quietly took off my bonnet and gloves, uninvited, and said
I would just step out to Bessie---who was, I dared say, in the
kitchen---and ask her to ascertain whether \Mrs{} Reed was disposed to
receive me or not to-night. I went, and having found Bessie and
despatched her on my errand, I proceeded to take further measures. It
had heretofore been my habit always to shrink from arrogance: received
as I had been to-day, I should, a year ago, have resolved to quit
Gateshead the very next morning; now, it was disclosed to me all at once
that that would be a foolish plan. I had taken a journey of a hundred
miles to see my aunt, and I must stay with her till she was better---or
dead: as to her daughters' pride or folly, I must put it on one side,
make myself independent of it. So I addressed the housekeeper; asked
her to show me a room, told her I should probably be a visitor here for
a week or two, had my trunk conveyed to my chamber, and followed it
thither myself: I met Bessie on the landing.

\enquote{Missis is awake,} said she; \enquote{I have told her you are
	here: come and let us see if she will know you.}

I did not need to be guided to the well-known room, to which I had so
often been summoned for chastisement or reprimand in former days. I
hastened before Bessie; I softly opened the door: a shaded light stood
on the table, for it was now getting dark. There was the great
four-post bed with amber hangings as of old; there the toilet-table, the
armchair, and the footstool, at which I had a hundred times been
sentenced to kneel, to ask pardon for offences by me uncommitted. I
looked into a certain corner near, half-expecting to see the slim
outline of a once dreaded switch which used to lurk there, waiting to
leap out imp-like and lace my quivering palm or shrinking neck. I
approached the bed; I opened the curtains and leant over the high-piled
pillows.

Well did I remember \Mrs{} Reed's face, and I eagerly sought the familiar
image. It is a happy thing that time quells the longings of vengeance
and hushes the promptings of rage and aversion. I had left this woman
in bitterness and hate, and I came back to her now with no other emotion
than a sort of ruth for her great sufferings, and a strong yearning to
forget and forgive all injuries---to be reconciled and clasp hands in
amity.

The well-known face was there: stern, relentless as ever---there was
that peculiar eye which nothing could melt, and the somewhat raised,
imperious, despotic eyebrow. How often had it lowered on me menace and
hate! and how the recollection of childhood's terrors and sorrows
revived as I traced its harsh line now! And yet I stooped down and
kissed her: she looked at me.

\enquote{Is this Jane Eyre?} she said.

\enquote{Yes, Aunt Reed. How are you, dear aunt?}

I had once vowed that I would never call her aunt again: I thought it no
sin to forget and break that vow now. My fingers had fastened on her
hand which lay outside the sheet: had she pressed mine kindly, I should
at that moment have experienced true pleasure. But unimpressionable
natures are not so soon softened, nor are natural antipathies so readily
eradicated. \Mrs{} Reed took her hand away, and, turning her face rather
from me, she remarked that the night was warm. Again she regarded me so
icily, I felt at once that her opinion of me---her feeling towards
me---was unchanged and unchangeable. I knew by her stony eye---opaque
to tenderness, indissoluble to tears---that she was resolved to consider
me bad to the last; because to believe me good would give her no
generous pleasure: only a sense of mortification.

I felt pain, and then I felt ire; and then I felt a determination to
subdue her---to be her mistress in spite both of her nature and her
will. My tears had risen, just as in childhood: I ordered them back to
their source. I brought a chair to the bed-head: I sat down and leaned
over the pillow.

\enquote{You sent for me,} I said, \enquote{and I am here; and it is my
	intention to stay till I see how you get on.}

\enquote{Oh, of course! You have seen my daughters?}

\enquote{Yes.}

\enquote{Well, you may tell them I wish you to stay till I can talk some
	things over with you I have on my mind: to-night it is too late, and I
	have a difficulty in recalling them. But there was something I wished
	to say---let me see---}

The wandering look and changed utterance told what wreck had taken place
in her once vigorous frame. Turning restlessly, she drew the bedclothes
round her; my elbow, resting on a corner of the quilt, fixed it down:
she was at once irritated.

\enquote{Sit up!} said she; \enquote{don't annoy me with holding the
	clothes fast. Are you Jane Eyre?}

\enquote{I am Jane Eyre.}

\enquote{I have had more trouble with that child than any one would
	believe. Such a burden to be left on my hands---and so much annoyance
	as she caused me, daily and hourly, with her incomprehensible
	disposition, and her sudden starts of temper, and her continual,
	unnatural watchings of one's movements! I declare she talked to me once
	like something mad, or like a fiend---no child ever spoke or looked as
	she did; I was glad to get her away from the house. What did they do
	with her at Lowood? The fever broke out there, and many of the pupils
	died. She, however, did not die: but I said she did---I wish she had
	died!}

\enquote{A strange wish, \Mrs{} Reed; why do you hate her so?}

\enquote{I had a dislike to her mother always; for she was my husband's
	only sister, and a great favourite with him: he opposed the family's
	disowning her when she made her low marriage; and when news came of her
	death, he wept like a simpleton. He would send for the baby; though I
	entreated him rather to put it out to nurse and pay for its
	maintenance. I hated it the first time I set my eyes on it---a sickly,
	whining, pining thing! It would wail in its cradle all night long---not
	screaming heartily like any other child, but whimpering and moaning.
	Reed pitied it; and he used to nurse it and notice it as if it had been
	his own: more, indeed, than he ever noticed his own at that age. He
	would try to make my children friendly to the little beggar: the
	darlings could not bear it, and he was angry with them when they showed
	their dislike. In his last illness, he had it brought continually to
	his bedside; and but an hour before he died, he bound me by vow to keep
	the creature. I would as soon have been charged with a pauper brat out
	of a workhouse: but he was weak, naturally weak. John does not at all
	resemble his father, and I am glad of it: John is like me and like my
	brothers---he is quite a Gibson. Oh, I wish he would cease tormenting
	me with letters for money? I have no more money to give him: we are
	getting poor. I must send away half the servants and shut up part of
	the house; or let it off. I can never submit to do that---yet how are
	we to get on? Two-thirds of my income goes in paying the interest of
	mortgages. John gambles dreadfully, and always loses---poor boy! He is
	beset by sharpers: John is sunk and degraded---his look is frightful---I
	feel ashamed for him when I see him.}

She was getting much excited. \enquote{I think I had better leave her
	now,} said I to Bessie, who stood on the other side of the bed.

\enquote{Perhaps you had, Miss: but she often talks in this way towards
	night---in the morning she is calmer.}

I rose. \enquote{Stop!} exclaimed \Mrs{} Reed, \enquote{there is another
	thing I wished to say. He threatens me---he continually threatens me
	with his own death, or mine: and I dream sometimes that I see him laid
	out with a great wound in his throat, or with a swollen and blackened
	face. I am come to a strange pass: I have heavy troubles. What is to
	be done? How is the money to be had?}

Bessie now endeavoured to persuade her to take a sedative draught: she
succeeded with difficulty. Soon after, \Mrs{} Reed grew more composed,
and sank into a dozing state. I then left her.

More than ten days elapsed before I had again any conversation with
her. She continued either delirious or lethargic; and the doctor
forbade everything which could painfully excite her. Meantime, I got on
as well as I could with Georgiana and Eliza. They were very cold,
indeed, at first. Eliza would sit half the day sewing, reading, or
writing, and scarcely utter a word either to me or her sister.
Georgiana would chatter nonsense to her canary bird by the hour, and
take no notice of me. But I was determined not to seem at a loss for
occupation or amusement: I had brought my drawing materials with me, and
they served me for both.

Provided with a case of pencils, and some sheets of paper, I used to
take a seat apart from them, near the window, and busy myself in
sketching fancy vignettes, representing any scene that happened
momentarily to shape itself in the ever-shifting kaleidoscope of
imagination: a glimpse of sea between two rocks; the rising moon, and a
ship crossing its disk; a group of reeds and water-flags, and a naiad's
head, crowned with lotus-flowers, rising out of them; an elf sitting in
a hedge-sparrow's nest, under a wreath of hawthorn-bloom.

One morning I fell to sketching a face: what sort of a face it was to
be, I did not care or know. I took a soft black pencil, gave it a broad
point, and worked away. Soon I had traced on the paper a broad and
prominent forehead and a square lower outline of visage: that contour
gave me pleasure; my fingers proceeded actively to fill it with
features. Strongly-marked horizontal eyebrows must be traced under that
brow; then followed, naturally, a well-defined nose, with a straight
ridge and full nostrils; then a flexible-looking mouth, by no means
narrow; then a firm chin, with a decided cleft down the middle of it: of
course, some black whiskers were wanted, and some jetty hair, tufted on
the temples, and waved above the forehead. Now for the eyes: I had left
them to the last, because they required the most careful working. I
drew them large; I shaped them well: the eyelashes I traced long and
sombre; the irids lustrous and large. \enquote{Good! but not quite the
	thing,} I thought, as I surveyed the effect: \enquote{they want more
	force and spirit;} and I wrought the shades blacker, that the lights
might flash more brilliantly---a happy touch or two secured success.
There, I had a friend's face under my gaze; and what did it signify that
those young ladies turned their backs on me? I looked at it; I smiled
at the speaking likeness: I was absorbed and content.

\enquote{Is that a portrait of some one you know?} asked Eliza, who had
approached me unnoticed. I responded that it was merely a fancy head,
and hurried it beneath the other sheets. Of course, I lied: it was, in
fact, a very faithful representation of \Mr{} Rochester. But what was
that to her, or to any one but myself? Georgiana also advanced to
look. The other drawings pleased her much, but she called that
\enquote{an ugly man.} They both seemed surprised at my skill. I
offered to sketch their portraits; and each, in turn, sat for a pencil
outline. Then Georgiana produced her album. I promised to contribute a
water-colour drawing: this put her at once into good humour. She
proposed a walk in the grounds. Before we had been out two hours, we
were deep in a confidential conversation: she had favoured me with a
description of the brilliant winter she had spent in London two seasons
ago---of the admiration she had there excited---the attention she had
received; and I even got hints of the titled conquest she had made. In
the course of the afternoon and evening these hints were enlarged on:
various soft conversations were reported, and sentimental scenes
represented; and, in short, a volume of a novel of fashionable life was
that day improvised by her for my benefit. The communications were
renewed from day to day: they always ran on the same theme---herself,
her loves, and woes. It was strange she never once adverted either to
her mother's illness, or her brother's death, or the present gloomy
state of the family prospects. Her mind seemed wholly taken up with
reminiscences of past gaiety, and aspirations after dissipations to
come. She passed about five minutes each day in her mother's sick-room,
and no more.

Eliza still spoke little: she had evidently no time to talk. I never
saw a busier person than she seemed to be; yet it was difficult to say
what she did: or rather, to discover any result of her diligence. She
had an alarm to call her up early. I know not how she occupied herself
before breakfast, but after that meal she divided her time into regular
portions, and each hour had its allotted task. Three times a day she
studied a little book, which I found, on inspection, was a Common Prayer
Book. I asked her once what was the great attraction of that volume,
and she said, \enquote{the Rubric.} Three hours she gave to stitching,
with gold thread, the border of a square crimson cloth, almost large
enough for a carpet. In answer to my inquiries after the use of this
article, she informed me it was a covering for the altar of a new church
lately erected near Gateshead. Two hours she devoted to her diary; two
to working by herself in the kitchen-garden; and one to the regulation
of her accounts. She seemed to want no company; no conversation. I
believe she was happy in her way: this routine sufficed for her; and
nothing annoyed her so much as the occurrence of any incident which
forced her to vary its clockwork regularity.

She told me one evening, when more disposed to be communicative than
usual, that John's conduct, and the threatened ruin of the family, had
been a source of profound affliction to her: but she had now, she said,
settled her mind, and formed her resolution. Her own fortune she had
taken care to secure; and when her mother died---and it was wholly
improbable, she tranquilly remarked, that she should either recover or
linger long---she would execute a long-cherished project: seek a
retirement where punctual habits would be permanently secured from
disturbance, and place safe barriers between herself and a frivolous
world. I asked if Georgiana would accompany her.

\enquote{Of course not. Georgiana and she had nothing in common: they
	never had had. She would not be burdened with her society for any
	consideration. Georgiana should take her own course; and she, Eliza,
	would take hers.}

Georgiana, when not unburdening her heart to me, spent most of her time
in lying on the sofa, fretting about the dulness of the house, and
wishing over and over again that her aunt Gibson would send her an
invitation up to town. \enquote{It would be so much better,} she said,
\enquote{if she could only get out of the way for a month or two, till
	all was over.} I did not ask what she meant by \enquote{all being
	over,} but I suppose she referred to the expected decease of her mother
and the gloomy sequel of funeral rites. Eliza generally took no more
notice of her sister's indolence and complaints than if no such
murmuring, lounging object had been before her. One day, however, as
she put away her account-book and unfolded her embroidery, she suddenly
took her up thus---

\enquote{Georgiana, a more vain and absurd animal than you was certainly
	never allowed to cumber the earth. You had no right to be born, for you
	make no use of life. Instead of living for, in, and with yourself, as a
	reasonable being ought, you seek only to fasten your feebleness on some
	other person's strength: if no one can be found willing to burden her or
	himself with such a fat, weak, puffy, useless thing, you cry out that
	you are ill-treated, neglected, miserable. Then, too, existence for you
	must be a scene of continual change and excitement, or else the world is
	a dungeon: you must be admired, you must be courted, you must be
	flattered---you must have music, dancing, and society---or you languish,
	you die away. Have you no sense to devise a system which will make you
	independent of all efforts, and all wills, but your own? Take one day;
	share it into sections; to each section apportion its task: leave no
	stray unemployed quarters of an hour, ten minutes, five
	minutes---include all; do each piece of business in its turn with
	method, with rigid regularity. The day will close almost before you are
	aware it has begun; and you are indebted to no one for helping you to
	get rid of one vacant moment: you have had to seek no one's company,
	conversation, sympathy, forbearance; you have lived, in short, as an
	independent being ought to do. Take this advice: the first and last I
	shall offer you; then you will not want me or any one else, happen what
	may. Neglect it---go on as heretofore, craving, whining, and
	idling---and suffer the results of your idiocy, however bad and
	insuperable they may be. I tell you this plainly; and listen: for
	though I shall no more repeat what I am now about to say, I shall
	steadily act on it. After my mother's death, I wash my hands of you:
	from the day her coffin is carried to the vault in Gateshead Church, you
	and I will be as separate as if we had never known each other. You need
	not think that because we chanced to be born of the same parents, I
	shall suffer you to fasten me down by even the feeblest claim: I can
	tell you this---if the whole human race, ourselves excepted, were swept
	away, and we two stood alone on the earth, I would leave you in the old
	world, and betake myself to the new.}

She closed her lips.

\enquote{You might have spared yourself the trouble of delivering that
	tirade,} answered Georgiana. \enquote{Everybody knows you are the most
	selfish, heartless creature in existence: and \emph{I} know your
	spiteful hatred towards me: I have had a specimen of it before in the
	trick you played me about Lord Edwin Vere: you could not bear me to be
	raised above you, to have a title, to be received into circles where you
	dare not show your face, and so you acted the spy and informer, and
	ruined my prospects for ever.} Georgiana took out her handkerchief and
blew her nose for an hour afterwards; Eliza sat cold, impassable, and
assiduously industrious.

True, generous feeling is made small account of by some, but here were
two natures rendered, the one intolerably acrid, the other despicably
savourless for the want of it. Feeling without judgment is a washy
draught indeed; but judgment untempered by feeling is too bitter and
husky a morsel for human deglutition.

It was a wet and windy afternoon: Georgiana had fallen asleep on the
sofa over the perusal of a novel; Eliza was gone to attend a saint's-day
service at the new church---for in matters of religion she was a rigid
formalist: no weather ever prevented the punctual discharge of what she
considered her devotional duties; fair or foul, she went to church
thrice every Sunday, and as often on week-days as there were prayers.

I bethought myself to go upstairs and see how the dying woman sped, who
lay there almost unheeded: the very servants paid her but a remittent
attention: the hired nurse, being little looked after, would slip out of
the room whenever she could. Bessie was faithful; but she had her own
family to mind, and could only come occasionally to the hall. I found
the sick-room unwatched, as I had expected: no nurse was there; the
patient lay still, and seemingly lethargic; her livid face sunk in the
pillows: the fire was dying in the grate. I renewed the fuel,
re-arranged the bedclothes, gazed awhile on her who could not now gaze
on me, and then I moved away to the window.

The rain beat strongly against the panes, the wind blew tempestuously:
\enquote{One lies there,} I thought, \enquote{who will soon be beyond
	the war of earthly elements. Whither will that spirit---now struggling
	to quit its material tenement---flit when at length released?}

In pondering the great mystery, I thought of Helen Burns, recalled her
dying words---her faith---her doctrine of the equality of disembodied
souls. I was still listening in thought to her well-remembered
tones---still picturing her pale and spiritual aspect, her wasted face
and sublime gaze, as she lay on her placid deathbed, and whispered her
longing to be restored to her divine Father's bosom---when a feeble
voice murmured from the couch behind: \enquote{Who is that?}

I knew \Mrs{} Reed had not spoken for days: was she reviving? I went up
to her.

\enquote{It is I, Aunt Reed.}

\enquote{Who---I?} was her answer. \enquote{Who are you?} looking at me
with surprise and a sort of alarm, but still not wildly. \enquote{You
	are quite a stranger to me---where is Bessie?}

\enquote{She is at the lodge, aunt.}

\enquote{Aunt,} she repeated. \enquote{Who calls me aunt? You are not
	one of the Gibsons; and yet I know you---that face, and the eyes and
	forehead, are quiet familiar to me: you are like---why, you are like
	Jane Eyre!}

I said nothing: I was afraid of occasioning some shock by declaring my
identity.

\enquote{Yet,} said she, \enquote{I am afraid it is a mistake: my
	thoughts deceive me. I wished to see Jane Eyre, and I fancy a likeness
	where none exists: besides, in eight years she must be so changed.} I
now gently assured her that I was the person she supposed and desired me
to be: and seeing that I was understood, and that her senses were quite
collected, I explained how Bessie had sent her husband to fetch me from
Thornfield.

\enquote{I am very ill, I know,} she said ere long. \enquote{I was
	trying to turn myself a few minutes since, and find I cannot move a
	limb. It is as well I should ease my mind before I die: what we think
	little of in health, burdens us at such an hour as the present is to
	me. Is the nurse here? or is there no one in the room but you?}

I assured her we were alone.

\enquote{Well, I have twice done you a wrong which I regret now. One
	was in breaking the promise which I gave my husband to bring you up as
	my own child; the other---} she stopped. \enquote{After all, it is of
	no great importance, perhaps,} she murmured to herself: \enquote{and
	then I may get better; and to humble myself so to her is painful.}

She made an effort to alter her position, but failed: her face changed;
she seemed to experience some inward sensation---the precursor, perhaps,
of the last pang.

\enquote{Well, I must get it over. Eternity is before me: I had better
	tell her.---Go to my dressing-case, open it, and take out a letter you
	will see there.}

I obeyed her directions. \enquote{Read the letter,} she said.

It was short, and thus conceived:---

\begin{quote}
	\needspace{7\baselineskip}
	\enquote{\textsc{Madam,}---Will you have the goodness to send me the address of my
		niece, Jane Eyre, and to tell me how she is? It is my intention to
		write shortly and desire her to come to me at Madeira. Providence has
		blessed my endeavours to secure a competency; and as I am unmarried and
		childless, I wish to adopt her during my life, and bequeath her at my
		death whatever I may have to leave.---I am, Madam, \etc, \etc,

		% Flush the next paragraph to the right (csquotes issue)
		\setlength{\parfillskip}{0pt}
		\setlength{\parindent}{0pt}
		\everypar{\hfill}

		\textsc{John Eyre,} Madeira.} %rem enq
\end{quote}

It was dated three years back.

\enquote{Why did I never hear of this?} I asked.

\enquote{Because I disliked you too fixedly and thoroughly ever to lend
	a hand in lifting you to prosperity. I could not forget your conduct to
	me, Jane---the fury with which you once turned on me; the tone in which
	you declared you abhorred me the worst of anybody in the world; the
	unchildlike look and voice with which you affirmed that the very thought
	of me made you sick, and asserted that I had treated you with miserable
	cruelty. I could not forget my own sensations when you thus started up
	and poured out the venom of your mind: I felt fear as if an animal that
	I had struck or pushed had looked up at me with human eyes and cursed me
	in a man's voice.---Bring me some water! Oh, make haste!}

\enquote{Dear \Mrs{} Reed,} said I, as I offered her the draught she
required, \enquote{think no more of all this, let it pass away from your
	mind. Forgive me for my passionate language: I was a child then; eight,
	nine years have passed since that day.}

She heeded nothing of what I said; but when she had tasted the water and
drawn breath, she went on thus---

\enquote{I tell you I could not forget it; and I took my revenge: for
	you to be adopted by your uncle, and placed in a state of ease and
	comfort, was what I could not endure. I wrote to him; I said I was
	sorry for his disappointment, but Jane Eyre was dead: she had died of
	typhus fever at Lowood. Now act as you please: write and contradict my
	assertion---expose my falsehood as soon as you like. You were born, I
	think, to be my torment: my last hour is racked by the recollection of a
	deed which, but for you, I should never have been tempted to commit.}

\enquote{If you could but be persuaded to think no more of it, aunt, and
	to regard me with kindness and forgiveness}

\enquote{You have a very bad disposition,} said she, \enquote{and one to
	this day I feel it impossible to understand: how for nine years you
	could be patient and quiescent under any treatment, and in the tenth
	break out all fire and violence, I can never comprehend.}

\enquote{My disposition is not so bad as you think: I am passionate, but
	not vindictive. Many a time, as a little child, I should have been glad
	to love you if you would have let me; and I long earnestly to be
	reconciled to you now: kiss me, aunt.}

I approached my cheek to her lips: she would not touch it. She said I
oppressed her by leaning over the bed, and again demanded water. As I
laid her down---for I raised her and supported her on my arm while she
drank---I covered her ice-cold and clammy hand with mine: the feeble
fingers shrank from my touch---the glazing eyes shunned my gaze.

\enquote{Love me, then, or hate me, as you will,} I said at last,
\enquote{you have my full and free forgiveness: ask now for God's, and
	be at peace.}

Poor, suffering woman! it was too late for her to make now the effort to
change her habitual frame of mind: living, she had ever hated
me---dying, she must hate me still.

The nurse now entered, and Bessie followed. I yet lingered half-an-hour
longer, hoping to see some sign of amity: but she gave none. She was
fast relapsing into stupor; nor did her mind again rally: at twelve
o'clock that night she died. I was not present to close her eyes, nor
were either of her daughters. They came to tell us the next morning
that all was over. She was by that time laid out. Eliza and I went to
look at her: Georgiana, who had burst out into loud weeping, said she
dared not go. There was stretched Sarah Reed's once robust and active
frame, rigid and still: her eye of flint was covered with its cold lid;
her brow and strong traits wore yet the impress of her inexorable soul.
A strange and solemn object was that corpse to me. I gazed on it with
gloom and pain: nothing soft, nothing sweet, nothing pitying, or
hopeful, or subduing did it inspire; only a grating anguish for
\emph{her} woes---not \emph{my} loss---and a sombre tearless dismay at
the fearfulness of death in such a form.

Eliza surveyed her parent calmly. After a silence of some minutes she
observed---

\enquote{With her constitution she should have lived to a good old age:
	her life was shortened by trouble.} And then a spasm constricted her
mouth for an instant: as it passed away she turned and left the room,
and so did I\@. Neither of us had dropt a tear.

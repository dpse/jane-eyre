\FChapter{Chapter Six}{6}

\Lettrine{T}{he} \textsc{next day} commenced as before, getting up and dressing by rushlight;
but this morning we were obliged to dispense with the ceremony of
washing; the water in the pitchers was frozen.  A change had taken place
in the weather the preceding evening, and a keen north-east wind,
whistling through the crevices of our bedroom windows all night long,
had made us shiver in our beds, and turned the contents of the ewers to
ice.

Before the long hour and a half of prayers and Bible-reading was over, I
felt ready to perish with cold.  Breakfast-time came at last, and this
morning the porridge was not burnt; the quality was eatable, the
quantity small.  How small my portion seemed!  I wished it had been
doubled.

In the course of the day I was enrolled a member of the fourth class,
and regular tasks and occupations were assigned me: hitherto, I had only
been a spectator of the proceedings at Lowood; I was now to become an
actor therein.  At first, being little accustomed to learn by heart, the
lessons appeared to me both long and difficult; the frequent change from
task to task, too, bewildered me; and I was glad when, about three
o'clock in the afternoon, Miss Smith put into my hands a border of
muslin two yards long, together with needle, thimble, \etc, and sent me
to sit in a quiet corner of the schoolroom, with directions to hem the
same.  At that hour most of the others were sewing likewise; but one
class still stood round Miss Scatcherd's chair reading, and as all was
quiet, the subject of their lessons could be heard, together with the
manner in which each girl acquitted herself, and the animadversions or
commendations of Miss Scatcherd on the performance.  It was English
history: among the readers I observed my acquaintance of the verandah:
at the commencement of the lesson, her place had been at the top of the
class, but for some error of pronunciation, or some inattention to
stops, she was suddenly sent to the very bottom.  Even in that obscure
position, Miss Scatcherd continued to make her an object of constant
notice: she was continually addressing to her such phrases as the
following:---

\enquote{Burns} (such it seems was her name: the girls here were all
called by their surnames, as boys are elsewhere), \enquote{Burns, you
	are standing on the side of your shoe; turn your toes out immediately.}
\enquote{Burns, you poke your chin most unpleasantly; draw it in.}
\enquote{Burns, I insist on your holding your head up; I will not have
	you before me in that attitude,} \etc{} \etc.

A chapter having been read through twice, the books were closed and the
girls examined.  The lesson had comprised part of the reign of Charles
I., and there were sundry questions about tonnage and poundage and
ship-money, which most of them appeared unable to answer; still, every
little difficulty was solved instantly when it reached Burns: her memory
seemed to have retained the substance of the whole lesson, and she was
ready with answers on every point.  I kept expecting that Miss Scatcherd
would praise her attention; but, instead of that, she suddenly cried
out---

\enquote{You dirty, disagreeable girl! you have never cleaned your nails
	this morning!}

Burns made no answer: I wondered at her silence.  \enquote{Why,} thought
I, \enquote{does she not explain that she could neither clean her nails
	nor wash her face, as the water was frozen?}

My attention was now called off by Miss Smith desiring me to hold a
skein of thread: while she was winding it, she talked to me from time to
time, asking whether I had ever been at school before, whether I could
mark, stitch, knit, \etc; till she dismissed me, I could not pursue my
observations on Miss Scatcherd's movements.  When I returned to my seat,
that lady was just delivering an order of which I did not catch the
import; but Burns immediately left the class, and going into the small
inner room where the books were kept, returned in half a minute,
carrying in her hand a bundle of twigs tied together at one end.  This
ominous tool she presented to Miss Scatcherd with a respectful curtesy;
then she quietly, and without being told, unloosed her pinafore, and the
teacher instantly and sharply inflicted on her neck a dozen strokes with
the bunch of twigs.  Not a tear rose to Burns' eye; and, while I paused
from my sewing, because my fingers quivered at this spectacle with a
sentiment of unavailing and impotent anger, not a feature of her pensive
face altered its ordinary expression.

\enquote{Hardened girl!} exclaimed Miss Scatcherd; \enquote{nothing can
	correct you of your slatternly habits: carry the rod away.}

Burns obeyed: I looked at her narrowly as she emerged from the
book-closet; she was just putting back her handkerchief into her pocket,
and the trace of a tear glistened on her thin cheek.

The play-hour in the evening I thought the pleasantest fraction of the
day at Lowood: the bit of bread, the draught of coffee swallowed at five
o'clock had revived vitality, if it had not satisfied hunger: the long
restraint of the day was slackened; the schoolroom felt warmer than in
the morning---its fires being allowed to burn a little more brightly, to
supply, in some measure, the place of candles, not yet introduced: the
ruddy gloaming, the licensed uproar, the confusion of many voices gave
one a welcome sense of liberty.

On the evening of the day on which I had seen Miss Scatcherd flog her
pupil, Burns, I wandered as usual among the forms and tables and
laughing groups without a companion, yet not feeling lonely: when I
passed the windows, I now and then lifted a blind, and looked out; it
snowed fast, a drift was already forming against the lower panes;
putting my ear close to the window, I could distinguish from the gleeful
tumult within, the disconsolate moan of the wind outside.

Probably, if I had lately left a good home and kind parents, this would
have been the hour when I should most keenly have regretted the
separation; that wind would then have saddened my heart; this obscure
chaos would have disturbed my peace! as it was, I derived from both a
strange excitement, and reckless and feverish, I wished the wind to howl
more wildly, the gloom to deepen to darkness, and the confusion to rise
to clamour.

Jumping over forms, and creeping under tables, I made my way to one of
the fire-places; there, kneeling by the high wire fender, I found Burns,
absorbed, silent, abstracted from all round her by the companionship of
a book, which she read by the dim glare of the embers.

\enquote{Is it still \enquote{Rasselas}?} I asked, coming behind her.

\enquote{Yes,} she said, \enquote{and I have just finished it.}

And in five minutes more she shut it up.  I was glad of this.
\enquote{Now,} thought I, \enquote{I can perhaps get her to talk.}  I
sat down by her on the floor.

\enquote{What is your name besides Burns?}

\enquote{Helen.}

\enquote{Do you come a long way from here?}

\enquote{I come from a place farther north, quite on the borders of
	Scotland.}

\enquote{Will you ever go back?}

\enquote{I hope so; but nobody can be sure of the future.}

\enquote{You must wish to leave Lowood?}

\enquote{No! why should I?  I was sent to Lowood to get an education;
	and it would be of no use going away until I have attained that object.}

\enquote{But that teacher, Miss Scatcherd, is so cruel to you?}

\enquote{Cruel?  Not at all!  She is severe: she dislikes my faults.}

\enquote{And if I were in your place I should dislike her; I should
	resist her.  If she struck me with that rod, I should get it from her
	hand; I should break it under her nose.}

\enquote{Probably you would do nothing of the sort: but if you did, \Mr{}
	Brocklehurst would expel you from the school; that would be a great
	grief to your relations.  It is far better to endure patiently a smart
	which nobody feels but yourself, than to commit a hasty action whose
	evil consequences will extend to all connected with you; and besides,
	the Bible bids us return good for evil.}

\enquote{But then it seems disgraceful to be flogged, and to be sent to
	stand in the middle of a room full of people; and you are such a great
	girl: I am far younger than you, and I could not bear it.}

\enquote{Yet it would be your duty to bear it, if you could not avoid it: it is
	weak and silly to say you \emph{cannot bear} what it is your fate to be
	required to bear.}

I heard her with wonder: I could not comprehend this doctrine of
endurance; and still less could I understand or sympathise with the
forbearance she expressed for her chastiser.  Still I felt that Helen
Burns considered things by a light invisible to my eyes.  I suspected
she might be right and I wrong; but I would not ponder the matter
deeply; like Felix, I put it off to a more convenient season.

\enquote{You say you have faults, Helen: what are they?  To me you seem
	very good.}

\enquote{Then learn from me, not to judge by appearances: I am, as Miss
	Scatcherd said, slatternly; I seldom put, and never keep, things, in
	order; I am careless; I forget rules; I read when I should learn my
	lessons; I have no method; and sometimes I say, like you, I cannot
	\emph{bear} to be subjected to systematic arrangements.  This is all
	very provoking to Miss Scatcherd, who is naturally neat, punctual, and
	particular.}

\enquote{And cross and cruel,} I added; but Helen Burns would not admit
my addition: she kept silence.

\enquote{Is Miss Temple as severe to you as Miss Scatcherd?}

At the utterance of Miss Temple's name, a soft smile flitted over her
grave face.

\enquote{Miss Temple is full of goodness; it pains her to be severe to
	any one, even the worst in the school: she sees my errors, and tells me
	of them gently; and, if I do anything worthy of praise, she gives me my
	meed liberally.  One strong proof of my wretchedly defective nature is,
	that even her expostulations, so mild, so rational, have not influence
	to cure me of my faults; and even her praise, though I value it most
	highly, cannot stimulate me to continued care and foresight.}

\enquote{That is curious,} said I, \enquote{it is so easy to be
	careful.}

\enquote{For \emph{you} I have no doubt it is.  I observed you in your class
	this morning, and saw you were closely attentive: your thoughts never
	seemed to wander while Miss Miller explained the lesson and questioned
	you.  Now, mine continually rove away; when I should be listening to
	Miss Scatcherd, and collecting all she says with assiduity, often I lose
	the very sound of her voice; I fall into a sort of dream.  Sometimes I
	think I am in Northumberland, and that the noises I hear round me are
	the bubbling of a little brook which runs through Deepden, near our
	house;---then, when it comes to my turn to reply, I have to be awakened;
	and having heard nothing of what was read for listening to the visionary
	brook, I have no answer ready.}

\enquote{Yet how well you replied this afternoon.}

\enquote{It was mere chance; the subject on which we had been reading
	had interested me.  This afternoon, instead of dreaming of Deepden, I
	was wondering how a man who wished to do right could act so unjustly and
	unwisely as Charles the First sometimes did; and I thought what a pity
	it was that, with his integrity and conscientiousness, he could see no
	farther than the prerogatives of the crown.  If he had but been able to
	look to a distance, and see how what they call the spirit of the age was
	tending!  Still, I like Charles---I respect him---I pity him, poor
	murdered king!  Yes, his enemies were the worst: they shed blood they
	had no right to shed.  How dared they kill him!}

Helen was talking to herself now: she had forgotten I could not very
well understand her---that I was ignorant, or nearly so, of the subject
she discussed.  I recalled her to my level.

\enquote{And when Miss Temple teaches you, do your thoughts wander
	then?}

\enquote{No, certainly, not often; because Miss Temple has generally
	something to say which is newer than my own reflections; her language is
	singularly agreeable to me, and the information she communicates is
	often just what I wished to gain.}

\enquote{Well, then, with Miss Temple you are good?}

\enquote{Yes, in a passive way: I make no effort; I follow as
	inclination guides me.  There is no merit in such goodness.}

\enquote{A great deal: you are good to those who are good to you.  It is
	all I ever desire to be.  If people were always kind and obedient to
	those who are cruel and unjust, the wicked people would have it all
	their own way: they would never feel afraid, and so they would never
	alter, but would grow worse and worse.  When we are struck at without a
	reason, we should strike back again very hard; I am sure we should---so
	hard as to teach the person who struck us never to do it again.}

\enquote{You will change your mind, I hope, when you grow older: as yet
	you are but a little untaught girl.}

\enquote{But I feel this, Helen; I must dislike those who, whatever I do
	to please them, persist in disliking me; I must resist those who punish
	me unjustly.  It is as natural as that I should love those who show me
	affection, or submit to punishment when I feel it is deserved.}

\enquote{Heathens and savage tribes hold that doctrine, but Christians
	and civilised nations disown it.}

\enquote{How?  I don't understand.}

\enquote{It is not violence that best overcomes hate---nor vengeance
	that most certainly heals injury.}

\enquote{What then?}

\enquote{Read the New Testament, and observe what Christ says, and how
	He acts; make His word your rule, and His conduct your example.}

\enquote{What does He say?}

\enquote{Love your enemies; bless them that curse you; do good to them
	that hate you and despitefully use you.}

\enquote{Then I should love \Mrs{} Reed, which I cannot do; I should bless
	her son John, which is impossible.}

In her turn, Helen Burns asked me to explain, and I proceeded forthwith
to pour out, in my own way, the tale of my sufferings and resentments.
Bitter and truculent when excited, I spoke as I felt, without reserve or
softening.

Helen heard me patiently to the end: I expected she would then make a
remark, but she said nothing.

\enquote{Well,} I asked impatiently, \enquote{is not \Mrs{} Reed a
	hard-hearted, bad woman?}

\enquote{She has been unkind to you, no doubt; because you see, she
	dislikes your cast of character, as Miss Scatcherd does mine; but how
	minutely you remember all she has done and said to you!  What a
	singularly deep impression her injustice seems to have made on your
	heart!  No ill-usage so brands its record on my feelings.  Would you not
	be happier if you tried to forget her severity, together with the
	passionate emotions it excited?  Life appears to me too short to be
	spent in nursing animosity or registering wrongs.  We are, and must be,
	one and all, burdened with faults in this world: but the time will soon
	come when, I trust, we shall put them off in putting off our corruptible
	bodies; when debasement and sin will fall from us with this cumbrous
	frame of flesh, and only the spark of the spirit will remain,---the
	impalpable principle of light and thought, pure as when it left the
	Creator to inspire the creature: whence it came it will return; perhaps
	again to be communicated to some being higher than man---perhaps to pass
	through gradations of glory, from the pale human soul to brighten to the
	seraph!  Surely it will never, on the contrary, be suffered to
	degenerate from man to fiend?  No; I cannot believe that: I hold another
	creed: which no one ever taught me, and which I seldom mention; but in
	which I delight, and to which I cling: for it extends hope to all: it
	makes Eternity a rest---a mighty home, not a terror and an abyss.
	Besides, with this creed, I can so clearly distinguish between the
	criminal and his crime; I can so sincerely forgive the first while I
	abhor the last: with this creed revenge never worries my heart,
	degradation never too deeply disgusts me, injustice never crushes me too
	low: I live in calm, looking to the end.}

Helen's head, always drooping, sank a little lower as she finished this
sentence.  I saw by her look she wished no longer to talk to me, but
rather to converse with her own thoughts.  She was not allowed much time
for meditation: a monitor, a great rough girl, presently came up,
exclaiming in a strong Cumberland accent---

\enquote{Helen Burns, if you don't go and put your drawer in order, and
	fold up your work this minute, I'll tell Miss Scatcherd to come and look
	at it!}

Helen sighed as her reverie fled, and getting up, obeyed the monitor
without reply as without delay.

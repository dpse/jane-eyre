\FChapter{Chapter Twenty-Seven}{27}

\Lettrine{S}{ome} \textsc{time} in the afternoon I raised my head, and looking round and
seeing the western sun gilding the sign of its decline on the wall, I
asked, \enquote{What am I to do?}

\zz
But the answer my mind gave---\enquote{Leave Thornfield at once}---was
so prompt, so dread, that I stopped my ears. I said I could not bear
such words now. \enquote{That I am not Edward Rochester's bride is the
	least part of my woe,} I alleged: \enquote{that I have wakened out of
	most glorious dreams, and found them all void and vain, is a horror I
	could bear and master; but that I must leave him decidedly, instantly,
	entirely, is intolerable. I cannot do it.}

But, then, a voice within me averred that I could do it and foretold
that I should do it. I wrestled with my own resolution: I wanted to be
weak that I might avoid the awful passage of further suffering I saw
laid out for me; and Conscience, turned tyrant, held Passion by the
throat, told her tauntingly, she had yet but dipped her dainty foot in
the slough, and swore that with that arm of iron he would thrust her
down to unsounded depths of agony.

\enquote{Let me be torn away,} then I cried. \enquote{Let another help
	me!}

\enquote{No; you shall tear yourself away, none shall help you: you
	shall yourself pluck out your right eye; yourself cut off your right
	hand: your heart shall be the victim, and you the priest to transfix
	it.}

I rose up suddenly, terror-struck at the solitude which so ruthless a
judge haunted,---at the silence which so awful a voice filled. My head
swam as I stood erect. I perceived that I was sickening from excitement
and inanition; neither meat nor drink had passed my lips that day, for I
had taken no breakfast. And, with a strange pang, I now reflected that,
long as I had been shut up here, no message had been sent to ask how I
was, or to invite me to come down: not even little Adèle had tapped at
the door; not even \Mrs{} Fairfax had sought me. \enquote{Friends always
	forget those whom fortune forsakes,} I murmured, as I undrew the bolt
and passed out. I stumbled over an obstacle: my head was still dizzy,
my sight was dim, and my limbs were feeble. I could not soon recover
myself. I fell, but not on to the ground: an outstretched arm caught
me. I looked up---I was supported by \Mr{} Rochester, who sat in a chair
across my chamber threshold.

\enquote{You come out at last,} he said. \enquote{Well, I have been
	waiting for you long, and listening: yet not one movement have I heard,
	nor one sob: five minutes more of that death-like hush, and I should
	have forced the lock like a burglar. So you shun me?---you shut
	yourself up and grieve alone! I would rather you had come and upbraided
	me with vehemence. You are passionate. I expected a scene of some
	kind. I was prepared for the hot rain of tears; only I wanted them to
	be shed on my breast: now a senseless floor has received them, or your
	drenched handkerchief. But I err: you have not wept at all! I see a
	white cheek and a faded eye, but no trace of tears. I suppose, then,
	your heart has been weeping blood?}

\enquote{Well, Jane! not a word of reproach? Nothing bitter---nothing
	poignant? Nothing to cut a feeling or sting a passion? You sit quietly
	where I have placed you, and regard me with a weary, passive look.}

\enquote{Jane, I never meant to wound you thus. If the man who had but
	one little ewe lamb that was dear to him as a daughter, that ate of his
	bread and drank of his cup, and lay in his bosom, had by some mistake
	slaughtered it at the shambles, he would not have rued his bloody
	blunder more than I now rue mine. Will you ever forgive me?}

Reader, I forgave him at the moment and on the spot. There was such
deep remorse in his eye, such true pity in his tone, such manly energy
in his manner; and besides, there was such unchanged love in his whole
look and mien---I forgave him all: yet not in words, not outwardly; only
at my heart's core.

\enquote{You know I am a scoundrel, Jane?} ere long he inquired
wistfully---wondering, I suppose, at my continued silence and tameness,
the result rather of weakness than of will.

\enquote{Yes, sir.}

\enquote{Then tell me so roundly and sharply---don't spare me.}

\enquote{I cannot: I am tired and sick. I want some water.} He heaved
a sort of shuddering sigh, and taking me in his arms, carried me
downstairs. At first I did not know to what room he had borne me; all
was cloudy to my glazed sight: presently I felt the reviving warmth of a
fire; for, summer as it was, I had become icy cold in my chamber. He
put wine to my lips; I tasted it and revived; then I ate something he
offered me, and was soon myself. I was in the library---sitting in his
chair---he was quite near. \enquote{If I could go out of life now,
	without too sharp a pang, it would be well for me,} I thought;
\enquote{then I should not have to make the effort of cracking my
	heart-strings in rending them from among \Mr{} Rochester's. I must leave
	him, it appears. I do not want to leave him---I cannot leave him.}

\enquote{How are you now, Jane?}

\enquote{Much better, sir; I shall be well soon.}

\enquote{Taste the wine again, Jane.}

I obeyed him; then he put the glass on the table, stood before me, and
looked at me attentively. Suddenly he turned away, with an inarticulate
exclamation, full of passionate emotion of some kind; he walked fast
through the room and came back; he stooped towards me as if to kiss me;
but I remembered caresses were now forbidden. I turned my face away and
put his aside.

\enquote{What!---How is this?} he exclaimed hastily. \enquote{Oh, I
	know! you won't kiss the husband of Bertha Mason? You consider my arms
	filled and my embraces appropriated?}

\enquote{At any rate, there is neither room nor claim for me, sir.}

\enquote{Why, Jane? I will spare you the trouble of much talking; I
	will answer for you---Because I have a wife already, you would
	reply.---I guess rightly?}

\enquote{Yes.}

\enquote{If you think so, you must have a strange opinion of me; you must
	regard me as a plotting profligate---a base and low rake who has been
	simulating disinterested love in order to draw you into a snare
	deliberately laid, and strip you of honour and rob you of self-respect.
	What do you say to that? I see you can say nothing in the first place,
	you are faint still, and have enough to do to draw your breath; in the
	second place, you cannot yet accustom yourself to accuse and revile me,
	and besides, the flood-gates of tears are opened, and they would rush
	out if you spoke much; and you have no desire to expostulate, to
	upbraid, to make a scene: you are thinking how \emph{to
		act}---\emph{talking} you consider is of no use. I know you---I am on
	my guard.}

\enquote{Sir, I do not wish to act against you,} I said; and my unsteady
voice warned me to curtail my sentence.

\enquote{Not in your sense of the word, but in mine you are scheming to
	destroy me. You have as good as said that I am a married man---as a
	married man you will shun me, keep out of my way: just now you have
	refused to kiss me. You intend to make yourself a complete stranger to
	me: to live under this roof only as Adèle's governess; if ever I say a
	friendly word to you, if ever a friendly feeling inclines you again to
	me, you will say,---\enquote{That man had nearly made me his mistress: I must be
		ice and rock to him;} and ice and rock you will accordingly become.}

I cleared and steadied my voice to reply: \enquote{All is changed about
	me, sir; I must change too---there is no doubt of that; and to avoid
	fluctuations of feeling, and continual combats with recollections and
	associations, there is only one way---Adèle must have a new governess,
	sir.}

\enquote{Oh, Adèle will go to school---I have settled that already; nor do I
	mean to torment you with the hideous associations and recollections of
	Thornfield Hall---this accursed place---this tent of Achan---this
	insolent vault, offering the ghastliness of living death to the light of
	the open sky---this narrow stone hell, with its one real fiend, worse
	than a legion of such as we imagine. Jane, you shall not stay here, nor
	will I\@. I was wrong ever to bring you to Thornfield Hall, knowing as I
	did how it was haunted. I charged them to conceal from you, before I
	ever saw you, all knowledge of the curse of the place; merely because I
	feared Adèle never would have a governess to stay if she knew with what
	inmate she was housed, and my plans would not permit me to remove the
	maniac elsewhere---though I possess an old house, Ferndean Manor, even
	more retired and hidden than this, where I could have lodged her safely
	enough, had not a scruple about the unhealthiness of the situation, in
	the heart of a wood, made my conscience recoil from the arrangement.
	Probably those damp walls would soon have eased me of her charge: but to
	each villain his own vice; and mine is not a tendency to indirect
	assassination, even of what I most hate.

	%rem enq
	Concealing the mad-woman's neighbourhood from you, however, was
	something like covering a child with a cloak and laying it down near a
	upas-tree: that demon's vicinage is poisoned, and always was. But I'll
	shut up Thornfield Hall: I'll nail up the front door and board the lower
	windows: I'll give \Mrs{} Poole two hundred a year to live here with
	\emph{my wife}, as you term that fearful hag: Grace will do much for
	money, and she shall have her son, the keeper at Grimsby Retreat, to
	bear her company and be at hand to give her aid in the paroxysms, when
	\emph{my wife} is prompted by her familiar to burn people in their beds
	at night, to stab them, to bite their flesh from their bones, and so
	on---}

\enquote{Sir,} I interrupted him, \enquote{you are inexorable for that
	unfortunate lady: you speak of her with hate---with vindictive
	antipathy. It is cruel---she cannot help being mad.}

\enquote{Jane, my little darling (so I will call you, for so you are),
	you don't know what you are talking about; you misjudge me again: it is
	not because she is mad I hate her. If you were mad, do you think I
	should hate you?}

\enquote{I do indeed, sir.}

\enquote{Then you are mistaken, and you know nothing about me, and
	nothing about the sort of love of which I am capable. Every atom of
	your flesh is as dear to me as my own: in pain and sickness it would
	still be dear. Your mind is my treasure, and if it were broken, it
	would be my treasure still: if you raved, my arms should confine you,
	and not a strait waistcoat---your grasp, even in fury, would have a
	charm for me: if you flew at me as wildly as that woman did this
	morning, I should receive you in an embrace, at least as fond as it
	would be restrictive. I should not shrink from you with disgust as I
	did from her: in your quiet moments you should have no watcher and no
	nurse but me; and I could hang over you with untiring tenderness, though
	you gave me no smile in return; and never weary of gazing into your
	eyes, though they had no longer a ray of recognition for me.---But why
	do I follow that train of ideas? I was talking of removing you from
	Thornfield. All, you know, is prepared for prompt departure: to-morrow
	you shall go. I only ask you to endure one more night under this roof,
	Jane; and then, farewell to its miseries and terrors for ever! I have a
	place to repair to, which will be a secure sanctuary from hateful
	reminiscences, from unwelcome intrusion---even from falsehood and
	slander.}

\enquote{And take Adèle with you, sir,} I interrupted; \enquote{she will
	be a companion for you.}

\enquote{What do you mean, Jane? I told you I would send Adèle to
	school; and what do I want with a child for a companion, and not my own
	child,---a French dancer's bastard? Why do you importune me about her!
	I say, why do you assign Adèle to me for a companion?}

\enquote{You spoke of a retirement, sir; and retirement and solitude are
	dull: too dull for you.}

\enquote{Solitude! solitude!} he reiterated with irritation. \enquote{I
	see I must come to an explanation. I don't know what sphynx-like
	expression is forming in your countenance. You are to share my
	solitude. Do you understand?}

I shook my head: it required a degree of courage, excited as he was
becoming, even to risk that mute sign of dissent. He had been walking
fast about the room, and he stopped, as if suddenly rooted to one spot.
He looked at me long and hard: I turned my eyes from him, fixed them on
the fire, and tried to assume and maintain a quiet, collected aspect.

\enquote{Now for the hitch in Jane's character,} he said at last,
speaking more calmly than from his look I had expected him to speak.
\enquote{The reel of silk has run smoothly enough so far; but I always
	knew there would come a knot and a puzzle: here it is. Now for
	vexation, and exasperation, and endless trouble! By God! I long to
	exert a fraction of Samson's strength, and break the entanglement like
	tow!}

He recommenced his walk, but soon again stopped, and this time just
before me.

\enquote{Jane! will you hear reason?} (he stooped and approached his
lips to my ear); \enquote{because, if you won't, I'll try violence.}
His voice was hoarse; his look that of a man who is just about to burst
an insufferable bond and plunge headlong into wild license. I saw that
in another moment, and with one impetus of frenzy more, I should be able
to do nothing with him. The present---the passing second of time---was
all I had in which to control and restrain him---a movement of
repulsion, flight, fear would have sealed my doom,---and his. But I was
not afraid: not in the least. I felt an inward power; a sense of
influence, which supported me. The crisis was perilous; but not without
its charm: such as the Indian, perhaps, feels when he slips over the
rapid in his canoe. I took hold of his clenched hand, loosened the
contorted fingers, and said to him, soothingly---

\enquote{Sit down; I'll talk to you as long as you like, and hear all
	you have to say, whether reasonable or unreasonable.}

He sat down: but he did not get leave to speak directly. I had been
struggling with tears for some time: I had taken great pains to repress
them, because I knew he would not like to see me weep. Now, however, I
considered it well to let them flow as freely and as long as they
liked. If the flood annoyed him, so much the better. So I gave way and
cried heartily.

Soon I heard him earnestly entreating me to be composed. I said I could
not while he was in such a passion.

\enquote{But I am not angry, Jane: I only love you too well; and you had
	steeled your little pale face with such a resolute, frozen look, I could
	not endure it. Hush, now, and wipe your eyes.}

His softened voice announced that he was subdued; so I, in my turn,
became calm. Now he made an effort to rest his head on my shoulder, but
I would not permit it. Then he would draw me to him: no.

\enquote{Jane! Jane!} he said, in such an accent of bitter sadness it
thrilled along every nerve I had; \enquote{you don't love me, then? It
	was only my station, and the rank of my wife, that you valued? Now that
	you think me disqualified to become your husband, you recoil from my
	touch as if I were some toad or ape.}

These words cut me: yet what could I do or I say? I ought probably to
have done or said nothing; but I was so tortured by a sense of remorse
at thus hurting his feelings, I could not control the wish to drop balm
where I had wounded.

\enquote{I \emph{do} love you,} I said, \enquote{more than ever: but I must
	not show or indulge the feeling: and this is the last time I must
	express it.}

\enquote{The last time, Jane! What! do you think you can live with me,
	and see me daily, and yet, if you still love me, be always cold and
	distant?}

\enquote{No, sir; that I am certain I could not; and therefore I see
	there is but one way: but you will be furious if I mention it.}

\enquote{Oh, mention it! If I storm, you have the art of weeping.}

\enquote{\Mr{} Rochester, I must leave you.}

\enquote{For how long, Jane? For a few minutes, while you smooth your
	hair---which is somewhat dishevelled; and bathe your face---which looks
	feverish?}

\enquote{I must leave Adèle and Thornfield. I must part with you for my
	whole life: I must begin a new existence among strange faces and strange
	scenes.}

\enquote{Of course: I told you you should. I pass over the madness
	about parting from me. You mean you must become a part of me. As to
	the new existence, it is all right: you shall yet be my wife: I am not
	married. You shall be \Mrs{} Rochester---both virtually and nominally. I
	shall keep only to you so long as you and I live. You shall go to a
	place I have in the south of France: a whitewashed villa on the shores
	of the Mediterranean. There you shall live a happy, and guarded, and
	most innocent life. Never fear that I wish to lure you into error---to
	make you my mistress. Why did you shake your head? Jane, you must be
	reasonable, or in truth I shall again become frantic.}

His voice and hand quivered: his large nostrils dilated; his eye blazed:
still I dared to speak.

\enquote{Sir, your wife is living: that is a fact acknowledged this
	morning by yourself. If I lived with you as you desire, I should then
	be your mistress: to say otherwise is sophistical---is false.}

\enquote{Jane, I am not a gentle-tempered man---you forget that: I am
	not long-enduring; I am not cool and dispassionate. Out of pity to me
	and yourself, put your finger on my pulse, feel how it throbs,
	and---beware!}

He bared his wrist, and offered it to me: the blood was forsaking his
cheek and lips, they were growing livid; I was distressed on all hands.
To agitate him thus deeply, by a resistance he so abhorred, was cruel:
to yield was out of the question. I did what human beings do
instinctively when they are driven to utter extremity---looked for aid
to one higher than man: the words \enquote{God help me!} burst
involuntarily from my lips.

\enquote{I am a fool!} cried \Mr{} Rochester suddenly. \enquote{I keep
	telling her I am not married, and do not explain to her why. I forget
	she knows nothing of the character of that woman, or of the
	circumstances attending my infernal union with her. Oh, I am certain
	Jane will agree with me in opinion, when she knows all that I know!
	Just put your hand in mine, Janet---that I may have the evidence of
	touch as well as sight, to prove you are near me---and I will in a few
	words show you the real state of the case. Can you listen to me?}

\enquote{Yes, sir; for hours if you will.}

\enquote{I ask only minutes. Jane, did you ever hear or know that I was
	not the eldest son of my house: that I had once a brother older than I?}

\enquote{I remember \Mrs{} Fairfax told me so once.}

\enquote{And did you ever hear that my father was an avaricious,
	grasping man?}

\enquote{I have understood something to that effect.}

\enquote{Well, Jane, being so, it was his resolution to keep the
	property together; he could not bear the idea of dividing his estate and
	leaving me a fair portion: all, he resolved, should go to my brother,
	Rowland. Yet as little could he endure that a son of his should be a
	poor man. I must be provided for by a wealthy marriage. He sought me a
	partner betimes. \Mr{} Mason, a West India planter and merchant, was his
	old acquaintance. He was certain his possessions were real and vast: he
	made inquiries. \Mr{} Mason, he found, had a son and daughter; and he
	learned from him that he could and would give the latter a fortune of
	thirty thousand pounds: that sufficed. When I left college, I was sent
	out to Jamaica, to espouse a bride already courted for me. My father
	said nothing about her money; but he told me Miss Mason was the boast of
	Spanish Town for her beauty: and this was no lie. I found her a fine
	woman, in the style of Blanche Ingram: tall, dark, and majestic. Her
	family wished to secure me because I was of a good race; and so did
	she. They showed her to me in parties, splendidly dressed. I seldom
	saw her alone, and had very little private conversation with her. She
	flattered me, and lavishly displayed for my pleasure her charms and
	accomplishments. All the men in her circle seemed to admire her and
	envy me. I was dazzled, stimulated: my senses were excited; and being
	ignorant, raw, and inexperienced, I thought I loved her. There is no
	folly so besotted that the idiotic rivalries of society, the prurience,
	the rashness, the blindness of youth, will not hurry a man to its
	commission. Her relatives encouraged me; competitors piqued me; she
	allured me: a marriage was achieved almost before I knew where I was.
	Oh, I have no respect for myself when I think of that act!---an agony of
	inward contempt masters me. I never loved, I never esteemed, I did not
	even know her. I was not sure of the existence of one virtue in her
	nature: I had marked neither modesty, nor benevolence, nor candour, nor
	refinement in her mind or manners---and, I married her:---gross,
	grovelling, mole-eyed blockhead that I was! With less sin I might
	have---But let me remember to whom I am speaking.}

\enquote{My bride's mother I had never seen: I understood she was dead.
	The honeymoon over, I learned my mistake; she was only mad, and shut up
	in a lunatic asylum. There was a younger brother, too---a complete dumb
	idiot. The elder one, whom you have seen (and whom I cannot hate,
	whilst I abhor all his kindred, because he has some grains of affection
	in his feeble mind, shown in the continued interest he takes in his
	wretched sister, and also in a dog-like attachment he once bore me),
	will probably be in the same state one day. My father and my brother
	Rowland knew all this; but they thought only of the thirty thousand
	pounds, and joined in the plot against me.}

\enquote{These were vile discoveries; but except for the treachery of
	concealment, I should have made them no subject of reproach to my wife,
	even when I found her nature wholly alien to mine, her tastes obnoxious
	to me, her cast of mind common, low, narrow, and singularly incapable of
	being led to anything higher, expanded to anything larger---when I found
	that I could not pass a single evening, nor even a single hour of the
	day with her in comfort; that kindly conversation could not be sustained
	between us, because whatever topic I started, immediately received from
	her a turn at once coarse and trite, perverse and imbecile---when I
	perceived that I should never have a quiet or settled household, because
	no servant would bear the continued outbreaks of her violent and
	unreasonable temper, or the vexations of her absurd, contradictory,
	exacting orders---even then I restrained myself: I eschewed upbraiding,
	I curtailed remonstrance; I tried to devour my repentance and disgust in
	secret; I repressed the deep antipathy I felt.

	%rem enq
	Jane, I will not trouble you with abominable details: some strong
	words shall express what I have to say. I lived with that woman
	upstairs four years, and before that time she had tried me indeed: her
	character ripened and developed with frightful rapidity; her vices
	sprang up fast and rank: they were so strong, only cruelty could check
	them, and I would not use cruelty. What a pigmy intellect she had, and
	what giant propensities! How fearful were the curses those propensities
	entailed on me! Bertha Mason, the true daughter of an infamous mother,
	dragged me through all the hideous and degrading agonies which must
	attend a man bound to a wife at once intemperate and unchaste.

	%rem enq
	My brother in the interval was dead, and at the end of the four years
	my father died too. I was rich enough now---yet poor to hideous
	indigence: a nature the most gross, impure, depraved I ever saw, was
	associated with mine, and called by the law and by society a part of
	me. And I could not rid myself of it by any legal proceedings: for the
	doctors now discovered that \emph{my wife} was mad---her excesses had
	prematurely developed the germs of insanity. Jane, you don't like my
	narrative; you look almost sick---shall I defer the rest to another
	day?}

\enquote{No, sir, finish it now; I pity you---I do earnestly pity you.}

\enquote{Pity, Jane, from some people is a noxious and insulting sort of
	tribute, which one is justified in hurling back in the teeth of those
	who offer it; but that is the sort of pity native to callous, selfish
	hearts; it is a hybrid, egotistical pain at hearing of woes, crossed
	with ignorant contempt for those who have endured them. But that is not
	your pity, Jane; it is not the feeling of which your whole face is full
	at this moment---with which your eyes are now almost overflowing---with
	which your heart is heaving---with which your hand is trembling in
	mine. Your pity, my darling, is the suffering mother of love: its
	anguish is the very natal pang of the divine passion. I accept it,
	Jane; let the daughter have free advent---my arms wait to receive her.}

\enquote{Now, sir, proceed; what did you do when you found she was mad?}

\enquote{Jane, I approached the verge of despair; a remnant of self-respect was
	all that intervened between me and the gulf. In the eyes of the world,
	I was doubtless covered with grimy dishonour; but I resolved to be clean
	in my own sight---and to the last I repudiated the contamination of her
	crimes, and wrenched myself from connection with her mental defects.
	Still, society associated my name and person with hers; I yet saw her
	and heard her daily: something of her breath (faugh!) mixed with the air
	I breathed; and besides, I remembered I had once been her husband---that
	recollection was then, and is now, inexpressibly odious to me; moreover,
	I knew that while she lived I could never be the husband of another and
	better wife; and, though five years my senior (her family and her father
	had lied to me even in the particular of her age), she was likely to
	live as long as I, being as robust in frame as she was infirm in mind.
	Thus, at the age of twenty-six, I was hopeless.

	%rem enq
	One night I had been awakened by her yells---(since the medical men
	had pronounced her mad, she had, of course, been shut up)---it was a
	fiery West Indian night; one of the description that frequently precede
	the hurricanes of those climates. Being unable to sleep in bed, I got
	up and opened the window. The air was like sulphur-steams---I could
	find no refreshment anywhere. Mosquitoes came buzzing in and hummed
	sullenly round the room; the sea, which I could hear from thence,
	rumbled dull like an earthquake---black clouds were casting up over it;
	the moon was setting in the waves, broad and red, like a hot
	cannon-ball---she threw her last bloody glance over a world quivering
	with the ferment of tempest. I was physically influenced by the
	atmosphere and scene, and my ears were filled with the curses the maniac
	still shrieked out; wherein she momentarily mingled my name with such a
	tone of demon-hate, with such language!---no professed harlot ever had a
	fouler vocabulary than she: though two rooms off, I heard every
	word---the thin partitions of the West India house opposing but slight
	obstruction to her wolfish cries.

	%rem enq
	\enquote{This life,} said I at last, \enquote{is hell: this is the
		air---those are the sounds of the bottomless pit! I have a right to
		deliver myself from it if I can. The sufferings of this mortal state
		will leave me with the heavy flesh that now cumbers my soul. Of the
		fanatic's burning eternity I have no fear: there is not a future state
		worse than this present one---let me break away, and go home to God!}

	%rem enq
	I said this whilst I knelt down at, and unlocked a trunk which
	contained a brace of loaded pistols: I mean to shoot myself. I only
	entertained the intention for a moment; for, not being insane, the
	crisis of exquisite and unalloyed despair, which had originated the wish
	and design of self-destruction, was past in a second.

	%rem enq
	A wind fresh from Europe blew over the ocean and rushed through the
	open casement: the storm broke, streamed, thundered, blazed, and the air
	grew pure. I then framed and fixed a resolution. While I walked under
	the dripping orange-trees of my wet garden, and amongst its drenched
	pomegranates and pine-apples, and while the refulgent dawn of the
	tropics kindled round me---I reasoned thus, Jane---and now listen; for
	it was true Wisdom that consoled me in that hour, and showed me the
	right path to follow.

	%rem enq
	The sweet wind from Europe was still whispering in the refreshed
	leaves, and the Atlantic was thundering in glorious liberty; my heart,
	dried up and scorched for a long time, swelled to the tone, and filled
	with living blood---my being longed for renewal---my soul thirsted for a
	pure draught. I saw hope revive---and felt regeneration possible. From
	a flowery arch at the bottom of my garden I gazed over the sea---bluer
	than the sky: the old world was beyond; clear prospects opened thus:---

	%rem enq
	\enquote{Go,} said Hope, \enquote{and live again in Europe: there it
		is not known what a sullied name you bear, nor what a filthy burden is
		bound to you. You may take the maniac with you to England; confine her
		with due attendance and precautions at Thornfield: then travel yourself
		to what clime you will, and form what new tie you like. That woman, who
		has so abused your long-suffering, so sullied your name, so outraged
		your honour, so blighted your youth, is not your wife, nor are you her
		husband. See that she is cared for as her condition demands, and you
		have done all that God and humanity require of you. Let her identity,
		her connection with yourself, be buried in oblivion: you are bound to
		impart them to no living being. Place her in safety and comfort:
		shelter her degradation with secrecy, and leave her.}

	%rem enq
	I acted precisely on this suggestion. My father and brother had not
	made my marriage known to their acquaintance; because, in the very first
	letter I wrote to apprise them of the union---having already begun to
	experience extreme disgust of its consequences, and, from the family
	character and constitution, seeing a hideous future opening to me---I
	added an urgent charge to keep it secret: and very soon the infamous
	conduct of the wife my father had selected for me was such as to make
	him blush to own her as his daughter-in-law. Far from desiring to
	publish the connection, he became as anxious to conceal it as myself.

	%rem enq
	To England, then, I conveyed her; a fearful voyage I had with
	such a monster in the vessel. Glad was I when I at last got her to
	Thornfield, and saw her safely lodged in that third-storey room, of
	whose secret inner cabinet she has now for ten years made a wild beast's
	den---a goblin's cell. I had some trouble in finding an attendant for
	her, as it was necessary to select one on whose fidelity dependence
	could be placed; for her ravings would inevitably betray my secret:
	besides, she had lucid intervals of days---sometimes weeks---which she
	filled up with abuse of me. At last I hired Grace Poole from the
	Grimbsy Retreat. She and the surgeon, Carter (who dressed Mason's
	wounds that night he was stabbed and worried), are the only two I have
	ever admitted to my confidence. \Mrs{} Fairfax may indeed have suspected
	something, but she could have gained no precise knowledge as to facts.
	Grace has, on the whole, proved a good keeper; though, owing partly to a
	fault of her own, of which it appears nothing can cure her, and which is
	incident to her harassing profession, her vigilance has been more than
	once lulled and baffled. The lunatic is both cunning and malignant; she
	has never failed to take advantage of her guardian's temporary lapses;
	once to secrete the knife with which she stabbed her brother, and twice
	to possess herself of the key of her cell, and issue therefrom in the
	night-time. On the first of these occasions, she perpetrated the
	attempt to burn me in my bed; on the second, she paid that ghastly visit
	to you. I thank Providence, who watched over you, that she then spent
	her fury on your wedding apparel, which perhaps brought back vague
	reminiscences of her own bridal days: but on what might have happened, I
	cannot endure to reflect. When I think of the thing which flew at my
	throat this morning, hanging its black and scarlet visage over the nest
	of my dove, my blood curdles---}

\enquote{And what, sir,} I asked, while he paused, \enquote{did you do
	when you had settled her here? Where did you go?}

\enquote{What did I do, Jane? I transformed myself into a
	will-o'-the-wisp. Where did I go? I pursued wanderings as wild as
	those of the March-spirit. I sought the Continent, and went devious
	through all its lands. My fixed desire was to seek and find a good and
	intelligent woman, whom I could love: a contrast to the fury I left at
	Thornfield---}

\enquote{But you could not marry, sir.}

\enquote{I had determined and was convinced that I could and ought. It
	was not my original intention to deceive, as I have deceived you. I
	meant to tell my tale plainly, and make my proposals openly: and it
	appeared to me so absolutely rational that I should be considered free
	to love and be loved, I never doubted some woman might be found willing
	and able to understand my case and accept me, in spite of the curse with
	which I was burdened.}

\enquote{Well, sir?}

\enquote{When you are inquisitive, Jane, you always make me smile. You
	open your eyes like an eager bird, and make every now and then a
	restless movement, as if answers in speech did not flow fast enough for
	you, and you wanted to read the tablet of one's heart. But before I go
	on, tell me what you mean by your \enquote{Well, sir?} It is a small
	phrase very frequent with you; and which many a time has drawn me on and
	on through interminable talk: I don't very well know why.}

\enquote{I mean,---What next? How did you proceed? What came of such
	an event?}

\enquote{Precisely! and what do you wish to know now?}

\enquote{Whether you found any one you liked: whether you asked her to
	marry you; and what she said.}

\enquote{I can tell you whether I found any one I liked, and whether I asked
	her to marry me: but what she said is yet to be recorded in the book of
	Fate. For ten long years I roved about, living first in one capital,
	then another: sometimes in \St{} Petersburg; oftener in Paris;
	occasionally in Rome, Naples, and Florence. Provided with plenty of
	money and the passport of an old name, I could choose my own society: no
	circles were closed against me. I sought my ideal of a woman amongst
	English ladies, French countesses, Italian signoras, and German
	gräfinnen. I could not find her. Sometimes, for a fleeting moment, I
	thought I caught a glance, heard a tone, beheld a form, which announced
	the realisation of my dream: but I was presently undeserved. You are
	not to suppose that I desired perfection, either of mind or person. I
	longed only for what suited me---for the antipodes of the Creole: and I
	longed vainly. Amongst them all I found not one whom, had I been ever
	so free, I---warned as I was of the risks, the horrors, the loathings of
	incongruous unions---would have asked to marry me. Disappointment made
	me reckless. I tried dissipation---never debauchery: that I hated, and
	hate. That was my Indian Messalina's attribute: rooted disgust at it
	and her restrained me much, even in pleasure. Any enjoyment that
	bordered on riot seemed to approach me to her and her vices, and I
	eschewed it.

	%rem enq
	Yet I could not live alone; so I tried the companionship of
	mistresses. The first I chose was Céline Varens---another of those
	steps which make a man spurn himself when he recalls them. You already
	know what she was, and how my liaison with her terminated. She had two
	successors: an Italian, Giacinta, and a German, Clara; both considered
	singularly handsome. What was their beauty to me in a few weeks?
	Giacinta was unprincipled and violent: I tired of her in three months.
	Clara was honest and quiet; but heavy, mindless, and unimpressible: not
	one whit to my taste. I was glad to give her a sufficient sum to set
	her up in a good line of business, and so get decently rid of her. But,
	Jane, I see by your face you are not forming a very favourable opinion
	of me just now. You think me an unfeeling, loose-principled rake: don't
	you?}

\enquote{I don't like you so well as I have done sometimes, indeed,
	sir. Did it not seem to you in the least wrong to live in that way,
	first with one mistress and then another? You talk of it as a mere
	matter of course.}

\enquote{It was with me; and I did not like it. It was a grovelling
	fashion of existence: I should never like to return to it. Hiring a
	mistress is the next worse thing to buying a slave: both are often by
	nature, and always by position, inferior: and to live familiarly with
	inferiors is degrading. I now hate the recollection of the time I
	passed with Céline, Giacinta, and Clara.}

I felt the truth of these words; and I drew from them the certain
inference, that if I were so far to forget myself and all the teaching
that had ever been instilled into me, as---under any pretext---with any
justification---through any temptation---to become the successor of
these poor girls, he would one day regard me with the same feeling which
now in his mind desecrated their memory. I did not give utterance to
this conviction: it was enough to feel it. I impressed it on my heart,
that it might remain there to serve me as aid in the time of trial.

\enquote{Now, Jane, why don't you say \enquote{Well, sir?} I have not done.
	You are looking grave. You disapprove of me still, I see. But let me
	come to the point. Last January, rid of all mistresses---in a harsh,
	bitter frame of mind, the result of a useless, roving, lonely
	life---corroded with disappointment, sourly disposed against all men,
	and especially against all womankind (for I began to regard the notion
	of an intellectual, faithful, loving woman as a mere dream), recalled by
	business, I came back to England.

	%rem enq
	On a frosty winter afternoon, I rode in sight of Thornfield Hall.
	Abhorred spot! I expected no peace---no pleasure there. On a stile in
	Hay Lane I saw a quiet little figure sitting by itself. I passed it as
	negligently as I did the pollard willow opposite to it: I had no
	presentiment of what it would be to me; no inward warning that the
	arbitress of my life---my genius for good or evil---waited there in
	humble guise. I did not know it, even when, on the occasion of
	Mesrour's accident, it came up and gravely offered me help. Childish
	and slender creature! It seemed as if a linnet had hopped to my foot
	and proposed to bear me on its tiny wing. I was surly; but the thing
	would not go: it stood by me with strange perseverance, and looked and
	spoke with a sort of authority. I must be aided, and by that hand: and
	aided I was.

	%rem enq
	When once I had pressed the frail shoulder, something new---a fresh
	sap and sense---stole into my frame. It was well I had learnt that this
	elf must return to me---that it belonged to my house down below---or I
	could not have felt it pass away from under my hand, and seen it vanish
	behind the dim hedge, without singular regret. I heard you come home
	that night, Jane, though probably you were not aware that I thought of
	you or watched for you. The next day I observed you---myself
	unseen---for half-an-hour, while you played with Adèle in the gallery.
	It was a snowy day, I recollect, and you could not go out of doors. I
	was in my room; the door was ajar: I could both listen and watch. Adèle
	claimed your outward attention for a while; yet I fancied your thoughts
	were elsewhere: but you were very patient with her, my little Jane; you
	talked to her and amused her a long time. When at last she left you,
	you lapsed at once into deep reverie: you betook yourself slowly to pace
	the gallery. Now and then, in passing a casement, you glanced out at
	the thick-falling snow; you listened to the sobbing wind, and again you
	paced gently on and dreamed. I think those day visions were not dark:
	there was a pleasurable illumination in your eye occasionally, a soft
	excitement in your aspect, which told of no bitter, bilious,
	hypochondriac brooding: your look revealed rather the sweet musings of
	youth when its spirit follows on willing wings the flight of Hope up and
	on to an ideal heaven. The voice of \Mrs{} Fairfax, speaking to a servant
	in the hall, wakened you: and how curiously you smiled to and at
	yourself, Janet! There was much sense in your smile: it was very
	shrewd, and seemed to make light of your own abstraction. It seemed to
	say---\enquote{My fine visions are all very well, but I must not forget they are
		absolutely unreal. I have a rosy sky and a green flowery Eden in my
		brain; but without, I am perfectly aware, lies at my feet a rough tract
		to travel, and around me gather black tempests to encounter.} You ran
	downstairs and demanded of \Mrs{} Fairfax some occupation: the weekly
	house accounts to make up, or something of that sort, I think it was. I
	was vexed with you for getting out of my sight.

	%rem eqn
	Impatiently I waited for evening, when I might summon you to my
	presence. An unusual---to me---a perfectly new character I suspected
	was yours: I desired to search it deeper and know it better. You
	entered the room with a look and air at once shy and independent: you
	were quaintly dressed---much as you are now. I made you talk: ere long
	I found you full of strange contrasts. Your garb and manner were
	restricted by rule; your air was often diffident, and altogether that of
	one refined by nature, but absolutely unused to society, and a good deal
	afraid of making herself disadvantageously conspicuous by some solecism
	or blunder; yet when addressed, you lifted a keen, a daring, and a
	glowing eye to your interlocutor's face: there was penetration and power
	in each glance you gave; when plied by close questions, you found ready
	and round answers. Very soon you seemed to get used to me: I believe
	you felt the existence of sympathy between you and your grim and cross
	master, Jane; for it was astonishing to see how quickly a certain
	pleasant ease tranquillised your manner: snarl as I would, you showed no
	surprise, fear, annoyance, or displeasure at my moroseness; you watched
	me, and now and then smiled at me with a simple yet sagacious grace I
	cannot describe. I was at once content and stimulated with what I saw:
	I liked what I had seen, and wished to see more. Yet, for a long time,
	I treated you distantly, and sought your company rarely. I was an
	intellectual epicure, and wished to prolong the gratification of making
	this novel and piquant acquaintance: besides, I was for a while troubled
	with a haunting fear that if I handled the flower freely its bloom would
	fade---the sweet charm of freshness would leave it. I did not then know
	that it was no transitory blossom, but rather the radiant resemblance of
	one, cut in an indestructible gem. Moreover, I wished to see whether
	you would seek me if I shunned you---but you did not; you kept in the
	schoolroom as still as your own desk and easel; if by chance I met you,
	you passed me as soon, and with as little token of recognition, as was
	consistent with respect. Your habitual expression in those days, Jane,
	was a thoughtful look; not despondent, for you were not sickly; but not
	buoyant, for you had little hope, and no actual pleasure. I wondered
	what you thought of me, or if you ever thought of me, and resolved to
	find this out.

	%rem enq
	I resumed my notice of you. There was something glad in your
	glance, and genial in your manner, when you conversed: I saw you had a
	social heart; it was the silent schoolroom---it was the tedium of your
	life---that made you mournful. I permitted myself the delight of being
	kind to you; kindness stirred emotion soon: your face became soft in
	expression, your tones gentle; I liked my name pronounced by your lips
	in a grateful happy accent. I used to enjoy a chance meeting with you,
	Jane, at this time: there was a curious hesitation in your manner: you
	glanced at me with a slight trouble---a hovering doubt: you did not know
	what my caprice might be---whether I was going to play the master and be
	stern, or the friend and be benignant. I was now too fond of you often
	to simulate the first whim; and, when I stretched my hand out cordially,
	such bloom and light and bliss rose to your young, wistful features, I
	had much ado often to avoid straining you then and there to my heart.}

\enquote{Don't talk any more of those days, sir,} I interrupted,
furtively dashing away some tears from my eyes; his language was torture
to me; for I knew what I must do---and do soon---and all these
reminiscences, and these revelations of his feelings only made my work
more difficult.

\enquote{No, Jane,} he returned: \enquote{what necessity is there to
	dwell on the Past, when the Present is so much surer---the Future so
	much brighter?}

I shuddered to hear the infatuated assertion.

\enquote{You see now how the case stands---do you not?} he continued.
\enquote{After a youth and manhood passed half in unutterable misery and half
	in dreary solitude, I have for the first time found what I can truly
	love---I have found you. You are my sympathy---my better self---my good
	angel. I am bound to you with a strong attachment. I think you good,
	gifted, lovely: a fervent, a solemn passion is conceived in my heart; it
	leans to you, draws you to my centre and spring of life, wraps my
	existence about you, and, kindling in pure, powerful flame, fuses you
	and me in one.

	%rem enq
	It was because I felt and knew this, that I resolved to marry you. To
	tell me that I had already a wife is empty mockery: you know now that I
	had but a hideous demon. I was wrong to attempt to deceive you; but I
	feared a stubbornness that exists in your character. I feared early
	instilled prejudice: I wanted to have you safe before hazarding
	confidences. This was cowardly: I should have appealed to your
	nobleness and magnanimity at first, as I do now---opened to you plainly
	my life of agony---described to you my hunger and thirst after a higher
	and worthier existence---shown to you, not my \emph{resolution} (that
	word is weak), but my resistless \emph{bent} to love faithfully and
	well, where I am faithfully and well loved in return. Then I should
	have asked you to accept my pledge of fidelity and to give me yours.
	Jane---give it me now.}

A pause.

\enquote{Why are you silent, Jane?}

I was experiencing an ordeal: a hand of fiery iron grasped my vitals.
Terrible moment: full of struggle, blackness, burning! Not a human
being that ever lived could wish to be loved better than I was loved;
and him who thus loved me I absolutely worshipped: and I must renounce
love and idol. One drear word comprised my intolerable
duty---\enquote{Depart!}

\enquote{Jane, you understand what I want of you? Just this
	promise---\enquote{I will be yours, \Mr{} Rochester.}}

\enquote{\Mr{} Rochester, I will \emph{not} be yours.}

Another long silence.

\enquote{Jane!} recommenced he, with a gentleness that broke me down
with grief, and turned me stone-cold with ominous terror---for this
still voice was the pant of a lion rising---\enquote{Jane, do you mean
	to go one way in the world, and to let me go another?}

\enquote{I do.}

\enquote{Jane} (bending towards and embracing me), \enquote{do you mean
	it now?}

\enquote{I do.}

\enquote{And now?} softly kissing my forehead and cheek.

\enquote{I do,} extricating myself from restraint rapidly and
completely.

\enquote{Oh, Jane, this is bitter! This---this is wicked. It would not
	be wicked to love me.}

\enquote{It would to obey you.}

A wild look raised his brows---crossed his features: he rose; but he
forebore yet. I laid my hand on the back of a chair for support: I
shook, I feared---but I resolved.

\enquote{One instant, Jane. Give one glance to my horrible life when
	you are gone. All happiness will be torn away with you. What then is
	left? For a wife I have but the maniac upstairs: as well might you
	refer me to some corpse in yonder churchyard. What shall I do, Jane?
	Where turn for a companion and for some hope?}

\enquote{Do as I do: trust in God and yourself. Believe in heaven.
	Hope to meet again there.}

\enquote{Then you will not yield?}

\enquote{No.}

\enquote{Then you condemn me to live wretched and to die accursed?} His
voice rose.

\enquote{I advise you to live sinless, and I wish you to die tranquil.}

\enquote{Then you snatch love and innocence from me? You fling me back
	on lust for a passion---vice for an occupation?}

\enquote{\Mr{} Rochester, I no more assign this fate to you than I grasp
	at it for myself. We were born to strive and endure---you as well as I:
	do so. You will forget me before I forget you.}

\enquote{You make me a liar by such language: you sully my honour. I
	declared I could not change: you tell me to my face I shall change
	soon. And what a distortion in your judgment, what a perversity in your
	ideas, is proved by your conduct! Is it better to drive a
	fellow-creature to despair than to transgress a mere human law, no man
	being injured by the breach? for you have neither relatives nor
	acquaintances whom you need fear to offend by living with me?}

This was true: and while he spoke my very conscience and reason turned
traitors against me, and charged me with crime in resisting him. They
spoke almost as loud as Feeling: and that clamoured wildly.
\enquote{Oh, comply!} it said. \enquote{Think of his misery; think of his
	danger---look at his state when left alone; remember his headlong
	nature; consider the recklessness following on despair---soothe him;
	save him; love him; tell him you love him and will be his. Who in the
	world cares for \emph{you}? or who will be injured by what you do?}

Still indomitable was the reply---\enquote{\emph{I} care for myself. The more
	solitary, the more friendless, the more unsustained I am, the more I
	will respect myself. I will keep the law given by God; sanctioned by
	man. I will hold to the principles received by me when I was sane, and
	not mad---as I am now. Laws and principles are not for the times when
	there is no temptation: they are for such moments as this, when body and
	soul rise in mutiny against their rigour; stringent are they; inviolate
	they shall be. If at my individual convenience I might break them, what
	would be their worth? They have a worth---so I have always believed;
	and if I cannot believe it now, it is because I am insane---quite
	insane: with my veins running fire, and my heart beating faster than I
	can count its throbs. Preconceived opinions, foregone determinations,
	are all I have at this hour to stand by: there I plant my foot.}

I did. \Mr{} Rochester, reading my countenance, saw I had done so. His
fury was wrought to the highest: he must yield to it for a moment,
whatever followed; he crossed the floor and seized my arm and grasped my
waist. He seemed to devour me with his flaming glance: physically, I
felt, at the moment, powerless as stubble exposed to the draught and
glow of a furnace: mentally, I still possessed my soul, and with it the
certainty of ultimate safety. The soul, fortunately, has an
interpreter---often an unconscious, but still a truthful
interpreter---in the eye. My eye rose to his; and while I looked in his
fierce face I gave an involuntary sigh; his gripe was painful, and my
over-taxed strength almost exhausted.

\enquote{Never,} said he, as he ground his teeth, \enquote{never was
	anything at once so frail and so indomitable. A mere reed she feels in
	my hand!} (And he shook me with the force of his hold.) \enquote{I
	could bend her with my finger and thumb: and what good would it do if I
	bent, if I uptore, if I crushed her? Consider that eye: consider the
	resolute, wild, free thing looking out of it, defying me, with more than
	courage---with a stern triumph. Whatever I do with its cage, I cannot
	get at it---the savage, beautiful creature! If I tear, if I rend the
	slight prison, my outrage will only let the captive loose. Conqueror I
	might be of the house; but the inmate would escape to heaven before I
	could call myself possessor of its clay dwelling-place. And it is you,
	spirit---with will and energy, and virtue and purity---that I want: not
	alone your brittle frame. Of yourself you could come with soft flight
	and nestle against my heart, if you would: seized against your will, you
	will elude the grasp like an essence---you will vanish ere I inhale your
	fragrance. Oh! come, Jane, come!}

As he said this, he released me from his clutch, and only looked at me.
The look was far worse to resist than the frantic strain: only an idiot,
however, would have succumbed now. I had dared and baffled his fury; I
must elude his sorrow: I retired to the door.

\enquote{You are going, Jane?}

\enquote{I am going, sir.}

\enquote{You are leaving me?}

\enquote{Yes.}

\enquote{You will not come? You will not be my comforter, my rescuer?
	My deep love, my wild woe, my frantic prayer, are all nothing to you?}

What unutterable pathos was in his voice! How hard it was to reiterate
firmly, \enquote{I am going.}

\enquote{Jane!}

\enquote{\Mr{} Rochester!}

\enquote{Withdraw, then,---I consent; but remember, you leave me here in
	anguish. Go up to your own room; think over all I have said, and, Jane,
	cast a glance on my sufferings---think of me.}

He turned away; he threw himself on his face on the sofa. \enquote{Oh,
	Jane! my hope---my love---my life!} broke in anguish from his lips.
Then came a deep, strong sob.

I had already gained the door; but, reader, I walked back---walked back
as determinedly as I had retreated. I knelt down by him; I turned his
face from the cushion to me; I kissed his cheek; I smoothed his hair
with my hand.

\enquote{God bless you, my dear master!} I said. \enquote{God keep you
	from harm and wrong---direct you, solace you---reward you well for your
	past kindness to me.}

\enquote{Little Jane's love would have been my best reward,} he
answered; \enquote{without it, my heart is broken. But Jane will give
	me her love: yes---nobly, generously.}

Up the blood rushed to his face; forth flashed the fire from his eyes;
erect he sprang; he held his arms out; but I evaded the embrace, and at
once quitted the room.

\enquote{Farewell!} was the cry of my heart as I left him. Despair
added, \enquote{Farewell for ever!}

\plainbreak{1}
\fancybreak{\includegraphics[height=1em]{symbols/divider.pdf}} %TODO improve
\plainbreak{1}

That night I never thought to sleep; but a slumber fell on me as soon as
I lay down in bed. I was transported in thought to the scenes of
childhood: I dreamt I lay in the red-room at Gateshead; that the night
was dark, and my mind impressed with strange fears. The light that long
ago had struck me into syncope, recalled in this vision, seemed
glidingly to mount the wall, and tremblingly to pause in the centre of
the obscured ceiling. I lifted up my head to look: the roof resolved to
clouds, high and dim; the gleam was such as the moon imparts to vapours
she is about to sever. I watched her come---watched with the strangest
anticipation; as though some word of doom were to be written on her
disk. She broke forth as never moon yet burst from cloud: a hand first
penetrated the sable folds and waved them away; then, not a moon, but a
white human form shone in the azure, inclining a glorious brow
earthward. It gazed and gazed on me. It spoke to my spirit:
immeasurably distant was the tone, yet so near, it whispered in my
heart---

\enquote{My daughter, flee temptation.}

\enquote{Mother, I will.}

So I answered after I had waked from the trance-like dream. It was yet
night, but July nights are short: soon after midnight, dawn comes.
\enquote{It cannot be too early to commence the task I have to fulfil,}
thought I\@. I rose: I was dressed; for I had taken off nothing but my
shoes. I knew where to find in my drawers some linen, a locket, a
ring. In seeking these articles, I encountered the beads of a pearl
necklace \Mr{} Rochester had forced me to accept a few days ago. I left
that; it was not mine: it was the visionary bride's who had melted in
air. The other articles I made up in a parcel; my purse, containing
twenty shillings (it was all I had), I put in my pocket: I tied on my
straw bonnet, pinned my shawl, took the parcel and my slippers, which I
would not put on yet, and stole from my room.

\enquote{Farewell, kind \Mrs{} Fairfax!} I whispered, as I glided past her
door. \enquote{Farewell, my darling Adèle!} I said, as I glanced
towards the nursery. No thought could be admitted of entering to
embrace her. I had to deceive a fine ear: for aught I knew it might now
be listening.

I would have got past \Mr{} Rochester's chamber without a pause; but my
heart momentarily stopping its beat at that threshold, my foot was
forced to stop also. No sleep was there: the inmate was walking
restlessly from wall to wall; and again and again he sighed while I
listened. There was a heaven---a temporary heaven---in this room for
me, if I chose: I had but to go in and to say---

\enquote{\Mr{} Rochester, I will love you and live with you through life
	till death,} and a fount of rapture would spring to my lips. I thought
of this.

That kind master, who could not sleep now, was waiting with impatience
for day. He would send for me in the morning; I should be gone. He
would have me sought for: vainly. He would feel himself forsaken; his
love rejected: he would suffer; perhaps grow desperate. I thought of
this too. My hand moved towards the lock: I caught it back, and glided
on.

Drearily I wound my way downstairs: I knew what I had to do, and I did
it mechanically. I sought the key of the side-door in the kitchen; I
sought, too, a phial of oil and a feather; I oiled the key and the
lock. I got some water, I got some bread: for perhaps I should have to
walk far; and my strength, sorely shaken of late, must not break down.
All this I did without one sound. I opened the door, passed out, shut
it softly. Dim dawn glimmered in the yard. The great gates were closed
and locked; but a wicket in one of them was only latched. Through that
I departed: it, too, I shut; and now I was out of Thornfield.

A mile off, beyond the fields, lay a road which stretched in the
contrary direction to Millcote; a road I had never travelled, but often
noticed, and wondered where it led: thither I bent my steps. No
reflection was to be allowed now: not one glance was to be cast back;
not even one forward. Not one thought was to be given either to the
past or the future. The first was a page so heavenly sweet---so deadly
sad---that to read one line of it would dissolve my courage and break
down my energy. The last was an awful blank: something like the world
when the deluge was gone by.

I skirted fields, and hedges, and lanes till after sunrise. I believe
it was a lovely summer morning: I know my shoes, which I had put on when
I left the house, were soon wet with dew. But I looked neither to
rising sun, nor smiling sky, nor wakening nature. He who is taken out
to pass through a fair scene to the scaffold, thinks not of the flowers
that smile on his road, but of the block and axe-edge; of the
disseverment of bone and vein; of the grave gaping at the end: and I
thought of drear flight and homeless wandering---and oh! with agony I
thought of what I left. I could not help it. I thought of him now---in
his room---watching the sunrise; hoping I should soon come to say I
would stay with him and be his. I longed to be his; I panted to return:
it was not too late; I could yet spare him the bitter pang of
bereavement. As yet my flight, I was sure, was undiscovered. I could
go back and be his comforter---his pride; his redeemer from misery,
perhaps from ruin. Oh, that fear of his self-abandonment---far worse
than my abandonment---how it goaded me! It was a barbed arrow-head in
my breast; it tore me when I tried to extract it; it sickened me when
remembrance thrust it farther in. Birds began singing in brake and
copse: birds were faithful to their mates; birds were emblems of love.
What was I? In the midst of my pain of heart and frantic effort of
principle, I abhorred myself. I had no solace from self-approbation:
none even from self-respect. I had injured---wounded---left my master.
I was hateful in my own eyes. Still I could not turn, nor retrace one
step. God must have led me on. As to my own will or conscience,
impassioned grief had trampled one and stifled the other. I was weeping
wildly as I walked along my solitary way: fast, fast I went like one
delirious. A weakness, beginning inwardly, extending to the limbs,
seized me, and I fell: I lay on the ground some minutes, pressing my
face to the wet turf. I had some fear---or hope---that here I should
die: but I was soon up; crawling forwards on my hands and knees, and
then again raised to my feet---as eager and as determined as ever to
reach the road.

When I got there, I was forced to sit to rest me under the hedge; and
while I sat, I heard wheels, and saw a coach come on. I stood up and
lifted my hand; it stopped. I asked where it was going: the driver
named a place a long way off, and where I was sure \Mr{} Rochester had no
connections. I asked for what sum he would take me there; he said
thirty shillings; I answered I had but twenty; well, he would try to
make it do. He further gave me leave to get into the inside, as the
vehicle was empty: I entered, was shut in, and it rolled on its way.

Gentle reader, may you never feel what I then felt! May your eyes never
shed such stormy, scalding, heart-wrung tears as poured from mine. May
you never appeal to Heaven in prayers so hopeless and so agonised as in
that hour left my lips; for never may you, like me, dread to be the
instrument of evil to what you wholly love.

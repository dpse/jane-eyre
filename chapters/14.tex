\FChapter{Chapter Fourteen}{14}

\Lettrine{F}{or} \textsc{several subsequent days} I saw little of \Mr{} Rochester. In the
mornings he seemed much engaged with business, and, in the afternoon,
gentlemen from Millcote or the neighbourhood called, and sometimes
stayed to dine with him. When his sprain was well enough to admit of
horse exercise, he rode out a good deal; probably to return these
visits, as he generally did not come back till late at night.

During this interval, even Adèle was seldom sent for to his presence,
and all my acquaintance with him was confined to an occasional rencontre
in the hall, on the stairs, or in the gallery, when he would sometimes
pass me haughtily and coldly, just acknowledging my presence by a
distant nod or a cool glance, and sometimes bow and smile with
gentlemanlike affability. His changes of mood did not offend me,
because I saw that I had nothing to do with their alternation; the ebb
and flow depended on causes quite disconnected with me.

One day he had had company to dinner, and had sent for my portfolio; in
order, doubtless, to exhibit its contents: the gentlemen went away
early, to attend a public meeting at Millcote, as \Mrs{} Fairfax informed
me; but the night being wet and inclement, \Mr{} Rochester did not
accompany them. Soon after they were gone he rang the bell: a message
came that I and Adèle were to go downstairs. I brushed Adèle's hair and
made her neat, and having ascertained that I was myself in my usual
Quaker trim, where there was nothing to retouch---all being too close
and plain, braided locks included, to admit of disarrangement---we
descended, Adèle wondering whether the \foreignlanguage{french}{\emph{petit coffre}} was at length
come; for, owing to some mistake, its arrival had hitherto been
delayed. She was gratified: there it stood, a little carton, on the
table when we entered the dining-room. She appeared to know it by
instinct.

\foreignquote{french}{Ma boite! ma boite!}\footnote{\enquote{My box! my box!}} exclaimed she, running towards it.

\enquote{Yes, there is your \foreignquote{french}{boite} at last: take it into a
	corner, you genuine daughter of Paris, and amuse yourself with
	disembowelling it,} said the deep and rather sarcastic voice of \Mr{}
Rochester, proceeding from the depths of an immense easy-chair at the
fireside. \enquote{And mind,} he continued, \enquote{don't bother me
	with any details of the anatomical process, or any notice of the
	condition of the entrails: let your operation be conducted in silence:
	\foreignlanguage{french}{tiens-toi tranquille, enfant; comprends-tu?}\footnote{\enquote{Keep quiet, child; do you understand?}}}

Adèle seemed scarcely to need the warning---she had already retired to a
sofa with her treasure, and was busy untying the cord which secured the
lid. Having removed this impediment, and lifted certain silvery
envelopes of tissue paper, she merely exclaimed---

\foreignquote{french}{Oh ciel! Que c'est beau!}\footnote{\enquote{Oh heavens! How lovely!}} and then remained absorbed in
ecstatic contemplation.

\enquote{Is Miss Eyre there?} now demanded the master, half rising from
his seat to look round to the door, near which I still stood.

\enquote{Ah! well, come forward; be seated here.} He drew a chair near
his own. \enquote{I am not fond of the prattle of children,} he
continued; \enquote{for, old bachelor as I am, I have no pleasant associations
	connected with their lisp. It would be intolerable to me to pass a
	whole evening \emph{tête-à-tête} with a brat. Don't draw that chair
	farther off, Miss Eyre; sit down exactly where I placed it---if you
	please, that is. Confound these civilities! I continually forget
	them. Nor do I particularly affect simple-minded old ladies.
	By-the-bye, I must have mine in mind; it won't do to neglect her; she is
	a Fairfax, or wed to one; and blood is said to be thicker than water.}

He rang, and despatched an invitation to \Mrs{} Fairfax, who soon arrived,
knitting-basket in hand.

\enquote{Good evening, madam; I sent to you for a charitable purpose. I
	have forbidden Adèle to talk to me about her presents, and she is
	bursting with repletion: have the goodness to serve her as auditress and
	interlocutrice; it will be one of the most benevolent acts you ever
	performed.}

Adèle, indeed, no sooner saw \Mrs{} Fairfax, than she summoned her to her
sofa, and there quickly filled her lap with the porcelain, the ivory,
the waxen contents of her \foreignquote{french}{boite;} pouring out, meantime,
explanations and raptures in such broken English as she was mistress of.

\enquote{Now I have performed the part of a good host,} pursued \Mr{}
Rochester, \enquote{put my guests into the way of amusing each other, I
	ought to be at liberty to attend to my own pleasure. Miss Eyre, draw
	your chair still a little farther forward: you are yet too far back; I
	cannot see you without disturbing my position in this comfortable chair,
	which I have no mind to do.}

I did as I was bid, though I would much rather have remained somewhat in
the shade; but \Mr{} Rochester had such a direct way of giving orders, it
seemed a matter of course to obey him promptly.

We were, as I have said, in the dining-room: the lustre, which had been
lit for dinner, filled the room with a festal breadth of light; the
large fire was all red and clear; the purple curtains hung rich and
ample before the lofty window and loftier arch; everything was still,
save the subdued chat of Adèle (she dared not speak loud), and, filling
up each pause, the beating of winter rain against the panes.

\Mr{} Rochester, as he sat in his damask-covered chair, looked different
to what I had seen him look before; not quite so stern---much less
gloomy. There was a smile on his lips, and his eyes sparkled, whether
with wine or not, I am not sure; but I think it very probable. He was,
in short, in his after-dinner mood; more expanded and genial, and also
more self-indulgent than the frigid and rigid temper of the morning;
still he looked preciously grim, cushioning his massive head against the
swelling back of his chair, and receiving the light of the fire on his
granite-hewn features, and in his great, dark eyes; for he had great,
dark eyes, and very fine eyes, too---not without a certain change in
their depths sometimes, which, if it was not softness, reminded you, at
least, of that feeling.

He had been looking two minutes at the fire, and I had been looking the
same length of time at him, when, turning suddenly, he caught my gaze
fastened on his physiognomy.

\enquote{You examine me, Miss Eyre,} said he: \enquote{do you think me
	handsome?}

I should, if I had deliberated, have replied to this question by
something conventionally vague and polite; but the answer somehow
slipped from my tongue before I was aware---\enquote{No, sir.}

\enquote{Ah! By my word! there is something singular about you,} said
he: \enquote{you have the air of a little \emph{nonnette}; quaint, quiet,
	grave, and simple, as you sit with your hands before you, and your eyes
	generally bent on the carpet (except, by-the-bye, when they are directed
	piercingly to my face; as just now, for instance); and when one asks you
	a question, or makes a remark to which you are obliged to reply, you rap
	out a round rejoinder, which, if not blunt, is at least brusque. What
	do you mean by it?}

\enquote{Sir, I was too plain; I beg your pardon. I ought to have
	replied that it was not easy to give an impromptu answer to a question
	about appearances; that tastes mostly differ; and that beauty is of
	little consequence, or something of that sort.}

\enquote{You ought to have replied no such thing. Beauty of little
	consequence, indeed! And so, under pretence of softening the previous
	outrage, of stroking and soothing me into placidity, you stick a sly
	penknife under my ear! Go on: what fault do you find with me, pray? I
	suppose I have all my limbs and all my features like any other man?}

\enquote{\Mr{} Rochester, allow me to disown my first answer: I intended
	no pointed repartee: it was only a blunder.}

\enquote{Just so: I think so: and you shall be answerable for it.
	Criticise me: does my forehead not please you?}

He lifted up the sable waves of hair which lay horizontally over his
brow, and showed a solid enough mass of intellectual organs, but an
abrupt deficiency where the suave sign of benevolence should have risen.

\enquote{Now, ma'am, am I a fool?}

\enquote{Far from it, sir. You would, perhaps, think me rude if I
	inquired in return whether you are a philanthropist?}

\enquote{There again! Another stick of the penknife, when she pretended
	to pat my head: and that is because I said I did not like the society of
	children and old women (low be it spoken!). No, young lady, I am not a
	general philanthropist; but I bear a conscience;} and he pointed to the
prominences which are said to indicate that faculty, and which,
fortunately for him, were sufficiently conspicuous; giving, indeed, a
marked breadth to the upper part of his head: \enquote{and, besides, I
	once had a kind of rude tenderness of heart. When I was as old as you,
	I was a feeling fellow enough, partial to the unfledged, unfostered, and
	unlucky; but Fortune has knocked me about since: she has even kneaded me
	with her knuckles, and now I flatter myself I am hard and tough as an
	India-rubber ball; pervious, though, through a chink or two still, and
	with one sentient point in the middle of the lump. Yes: does that leave
	hope for me?}

\enquote{Hope of what, sir?}

\enquote{Of my final re-transformation from India-rubber back to flesh?}

\enquote{Decidedly he has had too much wine,} I thought; and I did not
know what answer to make to his queer question: how could I tell whether
he was capable of being re-transformed?

\enquote{You looked very much puzzled, Miss Eyre; and though you are not
	pretty any more than I am handsome, yet a puzzled air becomes you;
	besides, it is convenient, for it keeps those searching eyes of yours
	away from my physiognomy, and busies them with the worsted flowers of
	the rug; so puzzle on. Young lady, I am disposed to be gregarious and
	communicative to-night.}

With this announcement he rose from his chair, and stood, leaning his
arm on the marble mantelpiece: in that attitude his shape was seen
plainly as well as his face; his unusual breadth of chest,
disproportionate almost to his length of limb. I am sure most people
would have thought him an ugly man; yet there was so much unconscious
pride in his port; so much ease in his demeanour; such a look of
complete indifference to his own external appearance; so haughty a
reliance on the power of other qualities, intrinsic or adventitious, to
atone for the lack of mere personal attractiveness, that, in looking at
him, one inevitably shared the indifference, and, even in a blind,
imperfect sense, put faith in the confidence.

\enquote{I am disposed to be gregarious and communicative to-night,} he
repeated, \enquote{and that is why I sent for you: the fire and the
	chandelier were not sufficient company for me; nor would Pilot have
	been, for none of these can talk. Adèle is a degree better, but still
	far below the mark; \Mrs{} Fairfax ditto; you, I am persuaded, can suit me
	if you will: you puzzled me the first evening I invited you down here.
	I have almost forgotten you since: other ideas have driven yours from my
	head; but to-night I am resolved to be at ease; to dismiss what
	importunes, and recall what pleases. It would please me now to draw you
	out---to learn more of you---therefore speak.}

Instead of speaking, I smiled; and not a very complacent or submissive
smile either.

\enquote{Speak,} he urged.

\enquote{What about, sir?}

\enquote{Whatever you like. I leave both the choice of subject and the
	manner of treating it entirely to yourself.}

Accordingly I sat and said nothing: \enquote{If he expects me to talk
	for the mere sake of talking and showing off, he will find he has
	addressed himself to the wrong person,} I thought.

\enquote{You are dumb, Miss Eyre.}

I was dumb still. He bent his head a little towards me, and with a
single hasty glance seemed to dive into my eyes.

\enquote{Stubborn?} he said, \enquote{and annoyed. Ah! it is
	consistent. I put my request in an absurd, almost insolent form. Miss
	Eyre, I beg your pardon. The fact is, once for all, I don't wish to
	treat you like an inferior: that is} (correcting himself), \enquote{I claim
	only such superiority as must result from twenty years' difference in
	age and a century's advance in experience. This is legitimate, \emph{\foreignlanguage{french}{et
			j'y tiens,}}\footnote{\emph{\enquote{And I insist on it.}}}
	as Adèle would say; and it is by virtue of this superiority,
	and this alone, that I desire you to have the goodness to talk to me a
	little now, and divert my thoughts, which are galled with dwelling on
	one point---cankering as a rusty nail.}

He had deigned an explanation, almost an apology, and I did not feel
insensible to his condescension, and would not seem so.

\enquote{I am willing to amuse you, if I can, sir---quite willing; but I
	cannot introduce a topic, because how do I know what will interest you?
	Ask me questions, and I will do my best to answer them.}

\enquote{Then, in the first place, do you agree with me that I have a
	right to be a little masterful, abrupt, perhaps exacting, sometimes, on
	the grounds I stated, namely, that I am old enough to be your father,
	and that I have battled through a varied experience with many men of
	many nations, and roamed over half the globe, while you have lived
	quietly with one set of people in one house?}

\enquote{Do as you please, sir.}

\enquote{That is no answer; or rather it is a very irritating, because a
	very evasive one. Reply clearly.}

\enquote{I don't think, sir, you have a right to command me, merely
	because you are older than I, or because you have seen more of the world
	than I have; your claim to superiority depends on the use you have made
	of your time and experience.}

\enquote{Humph! Promptly spoken. But I won't allow that, seeing that
	it would never suit my case, as I have made an indifferent, not to say a
	bad, use of both advantages. Leaving superiority out of the question,
	then, you must still agree to receive my orders now and then, without
	being piqued or hurt by the tone of command. Will you?}

I smiled: I thought to myself \Mr{} Rochester \emph{is} peculiar---he
seems to forget that he pays me £30 per annum for receiving his orders.

\enquote{The smile is very well,} said he, catching instantly the
passing expression; \enquote{but speak too.}

\enquote{I was thinking, sir, that very few masters would trouble
	themselves to inquire whether or not their paid subordinates were piqued
	and hurt by their orders.}

\enquote{Paid subordinates! What! you are my paid subordinate, are
	you? Oh yes, I had forgotten the salary! Well then, on that mercenary
	ground, will you agree to let me hector a little?}

\enquote{No, sir, not on that ground; but, on the ground that you did
	forget it, and that you care whether or not a dependent is comfortable
	in his dependency, I agree heartily.}

\enquote{And will you consent to dispense with a great many conventional
	forms and phrases, without thinking that the omission arises from
	insolence?}

\enquote{I am sure, sir, I should never mistake informality for
	insolence: one I rather like, the other nothing free-born would submit
	to, even for a salary.}

\enquote{Humbug! Most things free-born will submit to anything for a
	salary; therefore, keep to yourself, and don't venture on generalities
	of which you are intensely ignorant. However, I mentally shake hands
	with you for your answer, despite its inaccuracy; and as much for the
	manner in which it was said, as for the substance of the speech; the
	manner was frank and sincere; one does not often see such a manner: no,
	on the contrary, affectation, or coldness, or stupid, coarse-minded
	misapprehension of one's meaning are the usual rewards of candour. Not
	three in three thousand raw school-girl-governesses would have answered
	me as you have just done. But I don't mean to flatter you: if you are
	cast in a different mould to the majority, it is no merit of yours:
	Nature did it. And then, after all, I go too fast in my conclusions:
	for what I yet know, you may be no better than the rest; you may have
	intolerable defects to counterbalance your few good points.}

\enquote{And so may you,} I thought. My eye met his as the idea crossed
my mind: he seemed to read the glance, answering as if its import had
been spoken as well as imagined---

\enquote{Yes, yes, you are right,} said he; \enquote{I have plenty of
	faults of my own: I know it, and I don't wish to palliate them, I assure
	you. God wot I need not be too severe about others; I have a past
	existence, a series of deeds, a colour of life to contemplate within my
	own breast, which might well call my sneers and censures from my
	neighbours to myself. I started, or rather (for like other defaulters,
	I like to lay half the blame on ill fortune and adverse circumstances)
	was thrust on to a wrong tack at the age of one-and-twenty, and have
	never recovered the right course since: but I might have been very
	different; I might have been as good as you---wiser---almost as
	stainless. I envy you your peace of mind, your clean conscience, your
	unpolluted memory. Little girl, a memory without blot or contamination
	must be an exquisite treasure---an inexhaustible source of pure
	refreshment: is it not?}

\enquote{How was your memory when you were eighteen, sir?}

\enquote{All right then; limpid, salubrious: no gush of bilge water had
	turned it to fetid puddle. I was your equal at eighteen---quite your
	equal. Nature meant me to be, on the whole, a good man, Miss Eyre; one
	of the better kind, and you see I am not so. You would say you don't
	see it; at least I flatter myself I read as much in your eye (beware,
	by-the-bye, what you express with that organ; I am quick at interpreting
	its language). Then take my word for it,---I am not a villain: you are
	not to suppose that---not to attribute to me any such bad eminence; but,
	owing, I verily believe, rather to circumstances than to my natural
	bent, I am a trite commonplace sinner, hackneyed in all the poor petty
	dissipations with which the rich and worthless try to put on life. Do
	you wonder that I avow this to you? Know, that in the course of your
	future life you will often find yourself elected the involuntary
	confidant of your acquaintances' secrets: people will instinctively find
	out, as I have done, that it is not your forte to tell of yourself, but
	to listen while others talk of themselves; they will feel, too, that you
	listen with no malevolent scorn of their indiscretion, but with a kind
	of innate sympathy; not the less comforting and encouraging because it
	is very unobtrusive in its manifestations.}

\enquote{How do you know?---how can you guess all this, sir?}

\enquote{I know it well; therefore I proceed almost as freely as if I
	were writing my thoughts in a diary. You would say, I should have been
	superior to circumstances; so I should---so I should; but you see I was
	not. When fate wronged me, I had not the wisdom to remain cool: I
	turned desperate; then I degenerated. Now, when any vicious simpleton
	excites my disgust by his paltry ribaldry, I cannot flatter myself that
	I am better than he: I am forced to confess that he and I are on a
	level. I wish I had stood firm---God knows I do! Dread remorse when
	you are tempted to err, Miss Eyre; remorse is the poison of life.}

\enquote{Repentance is said to be its cure, sir.}

\enquote{It is not its cure. Reformation may be its cure; and I could
	reform---I have strength yet for that---if---but where is the use of
	thinking of it, hampered, burdened, cursed as I am? Besides, since
	happiness is irrevocably denied me, I have a right to get pleasure out
	of life: and I \emph{will} get it, cost what it may.}

\enquote{Then you will degenerate still more, sir.}

\enquote{Possibly: yet why should I, if I can get sweet, fresh
	pleasure? And I may get it as sweet and fresh as the wild honey the bee
	gathers on the moor.}

\enquote{It will sting---it will taste bitter, sir.}

\enquote{How do you know?---you never tried it. How very serious---how
	very solemn you look: and you are as ignorant of the matter as this
	cameo head} (taking one from the mantelpiece). \enquote{You have no
	right to preach to me, you neophyte, that have not passed the porch of
	life, and are absolutely unacquainted with its mysteries.}

\enquote{I only remind you of your own words, sir: you said error
	brought remorse, and you pronounced remorse the poison of existence.}

\enquote{And who talks of error now? I scarcely think the notion that
	flittered across my brain was an error. I believe it was an inspiration
	rather than a temptation: it was very genial, very soothing---I know
	that. Here it comes again! It is no devil, I assure you; or if it be,
	it has put on the robes of an angel of light. I think I must admit so
	fair a guest when it asks entrance to my heart.}

\enquote{Distrust it, sir; it is not a true angel.}

\enquote{Once more, how do you know? By what instinct do you pretend to
	distinguish between a fallen seraph of the abyss and a messenger from
	the eternal throne---between a guide and a seducer?}

\enquote{I judged by your countenance, sir, which was troubled when you
	said the suggestion had returned upon you. I feel sure it will work you
	more misery if you listen to it.}

\enquote{Not at all---it bears the most gracious message in the world:
	for the rest, you are not my conscience-keeper, so don't make yourself
	uneasy. Here, come in, bonny wanderer!}

He said this as if he spoke to a vision, viewless to any eye but his
own; then, folding his arms, which he had half extended, on his chest,
he seemed to enclose in their embrace the invisible being.

\enquote{Now,} he continued, again addressing me, \enquote{I have
	received the pilgrim---a disguised deity, as I verily believe. Already
	it has done me good: my heart was a sort of charnel; it will now be a
	shrine.}

\enquote{To speak truth, sir, I don't understand you at all: I cannot
	keep up the conversation, because it has got out of my depth. Only one
	thing, I know: you said you were not as good as you should like to be,
	and that you regretted your own imperfection;---one thing I can
	comprehend: you intimated that to have a sullied memory was a perpetual
	bane. It seems to me, that if you tried hard, you would in time find it
	possible to become what you yourself would approve; and that if from
	this day you began with resolution to correct your thoughts and actions,
	you would in a few years have laid up a new and stainless store of
	recollections, to which you might revert with pleasure.}

\enquote{Justly thought; rightly said, Miss Eyre; and, at this moment, I
	am paving hell with energy.}

\enquote{Sir?}

\enquote{I am laying down good intentions, which I believe durable as
	flint. Certainly, my associates and pursuits shall be other than they
	have been.}

\enquote{And better?}

\enquote{And better---so much better as pure ore is than foul dross.
	You seem to doubt me; I don't doubt myself: I know what my aim is, what
	my motives are; and at this moment I pass a law, unalterable as that of
	the Medes and Persians, that both are right.}

\enquote{They cannot be, sir, if they require a new statute to legalise
	them.}

\enquote{They are, Miss Eyre, though they absolutely require a new
	statute: unheard-of combinations of circumstances demand unheard-of
	rules.}

\enquote{That sounds a dangerous maxim, sir; because one can see at once
	that it is liable to abuse.}

\enquote{Sententious sage! so it is: but I swear by my household gods
	not to abuse it.}

\enquote{You are human and fallible.}

\enquote{I am: so are you---what then?}

\enquote{The human and fallible should not arrogate a power with which
	the divine and perfect alone can be safely intrusted.}

\enquote{What power?}

\enquote{That of saying of any strange, unsanctioned line of
	action,---\enquote{Let it be right.}}

\enquote{\enquote{Let it be right}---the very words: you have
	pronounced them.}

\enquote{\emph{May} it be right then,} I said, as I rose, deeming it useless
to continue a discourse which was all darkness to me; and, besides,
sensible that the character of my interlocutor was beyond my
penetration; at least, beyond its present reach; and feeling the
uncertainty, the vague sense of insecurity, which accompanies a
conviction of ignorance.

\enquote{Where are you going?}

\enquote{To put Adèle to bed: it is past her bedtime.}

\enquote{You are afraid of me, because I talk like a Sphynx.}

\enquote{Your language is enigmatical, sir: but though I am bewildered,
	I am certainly not afraid.}

\enquote{You \emph{are} afraid---your self-love dreads a blunder.}

\enquote{In that sense I do feel apprehensive---I have no wish to talk
	nonsense.}

\enquote{If you did, it would be in such a grave, quiet manner, I should
	mistake it for sense. Do you never laugh, Miss Eyre? Don't trouble
	yourself to answer---I see you laugh rarely; but you can laugh very
	merrily: believe me, you are not naturally austere, any more than I am
	naturally vicious. The Lowood constraint still clings to you somewhat;
	controlling your features, muffling your voice, and restricting your
	limbs; and you fear in the presence of a man and a brother---or father,
	or master, or what you will---to smile too gaily, speak too freely, or
	move too quickly: but, in time, I think you will learn to be natural
	with me, as I find it impossible to be conventional with you; and then
	your looks and movements will have more vivacity and variety than they
	dare offer now. I see at intervals the glance of a curious sort of bird
	through the close-set bars of a cage: a vivid, restless, resolute
	captive is there; were it but free, it would soar cloud-high. You are
	still bent on going?}

\enquote{It has struck nine, sir.}

\enquote{Never mind,---wait a minute: Adèle is not ready to go to bed
	yet. My position, Miss Eyre, with my back to the fire, and my face to
	the room, favours observation. While talking to you, I have also
	occasionally watched Adèle (I have my own reasons for thinking her a
	curious study,---reasons that I may, nay, that I shall, impart to you
	some day). She pulled out of her box, about ten minutes ago, a little
	pink silk frock; rapture lit her face as she unfolded it; coquetry runs
	in her blood, blends with her brains, and seasons the marrow of her
	bones. \foreignquote{french}{Il faut que je l'essaie!} cried she, \foreignquote{french}{et à
		l'instant même!}\footnote{\enquote{I've got to try it on!} \textelp{} \enquote{this very instant!}}
	and she rushed out of the room. She is now with
	Sophie, undergoing a robing process: in a few minutes she will re-enter;
	and I know what I shall see,---a miniature of Céline Varens, as she used
	to appear on the boards at the rising of---But never mind that.
	However, my tenderest feelings are about to receive a shock: such is my
	presentiment; stay now, to see whether it will be realised.}

Ere long, Adèle's little foot was heard tripping across the hall. She
entered, transformed as her guardian had predicted. A dress of
rose-coloured satin, very short, and as full in the skirt as it could be
gathered, replaced the brown frock she had previously worn; a wreath of
rosebuds circled her forehead; her feet were dressed in silk stockings
and small white satin sandals.

\foreignquote{french}{Est-ce que ma robe va bien?} cried she, bounding forwards;
\foreignquote{french}{et mes souliers? et mes bas? Tenez, je crois que je vais
	danser!}\footnote{\enquote{Does my dress suit me?} \textelp{} \enquote{and my shoes?
		and my stockings? Look, I think I am about to dance.}}

And spreading out her dress, she chasséed across the room till, having
reached \Mr{} Rochester, she wheeled lightly round before him on tip-toe,
then dropped on one knee at his feet, exclaiming---

\foreignquote{french}{Monsieur, je vous remercie mille fois de votre bonté;} then
rising, she added, \foreignquote{french}{C'est comme cela que maman faisait, n'est-ce
	pas, monsieur?}\footnote{\enquote{Sir, thank you a thousand times for your generosity.}
	\textelp{} \enquote{That's how mother used to do it, isn't it, sir?}}

\enquote{Pre-cise-ly!} was the answer; \enquote{and, \foreignquote{french}{comme
		cela,} she charmed my English gold out of my British breeches' pocket.
	I have been green, too, Miss Eyre,---ay, grass green: not a more vernal
	tint freshens you now than once freshened me. My Spring is gone,
	however, but it has left me that French floweret on my hands, which, in
	some moods, I would fain be rid of. Not valuing now the root whence it
	sprang; having found that it was of a sort which nothing but gold dust
	could manure, I have but half a liking to the blossom, especially when
	it looks so artificial as just now. I keep it and rear it rather on the
	Roman Catholic principle of expiating numerous sins, great or small, by
	one good work. I'll explain all this some day. Good-night.}

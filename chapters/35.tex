\FChapter{Chapter Thirty-Five}{35}

\Lettrine{H}{e} \textsc{did not leave} for Cambridge the next day, as he had said he would. 
He deferred his departure a whole week, and during that time he made me
feel what severe punishment a good yet stern, a conscientious yet
implacable man can inflict on one who has offended him. Without one
overt act of hostility, one upbraiding word, he contrived to impress me
momently with the conviction that I was put beyond the pale of his
favour.

Not that \St{} John harboured a spirit of unchristian vindictiveness---not
that he would have injured a hair of my head, if it had been fully in
his power to do so. Both by nature and principle, he was superior to
the mean gratification of vengeance: he had forgiven me for saying I
scorned him and his love, but he had not forgotten the words; and as
long as he and I lived he never would forget them. I saw by his look,
when he turned to me, that they were always written on the air between
me and him; whenever I spoke, they sounded in my voice to his ear, and
their echo toned every answer he gave me.

He did not abstain from conversing with me: he even called me as usual
each morning to join him at his desk; and I fear the corrupt man within
him had a pleasure unimparted to, and unshared by, the pure Christian,
in evincing with what skill he could, while acting and speaking
apparently just as usual, extract from every deed and every phrase the
spirit of interest and approval which had formerly communicated a
certain austere charm to his language and manner. To me, he was in
reality become no longer flesh, but marble; his eye was a cold, bright,
blue gem; his tongue a speaking instrument---nothing more.

All this was torture to me---refined, lingering torture. It kept up a
slow fire of indignation and a trembling trouble of grief, which
harassed and crushed me altogether. I felt how---if I were his wife,
this good man, pure as the deep sunless source, could soon kill me,
without drawing from my veins a single drop of blood, or receiving on
his own crystal conscience the faintest stain of crime. Especially I
felt this when I made any attempt to propitiate him. No ruth met my
ruth. \emph{He} experienced no suffering from estrangement---no
yearning after reconciliation; and though, more than once, my fast
falling tears blistered the page over which we both bent, they produced
no more effect on him than if his heart had been really a matter of
stone or metal. To his sisters, meantime, he was somewhat kinder than
usual: as if afraid that mere coldness would not sufficiently convince
me how completely I was banished and banned, he added the force of
contrast; and this I am sure he did not by force, but on principle.

The night before he left home, happening to see him walking in the
garden about sunset, and remembering, as I looked at him, that this man,
alienated as he now was, had once saved my life, and that we were near
relations, I was moved to make a last attempt to regain his friendship. 
I went out and approached him as he stood leaning over the little gate;
I spoke to the point at once.

\enquote{\St{} John, I am unhappy because you are still angry with me. 
Let us be friends.}

\enquote{I hope we are friends,} was the unmoved reply; while he still
watched the rising of the moon, which he had been contemplating as I
approached.

\enquote{No, \St{} John, we are not friends as we were. You know that.}

\enquote{Are we not? That is wrong. For my part, I wish you no ill and
all good.}

\enquote{I believe you, \St{} John; for I am sure you are incapable of
wishing any one ill; but, as I am your kinswoman, I should desire
somewhat more of affection than that sort of general philanthropy you
extend to mere strangers.}

\enquote{Of course,} he said. \enquote{Your wish is reasonable, and I
am far from regarding you as a stranger.}

This, spoken in a cool, tranquil tone, was mortifying and baffling
enough. Had I attended to the suggestions of pride and ire, I should
immediately have left him; but something worked within me more strongly
than those feelings could. I deeply venerated my cousin's talent and
principle. His friendship was of value to me: to lose it tried me
severely. I would not so soon relinquish the attempt to reconquer it.

\enquote{Must we part in this way, \St{} John? And when you go to India,
will you leave me so, without a kinder word than you have yet spoken?}

He now turned quite from the moon and faced me.

\enquote{When I go to India, Jane, will I leave you! What! do you not
go to India?}

\enquote{You said I could not unless I married you.}

\enquote{And you will not marry me! You adhere to that resolution?}

Reader, do you know, as I do, what terror those cold people can put into
the ice of their questions? How much of the fall of the avalanche is in
their anger? of the breaking up of the frozen sea in their displeasure?

\enquote{No. \St{} John, I will not marry you. I adhere to my
resolution.}

The avalanche had shaken and slid a little forward, but it did not yet
crash down.

\enquote{Once more, why this refusal?} he asked.

\enquote{Formerly,} I answered, \enquote{because you did not love me;
now, I reply, because you almost hate me. If I were to marry you, you
would kill me. You are killing me now.}

His lips and cheeks turned white---quite white.

\enquote{\emph{I should kill you}---\emph{I am killing you}? Your words are
such as ought not to be used: violent, unfeminine, and untrue. They
betray an unfortunate state of mind: they merit severe reproof: they
would seem inexcusable, but that it is the duty of man to forgive his
fellow even until seventy-and-seven times.}

I had finished the business now. While earnestly wishing to erase from
his mind the trace of my former offence, I had stamped on that tenacious
surface another and far deeper impression, I had burnt it in.

\enquote{Now you will indeed hate me,} I said. \enquote{It is useless
to attempt to conciliate you: I see I have made an eternal enemy of
you.}

A fresh wrong did these words inflict: the worse, because they touched
on the truth. That bloodless lip quivered to a temporary spasm. I knew
the steely ire I had whetted. I was heart-wrung.

\enquote{You utterly misinterpret my words,} I said, at once seizing his
hand: \enquote{I have no intention to grieve or pain you---indeed, I
have not.}

Most bitterly he smiled---most decidedly he withdrew his hand from
mine. \enquote{And now you recall your promise, and will not go to
India at all, I presume?} said he, after a considerable pause.

\enquote{Yes, I will, as your assistant,} I answered.

A very long silence succeeded. What struggle there was in him between
Nature and Grace in this interval, I cannot tell: only singular gleams
scintillated in his eyes, and strange shadows passed over his face. He
spoke at last.

\enquote{I before proved to you the absurdity of a single woman of your
age proposing to accompany abroad a single man of mine. I proved it to
you in such terms as, I should have thought, would have prevented your
ever again alluding to the plan. That you have done so, I regret---for
your sake.}

I interrupted him. Anything like a tangible reproach gave me courage at
once. \enquote{Keep to common sense, \St{} John: you are verging on
nonsense. You pretend to be shocked by what I have said. You are not
really shocked: for, with your superior mind, you cannot be either so
dull or so conceited as to misunderstand my meaning. I say again, I
will be your curate, if you like, but never your wife.}

Again he turned lividly pale; but, as before, controlled his passion
perfectly. He answered emphatically but calmly---

\enquote{A female curate, who is not my wife, would never suit me. With
me, then, it seems, you cannot go: but if you are sincere in your offer,
I will, while in town, speak to a married missionary, whose wife needs a
coadjutor. Your own fortune will make you independent of the Society's
aid; and thus you may still be spared the dishonour of breaking your
promise and deserting the band you engaged to join.}

Now I never had, as the reader knows, either given any formal promise or
entered into any engagement; and this language was all much too hard and
much too despotic for the occasion. I replied---

\enquote{There is no dishonour, no breach of promise, no desertion in
the case. I am not under the slightest obligation to go to India,
especially with strangers. With you I would have ventured much, because
I admire, confide in, and, as a sister, I love you; but I am convinced
that, go when and with whom I would, I should not live long in that
climate.}

\enquote{Ah! you are afraid of yourself,} he said, curling his lip.

\enquote{I am. God did not give me my life to throw away; and to do as
you wish me would, I begin to think, be almost equivalent to committing
suicide. Moreover, before I definitively resolve on quitting England, I
will know for certain whether I cannot be of greater use by remaining in
it than by leaving it.}

\enquote{What do you mean?}

\enquote{It would be fruitless to attempt to explain; but there is a
point on which I have long endured painful doubt, and I can go nowhere
till by some means that doubt is removed.}

\enquote{I know where your heart turns and to what it clings. The
interest you cherish is lawless and unconsecrated. Long since you ought
to have crushed it: now you should blush to allude to it. You think of
\Mr{} Rochester?}

It was true. I confessed it by silence.

\enquote{Are you going to seek \Mr{} Rochester?}

\enquote{I must find out what is become of him.}

\enquote{It remains for me, then,} he said, \enquote{to remember you in my
prayers, and to entreat God for you, in all earnestness, that you may
not indeed become a castaway. I had thought I recognised in you one of
the chosen. But God sees not as man sees: \emph{His} will be done---}

He opened the gate, passed through it, and strayed away down the glen. 
He was soon out of sight.

On re-entering the parlour, I found Diana standing at the window,
looking very thoughtful. Diana was a great deal taller than I: she put
her hand on my shoulder, and, stooping, examined my face.

\enquote{Jane,} she said, \enquote{you are always agitated and pale
now. I am sure there is something the matter. Tell me what business
\St{} John and you have on hands. I have watched you this half hour from
the window; you must forgive my being such a spy, but for a long time I
have fancied I hardly know what. \St{} John is a strange being---}

She paused---I did not speak: soon she resumed---

\enquote{That brother of mine cherishes peculiar views of some sort
respecting you, I am sure: he has long distinguished you by a notice and
interest he never showed to any one else---to what end? I wish he loved
you---does he, Jane?}

I put her cool hand to my hot forehead; \enquote{No, Die, not one whit.}

\enquote{Then why does he follow you so with his eyes, and get you so
frequently alone with him, and keep you so continually at his side? 
Mary and I had both concluded he wished you to marry him.}

\enquote{He does---he has asked me to be his wife.}

Diana clapped her hands. \enquote{That is just what we hoped and
thought! And you will marry him, Jane, won't you? And then he will
stay in England.}

\enquote{Far from that, Diana; his sole idea in proposing to me is to
procure a fitting fellow-labourer in his Indian toils.}

\enquote{What! He wishes you to go to India?}

\enquote{Yes.}

\enquote{Madness!} she exclaimed. \enquote{You would not live three
months there, I am certain. You never shall go: you have not consented,
have you, Jane?}

\enquote{I have refused to marry him---}

\enquote{And have consequently displeased him?} she suggested.

\enquote{Deeply: he will never forgive me, I fear: yet I offered to
accompany him as his sister.}

\enquote{It was frantic folly to do so, Jane. Think of the task you
undertook---one of incessant fatigue, where fatigue kills even the
strong, and you are weak. \St{} John---you know him---would urge you to
impossibilities: with him there would be no permission to rest during
the hot hours; and unfortunately, I have noticed, whatever he exacts,
you force yourself to perform. I am astonished you found courage to
refuse his hand. You do not love him then, Jane?}

\enquote{Not as a husband.}

\enquote{Yet he is a handsome fellow.}

\enquote{And I am so plain, you see, Die. We should never suit.}

\enquote{Plain! You? Not at all. You are much too pretty, as well as
too good, to be grilled alive in Calcutta.} And again she earnestly
conjured me to give up all thoughts of going out with her brother.

\enquote{I must indeed,} I said; \enquote{for when just now I repeated
the offer of serving him for a deacon, he expressed himself shocked at
my want of decency. He seemed to think I had committed an impropriety
in proposing to accompany him unmarried: as if I had not from the first
hoped to find in him a brother, and habitually regarded him as such.}

\enquote{What makes you say he does not love you, Jane?}

\enquote{You should hear himself on the subject. He has again and again
explained that it is not himself, but his office he wishes to mate. He
has told me I am formed for labour---not for love: which is true, no
doubt. But, in my opinion, if I am not formed for love, it follows that
I am not formed for marriage. Would it not be strange, Die, to be
chained for life to a man who regarded one but as a useful tool?}

\enquote{Insupportable---unnatural---out of the question!}

\enquote{And then,} I continued, \enquote{though I have only sisterly
affection for him now, yet, if forced to be his wife, I can imagine the
possibility of conceiving an inevitable, strange, torturing kind of love
for him, because he is so talented; and there is often a certain heroic
grandeur in his look, manner, and conversation. In that case, my lot
would become unspeakably wretched. He would not want me to love him;
and if I showed the feeling, he would make me sensible that it was a
superfluity, unrequired by him, unbecoming in me. I know he would.}

\enquote{And yet \St{} John is a good man,} said Diana.

\enquote{He is a good and a great man; but he forgets, pitilessly, the
feelings and claims of little people, in pursuing his own large views. 
It is better, therefore, for the insignificant to keep out of his way,
lest, in his progress, he should trample them down. Here he comes! I
will leave you, Diana.} And I hastened upstairs as I saw him entering
the garden.

But I was forced to meet him again at supper. During that meal he
appeared just as composed as usual. I had thought he would hardly speak
to me, and I was certain he had given up the pursuit of his matrimonial
scheme: the sequel showed I was mistaken on both points. He addressed
me precisely in his ordinary manner, or what had, of late, been his
ordinary manner---one scrupulously polite. No doubt he had invoked the
help of the Holy Spirit to subdue the anger I had roused in him, and now
believed he had forgiven me once more.

For the evening reading before prayers, he selected the twenty-first
chapter of Revelation. It was at all times pleasant to listen while
from his lips fell the words of the Bible: never did his fine voice
sound at once so sweet and full---never did his manner become so
impressive in its noble simplicity, as when he delivered the oracles of
God: and to-night that voice took a more solemn tone---that manner a
more thrilling meaning---as he sat in the midst of his household circle
(the May moon shining in through the uncurtained window, and rendering
almost unnecessary the light of the candle on the table): as he sat
there, bending over the great old Bible, and described from its page the
vision of the new heaven and the new earth---told how God would come to
dwell with men, how He would wipe away all tears from their eyes, and
promised that there should be no more death, neither sorrow nor crying,
nor any more pain, because the former things were passed away.

The succeeding words thrilled me strangely as he spoke them: especially
as I felt, by the slight, indescribable alteration in sound, that in
uttering them, his eye had turned on me.

\enquote{He that overcometh shall inherit all things; and I will be his
God, and he shall be my son. But,} was slowly, distinctly read,
\enquote{the fearful, the unbelieving, \etc, shall have their part in
the lake which burneth with fire and brimstone, which is the second
death.}

Henceforward, I knew what fate \St{} John feared for me.

A calm, subdued triumph, blent with a longing earnestness, marked his
enunciation of the last glorious verses of that chapter. The reader
believed his name was already written in the Lamb's book of life, and he
yearned after the hour which should admit him to the city to which the
kings of the earth bring their glory and honour; which has no need of
sun or moon to shine in it, because the glory of God lightens it, and
the Lamb is the light thereof.

In the prayer following the chapter, all his energy gathered---all his
stern zeal woke: he was in deep earnest, wrestling with God, and
resolved on a conquest. He supplicated strength for the weak-hearted;
guidance for wanderers from the fold: a return, even at the eleventh
hour, for those whom the temptations of the world and the flesh were
luring from the narrow path. He asked, he urged, he claimed the boon of
a brand snatched from the burning. Earnestness is ever deeply solemn:
first, as I listened to that prayer, I wondered at his; then, when it
continued and rose, I was touched by it, and at last awed. He felt the
greatness and goodness of his purpose so sincerely: others who heard him
plead for it, could not but feel it too.

The prayer over, we took leave of him: he was to go at a very early hour
in the morning. Diana and Mary having kissed him, left the room---in
compliance, I think, with a whispered hint from him: I tendered my hand,
and wished him a pleasant journey.

\enquote{Thank you, Jane. As I said, I shall return from Cambridge in a
fortnight: that space, then, is yet left you for reflection. If I
listened to human pride, I should say no more to you of marriage with
me; but I listen to my duty, and keep steadily in view my first aim---to
do all things to the glory of God. My Master was long-suffering: so
will I be. I cannot give you up to perdition as a vessel of wrath:
repent---resolve, while there is yet time. Remember, we are bid to work
while it is day---warned that \enquote{the night cometh when no man
shall work.} Remember the fate of Dives, who had his good things in
this life. God give you strength to choose that better part which shall
not be taken from you!}

He laid his hand on my head as he uttered the last words. He had spoken
earnestly, mildly: his look was not, indeed, that of a lover beholding
his mistress, but it was that of a pastor recalling his wandering
sheep---or better, of a guardian angel watching the soul for which he is
responsible. All men of talent, whether they be men of feeling or not;
whether they be zealots, or aspirants, or despots---provided only they
be sincere---have their sublime moments, when they subdue and rule. I
felt veneration for \St{} John---veneration so strong that its impetus
thrust me at once to the point I had so long shunned. I was tempted to
cease struggling with him---to rush down the torrent of his will into
the gulf of his existence, and there lose my own. I was almost as hard
beset by him now as I had been once before, in a different way, by
another. I was a fool both times. To have yielded then would have been
an error of principle; to have yielded now would have been an error of
judgment. So I think at this hour, when I look back to the crisis
through the quiet medium of time: I was unconscious of folly at the
instant.

I stood motionless under my hierophant's touch. My refusals were
forgotten---my fears overcome---my wrestlings paralysed. The
Impossible---\emph{\ie}, my marriage with \St{} John---was fast becoming
the Possible. All was changing utterly with a sudden sweep. Religion
called---Angels beckoned---God commanded---life rolled together like a
scroll---death's gates opening, showed eternity beyond: it seemed, that
for safety and bliss there, all here might be sacrificed in a second. 
The dim room was full of visions.

\enquote{Could you decide now?} asked the missionary. The inquiry was
put in gentle tones: he drew me to him as gently. Oh, that gentleness!
how far more potent is it than force! I could resist \St{} John's wrath:
I grew pliant as a reed under his kindness. Yet I knew all the time, if
I yielded now, I should not the less be made to repent, some day, of my
former rebellion. His nature was not changed by one hour of solemn
prayer: it was only elevated.

\enquote{I could decide if I were but certain,} I answered:
\enquote{were I but convinced that it is God's will I should marry you,
I could vow to marry you here and now---come afterwards what would!}

\enquote{My prayers are heard!} ejaculated \St{} John. He pressed his
hand firmer on my head, as if he claimed me: he surrounded me with his
arm, \emph{almost} as if he loved me (I say \emph{almost}---I knew the
difference---for I had felt what it was to be loved; but, like him, I
had now put love out of the question, and thought only of duty). I
contended with my inward dimness of vision, before which clouds yet
rolled. I sincerely, deeply, fervently longed to do what was right; and
only that. \enquote{Show me, show me the path!} I entreated of Heaven. 
I was excited more than I had ever been; and whether what followed was
the effect of excitement the reader shall judge.

All the house was still; for I believe all, except \St{} John and myself,
were now retired to rest. The one candle was dying out: the room was
full of moonlight. My heart beat fast and thick: I heard its throb. 
Suddenly it stood still to an inexpressible feeling that thrilled it
through, and passed at once to my head and extremities. The feeling was
not like an electric shock, but it was quite as sharp, as strange, as
startling: it acted on my senses as if their utmost activity hitherto
had been but torpor, from which they were now summoned and forced to
wake. They rose expectant: eye and ear waited while the flesh quivered
on my bones.

\enquote{What have you heard? What do you see?} asked \St{} John. I saw
nothing, but I heard a voice somewhere cry---

\enquote{Jane! Jane! Jane!}---nothing more.

\enquote{O God! what is it?} I gasped.

I might have said, \enquote{Where is it?} for it did not seem in the
room---nor in the house---nor in the garden; it did not come out of the
air---nor from under the earth---nor from overhead. I had heard
it---where, or whence, for ever impossible to know! And it was the
voice of a human being---a known, loved, well-remembered voice---that of
Edward Fairfax Rochester; and it spoke in pain and woe, wildly, eerily,
urgently.

\enquote{I am coming!} I cried. \enquote{Wait for me! Oh, I will
come!} I flew to the door and looked into the passage: it was dark. I
ran out into the garden: it was void.

\enquote{Where are you?} I exclaimed.

The hills beyond Marsh Glen sent the answer faintly
back---\enquote{Where are you?} I listened. The wind sighed low in the
firs: all was moorland loneliness and midnight hush.

\enquote{Down superstition!} I commented, as that spectre rose up black
by the black yew at the gate. \enquote{This is not thy deception, nor
thy witchcraft: it is the work of nature. She was roused, and did---no
miracle---but her best.}

I broke from \St{} John, who had followed, and would have detained me. It
was \emph{my} time to assume ascendency. \emph{My} powers were in play
and in force. I told him to forbear question or remark; I desired him
to leave me: I must and would be alone. He obeyed at once. Where there
is energy to command well enough, obedience never fails. I mounted to
my chamber; locked myself in; fell on my knees; and prayed in my way---a
different way to \St{} John's, but effective in its own fashion. I seemed
to penetrate very near a Mighty Spirit; and my soul rushed out in
gratitude at His feet. I rose from the thanksgiving---took a
resolve---and lay down, unscared, enlightened---eager but for the
daylight.

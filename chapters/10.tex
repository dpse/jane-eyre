\FChapter{Chapter Ten}{10}

\Lettrine{H}{itherto} \textsc{I have recorded} in detail the events of my insignificant
existence: to the first ten years of my life I have given almost as many
chapters. But this is not to be a regular autobiography. I am only
bound to invoke Memory where I know her responses will possess some
degree of interest; therefore I now pass a space of eight years almost
in silence: a few lines only are necessary to keep up the links of
connection.

When the typhus fever had fulfilled its mission of devastation at
Lowood, it gradually disappeared from thence; but not till its virulence
and the number of its victims had drawn public attention on the school. 
Inquiry was made into the origin of the scourge, and by degrees various
facts came out which excited public indignation in a high degree. The
unhealthy nature of the site; the quantity and quality of the children's
food; the brackish, fetid water used in its preparation; the pupils'
wretched clothing and accommodations---all these things were discovered,
and the discovery produced a result mortifying to \Mr{} Brocklehurst, but
beneficial to the institution.

Several wealthy and benevolent individuals in the county subscribed
largely for the erection of a more convenient building in a better
situation; new regulations were made; improvements in diet and clothing
introduced; the funds of the school were intrusted to the management of
a committee. \Mr{} Brocklehurst, who, from his wealth and family
connections, could not be overlooked, still retained the post of
treasurer; but he was aided in the discharge of his duties by gentlemen
of rather more enlarged and sympathising minds: his office of inspector,
too, was shared by those who knew how to combine reason with strictness,
comfort with economy, compassion with uprightness. The school, thus
improved, became in time a truly useful and noble institution. I
remained an inmate of its walls, after its regeneration, for eight
years: six as pupil, and two as teacher; and in both capacities I bear
my testimony to its value and importance.

During these eight years my life was uniform: but not unhappy, because
it was not inactive. I had the means of an excellent education placed
within my reach; a fondness for some of my studies, and a desire to
excel in all, together with a great delight in pleasing my teachers,
especially such as I loved, urged me on: I availed myself fully of the
advantages offered me. In time I rose to be the first girl of the first
class; then I was invested with the office of teacher; which I
discharged with zeal for two years: but at the end of that time I
altered.

Miss Temple, through all changes, had thus far continued superintendent
of the seminary: to her instruction I owed the best part of my
acquirements; her friendship and society had been my continual solace;
she had stood me in the stead of mother, governess, and, latterly,
companion. At this period she married, removed with her husband (a
clergyman, an excellent man, almost worthy of such a wife) to a distant
county, and consequently was lost to me.

From the day she left I was no longer the same: with her was gone every
settled feeling, every association that had made Lowood in some degree a
home to me. I had imbibed from her something of her nature and much of
her habits: more harmonious thoughts: what seemed better regulated
feelings had become the inmates of my mind. I had given in allegiance
to duty and order; I was quiet; I believed I was content: to the eyes of
others, usually even to my own, I appeared a disciplined and subdued
character.

But destiny, in the shape of the Rev.\@ \Mr{} Nasmyth, came between me and
Miss Temple: I saw her in her travelling dress step into a post-chaise,
shortly after the marriage ceremony; I watched the chaise mount the hill
and disappear beyond its brow; and then retired to my own room, and
there spent in solitude the greatest part of the half-holiday granted in
honour of the occasion.

I walked about the chamber most of the time. I imagined myself only to
be regretting my loss, and thinking how to repair it; but when my
reflections were concluded, and I looked up and found that the afternoon
was gone, and evening far advanced, another discovery dawned on me,
namely, that in the interval I had undergone a transforming process;
that my mind had put off all it had borrowed of Miss Temple---or rather
that she had taken with her the serene atmosphere I had been breathing
in her vicinity---and that now I was left in my natural element, and
beginning to feel the stirring of old emotions. It did not seem as if a
prop were withdrawn, but rather as if a motive were gone: it was not the
power to be tranquil which had failed me, but the reason for
tranquillity was no more. My world had for some years been in Lowood:
my experience had been of its rules and systems; now I remembered that
the real world was wide, and that a varied field of hopes and fears, of
sensations and excitements, awaited those who had courage to go forth
into its expanse, to seek real knowledge of life amidst its perils.

I went to my window, opened it, and looked out. There were the two
wings of the building; there was the garden; there were the skirts of
Lowood; there was the hilly horizon. My eye passed all other objects to
rest on those most remote, the blue peaks; it was those I longed to
surmount; all within their boundary of rock and heath seemed
prison-ground, exile limits. I traced the white road winding round the
base of one mountain, and vanishing in a gorge between two; how I longed
to follow it farther! I recalled the time when I had travelled that
very road in a coach; I remembered descending that hill at twilight; an
age seemed to have elapsed since the day which brought me first to
Lowood, and I had never quitted it since. My vacations had all been
spent at school: \Mrs{} Reed had never sent for me to Gateshead; neither
she nor any of her family had ever been to visit me. I had had no
communication by letter or message with the outer world: school-rules,
school-duties, school-habits and notions, and voices, and faces, and
phrases, and costumes, and preferences, and antipathies---such was what
I knew of existence. And now I felt that it was not enough; I tired of
the routine of eight years in one afternoon. I desired liberty; for
liberty I gasped; for liberty I uttered a prayer; it seemed scattered on
the wind then faintly blowing. I abandoned it and framed a humbler
supplication; for change, stimulus: that petition, too, seemed swept off
into vague space: \enquote{Then,} I cried, half desperate,
\enquote{grant me at least a new servitude!}

Here a bell, ringing the hour of supper, called me downstairs.

I was not free to resume the interrupted chain of my reflections till
bedtime: even then a teacher who occupied the same room with me kept me
from the subject to which I longed to recur, by a prolonged effusion of
small talk. How I wished sleep would silence her. It seemed as if,
could I but go back to the idea which had last entered my mind as I
stood at the window, some inventive suggestion would rise for my relief.

Miss Gryce snored at last; she was a heavy Welshwoman, and till now her
habitual nasal strains had never been regarded by me in any other light
than as a nuisance; to-night I hailed the first deep notes with
satisfaction; I was debarrassed of interruption; my half-effaced thought
instantly revived.

\enquote{A new servitude! There is something in that,} I soliloquised
(mentally, be it understood; I did not talk aloud), \enquote{I know
there is, because it does not sound too sweet; it is not like such words
as Liberty, Excitement, Enjoyment: delightful sounds truly; but no more
than sounds for me; and so hollow and fleeting that it is mere waste of
time to listen to them. But Servitude! That must be matter of fact. 
Any one may serve: I have served here eight years; now all I want is to
serve elsewhere. Can I not get so much of my own will? Is not the
thing feasible? Yes---yes---the end is not so difficult; if I had only
a brain active enough to ferret out the means of attaining it.}

I sat up in bed by way of arousing this said brain: it was a chilly
night; I covered my shoulders with a shawl, and then I proceeded
\emph{to think} again with all my might.

\enquote{What do I want? A new place, in a new house, amongst new
faces, under new circumstances: I want this because it is of no use
wanting anything better. How do people do to get a new place? They
apply to friends, I suppose: I have no friends. There are many others
who have no friends, who must look about for themselves and be their own
helpers; and what is their resource?}

I could not tell: nothing answered me; I then ordered my brain to find a
response, and quickly. It worked and worked faster: I felt the pulses
throb in my head and temples; but for nearly an hour it worked in chaos;
and no result came of its efforts. Feverish with vain labour, I got up
and took a turn in the room; undrew the curtain, noted a star or two,
shivered with cold, and again crept to bed.

A kind fairy, in my absence, had surely dropped the required suggestion
on my pillow; for as I lay down, it came quietly and naturally to my
mind.---\enquote{Those who want situations advertise; you must advertise in the
\emph{---shire Herald}.}

\enquote{How? I know nothing about advertising.}

Replies rose smooth and prompt now:---

\enquote{You must enclose the advertisement and the money to pay for it under a
cover directed to the editor of the \emph{Herald}; you must put it, the
first opportunity you have, into the post at Lowton; answers must be
addressed to J\,.E., at the post-office there; you can go and inquire in
about a week after you send your letter, if any are come, and act
accordingly.}

This scheme I went over twice, thrice; it was then digested in my mind;
I had it in a clear practical form: I felt satisfied, and fell asleep.

With earliest day, I was up: I had my advertisement written, enclosed,
and directed before the bell rang to rouse the school; it ran thus:---

\enquote{A young lady accustomed to tuition} (had I not been a teacher
two years?) \enquote{is desirous of meeting with a situation in a
private family where the children are under fourteen} (I thought that as
I was barely eighteen, it would not do to undertake the guidance of
pupils nearer my own age). \enquote{She is qualified to teach the usual
branches of a good English education, together with French, Drawing, and
Music} (in those days, reader, this now narrow catalogue of
accomplishments, would have been held tolerably comprehensive). 
\enquote{Address, J.\,E., Post-office, Lowton, ---shire.}

This document remained locked in my drawer all day: after tea, I asked
leave of the new superintendent to go to Lowton, in order to perform
some small commissions for myself and one or two of my fellow-teachers;
permission was readily granted; I went. It was a walk of two miles, and
the evening was wet, but the days were still long; I visited a shop or
two, slipped the letter into the post-office, and came back through
heavy rain, with streaming garments, but with a relieved heart.

The succeeding week seemed long: it came to an end at last, however,
like all sublunary things, and once more, towards the close of a
pleasant autumn day, I found myself afoot on the road to Lowton. A
picturesque track it was, by the way; lying along the side of the beck
and through the sweetest curves of the dale: but that day I thought more
of the letters, that might or might not be awaiting me at the little
burgh whither I was bound, than of the charms of lea and water.

My ostensible errand on this occasion was to get measured for a pair of
shoes; so I discharged that business first, and when it was done, I
stepped across the clean and quiet little street from the shoemaker's to
the post-office: it was kept by an old dame, who wore horn spectacles on
her nose, and black mittens on her hands.

\enquote{Are there any letters for J.\,E.?} I asked.

She peered at me over her spectacles, and then she opened a drawer and
fumbled among its contents for a long time, so long that my hopes began
to falter. At last, having held a document before her glasses for
nearly five minutes, she presented it across the counter, accompanying
the act by another inquisitive and mistrustful glance---it was for J.\,E\@.

\enquote{Is there only one?} I demanded.

\enquote{There are no more,} said she; and I put it in my pocket and
turned my face homeward: I could not open it then; rules obliged me to
be back by eight, and it was already half-past seven.

Various duties awaited me on my arrival. I had to sit with the girls
during their hour of study; then it was my turn to read prayers; to see
them to bed: afterwards I supped with the other teachers. Even when we
finally retired for the night, the inevitable Miss Gryce was still my
companion: we had only a short end of candle in our candlestick, and I
dreaded lest she should talk till it was all burnt out; fortunately,
however, the heavy supper she had eaten produced a soporific effect: she
was already snoring before I had finished undressing. There still
remained an inch of candle: I now took out my letter; the seal was an
initial F.; I broke it; the contents were brief.

\enquote{If J.\,E., who advertised in the \emph{---shire Herald} of last
Thursday, possesses the acquirements mentioned, and if she is in a
position to give satisfactory references as to character and competency,
a situation can be offered her where there is but one pupil, a little
girl, under ten years of age; and where the salary is thirty pounds per
annum. J.\,E. is requested to send references, name, address, and all
particulars to the direction:---

% manual enquote removed
\Mrs{} Fairfax, Thornfield, near Millcote, ---shire.}

I examined the document long: the writing was old-fashioned and rather
uncertain, like that of an elderly lady. This circumstance was
satisfactory: a private fear had haunted me, that in thus acting for
myself, and by my own guidance, I ran the risk of getting into some
scrape; and, above all things, I wished the result of my endeavours to
be respectable, proper, \emph{en règle}. I now felt that an elderly
lady was no bad ingredient in the business I had on hand. \Mrs{}
Fairfax! I saw her in a black gown and widow's cap; frigid, perhaps,
but not uncivil: a model of elderly English respectability. Thornfield!
that, doubtless, was the name of her house: a neat orderly spot, I was
sure; though I failed in my efforts to conceive a correct plan of the
premises. Millcote, ---shire; I brushed up my recollections of the map
of England, yes, I saw it; both the shire and the town. ---shire was
seventy miles nearer London than the remote county where I now resided:
that was a recommendation to me. I longed to go where there was life
and movement: Millcote was a large manufacturing town on the banks of
the A-; a busy place enough, doubtless: so much the better; it would be
a complete change at least. Not that my fancy was much captivated by
the idea of long chimneys and clouds of smoke---\enquote{but,} I argued,
\enquote{Thornfield will, probably, be a good way from the town.}

Here the socket of the candle dropped, and the wick went out.

Next day new steps were to be taken; my plans could no longer be
confined to my own breast; I must impart them in order to achieve their
success. Having sought and obtained an audience of the superintendent
during the noontide recreation, I told her I had a prospect of getting a
new situation where the salary would be double what I now received (for
at Lowood I only got £15 per annum); and requested she would break the
matter for me to \Mr{} Brocklehurst, or some of the committee, and
ascertain whether they would permit me to mention them as references. 
She obligingly consented to act as mediatrix in the matter. The next
day she laid the affair before \Mr{} Brocklehurst, who said that \Mrs{} Reed
must be written to, as she was my natural guardian. A note was
accordingly addressed to that lady, who returned for answer, that
\enquote{I might do as I pleased: she had long relinquished all
interference in my affairs.} This note went the round of the committee,
and at last, after what appeared to me most tedious delay, formal leave
was given me to better my condition if I could; and an assurance added,
that as I had always conducted myself well, both as teacher and pupil,
at Lowood, a testimonial of character and capacity, signed by the
inspectors of that institution, should forthwith be furnished me.

This testimonial I accordingly received in about a month, forwarded a
copy of it to \Mrs{} Fairfax, and got that lady's reply, stating that she
was satisfied, and fixing that day fortnight as the period for my
assuming the post of governess in her house.

I now busied myself in preparations: the fortnight passed rapidly. I
had not a very large wardrobe, though it was adequate to my wants; and
the last day sufficed to pack my trunk,---the same I had brought with me
eight years ago from Gateshead.

The box was corded, the card nailed on. In half-an-hour the carrier was
to call for it to take it to Lowton, whither I myself was to repair at
an early hour the next morning to meet the coach. I had brushed my
black stuff travelling-dress, prepared my bonnet, gloves, and muff;
sought in all my drawers to see that no article was left behind; and now
having nothing more to do, I sat down and tried to rest. I could not;
though I had been on foot all day, I could not now repose an instant; I
was too much excited. A phase of my life was closing to-night, a new
one opening to-morrow: impossible to slumber in the interval; I must
watch feverishly while the change was being accomplished.

\enquote{Miss,} said a servant who met me in the lobby, where I was
wandering like a troubled spirit, \enquote{a person below wishes to see
you.}

\enquote{The carrier, no doubt,} I thought, and ran downstairs without
inquiry. I was passing the back-parlour or teachers' sitting-room, the
door of which was half open, to go to the kitchen, when some one ran
out---

\enquote{It's her, I am sure!---I could have told her anywhere!} cried
the individual who stopped my progress and took my hand.

I looked: I saw a woman attired like a well-dressed servant, matronly,
yet still young; very good-looking, with black hair and eyes, and lively
complexion.

\enquote{Well, who is it?} she asked, in a voice and with a smile I half
recognised; \enquote{you've not quite forgotten me, I think, Miss Jane?}

In another second I was embracing and kissing her rapturously:
\enquote{Bessie! Bessie! Bessie!} that was all I said; whereat she
half laughed, half cried, and we both went into the parlour. By the
fire stood a little fellow of three years old, in plaid frock and
trousers.

\enquote{That is my little boy,} said Bessie directly.

\enquote{Then you are married, Bessie?}

\enquote{Yes; nearly five years since to Robert Leaven, the coachman;
and I've a little girl besides Bobby there, that I've christened Jane.}

\enquote{And you don't live at Gateshead?}

\enquote{I live at the lodge: the old porter has left.}

\enquote{Well, and how do they all get on? Tell me everything about
them, Bessie: but sit down first; and, Bobby, come and sit on my knee,
will you?} but Bobby preferred sidling over to his mother.

\enquote{You're not grown so very tall, Miss Jane, nor so very stout,}
continued \Mrs{} Leaven. \enquote{I dare say they've not kept you too
well at school: Miss Reed is the head and shoulders taller than you are;
and Miss Georgiana would make two of you in breadth.}

\enquote{Georgiana is handsome, I suppose, Bessie?}

\enquote{Very. She went up to London last winter with her mama, and
there everybody admired her, and a young lord fell in love with her: but
his relations were against the match; and---what do you think?---he and
Miss Georgiana made it up to run away; but they were found out and
stopped. It was Miss Reed that found them out: I believe she was
envious; and now she and her sister lead a cat and dog life together;
they are always quarrelling---}

\enquote{Well, and what of John Reed?}

\enquote{Oh, he is not doing so well as his mama could wish. He went to
college, and he got---plucked, I think they call it: and then his uncles
wanted him to be a barrister, and study the law: but he is such a
dissipated young man, they will never make much of him, I think.}

\enquote{What does he look like?}

\enquote{He is very tall: some people call him a fine-looking young man;
but he has such thick lips.}

\enquote{And \Mrs{} Reed?}

\enquote{Missis looks stout and well enough in the face, but I think
she's not quite easy in her mind: \Mr{} John's conduct does not please
her---he spends a deal of money.}

\enquote{Did she send you here, Bessie?}

\enquote{No, indeed: but I have long wanted to see you, and when I heard
that there had been a letter from you, and that you were going to
another part of the country, I thought I'd just set off, and get a look
at you before you were quite out of my reach.}

\enquote{I am afraid you are disappointed in me, Bessie.} I said this
laughing: I perceived that Bessie's glance, though it expressed regard,
did in no shape denote admiration.

\enquote{No, Miss Jane, not exactly: you are genteel enough; you look
like a lady, and it is as much as ever I expected of you: you were no
beauty as a child.}

I smiled at Bessie's frank answer: I felt that it was correct, but I
confess I was not quite indifferent to its import: at eighteen most
people wish to please, and the conviction that they have not an exterior
likely to second that desire brings anything but gratification.

\enquote{I dare say you are clever, though,} continued Bessie, by way of
solace. \enquote{What can you do? Can you play on the piano?}

\enquote{A little.}

There was one in the room; Bessie went and opened it, and then asked me
to sit down and give her a tune: I played a waltz or two, and she was
charmed.

\enquote{The Miss Reeds could not play as well!} said she exultingly. 
\enquote{I always said you would surpass them in learning: and can you
draw?}

\enquote{That is one of my paintings over the chimney-piece.} It was a
landscape in water colours, of which I had made a present to the
superintendent, in acknowledgment of her obliging mediation with the
committee on my behalf, and which she had framed and glazed.

\enquote{Well, that is beautiful, Miss Jane! It is as fine a picture as
any Miss Reed's drawing-master could paint, let alone the young ladies
themselves, who could not come near it: and have you learnt French?}

\enquote{Yes, Bessie, I can both read it and speak it.}

\enquote{And you can work on muslin and canvas?}

\enquote{I can.}

\enquote{Oh, you are quite a lady, Miss Jane! I knew you would be: you
will get on whether your relations notice you or not. There was
something I wanted to ask you. Have you ever heard anything from your
father's kinsfolk, the Eyres?}

\enquote{Never in my life.}

\enquote{Well, you know Missis always said they were poor and quite
despicable: and they may be poor; but I believe they are as much gentry
as the Reeds are; for one day, nearly seven years ago, a \Mr{} Eyre came
to Gateshead and wanted to see you; Missis said you were at school fifty
miles off; he seemed so much disappointed, for he could not stay: he was
going on a voyage to a foreign country, and the ship was to sail from
London in a day or two. He looked quite a gentleman, and I believe he
was your father's brother.}

\enquote{What foreign country was he going to, Bessie?}

\enquote{An island thousands of miles off, where they make wine---the
butler did tell me---}

\enquote{Madeira?} I suggested.

\enquote{Yes, that is it---that is the very word.}

\enquote{So he went?}

\enquote{Yes; he did not stay many minutes in the house: Missis was very
high with him; she called him afterwards a \enquote{sneaking
tradesman.} My Robert believes he was a wine-merchant.}

\enquote{Very likely,} I returned; \enquote{or perhaps clerk or agent to
a wine-merchant.}

Bessie and I conversed about old times an hour longer, and then she was
obliged to leave me: I saw her again for a few minutes the next morning
at Lowton, while I was waiting for the coach. We parted finally at the
door of the Brocklehurst Arms there: each went her separate way; she set
off for the brow of Lowood Fell to meet the conveyance which was to take
her back to Gateshead, I mounted the vehicle which was to bear me to new
duties and a new life in the unknown environs of Millcote.

\chapter{Preface}

A preface to the first edition of \enquote{\textsc{Jane Eyre}} being unnecessary,
I gave none: this second edition demands a few words both of
acknowledgment and miscellaneous remark.

My thanks are due in three quarters.

To the Public, for the indulgent ear it has inclined to a plain tale
with few pretensions.

To the Press, for the fair field its honest suffrage has opened to an
obscure aspirant.

To my Publishers, for the aid their tact, their energy, their practical
sense and frank liberality have afforded an unknown and unrecommended
Author.

The Press and the Public are but vague personifications for me, and I
must thank them in vague terms; but my Publishers are definite: so are
certain generous critics who have encouraged me as only large-hearted
and high-minded men know how to encourage a struggling stranger; to
them, \emph{\ie}, to my Publishers and the select Reviewers, I say
cordially, Gentlemen, I thank you from my heart.

Having thus acknowledged what I owe those who have aided and approved
me, I turn to another class; a small one, so far as I know, but not,
therefore, to be overlooked.  I mean the timorous or carping few who
doubt the tendency of such books as \enquote{\textsc{Jane Eyre:}} in whose eyes
whatever is unusual is wrong; whose ears detect in each protest against
bigotry---that parent of crime---an insult to piety, that regent of God
on earth.  I would suggest to such doubters certain obvious
distinctions; I would remind them of certain simple truths.

Conventionality is not morality.  Self-righteousness is not religion.
To attack the first is not to assail the last.  To pluck the mask from
the face of the Pharisee, is not to lift an impious hand to the Crown of
Thorns.

These things and deeds are diametrically opposed: they are as distinct
as is vice from virtue.  Men too often confound them: they should not be
confounded: appearance should not be mistaken for truth; narrow human
doctrines, that only tend to elate and magnify a few, should not be
substituted for the world-redeeming creed of Christ.  There is---I
repeat it---a difference; and it is a good, and not a bad action to mark
broadly and clearly the line of separation between them.

The world may not like to see these ideas dissevered, for it has been
accustomed to blend them; finding it convenient to make external show
pass for sterling worth---to let white-washed walls vouch for clean
shrines.  It may hate him who dares to scrutinise and expose---to rase
the gilding, and show base metal under it---to penetrate the sepulchre,
and reveal charnel relics: but hate as it will, it is indebted to him.

Ahab did not like Micaiah, because he never prophesied good concerning
him, but evil; probably he liked the sycophant son of Chenaannah better;
yet might Ahab have escaped a bloody death, had he but stopped his ears
to flattery, and opened them to faithful counsel.

There is a man in our own days whose words are not framed to tickle
delicate ears: who, to my thinking, comes before the great ones of
society, much as the son of Imlah came before the throned Kings of Judah
and Israel; and who speaks truth as deep, with a power as prophet-like
and as vital---a mien as dauntless and as daring.  Is the satirist of
\enquote{Vanity Fair} admired in high places?  I cannot tell; but I
think if some of those amongst whom he hurls the Greek fire of his
sarcasm, and over whom he flashes the levin-brand of his denunciation,
were to take his warnings in time---they or their seed might yet escape
a fatal Rimoth-Gilead.

Why have I alluded to this man?  I have alluded to him, Reader, because
I think I see in him an intellect profounder and more unique than his
contemporaries have yet recognised; because I regard him as the first
social regenerator of the day---as the very master of that working corps
who would restore to rectitude the warped system of things; because I
think no commentator on his writings has yet found the comparison that
suits him, the terms which rightly characterise his talent.  They say he
is like Fielding: they talk of his wit, humour, comic powers.  He
resembles Fielding as an eagle does a vulture: Fielding could stoop on
carrion, but Thackeray never does.  His wit is bright, his humour
attractive, but both bear the same relation to his serious genius that
the mere lambent sheet-lightning playing under the edge of the
summer-cloud does to the electric death-spark hid in its womb.  Finally,
I have alluded to \Mr{} Thackeray, because to him---if he will accept the
tribute of a total stranger---I have dedicated this second edition of \enquote{\textsc{Jane Eyre.}}

\plainbreak{2}

{\DTMsetstyle{en-US}\textit{\DTMdisplaydate{1847}{12}{21}{-1}} \hfill \textsc{Currer Bell}}

\chapter{Note to the Third Edition}

I avail myself of the opportunity which a third edition of \enquote{\textsc{Jane
		Eyre}} affords me, of again addressing a word to the Public, to explain
that my claim to the title of novelist rests on this one work alone.
If, therefore, the authorship of other works of fiction has been
attributed to me, an honour is awarded where it is not merited; and
consequently, denied where it is justly due.

This explanation will serve to rectify mistakes which may already have
been made, and to prevent future errors.

\plainbreak{2}

{\DTMsetstyle{en-US}\textit{\DTMdisplaydate{1848}{04}{13}{-1}} \hfill \textsc{Currer Bell}}

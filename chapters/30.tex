\FChapter{Chapter Thirty}{30}

\Lettrine{T}{he} \textsc{more I knew} of the inmates of Moor House, the better I liked them. 
In a few days I had so far recovered my health that I could sit up all
day, and walk out sometimes. I could join with Diana and Mary in all
their occupations; converse with them as much as they wished, and aid
them when and where they would allow me. There was a reviving pleasure
in this intercourse, of a kind now tasted by me for the first time---the
pleasure arising from perfect congeniality of tastes, sentiments, and
principles.

I liked to read what they liked to read: what they enjoyed, delighted
me; what they approved, I reverenced. They loved their sequestered
home. I, too, in the grey, small, antique structure, with its low roof,
its latticed casements, its mouldering walls, its avenue of aged
firs---all grown aslant under the stress of mountain winds; its garden,
dark with yew and holly---and where no flowers but of the hardiest
species would bloom---found a charm both potent and permanent. They
clung to the purple moors behind and around their dwelling---to the
hollow vale into which the pebbly bridle-path leading from their gate
descended, and which wound between fern-banks first, and then amongst a
few of the wildest little pasture-fields that ever bordered a wilderness
of heath, or gave sustenance to a flock of grey moorland sheep, with
their little mossy-faced lambs:---they clung to this scene, I say, with
a perfect enthusiasm of attachment. I could comprehend the feeling, and
share both its strength and truth. I saw the fascination of the
locality. I felt the consecration of its loneliness: my eye feasted on
the outline of swell and sweep---on the wild colouring communicated to
ridge and dell by moss, by heath-bell, by flower-sprinkled turf, by
brilliant bracken, and mellow granite crag. These details were just to
me what they were to them---so many pure and sweet sources of pleasure. 
The strong blast and the soft breeze; the rough and the halcyon day; the
hours of sunrise and sunset; the moonlight and the clouded night,
developed for me, in these regions, the same attraction as for
them---wound round my faculties the same spell that entranced theirs.

Indoors we agreed equally well. They were both more accomplished and
better read than I was; but with eagerness I followed in the path of
knowledge they had trodden before me. I devoured the books they lent
me: then it was full satisfaction to discuss with them in the evening
what I had perused during the day. Thought fitted thought; opinion met
opinion: we coincided, in short, perfectly.

If in our trio there was a superior and a leader, it was Diana. 
Physically, she far excelled me: she was handsome; she was vigorous. In
her animal spirits there was an affluence of life and certainty of flow,
such as excited my wonder, while it baffled my comprehension. I could
talk a while when the evening commenced, but the first gush of vivacity
and fluency gone, I was fain to sit on a stool at Diana's feet, to rest
my head on her knee, and listen alternately to her and Mary, while they
sounded thoroughly the topic on which I had but touched. Diana offered
to teach me German. I liked to learn of her: I saw the part of
instructress pleased and suited her; that of scholar pleased and suited
me no less. Our natures dovetailed: mutual affection---of the strongest
kind---was the result. They discovered I could draw: their pencils and
colour-boxes were immediately at my service. My skill, greater in this
one point than theirs, surprised and charmed them. Mary would sit and
watch me by the hour together: then she would take lessons; and a
docile, intelligent, assiduous pupil she made. Thus occupied, and
mutually entertained, days passed like hours, and weeks like days.

As to \Mr{} St John, the intimacy which had arisen so naturally and
rapidly between me and his sisters did not extend to him. One reason of
the distance yet observed between us was, that he was comparatively
seldom at home: a large proportion of his time appeared devoted to
visiting the sick and poor among the scattered population of his parish.

No weather seemed to hinder him in these pastoral excursions: rain or
fair, he would, when his hours of morning study were over, take his hat,
and, followed by his father's old pointer, Carlo, go out on his mission
of love or duty---I scarcely know in which light he regarded it. 
Sometimes, when the day was very unfavourable, his sisters would
expostulate. He would then say, with a peculiar smile, more solemn than
cheerful---

\enquote{And if I let a gust of wind or a sprinkling of rain turn me
aside from these easy tasks, what preparation would such sloth be for
the future I propose to myself?}

Diana and Mary's general answer to this question was a sigh, and some
minutes of apparently mournful meditation.

But besides his frequent absences, there was another barrier to
friendship with him: he seemed of a reserved, an abstracted, and even of
a brooding nature. Zealous in his ministerial labours, blameless in his
life and habits, he yet did not appear to enjoy that mental serenity,
that inward content, which should be the reward of every sincere
Christian and practical philanthropist. Often, of an evening, when he
sat at the window, his desk and papers before him, he would cease
reading or writing, rest his chin on his hand, and deliver himself up to
I know not what course of thought; but that it was perturbed and
exciting might be seen in the frequent flash and changeful dilation of
his eye.

I think, moreover, that Nature was not to him that treasury of delight
it was to his sisters. He expressed once, and but once in my hearing, a
strong sense of the rugged charm of the hills, and an inborn affection
for the dark roof and hoary walls he called his home; but there was more
of gloom than pleasure in the tone and words in which the sentiment was
manifested; and never did he seem to roam the moors for the sake of
their soothing silence---never seek out or dwell upon the thousand
peaceful delights they could yield.

Incommunicative as he was, some time elapsed before I had an opportunity
of gauging his mind. I first got an idea of its calibre when I heard
him preach in his own church at Morton. I wish I could describe that
sermon: but it is past my power. I cannot even render faithfully the
effect it produced on me.

It began calm---and indeed, as far as delivery and pitch of voice went,
it was calm to the end: an earnestly felt, yet strictly restrained zeal
breathed soon in the distinct accents, and prompted the nervous
language. This grew to force---compressed, condensed, controlled. The
heart was thrilled, the mind astonished, by the power of the preacher:
neither were softened. Throughout there was a strange bitterness; an
absence of consolatory gentleness; stern allusions to Calvinistic
doctrines---election, predestination, reprobation---were frequent; and
each reference to these points sounded like a sentence pronounced for
doom. When he had done, instead of feeling better, calmer, more
enlightened by his discourse, I experienced an inexpressible sadness;
for it seemed to me---I know not whether equally so to others---that the
eloquence to which I had been listening had sprung from a depth where
lay turbid dregs of disappointment---where moved troubling impulses of
insatiate yearnings and disquieting aspirations. I was sure \St{} John
Rivers---pure-lived, conscientious, zealous as he was---had not yet
found that peace of God which passeth all understanding: he had no more
found it, I thought, than had I with my concealed and racking regrets
for my broken idol and lost elysium---regrets to which I have latterly
avoided referring, but which possessed me and tyrannised over me
ruthlessly.

Meantime a month was gone. Diana and Mary were soon to leave Moor
House, and return to the far different life and scene which awaited
them, as governesses in a large, fashionable, south-of-England city,
where each held a situation in families by whose wealthy and haughty
members they were regarded only as humble dependants, and who neither
knew nor sought out their innate excellences, and appreciated only their
acquired accomplishments as they appreciated the skill of their cook or
the taste of their waiting-woman. \Mr{} \St{} John had said nothing to me
yet about the employment he had promised to obtain for me; yet it became
urgent that I should have a vocation of some kind. One morning, being
left alone with him a few minutes in the parlour, I ventured to approach
the window-recess---which his table, chair, and desk consecrated as a
kind of study---and I was going to speak, though not very well knowing
in what words to frame my inquiry---for it is at all times difficult to
break the ice of reserve glassing over such natures as his---when he
saved me the trouble by being the first to commence a dialogue.

Looking up as I drew near---\enquote{You have a question to ask of me?}
he said.

\enquote{Yes; I wish to know whether you have heard of any service I can
offer myself to undertake?}

\enquote{I found or devised something for you three weeks ago; but as
you seemed both useful and happy here---as my sisters had evidently
become attached to you, and your society gave them unusual pleasure---I
deemed it inexpedient to break in on your mutual comfort till their
approaching departure from Marsh End should render yours necessary.}

\enquote{And they will go in three days now?} I said.

\enquote{Yes; and when they go, I shall return to the parsonage at
Morton: Hannah will accompany me; and this old house will be shut up.}

I waited a few moments, expecting he would go on with the subject first
broached: but he seemed to have entered another train of reflection: his
look denoted abstraction from me and my business. I was obliged to
recall him to a theme which was of necessity one of close and anxious
interest to me.

\enquote{What is the employment you had in view, \Mr{} Rivers? I hope
this delay will not have increased the difficulty of securing it.}

\enquote{Oh, no; since it is an employment which depends only on me to
give, and you to accept.}

He again paused: there seemed a reluctance to continue. I grew
impatient: a restless movement or two, and an eager and exacting glance
fastened on his face, conveyed the feeling to him as effectually as
words could have done, and with less trouble.

\enquote{You need be in no hurry to hear,} he said: \enquote{let me
frankly tell you, I have nothing eligible or profitable to suggest. 
Before I explain, recall, if you please, my notice, clearly given, that
if I helped you, it must be as the blind man would help the lame. I am
poor; for I find that, when I have paid my father's debts, all the
patrimony remaining to me will be this crumbling grange, the row of
scathed firs behind, and the patch of moorish soil, with the yew-trees
and holly-bushes in front. I am obscure: Rivers is an old name; but of
the three sole descendants of the race, two earn the dependant's crust
among strangers, and the third considers himself an alien from his
native country---not only for life, but in death. Yes, and deems, and
is bound to deem, himself honoured by the lot, and aspires but after the
day when the cross of separation from fleshly ties shall be laid on his
shoulders, and when the Head of that church-militant of whose humblest
members he is one, shall give the word, \enquote{Rise, follow Me!}}

\St{} John said these words as he pronounced his sermons, with a quiet,
deep voice; with an unflushed cheek, and a coruscating radiance of
glance. He resumed---

\enquote{And since I am myself poor and obscure, I can offer you but a service
of poverty and obscurity. \emph{You} may even think it degrading---for
I see now your habits have been what the world calls refined: your
tastes lean to the ideal, and your society has at least been amongst the
educated; but \emph{I} consider that no service degrades which can
better our race. I hold that the more arid and unreclaimed the soil
where the Christian labourer's task of tillage is appointed him---the
scantier the meed his toil brings---the higher the honour. His, under
such circumstances, is the destiny of the pioneer; and the first
pioneers of the Gospel were the Apostles---their captain was Jesus, the
Redeemer, Himself.}

\enquote{Well?} I said, as he again paused---\enquote{proceed.}

He looked at me before he proceeded: indeed, he seemed leisurely to read
my face, as if its features and lines were characters on a page. The
conclusions drawn from this scrutiny he partially expressed in his
succeeding observations.

\enquote{I believe you will accept the post I offer you,} said he,
\enquote{and hold it for a while: not permanently, though: any more than
I could permanently keep the narrow and narrowing---the tranquil, hidden
office of English country incumbent; for in your nature is an alloy as
detrimental to repose as that in mine, though of a different kind.}

\enquote{Do explain,} I urged, when he halted once more.

\enquote{I will; and you shall hear how poor the proposal is,---how
trivial---how cramping. I shall not stay long at Morton, now that my
father is dead, and that I am my own master. I shall leave the place
probably in the course of a twelve-month; but while I do stay, I will
exert myself to the utmost for its improvement. Morton, when I came to
it two years ago, had no school: the children of the poor were excluded
from every hope of progress. I established one for boys: I mean now to
open a second school for girls. I have hired a building for the
purpose, with a cottage of two rooms attached to it for the mistress's
house. Her salary will be thirty pounds a year: her house is already
furnished, very simply, but sufficiently, by the kindness of a lady,
Miss Oliver; the only daughter of the sole rich man in my parish---\Mr{}
Oliver, the proprietor of a needle-factory and iron-foundry in the
valley. The same lady pays for the education and clothing of an orphan
from the workhouse, on condition that she shall aid the mistress in such
menial offices connected with her own house and the school as her
occupation of teaching will prevent her having time to discharge in
person. Will you be this mistress?}

He put the question rather hurriedly; he seemed half to expect an
indignant, or at least a disdainful rejection of the offer: not knowing
all my thoughts and feelings, though guessing some, he could not tell in
what light the lot would appear to me. In truth it was humble---but
then it was sheltered, and I wanted a safe asylum: it was plodding---but
then, compared with that of a governess in a rich house, it was
independent; and the fear of servitude with strangers entered my soul
like iron: it was not ignoble---not unworthy---not mentally degrading, I
made my decision.

\enquote{I thank you for the proposal, \Mr{} Rivers, and I accept it with
all my heart.}

\enquote{But you comprehend me?} he said. \enquote{It is a village
school: your scholars will be only poor girls---cottagers' children---at
the best, farmers' daughters. Knitting, sewing, reading, writing,
ciphering, will be all you will have to teach. What will you do with
your accomplishments? What, with the largest portion of your
mind---sentiments---tastes?}

\enquote{Save them till they are wanted. They will keep.}

\enquote{You know what you undertake, then?}

\enquote{I do.}

He now smiled: and not a bitter or a sad smile, but one well pleased and
deeply gratified.

\enquote{And when will you commence the exercise of your function?}

\enquote{I will go to my house to-morrow, and open the school, if you
like, next week.}

\enquote{Very well: so be it.}

He rose and walked through the room. Standing still, he again looked at
me. He shook his head.

\enquote{What do you disapprove of, \Mr{} Rivers?} I asked.

\enquote{You will not stay at Morton long: no, no!}

\enquote{Why? What is your reason for saying so?}

\enquote{I read it in your eye; it is not of that description which
promises the maintenance of an even tenor in life.}

\enquote{I am not ambitious.}

He started at the word \enquote{ambitious.} He repeated, \enquote{No. 
What made you think of ambition? Who is ambitious? I know I am: but
how did you find it out?}

\enquote{I was speaking of myself.}

\enquote{Well, if you are not ambitious, you are---} He paused.

\enquote{What?}

\enquote{I was going to say, impassioned: but perhaps you would have
misunderstood the word, and been displeased. I mean, that human
affections and sympathies have a most powerful hold on you. I am sure
you cannot long be content to pass your leisure in solitude, and to
devote your working hours to a monotonous labour wholly void of
stimulus: any more than I can be content,} he added, with emphasis,
\enquote{to live here buried in morass, pent in with mountains---my
nature, that God gave me, contravened; my faculties, heaven-bestowed,
paralysed---made useless. You hear now how I contradict myself. I, who
preached contentment with a humble lot, and justified the vocation even
of hewers of wood and drawers of water in God's service---I, His
ordained minister, almost rave in my restlessness. Well, propensities
and principles must be reconciled by some means.}

He left the room. In this brief hour I had learnt more of him than in
the whole previous month: yet still he puzzled me.

Diana and Mary Rivers became more sad and silent as the day approached
for leaving their brother and their home. They both tried to appear as
usual; but the sorrow they had to struggle against was one that could
not be entirely conquered or concealed. Diana intimated that this would
be a different parting from any they had ever yet known. It would
probably, as far as \St{} John was concerned, be a parting for years: it
might be a parting for life.

\enquote{He will sacrifice all to his long-framed resolves,} she said:
\enquote{natural affection and feelings more potent still. \St{} John
looks quiet, Jane; but he hides a fever in his vitals. You would think
him gentle, yet in some things he is inexorable as death; and the worst
of it is, my conscience will hardly permit me to dissuade him from his
severe decision: certainly, I cannot for a moment blame him for it. It
is right, noble, Christian: yet it breaks my heart!} And the tears
gushed to her fine eyes. Mary bent her head low over her work.

\enquote{We are now without father: we shall soon be without home and
brother,} she murmured.

At that moment a little accident supervened, which seemed decreed by
fate purposely to prove the truth of the adage, that
\enquote{misfortunes never come singly,} and to add to their distresses
the vexing one of the slip between the cup and the lip. \St{} John passed
the window reading a letter. He entered.

\enquote{Our uncle John is dead,} said he.

Both the sisters seemed struck: not shocked or appalled; the tidings
appeared in their eyes rather momentous than afflicting.

\enquote{Dead?} repeated Diana.

\enquote{Yes.}

She riveted a searching gaze on her brother's face. \enquote{And what
then?} she demanded, in a low voice.

\enquote{What then, Die?} he replied, maintaining a marble immobility of
feature. \enquote{What then? Why---nothing. Read.}

He threw the letter into her lap. She glanced over it, and handed it to
Mary. Mary perused it in silence, and returned it to her brother. All
three looked at each other, and all three smiled---a dreary, pensive
smile enough.

\enquote{Amen! We can yet live,} said Diana at last.

\enquote{At any rate, it makes us no worse off than we were before,}
remarked Mary.

\enquote{Only it forces rather strongly on the mind the picture of what
\emph{might have been},} said \Mr{} Rivers, \enquote{and contrasts it somewhat
too vividly with what \emph{is}.}

He folded the letter, locked it in his desk, and again went out.

For some minutes no one spoke. Diana then turned to me.

\enquote{Jane, you will wonder at us and our mysteries,} she said,
\enquote{and think us hard-hearted beings not to be more moved at the
death of so near a relation as an uncle; but we have never seen him or
known him. He was my mother's brother. My father and he quarrelled
long ago. It was by his advice that my father risked most of his
property in the speculation that ruined him. Mutual recrimination
passed between them: they parted in anger, and were never reconciled. 
My uncle engaged afterwards in more prosperous undertakings: it appears
he realised a fortune of twenty thousand pounds. He was never married,
and had no near kindred but ourselves and one other person, not more
closely related than we. My father always cherished the idea that he
would atone for his error by leaving his possessions to us; that letter
informs us that he has bequeathed every penny to the other relation,
with the exception of thirty guineas, to be divided between \St{} John,
Diana, and Mary Rivers, for the purchase of three mourning rings. He
had a right, of course, to do as he pleased: and yet a momentary damp is
cast on the spirits by the receipt of such news. Mary and I would have
esteemed ourselves rich with a thousand pounds each; and to \St{} John
such a sum would have been valuable, for the good it would have enabled
him to do.}

This explanation given, the subject was dropped, and no further
reference made to it by either \Mr{} Rivers or his sisters. The next day
I left Marsh End for Morton. The day after, Diana and Mary quitted it
for distant B-. In a week, \Mr{} Rivers and Hannah repaired to the
parsonage: and so the old grange was abandoned.

\FChapter{Chapter Seventeen}{17}

\Lettrine{A}{} \textsc{week passed,} and no news arrived of \Mr{} Rochester: ten days, and still
he did not come. \Mrs{} Fairfax said she should not be surprised if he
were to go straight from the Leas to London, and thence to the
Continent, and not show his face again at Thornfield for a year to come;
he had not unfrequently quitted it in a manner quite as abrupt and
unexpected. When I heard this, I was beginning to feel a strange chill
and failing at the heart. I was actually permitting myself to
experience a sickening sense of disappointment; but rallying my wits,
and recollecting my principles, I at once called my sensations to order;
and it was wonderful how I got over the temporary blunder---how I
cleared up the mistake of supposing \Mr{} Rochester's movements a matter
in which I had any cause to take a vital interest. Not that I humbled
myself by a slavish notion of inferiority: on the contrary, I just
said---

\enquote{You have nothing to do with the master of Thornfield, further
	than to receive the salary he gives you for teaching his protégée, and
	to be grateful for such respectful and kind treatment as, if you do your
	duty, you have a right to expect at his hands. Be sure that is the only
	tie he seriously acknowledges between you and him; so don't make him the
	object of your fine feelings, your raptures, agonies, and so forth. He
	is not of your order: keep to your caste, and be too self-respecting to
	lavish the love of the whole heart, soul, and strength, where such a
	gift is not wanted and would be despised.}

I went on with my day's business tranquilly; but ever and anon vague
suggestions kept wandering across my brain of reasons why I should quit
Thornfield; and I kept involuntarily framing advertisements and
pondering conjectures about new situations: these thoughts I did not
think to check; they might germinate and bear fruit if they could.

\Mr{} Rochester had been absent upwards of a fortnight, when the post
brought \Mrs{} Fairfax a letter.

\enquote{It is from the master,} said she, as she looked at the
direction. \enquote{Now I suppose we shall know whether we are to
	expect his return or not.}

And while she broke the seal and perused the document, I went on taking
my coffee (we were at breakfast): it was hot, and I attributed to that
circumstance a fiery glow which suddenly rose to my face. Why my hand
shook, and why I involuntarily spilt half the contents of my cup into my
saucer, I did not choose to consider.

\enquote{Well, I sometimes think we are too quiet; but we run a chance
	of being busy enough now: for a little while at least,} said \Mrs{}
Fairfax, still holding the note before her spectacles.

Ere I permitted myself to request an explanation, I tied the string of
Adèle's pinafore, which happened to be loose: having helped her also to
another bun and refilled her mug with milk, I said, nonchalantly---

\enquote{\Mr{} Rochester is not likely to return soon, I suppose?}

\enquote{Indeed he is---in three days, he says: that will be next
	Thursday; and not alone either. I don't know how many of the fine
	people at the Leas are coming with him: he sends directions for all the
	best bedrooms to be prepared; and the library and drawing-rooms are to
	be cleaned out; I am to get more kitchen hands from the George Inn, at
	Millcote, and from wherever else I can; and the ladies will bring their
	maids and the gentlemen their valets: so we shall have a full house of
	it.} And \Mrs{} Fairfax swallowed her breakfast and hastened away to
commence operations.

The three days were, as she had foretold, busy enough. I had thought
all the rooms at Thornfield beautifully clean and well arranged; but it
appears I was mistaken. Three women were got to help; and such
scrubbing, such brushing, such washing of paint and beating of carpets,
such taking down and putting up of pictures, such polishing of mirrors
and lustres, such lighting of fires in bedrooms, such airing of sheets
and feather-beds on hearths, I never beheld, either before or since.
Adèle ran quite wild in the midst of it: the preparations for company
and the prospect of their arrival, seemed to throw her into ecstasies.
She would have Sophie to look over all her \foreignquote{french}{toilettes,} as she
called frocks; to furbish up any that were \foreignquote{french}{\emph{passées},} and to
air and arrange the new. For herself, she did nothing but caper about
in the front chambers, jump on and off the bedsteads, and lie on the
mattresses and piled-up bolsters and pillows before the enormous fires
roaring in the chimneys. From school duties she was exonerated: \Mrs{}
Fairfax had pressed me into her service, and I was all day in the
storeroom, helping (or hindering) her and the cook; learning to make
custards and cheese-cakes and French pastry, to truss game and garnish
desert-dishes.

The party were expected to arrive on Thursday afternoon, in time for
dinner at six. During the intervening period I had no time to nurse
chimeras; and I believe I was as active and gay as anybody---Adèle
excepted. Still, now and then, I received a damping check to my
cheerfulness; and was, in spite of myself, thrown back on the region of
doubts and portents, and dark conjectures. This was when I chanced to
see the third-storey staircase door (which of late had always been kept
locked) open slowly, and give passage to the form of Grace Poole, in
prim cap, white apron, and handkerchief; when I watched her glide along
the gallery, her quiet tread muffled in a list slipper; when I saw her
look into the bustling, topsy-turvy bedrooms,---just say a word,
perhaps, to the charwoman about the proper way to polish a grate, or
clean a marble mantelpiece, or take stains from papered walls, and then
pass on. She would thus descend to the kitchen once a day, eat her
dinner, smoke a moderate pipe on the hearth, and go back, carrying her
pot of porter with her, for her private solace, in her own gloomy, upper
haunt. Only one hour in the twenty-four did she pass with her
fellow-servants below; all the rest of her time was spent in some
low-ceiled, oaken chamber of the second storey: there she sat and
sewed---and probably laughed drearily to herself,---as companionless as
a prisoner in his dungeon.

The strangest thing of all was, that not a soul in the house, except me,
noticed her habits, or seemed to marvel at them: no one discussed her
position or employment; no one pitied her solitude or isolation. I
once, indeed, overheard part of a dialogue between Leah and one of the
charwomen, of which Grace formed the subject. Leah had been saying
something I had not caught, and the charwoman remarked---

\enquote{She gets good wages, I guess?}

\enquote{Yes,} said Leah; \enquote{I wish I had as good; not that mine
	are to complain of,---there's no stinginess at Thornfield; but they're
	not one fifth of the sum \Mrs{} Poole receives. And she is laying by: she
	goes every quarter to the bank at Millcote. I should not wonder but she
	has saved enough to keep her independent if she liked to leave; but I
	suppose she's got used to the place; and then she's not forty yet, and
	strong and able for anything. It is too soon for her to give up
	business.}

\enquote{She is a good hand, I daresay,} said the charwoman.

\enquote{Ah!---she understands what she has to do,---nobody better,}
rejoined Leah significantly; \enquote{and it is not every one could fill
	her shoes---not for all the money she gets.}

\enquote{That it is not!} was the reply. \enquote{I wonder whether the
	master---}

The charwoman was going on; but here Leah turned and perceived me, and
she instantly gave her companion a nudge.

\enquote{Doesn't she know?} I heard the woman whisper.

Leah shook her head, and the conversation was of course dropped. All I
had gathered from it amounted to this,---that there was a mystery at
Thornfield; and that from participation in that mystery I was purposely
excluded.

Thursday came: all work had been completed the previous evening; carpets
were laid down, bed-hangings festooned, radiant white counterpanes
spread, toilet tables arranged, furniture rubbed, flowers piled in
vases: both chambers and saloons looked as fresh and bright as hands
could make them. The hall, too, was scoured; and the great carved
clock, as well as the steps and banisters of the staircase, were
polished to the brightness of glass; in the dining-room, the sideboard
flashed resplendent with plate; in the drawing-room and boudoir, vases
of exotics bloomed on all sides.

Afternoon arrived: \Mrs{} Fairfax assumed her best black satin gown, her
gloves, and her gold watch; for it was her part to receive the
company,---to conduct the ladies to their rooms, \etc. Adèle, too, would
be dressed: though I thought she had little chance of being introduced
to the party that day at least. However, to please her, I allowed
Sophie to apparel her in one of her short, full muslin frocks. For
myself, I had no need to make any change; I should not be called upon to
quit my sanctum of the schoolroom; for a sanctum it was now become to
me,---\enquote{a very pleasant refuge in time of trouble.}

It had been a mild, serene spring day---one of those days which, towards
the end of March or the beginning of April, rise shining over the earth
as heralds of summer. It was drawing to an end now; but the evening was
even warm, and I sat at work in the schoolroom with the window open.

\enquote{It gets late,} said \Mrs{} Fairfax, entering in rustling state.
\enquote{I am glad I ordered dinner an hour after the time \Mr{} Rochester
	mentioned; for it is past six now. I have sent John down to the gates
	to see if there is anything on the road: one can see a long way from
	thence in the direction of Millcote.} She went to the window.
\enquote{Here he is!} said she. \enquote{Well, John} (leaning out),
\enquote{any news?}

\enquote{They're coming, ma'am,} was the answer. \enquote{They'll be
	here in ten minutes.}

Adèle flew to the window. I followed, taking care to stand on one side,
so that, screened by the curtain, I could see without being seen.

The ten minutes John had given seemed very long, but at last wheels were
heard; four equestrians galloped up the drive, and after them came two
open carriages. Fluttering veils and waving plumes filled the vehicles;
two of the cavaliers were young, dashing-looking gentlemen; the third
was \Mr{} Rochester, on his black horse, Mesrour, Pilot bounding before
him; at his side rode a lady, and he and she were the first of the
party. Her purple riding-habit almost swept the ground, her veil
streamed long on the breeze; mingling with its transparent folds, and
gleaming through them, shone rich raven ringlets.

\enquote{Miss Ingram!} exclaimed \Mrs{} Fairfax, and away she hurried to
her post below.

The cavalcade, following the sweep of the drive, quickly turned the
angle of the house, and I lost sight of it. Adèle now petitioned to go
down; but I took her on my knee, and gave her to understand that she
must not on any account think of venturing in sight of the ladies,
either now or at any other time, unless expressly sent for: that \Mr{}
Rochester would be very angry, \etc. \enquote{Some natural tears she
	shed} on being told this; but as I began to look very grave, she
consented at last to wipe them.

A joyous stir was now audible in the hall: gentlemen's deep tones and
ladies' silvery accents blent harmoniously together, and distinguishable
above all, though not loud, was the sonorous voice of the master of
Thornfield Hall, welcoming his fair and gallant guests under its roof.
Then light steps ascended the stairs; and there was a tripping through
the gallery, and soft cheerful laughs, and opening and closing doors,
and, for a time, a hush.

\foreignquote{french}{Elles changent de toilettes,}\footnote{\enquote{They are changing their clothes.}}
said Adèle; who, listening attentively, had followed every movement; and she sighed.

\foreignquote{french}{Chez maman,} said she, \foreignquote{french}{quand il y avait du monde, je
	le suivais partout, au salon et à leurs chambres; souvent je regardais
	les femmes de chambre coiffer et habiller les dames, et c'était si
	amusant: comme cela on apprend.}\footnote{\enquote{At my mother's,} \textelp{}
	\enquote{when there were guests, I used to follow them everywhere, from the
		drawing room to their bedrooms; often I used to watch the maids arrange
		the ladies' hair and dress them, and that was so interesting: you can
		learn from that.}}

\enquote{Don't you feel hungry, Adèle?}

\foreignquote{french}{Mais oui, mademoiselle: voilà cinq ou six heures que nous
	n'avons pas mangé.}\footnote{\enquote{Oh yes, miss: it's been five or six hours since we've eaten.}}

\enquote{Well now, while the ladies are in their rooms, I will venture
	down and get you something to eat.}

And issuing from my asylum with precaution, I sought a back-stairs which
conducted directly to the kitchen. All in that region was fire and
commotion; the soup and fish were in the last stage of projection, and
the cook hung over her crucibles in a frame of mind and body threatening
spontaneous combustion. In the servants' hall two coachmen and three
gentlemen's gentlemen stood or sat round the fire; the abigails, I
suppose, were upstairs with their mistresses; the new servants, that had
been hired from Millcote, were bustling about everywhere. Threading
this chaos, I at last reached the larder; there I took possession of a
cold chicken, a roll of bread, some tarts, a plate or two and a knife
and fork: with this booty I made a hasty retreat. I had regained the
gallery, and was just shutting the back-door behind me, when an
accelerated hum warned me that the ladies were about to issue from their
chambers. I could not proceed to the schoolroom without passing some of
their doors, and running the risk of being surprised with my cargo of
victualage; so I stood still at this end, which, being windowless, was
dark: quite dark now, for the sun was set and twilight gathering.

Presently the chambers gave up their fair tenants one after another:
each came out gaily and airily, with dress that gleamed lustrous through
the dusk. For a moment they stood grouped together at the other
extremity of the gallery, conversing in a key of sweet subdued vivacity:
they then descended the staircase almost as noiselessly as a bright mist
rolls down a hill. Their collective appearance had left on me an
impression of high-born elegance, such as I had never before received.

I found Adèle peeping through the schoolroom door, which she held ajar.
\enquote{What beautiful ladies!} cried she in English. \enquote{Oh, I
	wish I might go to them! Do you think \Mr{} Rochester will send for us
	by-and-bye, after dinner?}

\enquote{No, indeed, I don't; \Mr{} Rochester has something else to think
	about. Never mind the ladies to-night; perhaps you will see them
	to-morrow: here is your dinner.}

She was really hungry, so the chicken and tarts served to divert her
attention for a time. It was well I secured this forage, or both she,
I, and Sophie, to whom I conveyed a share of our repast, would have run
a chance of getting no dinner at all: every one downstairs was too much
engaged to think of us. The dessert was not carried out till after nine
and at ten footmen were still running to and fro with trays and
coffee-cups. I allowed Adèle to sit up much later than usual; for she
declared she could not possibly go to sleep while the doors kept opening
and shutting below, and people bustling about. Besides, she added, a
message might possibly come from \Mr{} Rochester when she was undressed;
\foreignquote{french}{et alors quel dommage!}\footnote{\enquote{And then what a shame!}}

I told her stories as long as she would listen to them; and then for a
change I took her out into the gallery. The hall lamp was now lit, and
it amused her to look over the balustrade and watch the servants passing
backwards and forwards. When the evening was far advanced, a sound of
music issued from the drawing-room, whither the piano had been removed;
Adèle and I sat down on the top step of the stairs to listen. Presently
a voice blent with the rich tones of the instrument; it was a lady who
sang, and very sweet her notes were. The solo over, a duet followed,
and then a glee: a joyous conversational murmur filled up the
intervals. I listened long: suddenly I discovered that my ear was
wholly intent on analysing the mingled sounds, and trying to
discriminate amidst the confusion of accents those of \Mr{} Rochester; and
when it caught them, which it soon did, it found a further task in
framing the tones, rendered by distance inarticulate, into words.

The clock struck eleven. I looked at Adèle, whose head leant against my
shoulder; her eyes were waxing heavy, so I took her up in my arms and
carried her off to bed. It was near one before the gentlemen and ladies
sought their chambers.

The next day was as fine as its predecessor: it was devoted by the party
to an excursion to some site in the neighbourhood. They set out early
in the forenoon, some on horseback, the rest in carriages; I witnessed
both the departure and the return. Miss Ingram, as before, was the only
lady equestrian; and, as before, \Mr{} Rochester galloped at her side; the
two rode a little apart from the rest. I pointed out this circumstance
to \Mrs{} Fairfax, who was standing at the window with me---

\enquote{You said it was not likely they should think of being married,}
said I, \enquote{but you see \Mr{} Rochester evidently prefers her to any
	of the other ladies.}

\enquote{Yes, I daresay: no doubt he admires her.}

\enquote{And she him,} I added; \enquote{look how she leans her head
	towards him as if she were conversing confidentially; I wish I could see
	her face; I have never had a glimpse of it yet.}

\enquote{You will see her this evening,} answered \Mrs{} Fairfax.
\enquote{I happened to remark to \Mr{} Rochester how much Adèle wished to
	be introduced to the ladies, and he said: \enquote{Oh! let her come
		into the drawing-room after dinner; and request Miss Eyre to accompany
		her.}}

\enquote{Yes; he said that from mere politeness: I need not go, I am
	sure,} I answered.

\enquote{Well, I observed to him that as you were unused to company, I
	did not think you would like appearing before so gay a party---all
	strangers; and he replied, in his quick way---\enquote{Nonsense! If she
		objects, tell her it is my particular wish; and if she resists, say I
		shall come and fetch her in case of contumacy.}}

\enquote{I will not give him that trouble,} I answered. \enquote{I will
	go, if no better may be; but I don't like it. Shall you be there, \Mrs{}
	Fairfax?}

\enquote{No; I pleaded off, and he admitted my plea. I'll tell you how
	to manage so as to avoid the embarrassment of making a formal entrance,
	which is the most disagreeable part of the business. You must go into
	the drawing-room while it is empty, before the ladies leave the
	dinner-table; choose your seat in any quiet nook you like; you need not
	stay long after the gentlemen come in, unless you please: just let \Mr{}
	Rochester see you are there and then slip away---nobody will notice
	you.}

\enquote{Will these people remain long, do you think?}

\enquote{Perhaps two or three weeks, certainly not more. After the
	Easter recess, Sir George Lynn, who was lately elected member for
	Millcote, will have to go up to town and take his seat; I daresay \Mr{}
	Rochester will accompany him: it surprises me that he has already made
	so protracted a stay at Thornfield.}

It was with some trepidation that I perceived the hour approach when I
was to repair with my charge to the drawing-room. Adèle had been in a
state of ecstasy all day, after hearing she was to be presented to the
ladies in the evening; and it was not till Sophie commenced the
operation of dressing her that she sobered down. Then the importance of
the process quickly steadied her, and by the time she had her curls
arranged in well-smoothed, drooping clusters, her pink satin frock put
on, her long sash tied, and her lace mittens adjusted, she looked as
grave as any judge. No need to warn her not to disarrange her attire:
when she was dressed, she sat demurely down in her little chair, taking
care previously to lift up the satin skirt for fear she should crease
it, and assured me she would not stir thence till I was ready. This I
quickly was: my best dress (the silver-grey one, purchased for Miss
Temple's wedding, and never worn since) was soon put on; my hair was
soon smoothed; my sole ornament, the pearl brooch, soon assumed. We
descended.

Fortunately there was another entrance to the drawing-room than that
through the saloon where they were all seated at dinner. We found the
apartment vacant; a large fire burning silently on the marble hearth,
and wax candles shining in bright solitude, amid the exquisite flowers
with which the tables were adorned. The crimson curtain hung before the
arch: slight as was the separation this drapery formed from the party in
the adjoining saloon, they spoke in so low a key that nothing of their
conversation could be distinguished beyond a soothing murmur.

Adèle, who appeared to be still under the influence of a most
solemnising impression, sat down, without a word, on the footstool I
pointed out to her. I retired to a window-seat, and taking a book from
a table near, endeavoured to read. Adèle brought her stool to my feet;
ere long she touched my knee.

\enquote{What is it, Adèle?}

\foreignquote{french}{Est-ce que je ne puis pas prendrie une seule de ces fleurs
	magnifiques, mademoiselle? Seulement pour completer ma toilette.}\footnote{
	\enquote{May I take one of these glorious flowers, miss? Just to complete my outfit.}}

\enquote{You think too much of your \foreignquote{french}{toilette,} Adèle: but you
	may have a flower.} And I took a rose from a vase and fastened it in
her sash. She sighed a sigh of ineffable satisfaction, as if her cup of
happiness were now full. I turned my face away to conceal a smile I
could not suppress: there was something ludicrous as well as painful in
the little Parisienne's earnest and innate devotion to matters of dress.

A soft sound of rising now became audible; the curtain was swept back
from the arch; through it appeared the dining-room, with its lit lustre
pouring down light on the silver and glass of a magnificent
dessert-service covering a long table; a band of ladies stood in the
opening; they entered, and the curtain fell behind them.

There were but eight; yet, somehow, as they flocked in, they gave the
impression of a much larger number. Some of them were very tall; many
were dressed in white; and all had a sweeping amplitude of array that
seemed to magnify their persons as a mist magnifies the moon. I rose
and curtseyed to them: one or two bent their heads in return, the others
only stared at me.

They dispersed about the room, reminding me, by the lightness and
buoyancy of their movements, of a flock of white plumy birds. Some of
them threw themselves in half-reclining positions on the sofas and
ottomans: some bent over the tables and examined the flowers and books:
the rest gathered in a group round the fire: all talked in a low but
clear tone which seemed habitual to them. I knew their names
afterwards, and may as well mention them now.

First, there was \Mrs{} Eshton and two of her daughters. She had
evidently been a handsome woman, and was well preserved still. Of her
daughters, the eldest, Amy, was rather little: naive, and child-like in
face and manner, and piquant in form; her white muslin dress and blue
sash became her well. The second, Louisa, was taller and more elegant
in figure; with a very pretty face, of that order the French term
\foreignlanguage{french}{\emph{minois chiffoné:}}\footnote{
	Literally \emph{\enquote{crumpled little face;}} someone who is attractive in an unusual way.} both sisters were fair as lilies. %TODO include colon?

Lady Lynn was a large and stout personage of about forty, very erect,
very haughty-looking, richly dressed in a satin robe of changeful sheen:
her dark hair shone glossily under the shade of an azure plume, and
within the circlet of a band of gems.

\Mrs{} Colonel Dent was less showy; but, I thought, more lady-like. She
had a slight figure, a pale, gentle face, and fair hair. Her black
satin dress, her scarf of rich foreign lace, and her pearl ornaments,
pleased me better than the rainbow radiance of the titled dame.

But the three most distinguished---partly, perhaps, because the tallest
figures of the band---were the Dowager Lady Ingram and her daughters,
Blanche and Mary. They were all three of the loftiest stature of
women. The Dowager might be between forty and fifty: her shape was
still fine; her hair (by candle-light at least) still black; her teeth,
too, were still apparently perfect. Most people would have termed her a
splendid woman of her age: and so she was, no doubt, physically
speaking; but then there was an expression of almost insupportable
haughtiness in her bearing and countenance. She had Roman features and
a double chin, disappearing into a throat like a pillar: these features
appeared to me not only inflated and darkened, but even furrowed with
pride; and the chin was sustained by the same principle, in a position
of almost preternatural erectness. She had, likewise, a fierce and a
hard eye: it reminded me of \Mrs{} Reed's; she mouthed her words in
speaking; her voice was deep, its inflections very pompous, very
dogmatical,---very intolerable, in short. A crimson velvet robe, and a
shawl turban of some gold-wrought Indian fabric, invested her (I suppose
she thought) with a truly imperial dignity.

Blanche and Mary were of equal stature,---straight and tall as poplars.
Mary was too slim for her height, but Blanche was moulded like a Dian.
I regarded her, of course, with special interest. First, I wished to
see whether her appearance accorded with \Mrs{} Fairfax's description;
secondly, whether it at all resembled the fancy miniature I had painted
of her; and thirdly---it will out!---whether it were such as I should
fancy likely to suit \Mr{} Rochester's taste.

As far as person went, she answered point for point, both to my picture
and \Mrs{} Fairfax's description. The noble bust, the sloping shoulders,
the graceful neck, the dark eyes and black ringlets were all
there;---but her face? Her face was like her mother's; a youthful
unfurrowed likeness: the same low brow, the same high features, the same
pride. It was not, however, so saturnine a pride! she laughed
continually; her laugh was satirical, and so was the habitual expression
of her arched and haughty lip.

Genius is said to be self-conscious. I cannot tell whether Miss Ingram
was a genius, but she was self-conscious---remarkably self-conscious
indeed. She entered into a discourse on botany with the gentle \Mrs{}
Dent. It seemed \Mrs{} Dent had not studied that science: though, as she
said, she liked flowers, \enquote{especially wild ones;} Miss Ingram
had, and she ran over its vocabulary with an air. I presently perceived
she was (what is vernacularly termed) \emph{trailing} \Mrs{} Dent; that
is, playing on her ignorance---her \emph{trail} might be clever, but it
was decidedly not good-natured. She played: her execution was
brilliant; she sang: her voice was fine; she talked French apart to her
mamma; and she talked it well, with fluency and with a good accent.

Mary had a milder and more open countenance than Blanche; softer
features too, and a skin some shades fairer (Miss Ingram was dark as a
Spaniard)---but Mary was deficient in life: her face lacked expression,
her eye lustre; she had nothing to say, and having once taken her seat,
remained fixed like a statue in its niche. The sisters were both
attired in spotless white.

And did I now think Miss Ingram such a choice as \Mr{} Rochester would be
likely to make? I could not tell---I did not know his taste in female
beauty. If he liked the majestic, she was the very type of majesty:
then she was accomplished, sprightly. Most gentlemen would admire her,
I thought; and that he \emph{did} admire her, I already seemed to have
obtained proof: to remove the last shade of doubt, it remained but to
see them together.

You are not to suppose, reader, that Adèle has all this time been
sitting motionless on the stool at my feet: no; when the ladies entered,
she rose, advanced to meet them, made a stately reverence, and said with
gravity---

\foreignquote{french}{Bon jour, mesdames.}

And Miss Ingram had looked down at her with a mocking air, and
exclaimed, \enquote{Oh, what a little puppet!}

Lady Lynn had remarked, \enquote{It is \Mr{} Rochester's ward, I
	suppose---the little French girl he was speaking of.}

\Mrs{} Dent had kindly taken her hand, and given her a kiss.

Amy and Louisa Eshton had cried out simultaneously---\enquote{What a
	love of a child!}

And then they had called her to a sofa, where she now sat, ensconced
between them, chattering alternately in French and broken English;
absorbing not only the young ladies' attention, but that of \Mrs{} Eshton
and Lady Lynn, and getting spoilt to her heart's content.

At last coffee is brought in, and the gentlemen are summoned. I sit in
the shade---if any shade there be in this brilliantly-lit apartment; the
window-curtain half hides me. Again the arch yawns; they come. The
collective appearance of the gentlemen, like that of the ladies, is very
imposing: they are all costumed in black; most of them are tall, some
young. Henry and Frederick Lynn are very dashing sparks indeed; and
Colonel Dent is a fine soldierly man. \Mr{} Eshton, the magistrate of the
district, is gentleman-like: his hair is quite white, his eyebrows and
whiskers still dark, which gives him something of the appearance of a
\foreignquote{french}{père noble de théâtre.}\footnote{
	\enquote{A father of the theatre.}} Lord Ingram, like his sisters, is very
tall; like them, also, he is handsome; but he shares Mary's apathetic
and listless look: he seems to have more length of limb than vivacity of
blood or vigour of brain.

And where is \Mr{} Rochester?

He comes in last: I am not looking at the arch, yet I see him enter. I
try to concentrate my attention on those netting-needles, on the meshes
of the purse I am forming---I wish to think only of the work I have in
my hands, to see only the silver beads and silk threads that lie in my
lap; whereas, I distinctly behold his figure, and I inevitably recall
the moment when I last saw it; just after I had rendered him, what he
deemed, an essential service, and he, holding my hand, and looking down
on my face, surveyed me with eyes that revealed a heart full and eager
to overflow; in whose emotions I had a part. How near had I approached
him at that moment! What had occurred since, calculated to change his
and my relative positions? Yet now, how distant, how far estranged we
were! So far estranged, that I did not expect him to come and speak to
me. I did not wonder, when, without looking at me, he took a seat at
the other side of the room, and began conversing with some of the
ladies.

No sooner did I see that his attention was riveted on them, and that I
might gaze without being observed, than my eyes were drawn involuntarily
to his face; I could not keep their lids under control: they would rise,
and the irids would fix on him. I looked, and had an acute pleasure in
looking,---a precious yet poignant pleasure; pure gold, with a steely
point of agony: a pleasure like what the thirst-perishing man might feel
who knows the well to which he has crept is poisoned, yet stoops and
drinks divine draughts nevertheless.

Most true is it that \enquote{beauty is in the eye of the gazer.} My
master's colourless, olive face, square, massive brow, broad and jetty
eyebrows, deep eyes, strong features, firm, grim mouth,---all energy,
decision, will,---were not beautiful, according to rule; but they were
more than beautiful to me; they were full of an interest, an influence
that quite mastered me,---that took my feelings from my own power and
fettered them in his. I had not intended to love him; the reader knows
I had wrought hard to extirpate from my soul the germs of love there
detected; and now, at the first renewed view of him, they spontaneously
arrived, green and strong! He made me love him without looking at me.

I compared him with his guests. What was the gallant grace of the
Lynns, the languid elegance of Lord Ingram,---even the military
distinction of Colonel Dent, contrasted with his look of native pith and
genuine power? I had no sympathy in their appearance, their expression:
yet I could imagine that most observers would call them attractive,
handsome, imposing; while they would pronounce \Mr{} Rochester at once
harsh-featured and melancholy-looking. I saw them smile, laugh---it was
nothing; the light of the candles had as much soul in it as their smile;
the tinkle of the bell as much significance as their laugh. I saw \Mr{}
Rochester smile:---his stern features softened; his eye grew both
brilliant and gentle, its ray both searching and sweet. He was talking,
at the moment, to Louisa and Amy Eshton. I wondered to see them receive
with calm that look which seemed to me so penetrating: I expected their
eyes to fall, their colour to rise under it; yet I was glad when I found
they were in no sense moved. \enquote{He is not to them what he is to
	me,} I thought: \enquote{he is not of their kind. I believe he is of
	mine;---I am sure he is---I feel akin to him---I understand the language
	of his countenance and movements: though rank and wealth sever us
	widely, I have something in my brain and heart, in my blood and nerves,
	that assimilates me mentally to him. Did I say, a few days since, that
	I had nothing to do with him but to receive my salary at his hands? Did
	I forbid myself to think of him in any other light than as a paymaster?
	Blasphemy against nature! Every good, true, vigorous feeling I have
	gathers impulsively round him. I know I must conceal my sentiments: I
	must smother hope; I must remember that he cannot care much for me. For
	when I say that I am of his kind, I do not mean that I have his force to
	influence, and his spell to attract; I mean only that I have certain
	tastes and feelings in common with him. I must, then, repeat
	continually that we are for ever sundered:---and yet, while I breathe
	and think, I must love him.}

Coffee is handed. The ladies, since the gentlemen entered, have become
lively as larks; conversation waxes brisk and merry. Colonel Dent and
\Mr{} Eshton argue on politics; their wives listen. The two proud
dowagers, Lady Lynn and Lady Ingram, confabulate together. Sir
George---whom, by-the-bye, I have forgotten to describe,---a very big,
and very fresh-looking country gentleman, stands before their sofa,
coffee-cup in hand, and occasionally puts in a word. \Mr{} Frederick Lynn
has taken a seat beside Mary Ingram, and is showing her the engravings
of a splendid volume: she looks, smiles now and then, but apparently
says little. The tall and phlegmatic Lord Ingram leans with folded arms
on the chair-back of the little and lively Amy Eshton; she glances up at
him, and chatters like a wren: she likes him better than she does \Mr{}
Rochester. Henry Lynn has taken possession of an ottoman at the feet of
Louisa: Adèle shares it with him: he is trying to talk French with her,
and Louisa laughs at his blunders. With whom will Blanche Ingram pair?
She is standing alone at the table, bending gracefully over an album.
She seems waiting to be sought; but she will not wait too long: she
herself selects a mate.

\Mr{} Rochester, having quitted the Eshtons, stands on the hearth as
solitary as she stands by the table: she confronts him, taking her
station on the opposite side of the mantelpiece.

\enquote{\Mr{} Rochester, I thought you were not fond of children?}

\enquote{Nor am I\@.}

\enquote{Then, what induced you to take charge of such a little doll as
	that?} (pointing to Adèle). \enquote{Where did you pick her up?}

\enquote{I did not pick her up; she was left on my hands.}

\enquote{You should have sent her to school.}

\enquote{I could not afford it: schools are so dear.}

\enquote{Why, I suppose you have a governess for her: I saw a person
	with her just now---is she gone? Oh, no! there she is still, behind the
	window-curtain. You pay her, of course; I should think it quite as
	expensive,---more so; for you have them both to keep in addition.}

I feared---or should I say, hoped?---the allusion to me would make \Mr{}
Rochester glance my way; and I involuntarily shrank farther into the
shade: but he never turned his eyes.

\enquote{I have not considered the subject,} said he indifferently,
looking straight before him.

\enquote{No, you men never do consider economy and common sense. You
	should hear mama on the chapter of governesses: Mary and I have had, I
	should think, a dozen at least in our day; half of them detestable and
	the rest ridiculous, and all incubi---were they not, mama?}

\enquote{Did you speak, my own?}

The young lady thus claimed as the dowager's special property,
reiterated her question with an explanation.

\enquote{My dearest, don't mention governesses; the word makes me
	nervous. I have suffered a martyrdom from their incompetency and
	caprice. I thank Heaven I have now done with them!}

\Mrs{} Dent here bent over to the pious lady and whispered something in
her ear; I suppose, from the answer elicited, it was a reminder that one
of the anathematised race was present.

\foreignquote{french}{Tant pis!}\footnote{\enquote{Never mind!}} said her Ladyship, \enquote{I hope it may do her
	good!} Then, in a lower tone, but still loud enough for me to hear,
\enquote{I noticed her; I am a judge of physiognomy, and in hers I see
	all the faults of her class.}

\enquote{What are they, madam?} inquired \Mr{} Rochester aloud.

\enquote{I will tell you in your private ear,} replied she, wagging her
turban three times with portentous significancy.

\enquote{But my curiosity will be past its appetite; it craves food
	now.}

\enquote{Ask Blanche; she is nearer you than I\@.}

\enquote{Oh, don't refer him to me, mama! I have just one word to say
	of the whole tribe; they are a nuisance. Not that I ever suffered much
	from them; I took care to turn the tables. What tricks Theodore and I
	used to play on our Miss Wilsons, and \Mrs{} Greys, and Madame Jouberts!
	Mary was always too sleepy to join in a plot with spirit. The best fun
	was with Madame Joubert: Miss Wilson was a poor sickly thing, lachrymose
	and low-spirited, not worth the trouble of vanquishing, in short; and
	\Mrs{} Grey was coarse and insensible; no blow took effect on her. But
	poor Madame Joubert! I see her yet in her raging passions, when we had
	driven her to extremities---spilt our tea, crumbled our bread and
	butter, tossed our books up to the ceiling, and played a charivari with
	the ruler and desk, the fender and fire-irons. Theodore, do you
	remember those merry days?}

\enquote{Yaas, to be sure I do,} drawled Lord Ingram; \enquote{and the
	poor old stick used to cry out \enquote{Oh you villains childs!}---and
	then we sermonised her on the presumption of attempting to teach such
	clever blades as we were, when she was herself so ignorant.}

\enquote{We did; and, Tedo, you know, I helped you in prosecuting (or
	persecuting) your tutor, whey-faced \Mr{} Vining---the parson in the pip,
	as we used to call him. He and Miss Wilson took the liberty of falling
	in love with each other---at least Tedo and I thought so; we surprised
	sundry tender glances and sighs which we interpreted as tokens of
	\enquote{la belle passion,} and I promise you the public soon had the
	benefit of our discovery; we employed it as a sort of lever to hoist our
	dead-weights from the house. Dear mama, there, as soon as she got an
	inkling of the business, found out that it was of an immoral tendency.
	Did you not, my lady-mother?}

\enquote{Certainly, my best. And I was quite right: depend on that:
	there are a thousand reasons why liaisons between governesses and tutors
	should never be tolerated a moment in any well-regulated house;
	firstly---}

\enquote{Oh, gracious, mama! Spare us the enumeration! \foreignquote{french}{\emph{Au reste,}} we %TODO include comma
	all know them: danger of bad example to innocence of childhood;
	distractions and consequent neglect of duty on the part of the
	attached---mutual alliance and reliance; confidence thence
	resulting---insolence accompanying---mutiny and general blow-up. Am I
	right, Baroness Ingram, of Ingram Park?}

\enquote{My lily-flower, you are right now, as always.}

\enquote{Then no more need be said: change the subject.}

Amy Eshton, not hearing or not heeding this dictum, joined in with her
soft, infantine tone: \enquote{Louisa and I used to quiz our governess
	too; but she was such a good creature, she would bear anything: nothing
	put her out. She was never cross with us; was she, Louisa?}

\enquote{No, never: we might do what we pleased; ransack her desk and
	her workbox, and turn her drawers inside out; and she was so
	good-natured, she would give us anything we asked for.}

\enquote{I suppose, now,} said Miss Ingram, curling her lip
sarcastically, \enquote{we shall have an abstract of the memoirs of all
	the governesses extant: in order to avert such a visitation, I again
	move the introduction of a new topic. \Mr{} Rochester, do you second my
	motion?}

\enquote{Madam, I support you on this point, as on every other.}

\enquote{Then on me be the onus of bringing it forward. Signior
	Eduardo, are you in voice to-night?}

\enquote{Donna Bianca, if you command it, I will be.}

\enquote{Then, signior, I lay on you my sovereign behest to furbish up
	your lungs and other vocal organs, as they will be wanted on my royal
	service.}

\enquote{Who would not be the Rizzio of so divine a Mary?}

\enquote{A fig for Rizzio!} cried she, tossing her head with all its
curls, as she moved to the piano. \enquote{It is my opinion the fiddler
	David must have been an insipid sort of fellow; I like black Bothwell
	better: to my mind a man is nothing without a spice of the devil in him;
	and history may say what it will of James Hepburn, but I have a notion,
	he was just the sort of wild, fierce, bandit hero whom I could have
	consented to gift with my hand.}

\enquote{Gentlemen, you hear! Now which of you most resembles
	Bothwell?} cried \Mr{} Rochester.

\enquote{I should say the preference lies with you,} responded Colonel
Dent.

\enquote{On my honour, I am much obliged to you,} was the reply.

Miss Ingram, who had now seated herself with proud grace at the piano,
spreading out her snowy robes in queenly amplitude, commenced a
brilliant prelude; talking meantime. She appeared to be on her high
horse to-night; both her words and her air seemed intended to excite not
only the admiration, but the amazement of her auditors: she was
evidently bent on striking them as something very dashing and daring
indeed.

\enquote{Oh, I am so sick of the young men of the present day!}
exclaimed she, rattling away at the instrument. \enquote{Poor, puny things,
	not fit to stir a step beyond papa's park gates: nor to go even so far
	without mama's permission and guardianship! Creatures so absorbed in
	care about their pretty faces, and their white hands, and their small
	feet; as if a man had anything to do with beauty! As if loveliness were
	not the special prerogative of woman---her legitimate appanage and
	heritage! I grant an ugly \emph{woman} is a blot on the fair face of
	creation; but as to the \emph{gentlemen}, let them be solicitous to
	possess only strength and valour: let their motto be:---Hunt, shoot, and
	fight: the rest is not worth a fillip. Such should be my device, were I
	a man.}

\enquote{Whenever I marry,} she continued after a pause which none
interrupted, \enquote{I am resolved my husband shall not be a rival, but
	a foil to me. I will suffer no competitor near the throne; I shall
	exact an undivided homage: his devotions shall not be shared between me
	and the shape he sees in his mirror. \Mr{} Rochester, now sing, and I
	will play for you.}

\enquote{I am all obedience,} was the response.

\enquote{Here then is a Corsair-song. Know that I doat on Corsairs; and for
	that reason, sing it \emph{con spirito}.}

\enquote{Commands from Miss Ingram's lips would put spirit into a mug of
	milk and water.}

\enquote{Take care, then: if you don't please me, I will shame you by showing
	how such things \emph{should} be done.}

\enquote{That is offering a premium on incapacity: I shall now endeavour
	to fail.}

\enquote{\foreignlanguage{french}{Gardez-vous en bien!}\footnote{\enquote{Take Care!} (A term from fencing.)} If you err wilfully, I shall devise a
	proportionate punishment.}

\enquote{Miss Ingram ought to be clement, for she has it in her power to
	inflict a chastisement beyond mortal endurance.}

\enquote{Ha! explain!} commanded the lady.

\enquote{Pardon me, madam: no need of explanation; your own fine sense
	must inform you that one of your frowns would be a sufficient substitute
	for capital punishment.}

\enquote{Sing!} said she, and again touching the piano, she commenced an
accompaniment in spirited style.

\enquote{Now is my time to slip away,} thought I: but the tones that
then severed the air arrested me. \Mrs{} Fairfax had said \Mr{} Rochester
possessed a fine voice: he did---a mellow, powerful bass, into which he
threw his own feeling, his own force; finding a way through the ear to
the heart, and there waking sensation strangely. I waited till the last
deep and full vibration had expired---till the tide of talk, checked an
instant, had resumed its flow; I then quitted my sheltered corner and
made my exit by the side-door, which was fortunately near. Thence a
narrow passage led into the hall: in crossing it, I perceived my sandal
was loose; I stopped to tie it, kneeling down for that purpose on the
mat at the foot of the staircase. I heard the dining-room door unclose;
a gentleman came out; rising hastily, I stood face to face with him: it
was \Mr{} Rochester.

\enquote{How do you do?} he asked.

\enquote{I am very well, sir.}

\enquote{Why did you not come and speak to me in the room?}

I thought I might have retorted the question on him who put it: but I
would not take that freedom. I answered---

\enquote{I did not wish to disturb you, as you seemed engaged, sir.}

\enquote{What have you been doing during my absence?}

\enquote{Nothing particular; teaching Adèle as usual.}

\enquote{And getting a good deal paler than you were---as I saw at first
	sight. What is the matter?}

\enquote{Nothing at all, sir.}

\enquote{Did you take any cold that night you half drowned me?}

\enquote{Not the least.}

\enquote{Return to the drawing-room: you are deserting too early.}

\enquote{I am tired, sir.}

He looked at me for a minute.

\enquote{And a little depressed,} he said. \enquote{What about? Tell
	me.}

\enquote{Nothing---nothing, sir. I am not depressed.}

\enquote{But I affirm that you are: so much depressed that a few more
	words would bring tears to your eyes---indeed, they are there now,
	shining and swimming; and a bead has slipped from the lash and fallen on
	to the flag. If I had time, and was not in mortal dread of some prating
	prig of a servant passing, I would know what all this means. Well,
	to-night I excuse you; but understand that so long as my visitors stay,
	I expect you to appear in the drawing-room every evening; it is my wish;
	don't neglect it. Now go, and send Sophie for Adèle. Good-night,
	my---} He stopped, bit his lip, and abruptly left me.
\FChapter{Chapter Seven}{7}

\Lettrine{M}{y} \textsc{first quarter} at Lowood seemed an age; and not the golden age either;
it comprised an irksome struggle with difficulties in habituating myself
to new rules and unwonted tasks. The fear of failure in these points
harassed me worse than the physical hardships of my lot; though these
were no trifles.

During January, February, and part of March, the deep snows, and, after
their melting, the almost impassable roads, prevented our stirring
beyond the garden walls, except to go to church; but within these limits
we had to pass an hour every day in the open air. Our clothing was
insufficient to protect us from the severe cold: we had no boots, the
snow got into our shoes and melted there: our ungloved hands became
numbed and covered with chilblains, as were our feet: I remember well
the distracting irritation I endured from this cause every evening, when
my feet inflamed; and the torture of thrusting the swelled, raw, and
stiff toes into my shoes in the morning. Then the scanty supply of food
was distressing: with the keen appetites of growing children, we had
scarcely sufficient to keep alive a delicate invalid. From this
deficiency of nourishment resulted an abuse, which pressed hardly on the
younger pupils: whenever the famished great girls had an opportunity,
they would coax or menace the little ones out of their portion. Many a
time I have shared between two claimants the precious morsel of brown
bread distributed at tea-time; and after relinquishing to a third half
the contents of my mug of coffee, I have swallowed the remainder with an
accompaniment of secret tears, forced from me by the exigency of hunger.

Sundays were dreary days in that wintry season. We had to walk two
miles to Brocklebridge Church, where our patron officiated. We set out
cold, we arrived at church colder: during the morning service we became
almost paralysed. It was too far to return to dinner, and an allowance
of cold meat and bread, in the same penurious proportion observed in our
ordinary meals, was served round between the services.

At the close of the afternoon service we returned by an exposed and
hilly road, where the bitter winter wind, blowing over a range of snowy
summits to the north, almost flayed the skin from our faces.

I can remember Miss Temple walking lightly and rapidly along our
drooping line, her plaid cloak, which the frosty wind fluttered,
gathered close about her, and encouraging us, by precept and example, to
keep up our spirits, and march forward, as she said, \enquote{like
	stalwart soldiers.} The other teachers, poor things, were generally
themselves too much dejected to attempt the task of cheering others.

How we longed for the light and heat of a blazing fire when we got
back! But, to the little ones at least, this was denied: each hearth in
the schoolroom was immediately surrounded by a double row of great
girls, and behind them the younger children crouched in groups, wrapping
their starved arms in their pinafores.

A little solace came at tea-time, in the shape of a double ration of
bread---a whole, instead of a half, slice---with the delicious addition
of a thin scrape of butter: it was the hebdomadal treat to which we all
looked forward from Sabbath to Sabbath. I generally contrived to
reserve a moiety of this bounteous repast for myself; but the remainder
I was invariably obliged to part with.

The Sunday evening was spent in repeating, by heart, the Church
Catechism, and the fifth, sixth, and seventh chapters of \St{} Matthew;
and in listening to a long sermon, read by Miss Miller, whose
irrepressible yawns attested her weariness. A frequent interlude of
these performances was the enactment of the part of Eutychus by some
half-dozen of little girls, who, overpowered with sleep, would fall
down, if not out of the third loft, yet off the fourth form, and be
taken up half dead. The remedy was, to thrust them forward into the
centre of the schoolroom, and oblige them to stand there till the sermon
was finished. Sometimes their feet failed them, and they sank together
in a heap; they were then propped up with the monitors' high stools.

I have not yet alluded to the visits of \Mr{} Brocklehurst; and indeed
that gentleman was from home during the greater part of the first month
after my arrival; perhaps prolonging his stay with his friend the
archdeacon: his absence was a relief to me. I need not say that I had
my own reasons for dreading his coming: but come he did at last.

One afternoon (I had then been three weeks at Lowood), as I was sitting
with a slate in my hand, puzzling over a sum in long division, my eyes,
raised in abstraction to the window, caught sight of a figure just
passing: I recognised almost instinctively that gaunt outline; and when,
two minutes after, all the school, teachers included, rose \emph{en
	masse}, it was not necessary for me to look up in order to ascertain
whose entrance they thus greeted. A long stride measured the
schoolroom, and presently beside Miss Temple, who herself had risen,
stood the same black column which had frowned on me so ominously from
the hearthrug of Gateshead. I now glanced sideways at this piece of
architecture. Yes, I was right: it was \Mr{} Brocklehurst, buttoned up in
a surtout, and looking longer, narrower, and more rigid than ever.

I had my own reasons for being dismayed at this apparition; too well I
remembered the perfidious hints given by \Mrs{} Reed about my disposition,
\etc; the promise pledged by \Mr{} Brocklehurst to apprise Miss Temple and
the teachers of my vicious nature. All along I had been dreading the
fulfilment of this promise,---I had been looking out daily for the
\enquote{Coming Man,} whose information respecting my past life and
conversation was to brand me as a bad child for ever: now there he was.

He stood at Miss Temple's side; he was speaking low in her ear: I did
not doubt he was making disclosures of my villainy; and I watched her
eye with painful anxiety, expecting every moment to see its dark orb
turn on me a glance of repugnance and contempt. I listened too; and as
I happened to be seated quite at the top of the room, I caught most of
what he said: its import relieved me from immediate apprehension.

\enquote{I suppose, Miss Temple, the thread I bought at Lowton will do;
	it struck me that it would be just of the quality for the calico
	chemises, and I sorted the needles to match. You may tell Miss Smith
	that I forgot to make a memorandum of the darning needles, but she shall
	have some papers sent in next week; and she is not, on any account, to
	give out more than one at a time to each pupil: if they have more, they
	are apt to be careless and lose them. And, O ma'am! I wish the woollen
	stockings were better looked to!---when I was here last, I went into the
	kitchen-garden and examined the clothes drying on the line; there was a
	quantity of black hose in a very bad state of repair: from the size of
	the holes in them I was sure they had not been well mended from time to
	time.}

He paused.

\enquote{Your directions shall be attended to, sir,} said Miss Temple.

\enquote{And, ma'am,} he continued, \enquote{the laundress tells me some
	of the girls have two clean tuckers in the week: it is too much; the
	rules limit them to one.}

\enquote{I think I can explain that circumstance, sir. Agnes and
	Catherine Johnstone were invited to take tea with some friends at Lowton
	last Thursday, and I gave them leave to put on clean tuckers for the
	occasion.}

\Mr{} Brocklehurst nodded.

\enquote{Well, for once it may pass; but please not to let the
	circumstance occur too often. And there is another thing which
	surprised me; I find, in settling accounts with the housekeeper, that a
	lunch, consisting of bread and cheese, has twice been served out to the
	girls during the past fortnight. How is this? I looked over the
	regulations, and I find no such meal as lunch mentioned. Who introduced
	this innovation? and by what authority?}

\enquote{I must be responsible for the circumstance, sir,} replied Miss
Temple: \enquote{the breakfast was so ill prepared that the pupils could
	not possibly eat it; and I dared not allow them to remain fasting till
	dinner-time.}

\enquote{Madam, allow me an instant. You are aware that my plan in
	bringing up these girls is, not to accustom them to habits of luxury and
	indulgence, but to render them hardy, patient, self-denying. Should any
	little accidental disappointment of the appetite occur, such as the
	spoiling of a meal, the under or the over dressing of a dish, the
	incident ought not to be neutralised by replacing with something more
	delicate the comfort lost, thus pampering the body and obviating the aim
	of this institution; it ought to be improved to the spiritual
	edification of the pupils, by encouraging them to evince fortitude under
	temporary privation. A brief address on those occasions would not be
	mistimed, wherein a judicious instructor would take the opportunity of
	referring to the sufferings of the primitive Christians; to the torments
	of martyrs; to the exhortations of our blessed Lord Himself, calling
	upon His disciples to take up their cross and follow Him; to His
	warnings that man shall not live by bread alone, but by every word that
	proceedeth out of the mouth of God; to His divine consolations, \enquote{If ye
		suffer hunger or thirst for My sake, happy are ye.} Oh, madam, when you
	put bread and cheese, instead of burnt porridge, into these children's
	mouths, you may indeed feed their vile bodies, but you little think how
	you starve their immortal souls!}

\Mr{} Brocklehurst again paused---perhaps overcome by his feelings. Miss
Temple had looked down when he first began to speak to her; but she now
gazed straight before her, and her face, naturally pale as marble,
appeared to be assuming also the coldness and fixity of that material;
especially her mouth, closed as if it would have required a sculptor's
chisel to open it, and her brow settled gradually into petrified
severity.

Meantime, \Mr{} Brocklehurst, standing on the hearth with his hands behind
his back, majestically surveyed the whole school. Suddenly his eye gave
a blink, as if it had met something that either dazzled or shocked its
pupil; turning, he said in more rapid accents than he had hitherto
used---

\enquote{Miss Temple, Miss Temple, what---\emph{what} is that girl with curled
	hair? Red hair, ma'am, curled---curled all over?} And extending his
cane he pointed to the awful object, his hand shaking as he did so.

\enquote{It is Julia Severn,} replied Miss Temple, very quietly.

\enquote{Julia Severn, ma'am! And why has she, or any other, curled
	hair? Why, in defiance of every precept and principle of this house,
	does she conform to the world so openly---here in an evangelical,
	charitable establishment---as to wear her hair one mass of curls?}

\enquote{Julia's hair curls naturally,} returned Miss Temple, still more
quietly.

\enquote{Naturally! Yes, but we are not to conform to nature; I wish
	these girls to be the children of Grace: and why that abundance? I have
	again and again intimated that I desire the hair to be arranged closely,
	modestly, plainly. Miss Temple, that girl's hair must be cut off
	entirely; I will send a barber to-morrow: and I see others who have far
	too much of the excrescence---that tall girl, tell her to turn round.
	Tell all the first form to rise up and direct their faces to the wall.}

Miss Temple passed her handkerchief over her lips, as if to smooth away
the involuntary smile that curled them; she gave the order, however, and
when the first class could take in what was required of them, they
obeyed. Leaning a little back on my bench, I could see the looks and
grimaces with which they commented on this manoeuvre: it was a pity \Mr{}
Brocklehurst could not see them too; he would perhaps have felt that,
whatever he might do with the outside of the cup and platter, the inside
was further beyond his interference than he imagined.

He scrutinised the reverse of these living medals some five minutes,
then pronounced sentence. These words fell like the knell of doom---

\enquote{All those top-knots must be cut off.}

Miss Temple seemed to remonstrate.

\enquote{Madam,} he pursued, \enquote{I have a Master to serve whose
	kingdom is not of this world: my mission is to mortify in these girls
	the lusts of the flesh; to teach them to clothe themselves with
	shame-facedness and sobriety, not with braided hair and costly apparel;
	and each of the young persons before us has a string of hair twisted in
	plaits which vanity itself might have woven; these, I repeat, must be
	cut off; think of the time wasted, of---}

\Mr{} Brocklehurst was here interrupted: three other visitors, ladies, now
entered the room. They ought to have come a little sooner to have heard
his lecture on dress, for they were splendidly attired in velvet, silk,
and furs. The two younger of the trio (fine girls of sixteen and
seventeen) had grey beaver hats, then in fashion, shaded with ostrich
plumes, and from under the brim of this graceful head-dress fell a
profusion of light tresses, elaborately curled; the elder lady was
enveloped in a costly velvet shawl, trimmed with ermine, and she wore a
false front of French curls.

These ladies were deferentially received by Miss Temple, as \Mrs{} and the
Misses Brocklehurst, and conducted to seats of honour at the top of the
room. It seems they had come in the carriage with their reverend
relative, and had been conducting a rummaging scrutiny of the room
upstairs, while he transacted business with the housekeeper, questioned
the laundress, and lectured the superintendent. They now proceeded to
address divers remarks and reproofs to Miss Smith, who was charged with
the care of the linen and the inspection of the dormitories: but I had
no time to listen to what they said; other matters called off and
enchanted my attention.

Hitherto, while gathering up the discourse of \Mr{} Brocklehurst and Miss
Temple, I had not, at the same time, neglected precautions to secure my
personal safety; which I thought would be effected, if I could only
elude observation. To this end, I had sat well back on the form, and
while seeming to be busy with my sum, had held my slate in such a manner
as to conceal my face: I might have escaped notice, had not my
treacherous slate somehow happened to slip from my hand, and falling
with an obtrusive crash, directly drawn every eye upon me; I knew it was
all over now, and, as I stooped to pick up the two fragments of slate, I
rallied my forces for the worst. It came.

\enquote{A careless girl!} said \Mr{} Brocklehurst, and immediately
after---\enquote{It is the new pupil, I perceive.} And before I could
draw breath, \enquote{I must not forget I have a word to say respecting
	her.} Then aloud: how loud it seemed to me! \enquote{Let the child who
	broke her slate come forward!}

Of my own accord I could not have stirred; I was paralysed: but the two
great girls who sit on each side of me, set me on my legs and pushed me
towards the dread judge, and then Miss Temple gently assisted me to his
very feet, and I caught her whispered counsel---

\enquote{Don't be afraid, Jane, I saw it was an accident; you shall not
	be punished.}

The kind whisper went to my heart like a dagger.

\enquote{Another minute, and she will despise me for a hypocrite,}
thought I; and an impulse of fury against Reed, Brocklehurst, and Co.
bounded in my pulses at the conviction. I was no Helen Burns.

\enquote{Fetch that stool,} said \Mr{} Brocklehurst, pointing to a very
high one from which a monitor had just risen: it was brought.

\enquote{Place the child upon it.}

And I was placed there, by whom I don't know: I was in no condition to
note particulars; I was only aware that they had hoisted me up to the
height of \Mr{} Brocklehurst's nose, that he was within a yard of me, and
that a spread of shot orange and purple silk pelisses and a cloud of
silvery plumage extended and waved below me.

\Mr{} Brocklehurst hemmed.

\enquote{Ladies,} said he, turning to his family, \enquote{Miss Temple,
	teachers, and children, you all see this girl?}

Of course they did; for I felt their eyes directed like burning-glasses
against my scorched skin.

\enquote{You see she is yet young; you observe she possesses the
	ordinary form of childhood; God has graciously given her the shape that
	He has given to all of us; no signal deformity points her out as a
	marked character. Who would think that the Evil One had already found a
	servant and agent in her? Yet such, I grieve to say, is the case.}

A pause---in which I began to steady the palsy of my nerves, and to feel
that the Rubicon was passed; and that the trial, no longer to be
shirked, must be firmly sustained.

\enquote{My dear children,} pursued the black marble clergyman, with
pathos, \enquote{this is a sad, a melancholy occasion; for it becomes my
	duty to warn you, that this girl, who might be one of God's own lambs,
	is a little castaway: not a member of the true flock, but evidently an
	interloper and an alien. You must be on your guard against her; you
	must shun her example; if necessary, avoid her company, exclude her from
	your sports, and shut her out from your converse. Teachers, you must
	watch her: keep your eyes on her movements, weigh well her words,
	scrutinise her actions, punish her body to save her soul: if, indeed,
	such salvation be possible, for (my tongue falters while I tell it) this
	girl, this child, the native of a Christian land, worse than many a
	little heathen who says its prayers to Brahma and kneels before
	Juggernaut---this girl is---a liar!}

Now came a pause of ten minutes, during which I, by this time in perfect
possession of my wits, observed all the female Brocklehursts produce
their pocket-handkerchiefs and apply them to their optics, while the
elderly lady swayed herself to and fro, and the two younger ones
whispered, \enquote{How shocking!} \Mr{} Brocklehurst resumed.

\enquote{This I learned from her benefactress; from the pious and
	charitable lady who adopted her in her orphan state, reared her as her
	own daughter, and whose kindness, whose generosity the unhappy girl
	repaid by an ingratitude so bad, so dreadful, that at last her excellent
	patroness was obliged to separate her from her own young ones, fearful
	lest her vicious example should contaminate their purity: she has sent
	her here to be healed, even as the Jews of old sent their diseased to
	the troubled pool of Bethesda; and, teachers, superintendent, I beg of
	you not to allow the waters to stagnate round her.}

With this sublime conclusion, \Mr{} Brocklehurst adjusted the top button
of his surtout, muttered something to his family, who rose, bowed to
Miss Temple, and then all the great people sailed in state from the
room. Turning at the door, my judge said---

\enquote{Let her stand half-an-hour longer on that stool, and let no one
	speak to her during the remainder of the day.}

There was I, then, mounted aloft; I, who had said I could not bear the
shame of standing on my natural feet in the middle of the room, was now
exposed to general view on a pedestal of infamy. What my sensations
were no language can describe; but just as they all rose, stifling my
breath and constricting my throat, a girl came up and passed me: in
passing, she lifted her eyes. What a strange light inspired them! What
an extraordinary sensation that ray sent through me! How the new
feeling bore me up! It was as if a martyr, a hero, had passed a slave
or victim, and imparted strength in the transit. I mastered the rising
hysteria, lifted up my head, and took a firm stand on the stool. Helen
Burns asked some slight question about her work of Miss Smith, was
chidden for the triviality of the inquiry, returned to her place, and
smiled at me as she again went by. What a smile! I remember it now,
and I know that it was the effluence of fine intellect, of true courage;
it lit up her marked lineaments, her thin face, her sunken grey eye,
like a reflection from the aspect of an angel. Yet at that moment Helen
Burns wore on her arm \enquote{the untidy badge;} scarcely an hour ago I
had heard her condemned by Miss Scatcherd to a dinner of bread and water
on the morrow because she had blotted an exercise in copying it out.
Such is the imperfect nature of man! such spots are there on the disc of
the clearest planet; and eyes like Miss Scatcherd's can only see those
minute defects, and are blind to the full brightness of the orb.

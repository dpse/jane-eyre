\FChapter{Chapter Eight}{8}

\Lettrine{E}{re} \textsc{the half-hour} ended, five o'clock struck; school was dismissed, and
all were gone into the refectory to tea. I now ventured to descend: it
was deep dusk; I retired into a corner and sat down on the floor. The
spell by which I had been so far supported began to dissolve; reaction
took place, and soon, so overwhelming was the grief that seized me, I
sank prostrate with my face to the ground. Now I wept: Helen Burns was
not here; nothing sustained me; left to myself I abandoned myself, and
my tears watered the boards. I had meant to be so good, and to do so
much at Lowood: to make so many friends, to earn respect and win
affection. Already I had made visible progress: that very morning I had
reached the head of my class; Miss Miller had praised me warmly; Miss
Temple had smiled approbation; she had promised to teach me drawing, and
to let me learn French, if I continued to make similar improvement two
months longer: and then I was well received by my fellow-pupils; treated
as an equal by those of my own age, and not molested by any; now, here I
lay again crushed and trodden on; and could I ever rise more?

\enquote{Never,} I thought; and ardently I wished to die. While sobbing
out this wish in broken accents, some one approached: I started
up---again Helen Burns was near me; the fading fires just showed her
coming up the long, vacant room; she brought my coffee and bread.

\enquote{Come, eat something,} she said; but I put both away from me,
feeling as if a drop or a crumb would have choked me in my present
condition. Helen regarded me, probably with surprise: I could not now
abate my agitation, though I tried hard; I continued to weep aloud. She
sat down on the ground near me, embraced her knees with her arms, and
rested her head upon them; in that attitude she remained silent as an
Indian. I was the first who spoke---

\enquote{Helen, why do you stay with a girl whom everybody believes to
	be a liar?}

\enquote{Everybody, Jane? Why, there are only eighty people who have
	heard you called so, and the world contains hundreds of millions.}

\enquote{But what have I to do with millions? The eighty, I know,
	despise me.}

\enquote{Jane, you are mistaken: probably not one in the school either
	despises or dislikes you: many, I am sure, pity you much.}

\enquote{How can they pity me after what \Mr{} Brocklehurst has said?}

\enquote{\Mr{} Brocklehurst is not a god: nor is he even a great and
	admired man: he is little liked here; he never took steps to make
	himself liked. Had he treated you as an especial favourite, you would
	have found enemies, declared or covert, all around you; as it is, the
	greater number would offer you sympathy if they dared. Teachers and
	pupils may look coldly on you for a day or two, but friendly feelings
	are concealed in their hearts; and if you persevere in doing well, these
	feelings will ere long appear so much the more evidently for their
	temporary suppression. Besides, Jane}---she paused.

\enquote{Well, Helen?} said I, putting my hand into hers: she chafed my
fingers gently to warm them, and went on---

\enquote{If all the world hated you, and believed you wicked, while your
	own conscience approved you, and absolved you from guilt, you would not
	be without friends.}

\enquote{No; I know I should think well of myself; but that is not
	enough: if others don't love me I would rather die than live---I cannot
	bear to be solitary and hated, Helen. Look here; to gain some real
	affection from you, or Miss Temple, or any other whom I truly love, I
	would willingly submit to have the bone of my arm broken, or to let a
	bull toss me, or to stand behind a kicking horse, and let it dash its
	hoof at my chest---}

\enquote{Hush, Jane! you think too much of the love of human beings; you
	are too impulsive, too vehement; the sovereign hand that created your
	frame, and put life into it, has provided you with other resources than
	your feeble self, or than creatures feeble as you. Besides this earth,
	and besides the race of men, there is an invisible world and a kingdom
	of spirits: that world is round us, for it is everywhere; and those
	spirits watch us, for they are commissioned to guard us; and if we were
	dying in pain and shame, if scorn smote us on all sides, and hatred
	crushed us, angels see our tortures, recognise our innocence (if
	innocent we be: as I know you are of this charge which \Mr{} Brocklehurst
	has weakly and pompously repeated at second-hand from \Mrs{} Reed; for I
	read a sincere nature in your ardent eyes and on your clear front), and
	God waits only the separation of spirit from flesh to crown us with a
	full reward. Why, then, should we ever sink overwhelmed with distress,
	when life is so soon over, and death is so certain an entrance to
	happiness---to glory?}

I was silent; Helen had calmed me; but in the tranquillity she imparted
there was an alloy of inexpressible sadness. I felt the impression of
woe as she spoke, but I could not tell whence it came; and when, having
done speaking, she breathed a little fast and coughed a short cough, I
momentarily forgot my own sorrows to yield to a vague concern for her.

Resting my head on Helen's shoulder, I put my arms round her waist; she
drew me to her, and we reposed in silence. We had not sat long thus,
when another person came in. Some heavy clouds, swept from the sky by a
rising wind, had left the moon bare; and her light, streaming in through
a window near, shone full both on us and on the approaching figure,
which we at once recognised as Miss Temple.

\enquote{I came on purpose to find you, Jane Eyre,} said she; \enquote{I
	want you in my room; and as Helen Burns is with you, she may come too.}

We went; following the superintendent's guidance, we had to thread some
intricate passages, and mount a staircase before we reached her
apartment; it contained a good fire, and looked cheerful. Miss Temple
told Helen Burns to be seated in a low arm-chair on one side of the
hearth, and herself taking another, she called me to her side.

\enquote{Is it all over?} she asked, looking down at my face.
\enquote{Have you cried your grief away?}

\enquote{I am afraid I never shall do that.}

\enquote{Why?}

\enquote{Because I have been wrongly accused; and you, ma'am, and
	everybody else, will now think me wicked.}

\enquote{We shall think you what you prove yourself to be, my child.
	Continue to act as a good girl, and you will satisfy us.}

\enquote{Shall I, Miss Temple?}

\enquote{You will,} said she, passing her arm round me. \enquote{And
	now tell me who is the lady whom \Mr{} Brocklehurst called your
	benefactress?}

\enquote{\Mrs{} Reed, my uncle's wife. My uncle is dead, and he left me
	to her care.}

\enquote{Did she not, then, adopt you of her own accord?}

\enquote{No, ma'am; she was sorry to have to do it: but my uncle, as I
	have often heard the servants say, got her to promise before he died
	that she would always keep me.}

\enquote{Well now, Jane, you know, or at least I will tell you, that
	when a criminal is accused, he is always allowed to speak in his own
	defence. You have been charged with falsehood; defend yourself to me as
	well as you can. Say whatever your memory suggests is true; but add
	nothing and exaggerate nothing.}

I resolved, in the depth of my heart, that I would be most
moderate---most correct; and, having reflected a few minutes in order to
arrange coherently what I had to say, I told her all the story of my sad
childhood. Exhausted by emotion, my language was more subdued than it
generally was when it developed that sad theme; and mindful of Helen's
warnings against the indulgence of resentment, I infused into the
narrative far less of gall and wormwood than ordinary. Thus restrained
and simplified, it sounded more credible: I felt as I went on that Miss
Temple fully believed me.

In the course of the tale I had mentioned \Mr{} Lloyd as having come to
see me after the fit: for I never forgot the, to me, frightful episode
of the red-room: in detailing which, my excitement was sure, in some
degree, to break bounds; for nothing could soften in my recollection the
spasm of agony which clutched my heart when \Mrs{} Reed spurned my wild
supplication for pardon, and locked me a second time in the dark and
haunted chamber.

I had finished: Miss Temple regarded me a few minutes in silence; she
then said---

\enquote{I know something of \Mr{} Lloyd; I shall write to him; if his
	reply agrees with your statement, you shall be publicly cleared from
	every imputation; to me, Jane, you are clear now.}

She kissed me, and still keeping me at her side (where I was well
contented to stand, for I derived a child's pleasure from the
contemplation of her face, her dress, her one or two ornaments, her
white forehead, her clustered and shining curls, and beaming dark eyes),
she proceeded to address Helen Burns.

\enquote{How are you to-night, Helen? Have you coughed much to-day?}

\enquote{Not quite so much, I think, ma'am.}

\enquote{And the pain in your chest?}

\enquote{It is a little better.}

Miss Temple got up, took her hand and examined her pulse; then she
returned to her own seat: as she resumed it, I heard her sigh low. She
was pensive a few minutes, then rousing herself, she said cheerfully---

\enquote{But you two are my visitors to-night; I must treat you as
	such.} She rang her bell.

\enquote{Barbara,} she said to the servant who answered it, \enquote{I
	have not yet had tea; bring the tray and place cups for these two young
	ladies.}

And a tray was soon brought. How pretty, to my eyes, did the china cups
and bright teapot look, placed on the little round table near the fire!
How fragrant was the steam of the beverage, and the scent of the toast!
of which, however, I, to my dismay (for I was beginning to be hungry)
discerned only a very small portion: Miss Temple discerned it too.

\enquote{Barbara,} said she, \enquote{can you not bring a little more
	bread and butter? There is not enough for three.}

Barbara went out: she returned soon---

\enquote{Madam, \Mrs{} Harden says she has sent up the usual quantity.}

\Mrs{} Harden, be it observed, was the housekeeper: a woman after \Mr{}
Brocklehurst's own heart, made up of equal parts of whalebone and iron.

\enquote{Oh, very well!} returned Miss Temple; \enquote{we must make it
	do, Barbara, I suppose.} And as the girl withdrew she added, smiling,
\enquote{Fortunately, I have it in my power to supply deficiencies for
	this once.}

Having invited Helen and me to approach the table, and placed before
each of us a cup of tea with one delicious but thin morsel of toast, she
got up, unlocked a drawer, and taking from it a parcel wrapped in paper,
disclosed presently to our eyes a good-sized seed-cake.

\enquote{I meant to give each of you some of this to take with you,}
said she, \enquote{but as there is so little toast, you must have it
	now,} and she proceeded to cut slices with a generous hand.

We feasted that evening as on nectar and ambrosia; and not the least
delight of the entertainment was the smile of gratification with which
our hostess regarded us, as we satisfied our famished appetites on the
delicate fare she liberally supplied.

Tea over and the tray removed, she again summoned us to the fire; we sat
one on each side of her, and now a conversation followed between her and
Helen, which it was indeed a privilege to be admitted to hear.

Miss Temple had always something of serenity in her air, of state in her
mien, of refined propriety in her language, which precluded deviation
into the ardent, the excited, the eager: something which chastened the
pleasure of those who looked on her and listened to her, by a
controlling sense of awe; and such was my feeling now: but as to Helen
Burns, I was struck with wonder.

The refreshing meal, the brilliant fire, the presence and kindness of
her beloved instructress, or, perhaps, more than all these, something in
her own unique mind, had roused her powers within her. They woke, they
kindled: first, they glowed in the bright tint of her cheek, which till
this hour I had never seen but pale and bloodless; then they shone in
the liquid lustre of her eyes, which had suddenly acquired a beauty more
singular than that of Miss Temple's---a beauty neither of fine colour
nor long eyelash, nor pencilled brow, but of meaning, of movement, of
radiance. Then her soul sat on her lips, and language flowed, from what
source I cannot tell. Has a girl of fourteen a heart large enough,
vigorous enough, to hold the swelling spring of pure, full, fervid
eloquence? Such was the characteristic of Helen's discourse on that, to
me, memorable evening; her spirit seemed hastening to live within a very
brief span as much as many live during a protracted existence.

They conversed of things I had never heard of; of nations and times
past; of countries far away; of secrets of nature discovered or guessed
at: they spoke of books: how many they had read! What stores of
knowledge they possessed! Then they seemed so familiar with French
names and French authors: but my amazement reached its climax when Miss
Temple asked Helen if she sometimes snatched a moment to recall the
Latin her father had taught her, and taking a book from a shelf, bade
her read and construe a page of Virgil; and Helen obeyed, my organ of
veneration expanding at every sounding line. She had scarcely finished
ere the bell announced bedtime! no delay could be admitted; Miss Temple
embraced us both, saying, as she drew us to her heart---

\enquote{God bless you, my children!}

Helen she held a little longer than me: she let her go more reluctantly;
it was Helen her eye followed to the door; it was for her she a second
time breathed a sad sigh; for her she wiped a tear from her cheek.

On reaching the bedroom, we heard the voice of Miss Scatcherd: she was
examining drawers; she had just pulled out Helen Burns's, and when we
entered Helen was greeted with a sharp reprimand, and told that
to-morrow she should have half-a-dozen of untidily folded articles
pinned to her shoulder.

\enquote{My things were indeed in shameful disorder,} murmured Helen to
me, in a low voice: \enquote{I intended to have arranged them, but I
	forgot.}

Next morning, Miss Scatcherd wrote in conspicuous characters on a piece
of pasteboard the word \enquote{Slattern,} and bound it like a
phylactery round Helen's large, mild, intelligent, and benign-looking
forehead. She wore it till evening, patient, unresentful, regarding it
as a deserved punishment. The moment Miss Scatcherd withdrew after
afternoon school, I ran to Helen, tore it off, and thrust it into the
fire: the fury of which she was incapable had been burning in my soul
all day, and tears, hot and large, had continually been scalding my
cheek; for the spectacle of her sad resignation gave me an intolerable
pain at the heart.

About a week subsequently to the incidents above narrated, Miss Temple,
who had written to \Mr{} Lloyd, received his answer: it appeared that what
he said went to corroborate my account. Miss Temple, having assembled
the whole school, announced that inquiry had been made into the charges
alleged against Jane Eyre, and that she was most happy to be able to
pronounce her completely cleared from every imputation. The teachers
then shook hands with me and kissed me, and a murmur of pleasure ran
through the ranks of my companions.

Thus relieved of a grievous load, I from that hour set to work afresh,
resolved to pioneer my way through every difficulty: I toiled hard, and
my success was proportionate to my efforts; my memory, not naturally
tenacious, improved with practice; exercise sharpened my wits; in a few
weeks I was promoted to a higher class; in less than two months I was
allowed to commence French and drawing. I learned the first two tenses
of the verb \emph{Etre}, and sketched my first cottage (whose walls,
by-the-bye, outrivalled in slope those of the leaning tower of Pisa), on
the same day. That night, on going to bed, I forgot to prepare in
imagination the Barmecide supper of hot roast potatoes, or white bread
and new milk, with which I was wont to amuse my inward cravings: I
feasted instead on the spectacle of ideal drawings, which I saw in the
dark; all the work of my own hands: freely pencilled houses and trees,
picturesque rocks and ruins, Cuyp-like groups of cattle, sweet paintings
of butterflies hovering over unblown roses, of birds picking at ripe
cherries, of wren's nests enclosing pearl-like eggs, wreathed about with
young ivy sprays. I examined, too, in thought, the possibility of my
ever being able to translate currently a certain little French story
which Madame Pierrot had that day shown me; nor was that problem solved
to my satisfaction ere I fell sweetly asleep.

Well has Solomon said---\enquote{Better is a dinner of herbs where love
	is, than a stalled ox and hatred therewith.}

I would not now have exchanged Lowood with all its privations for
Gateshead and its daily luxuries.

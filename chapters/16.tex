\FChapter{Chapter Sixteen}{16}

\Lettrine{I}{} \textsc{both wished} and feared to see \Mr{} Rochester on the day which followed
this sleepless night: I wanted to hear his voice again, yet feared to
meet his eye.  During the early part of the morning, I momentarily
expected his coming; he was not in the frequent habit of entering the
schoolroom, but he did step in for a few minutes sometimes, and I had
the impression that he was sure to visit it that day.

But the morning passed just as usual: nothing happened to interrupt the
quiet course of Adèle's studies; only soon after breakfast, I heard some
bustle in the neighbourhood of \Mr{} Rochester's chamber, \Mrs{} Fairfax's
voice, and Leah's, and the cook's---that is, John's wife---and even
John's own gruff tones.  There were exclamations of \enquote{What a
	mercy master was not burnt in his bed!}  \enquote{It is always dangerous
	to keep a candle lit at night.}  \enquote{How providential that he had
	presence of mind to think of the water-jug!}  \enquote{I wonder he waked
	nobody!}  \enquote{It is to be hoped he will not take cold with sleeping
	on the library sofa,} \etc.

To much confabulation succeeded a sound of scrubbing and setting to
rights; and when I passed the room, in going downstairs to dinner, I saw
through the open door that all was again restored to complete order;
only the bed was stripped of its hangings.  Leah stood up in the
window-seat, rubbing the panes of glass dimmed with smoke.  I was about
to address her, for I wished to know what account had been given of the
affair: but, on advancing, I saw a second person in the chamber---a
woman sitting on a chair by the bedside, and sewing rings to new
curtains.  That woman was no other than Grace Poole.

There she sat, staid and taciturn-looking, as usual, in her brown stuff
gown, her check apron, white handkerchief, and cap.  She was intent on
her work, in which her whole thoughts seemed absorbed: on her hard
forehead, and in her commonplace features, was nothing either of the
paleness or desperation one would have expected to see marking the
countenance of a woman who had attempted murder, and whose intended
victim had followed her last night to her lair, and (as I believed),
charged her with the crime she wished to perpetrate.  I was
amazed---confounded.  She looked up, while I still gazed at her: no
start, no increase or failure of colour betrayed emotion, consciousness
of guilt, or fear of detection.  She said \enquote{Good morning, Miss,}
in her usual phlegmatic and brief manner; and taking up another ring and
more tape, went on with her sewing.

\enquote{I will put her to some test,} thought I: \enquote{such absolute
	impenetrability is past comprehension.}

\enquote{Good morning, Grace,} I said.  \enquote{Has anything happened
	here?  I thought I heard the servants all talking together a while ago.}

\enquote{Only master had been reading in his bed last night; he fell
	asleep with his candle lit, and the curtains got on fire; but,
	fortunately, he awoke before the bed-clothes or the wood-work caught,
	and contrived to quench the flames with the water in the ewer.}

\enquote{A strange affair!} I said, in a low voice: then, looking at her
fixedly---\enquote{Did \Mr{} Rochester wake nobody?  Did no one hear him
	move?}

She again raised her eyes to me, and this time there was something of
consciousness in their expression.  She seemed to examine me warily;
then she answered---

\enquote{The servants sleep so far off, you know, Miss, they would not
	be likely to hear.  \Mrs{} Fairfax's room and yours are the nearest to
	master's; but \Mrs{} Fairfax said she heard nothing: when people get
	elderly, they often sleep heavy.}  She paused, and then added, with a
sort of assumed indifference, but still in a marked and significant
tone---\enquote{But you are young, Miss; and I should say a light
	sleeper: perhaps you may have heard a noise?}

\enquote{I did,} said I, dropping my voice, so that Leah, who was still
polishing the panes, could not hear me, \enquote{and at first I thought
	it was Pilot: but Pilot cannot laugh; and I am certain I heard a laugh,
	and a strange one.}

She took a new needleful of thread, waxed it carefully, threaded her
needle with a steady hand, and then observed, with perfect composure---

\enquote{It is hardly likely master would laugh, I should think, Miss,
	when he was in such danger: You must have been dreaming.}

\enquote{I was not dreaming,} I said, with some warmth, for her brazen
coolness provoked me.  Again she looked at me; and with the same
scrutinising and conscious eye.

\enquote{Have you told master that you heard a laugh?} she inquired.

\enquote{I have not had the opportunity of speaking to him this
	morning.}

\enquote{You did not think of opening your door and looking out into the
	gallery?} she further asked.

She appeared to be cross-questioning me, attempting to draw from me
information unawares.  The idea struck me that if she discovered I knew
or suspected her guilt, she would be playing of some of her malignant
pranks on me; I thought it advisable to be on my guard.

\enquote{On the contrary,} said I, \enquote{I bolted my door.}

\enquote{Then you are not in the habit of bolting your door every night
	before you get into bed?}

\enquote{Fiend! she wants to know my habits, that she may lay her plans
	accordingly!}  Indignation again prevailed over prudence: I replied
sharply, \enquote{Hitherto I have often omitted to fasten the bolt: I
	did not think it necessary.  I was not aware any danger or annoyance was
	to be dreaded at Thornfield Hall: but in future} (and I laid marked
stress on the words) \enquote{I shall take good care to make all secure
	before I venture to lie down.}

\enquote{It will be wise so to do,} was her answer: \enquote{this
	neighbourhood is as quiet as any I know, and I never heard of the hall
	being attempted by robbers since it was a house; though there are
	hundreds of pounds' worth of plate in the plate-closet, as is well
	known.  And you see, for such a large house, there are very few
	servants, because master has never lived here much; and when he does
	come, being a bachelor, he needs little waiting on: but I always think
	it best to err on the safe side; a door is soon fastened, and it is as
	well to have a drawn bolt between one and any mischief that may be
	about.  A deal of people, Miss, are for trusting all to Providence; but
	I say Providence will not dispense with the means, though He often
	blesses them when they are used discreetly.}  And here she closed her
harangue: a long one for her, and uttered with the demureness of a
Quakeress.

I still stood absolutely dumfoundered at what appeared to me her
miraculous self-possession and most inscrutable hypocrisy, when the cook
entered.

\enquote{\Mrs{} Poole,} said she, addressing Grace, \enquote{the servants'
	dinner will soon be ready: will you come down?}

\enquote{No; just put my pint of porter and bit of pudding on a tray,
	and I'll carry it upstairs.}

\enquote{You'll have some meat?}

\enquote{Just a morsel, and a taste of cheese, that's all.}

\enquote{And the sago?}

\enquote{Never mind it at present: I shall be coming down before
	teatime: I'll make it myself.}

The cook here turned to me, saying that \Mrs{} Fairfax was waiting for me:
so I departed.

I hardly heard \Mrs{} Fairfax's account of the curtain conflagration
during dinner, so much was I occupied in puzzling my brains over the
enigmatical character of Grace Poole, and still more in pondering the
problem of her position at Thornfield and questioning why she had not
been given into custody that morning, or, at the very least, dismissed
from her master's service.  He had almost as much as declared his
conviction of her criminality last night: what mysterious cause withheld
him from accusing her?  Why had he enjoined me, too, to secrecy?  It was
strange: a bold, vindictive, and haughty gentleman seemed somehow in the
power of one of the meanest of his dependants; so much in her power,
that even when she lifted her hand against his life, he dared not openly
charge her with the attempt, much less punish her for it.

Had Grace been young and handsome, I should have been tempted to think
that tenderer feelings than prudence or fear influenced \Mr{} Rochester in
her behalf; but, hard-favoured and matronly as she was, the idea could
not be admitted.  \enquote{Yet,} I reflected, \enquote{she has been
	young once; her youth would be contemporary with her master's: \Mrs{}
	Fairfax told me once, she had lived here many years.  I don't think she
	can ever have been pretty; but, for aught I know, she may possess
	originality and strength of character to compensate for the want of
	personal advantages.  \Mr{} Rochester is an amateur of the decided and
	eccentric: Grace is eccentric at least.  What if a former caprice (a
	freak very possible to a nature so sudden and headstrong as his) has
	delivered him into her power, and she now exercises over his actions a
	secret influence, the result of his own indiscretion, which he cannot
	shake off, and dare not disregard?}  But, having reached this point of
conjecture, \Mrs{} Poole's square, flat figure, and uncomely, dry, even
coarse face, recurred so distinctly to my mind's eye, that I thought,
\enquote{No; impossible! my supposition cannot be correct.  Yet,}
suggested the secret voice which talks to us in our own hearts,
\enquote{you are not beautiful either, and perhaps \Mr{} Rochester
	approves you: at any rate, you have often felt as if he did; and last
	night---remember his words; remember his look; remember his voice!}

I well remembered all; language, glance, and tone seemed at the moment
vividly renewed.  I was now in the schoolroom; Adèle was drawing; I bent
over her and directed her pencil.  She looked up with a sort of start.

\foreignquote{french}{Qu' avez-vous, mademoiselle?} said she.  \foreignquote{french}{Vos doigts
	tremblent comme la feuille, et vos joues sont rouges: mais, rouges comme
	des cerises!}\footnote{\enquote{What's wrong, miss?} \textelp{} \enquote{Your
		fingers are trembling like leaves, and your cheeks are red: they're red as cherries!}}

\enquote{I am hot, Adèle, with stooping!}  She went on sketching; I went
on thinking.

I hastened to drive from my mind the hateful notion I had been
conceiving respecting Grace Poole; it disgusted me.  I compared myself
with her, and found we were different.  Bessie Leaven had said I was
quite a lady; and she spoke truth---I was a lady.  And now I looked much
better than I did when Bessie saw me; I had more colour and more flesh,
more life, more vivacity, because I had brighter hopes and keener
enjoyments.

\enquote{Evening approaches,} said I, as I looked towards the window.
\enquote{I have never heard \Mr{} Rochester's voice or step in the house
	to-day; but surely I shall see him before night: I feared the meeting in
	the morning; now I desire it, because expectation has been so long
	baffled that it is grown impatient.}

When dusk actually closed, and when Adèle left me to go and play in the
nursery with Sophie, I did most keenly desire it.  I listened for the
bell to ring below; I listened for Leah coming up with a message; I
fancied sometimes I heard \Mr{} Rochester's own tread, and I turned to the
door, expecting it to open and admit him.  The door remained shut;
darkness only came in through the window.  Still it was not late; he
often sent for me at seven and eight o'clock, and it was yet but six.
Surely I should not be wholly disappointed to-night, when I had so many
things to say to him!  I wanted again to introduce the subject of Grace
Poole, and to hear what he would answer; I wanted to ask him plainly if
he really believed it was she who had made last night's hideous attempt;
and if so, why he kept her wickedness a secret.  It little mattered
whether my curiosity irritated him; I knew the pleasure of vexing and
soothing him by turns; it was one I chiefly delighted in, and a sure
instinct always prevented me from going too far; beyond the verge of
provocation I never ventured; on the extreme brink I liked well to try
my skill.  Retaining every minute form of respect, every propriety of my
station, I could still meet him in argument without fear or uneasy
restraint; this suited both him and me.

A tread creaked on the stairs at last.  Leah made her appearance; but it
was only to intimate that tea was ready in \Mrs{} Fairfax's room.  Thither
I repaired, glad at least to go downstairs; for that brought me, I
imagined, nearer to \Mr{} Rochester's presence.

\enquote{You must want your tea,} said the good lady, as I joined her;
\enquote{you ate so little at dinner.  I am afraid,} she continued,
\enquote{you are not well to-day: you look flushed and feverish.}

\enquote{Oh, quite well!  I never felt better.}

\enquote{Then you must prove it by evincing a good appetite; will you
	fill the teapot while I knit off this needle?}  Having completed her
task, she rose to draw down the blind, which she had hitherto kept up,
by way, I suppose, of making the most of daylight, though dusk was now
fast deepening into total obscurity.

\enquote{It is fair to-night,} said she, as she looked through the
panes, \enquote{though not starlight; \Mr{} Rochester has, on the whole,
	had a favourable day for his journey.}

\enquote{Journey!---Is \Mr{} Rochester gone anywhere?  I did not know he
	was out.}

\enquote{Oh, he set off the moment he had breakfasted!  He is gone to
	the Leas, \Mr{} Eshton's place, ten miles on the other side Millcote.  I
	believe there is quite a party assembled there; Lord Ingram, Sir George
	Lynn, Colonel Dent, and others.}

\enquote{Do you expect him back to-night?}

\enquote{No---nor to-morrow either; I should think he is very likely to
	stay a week or more: when these fine, fashionable people get together,
	they are so surrounded by elegance and gaiety, so well provided with all
	that can please and entertain, they are in no hurry to separate.
	Gentlemen especially are often in request on such occasions; and \Mr{}
	Rochester is so talented and so lively in society, that I believe he is
	a general favourite: the ladies are very fond of him; though you would
	not think his appearance calculated to recommend him particularly in
	their eyes: but I suppose his acquirements and abilities, perhaps his
	wealth and good blood, make amends for any little fault of look.}

\enquote{Are there ladies at the Leas?}

\enquote{There are \Mrs{} Eshton and her three daughters---very elegant
	young ladies indeed; and there are the Honourable Blanche and Mary
	Ingram, most beautiful women, I suppose: indeed I have seen Blanche, six
	or seven years since, when she was a girl of eighteen.  She came here to
	a Christmas ball and party \Mr{} Rochester gave.  You should have seen the
	dining-room that day---how richly it was decorated, how brilliantly lit
	up!  I should think there were fifty ladies and gentlemen present---all
	of the first county families; and Miss Ingram was considered the belle
	of the evening.}

\enquote{You saw her, you say, \Mrs{} Fairfax: what was she like?}

\enquote{Yes, I saw her.  The dining-room doors were thrown open; and,
	as it was Christmas-time, the servants were allowed to assemble in the
	hall, to hear some of the ladies sing and play.  \Mr{} Rochester would
	have me to come in, and I sat down in a quiet corner and watched them.
	I never saw a more splendid scene: the ladies were magnificently
	dressed; most of them---at least most of the younger ones---looked
	handsome; but Miss Ingram was certainly the queen.}

\enquote{And what was she like?}

\enquote{Tall, fine bust, sloping shoulders; long, graceful neck: olive
	complexion, dark and clear; noble features; eyes rather like \Mr{}
	Rochester's: large and black, and as brilliant as her jewels.  And then
	she had such a fine head of hair; raven-black and so becomingly
	arranged: a crown of thick plaits behind, and in front the longest, the
	glossiest curls I ever saw.  She was dressed in pure white; an
	amber-coloured scarf was passed over her shoulder and across her breast,
	tied at the side, and descending in long, fringed ends below her knee.
	She wore an amber-coloured flower, too, in her hair: it contrasted well
	with the jetty mass of her curls.}

\enquote{She was greatly admired, of course?}

\enquote{Yes, indeed: and not only for her beauty, but for her
	accomplishments.  She was one of the ladies who sang: a gentleman
	accompanied her on the piano.  She and \Mr{} Rochester sang a duet.}

\enquote{\Mr{} Rochester?  I was not aware he could sing.}

\enquote{Oh! he has a fine bass voice, and an excellent taste for
	music.}

\enquote{And Miss Ingram: what sort of a voice had she?}

\enquote{A very rich and powerful one: she sang delightfully; it was a
	treat to listen to her;---and she played afterwards.  I am no judge of
	music, but \Mr{} Rochester is; and I heard him say her execution was
	remarkably good.}

\enquote{And this beautiful and accomplished lady, she is not yet
	married?}

\enquote{It appears not: I fancy neither she nor her sister have very
	large fortunes.  Old Lord Ingram's estates were chiefly entailed, and
	the eldest son came in for everything almost.}

\enquote{But I wonder no wealthy nobleman or gentleman has taken a fancy
	to her: \Mr{} Rochester, for instance.  He is rich, is he not?}

\enquote{Oh! yes.  But you see there is a considerable difference in
	age: \Mr{} Rochester is nearly forty; she is but twenty-five.}

\enquote{What of that?  More unequal matches are made every day.}

\enquote{True: yet I should scarcely fancy \Mr{} Rochester would entertain
	an idea of the sort.  But you eat nothing: you have scarcely tasted
	since you began tea.}

\enquote{No: I am too thirsty to eat.  Will you let me have another
	cup?}

I was about again to revert to the probability of a union between \Mr{}
Rochester and the beautiful Blanche; but Adèle came in, and the
conversation was turned into another channel.

When once more alone, I reviewed the information I had got; looked into
my heart, examined its thoughts and feelings, and endeavoured to bring
back with a strict hand such as had been straying through imagination's
boundless and trackless waste, into the safe fold of common sense.

Arraigned at my own bar, Memory having given her evidence of the hopes,
wishes, sentiments I had been cherishing since last night---of the
general state of mind in which I had indulged for nearly a fortnight
past; Reason having come forward and told, in her own quiet way a plain,
unvarnished tale, showing how I had rejected the real, and rabidly
devoured the ideal;---I pronounced judgment to this effect:---

That a greater fool than Jane Eyre had never breathed the breath of
life; that a more fantastic idiot had never surfeited herself on sweet
lies, and swallowed poison as if it were nectar.

\enquote{\emph{You},} I said, \enquote{a favourite with \Mr{} Rochester?  \emph{You}
	gifted with the power of pleasing him?  \emph{You} of importance to him
	in any way?  Go! your folly sickens me.  And you have derived pleasure
	from occasional tokens of preference---equivocal tokens shown by a
	gentleman of family and a man of the world to a dependent and a novice.
	How dared you?  Poor stupid dupe!---Could not even self-interest make
	you wiser? You repeated to yourself this morning the brief scene of last
	night?---Cover your face and be ashamed!  He said something in praise of
	your eyes, did he?  Blind puppy!  Open their bleared lids and look on
	your own accursed senselessness!  It does good to no woman to be
	flattered by her superior, who cannot possibly intend to marry her; and
	it is madness in all women to let a secret love kindle within them,
	which, if unreturned and unknown, must devour the life that feeds it;
	and, if discovered and responded to, must lead,
	\emph{ignis-fatuus}-like, into miry wilds whence there is no
	extrication.

	%rem enq
	Listen, then, Jane Eyre, to your sentence: to-morrow, place the glass
	before you, and draw in chalk your own picture, faithfully, without
	softening one defect; omit no harsh line, smooth away no displeasing
	irregularity; write under it, \enquote{Portrait of a Governess,
		disconnected, poor, and plain.}

	%rem enq
	Afterwards, take a piece of smooth ivory---you have one prepared in
	your drawing-box: take your palette, mix your freshest, finest, clearest
	tints; choose your most delicate camel-hair pencils; delineate carefully
	the loveliest face you can imagine; paint it in your softest shades and
	sweetest lines, according to the description given by \Mrs{} Fairfax of
	Blanche Ingram; remember the raven ringlets, the oriental eye;---What!
	you revert to \Mr{} Rochester as a model!  Order!  No snivel!---no
	sentiment!---no regret!  I will endure only sense and resolution.
	Recall the august yet harmonious lineaments, the Grecian neck and bust;
	let the round and dazzling arm be visible, and the delicate hand; omit
	neither diamond ring nor gold bracelet; portray faithfully the attire,
	aërial lace and glistening satin, graceful scarf and golden rose; call
	it \enquote{Blanche, an accomplished lady of rank.}

	%rem enq
	Whenever, in future, you should chance to fancy \Mr{} Rochester
	thinks well of you, take out these two pictures and compare them: say,
	\enquote{\Mr{} Rochester might probably win that noble lady's love, if he
		chose to strive for it; is it likely he would waste a serious thought on
		this indigent and insignificant plebeian?}}

\enquote{I'll do it,} I resolved: and having framed this determination,
I grew calm, and fell asleep.

I kept my word.  An hour or two sufficed to sketch my own portrait in
crayons; and in less than a fortnight I had completed an ivory miniature
of an imaginary Blanche Ingram.  It looked a lovely face enough, and
when compared with the real head in chalk, the contrast was as great as
self-control could desire.  I derived benefit from the task: it had kept
my head and hands employed, and had given force and fixedness to the new
impressions I wished to stamp indelibly on my heart.

Ere long, I had reason to congratulate myself on the course of wholesome
discipline to which I had thus forced my feelings to submit.  Thanks to
it, I was able to meet subsequent occurrences with a decent calm, which,
had they found me unprepared, I should probably have been unequal to
maintain, even externally.

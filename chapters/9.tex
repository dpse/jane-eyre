\FChapter{Chapter Nine}{9}

\Lettrine{B}{ut} \textsc{the privations,} or rather the hardships, of Lowood lessened. Spring
drew on: she was indeed already come; the frosts of winter had ceased;
its snows were melted, its cutting winds ameliorated. My wretched feet,
flayed and swollen to lameness by the sharp air of January, began to
heal and subside under the gentler breathings of April; the nights and
mornings no longer by their Canadian temperature froze the very blood in
our veins; we could now endure the play-hour passed in the garden:
sometimes on a sunny day it began even to be pleasant and genial, and a
greenness grew over those brown beds, which, freshening daily, suggested
the thought that Hope traversed them at night, and left each morning
brighter traces of her steps. Flowers peeped out amongst the leaves;
snow-drops, crocuses, purple auriculas, and golden-eyed pansies. On
Thursday afternoons (half-holidays) we now took walks, and found still
sweeter flowers opening by the wayside, under the hedges.

I discovered, too, that a great pleasure, an enjoyment which the horizon
only bounded, lay all outside the high and spike-guarded walls of our
garden: this pleasure consisted in prospect of noble summits girdling a
great hill-hollow, rich in verdure and shadow; in a bright beck, full of
dark stones and sparkling eddies. How different had this scene looked
when I viewed it laid out beneath the iron sky of winter, stiffened in
frost, shrouded with snow!---when mists as chill as death wandered to
the impulse of east winds along those purple peaks, and rolled down
\enquote{ing} and holm till they blended with the frozen fog of the
beck! That beck itself was then a torrent, turbid and curbless: it tore
asunder the wood, and sent a raving sound through the air, often
thickened with wild rain or whirling sleet; and for the forest on its
banks, \emph{that} showed only ranks of skeletons.

April advanced to May: a bright serene May it was; days of blue sky,
placid sunshine, and soft western or southern gales filled up its
duration. And now vegetation matured with vigour; Lowood shook loose
its tresses; it became all green, all flowery; its great elm, ash, and
oak skeletons were restored to majestic life; woodland plants sprang up
profusely in its recesses; unnumbered varieties of moss filled its
hollows, and it made a strange ground-sunshine out of the wealth of its
wild primrose plants: I have seen their pale gold gleam in overshadowed
spots like scatterings of the sweetest lustre. All this I enjoyed often
and fully, free, unwatched, and almost alone: for this unwonted liberty
and pleasure there was a cause, to which it now becomes my task to
advert.

Have I not described a pleasant site for a dwelling, when I speak of it
as bosomed in hill and wood, and rising from the verge of a stream? 
Assuredly, pleasant enough: but whether healthy or not is another
question.

That forest-dell, where Lowood lay, was the cradle of fog and fog-bred
pestilence; which, quickening with the quickening spring, crept into the
Orphan Asylum, breathed typhus through its crowded schoolroom and
dormitory, and, ere May arrived, transformed the seminary into an
hospital.

Semi-starvation and neglected colds had predisposed most of the pupils
to receive infection: forty-five out of the eighty girls lay ill at one
time. Classes were broken up, rules relaxed. The few who continued
well were allowed almost unlimited license; because the medical
attendant insisted on the necessity of frequent exercise to keep them in
health: and had it been otherwise, no one had leisure to watch or
restrain them. Miss Temple's whole attention was absorbed by the
patients: she lived in the sick-room, never quitting it except to snatch
a few hours' rest at night. The teachers were fully occupied with
packing up and making other necessary preparations for the departure of
those girls who were fortunate enough to have friends and relations able
and willing to remove them from the seat of contagion. Many, already
smitten, went home only to die: some died at the school, and were buried
quietly and quickly, the nature of the malady forbidding delay.

While disease had thus become an inhabitant of Lowood, and death its
frequent visitor; while there was gloom and fear within its walls; while
its rooms and passages steamed with hospital smells, the drug and the
pastille striving vainly to overcome the effluvia of mortality, that
bright May shone unclouded over the bold hills and beautiful woodland
out of doors. Its garden, too, glowed with flowers: hollyhocks had
sprung up tall as trees, lilies had opened, tulips and roses were in
bloom; the borders of the little beds were gay with pink thrift and
crimson double daisies; the sweetbriars gave out, morning and evening,
their scent of spice and apples; and these fragrant treasures were all
useless for most of the inmates of Lowood, except to furnish now and
then a handful of herbs and blossoms to put in a coffin.

But I, and the rest who continued well, enjoyed fully the beauties of
the scene and season; they let us ramble in the wood, like gipsies, from
morning till night; we did what we liked, went where we liked: we lived
better too. \Mr{} Brocklehurst and his family never came near Lowood now:
household matters were not scrutinised into; the cross housekeeper was
gone, driven away by the fear of infection; her successor, who had been
matron at the Lowton Dispensary, unused to the ways of her new abode,
provided with comparative liberality. Besides, there were fewer to
feed; the sick could eat little; our breakfast-basins were better
filled; when there was no time to prepare a regular dinner, which often
happened, she would give us a large piece of cold pie, or a thick slice
of bread and cheese, and this we carried away with us to the wood, where
we each chose the spot we liked best, and dined sumptuously.

My favourite seat was a smooth and broad stone, rising white and dry
from the very middle of the beck, and only to be got at by wading
through the water; a feat I accomplished barefoot. The stone was just
broad enough to accommodate, comfortably, another girl and me, at that
time my chosen comrade---one Mary Ann Wilson; a shrewd, observant
personage, whose society I took pleasure in, partly because she was
witty and original, and partly because she had a manner which set me at
my ease. Some years older than I, she knew more of the world, and could
tell me many things I liked to hear: with her my curiosity found
gratification: to my faults also she gave ample indulgence, never
imposing curb or rein on anything I said. She had a turn for narrative,
I for analysis; she liked to inform, I to question; so we got on
swimmingly together, deriving much entertainment, if not much
improvement, from our mutual intercourse.

And where, meantime, was Helen Burns? Why did I not spend these sweet
days of liberty with her? Had I forgotten her? or was I so worthless as
to have grown tired of her pure society? Surely the Mary Ann Wilson I
have mentioned was inferior to my first acquaintance: she could only
tell me amusing stories, and reciprocate any racy and pungent gossip I
chose to indulge in; while, if I have spoken truth of Helen, she was
qualified to give those who enjoyed the privilege of her converse a
taste of far higher things.

True, reader; and I knew and felt this: and though I am a defective
being, with many faults and few redeeming points, yet I never tired of
Helen Burns; nor ever ceased to cherish for her a sentiment of
attachment, as strong, tender, and respectful as any that ever animated
my heart. How could it be otherwise, when Helen, at all times and under
all circumstances, evinced for me a quiet and faithful friendship, which
ill-humour never soured, nor irritation never troubled? But Helen was
ill at present: for some weeks she had been removed from my sight to I
knew not what room upstairs. She was not, I was told, in the hospital
portion of the house with the fever patients; for her complaint was
consumption, not typhus: and by consumption I, in my ignorance,
understood something mild, which time and care would be sure to
alleviate.

I was confirmed in this idea by the fact of her once or twice coming
downstairs on very warm sunny afternoons, and being taken by Miss Temple
into the garden; but, on these occasions, I was not allowed to go and
speak to her; I only saw her from the schoolroom window, and then not
distinctly; for she was much wrapped up, and sat at a distance under the
verandah.

One evening, in the beginning of June, I had stayed out very late with
Mary Ann in the wood; we had, as usual, separated ourselves from the
others, and had wandered far; so far that we lost our way, and had to
ask it at a lonely cottage, where a man and woman lived, who looked
after a herd of half-wild swine that fed on the mast in the wood. When
we got back, it was after moonrise: a pony, which we knew to be the
surgeon's, was standing at the garden door. Mary Ann remarked that she
supposed some one must be very ill, as \Mr{} Bates had been sent for at
that time of the evening. She went into the house; I stayed behind a
few minutes to plant in my garden a handful of roots I had dug up in the
forest, and which I feared would wither if I left them till the
morning. This done, I lingered yet a little longer: the flowers smelt
so sweet as the dew fell; it was such a pleasant evening, so serene, so
warm; the still glowing west promised so fairly another fine day on the
morrow; the moon rose with such majesty in the grave east. I was noting
these things and enjoying them as a child might, when it entered my mind
as it had never done before:---

\enquote{How sad to be lying now on a sick bed, and to be in danger of
dying! This world is pleasant---it would be dreary to be called from
it, and to have to go who knows where?}

And then my mind made its first earnest effort to comprehend what had
been infused into it concerning heaven and hell; and for the first time
it recoiled, baffled; and for the first time glancing behind, on each
side, and before it, it saw all round an unfathomed gulf: it felt the
one point where it stood---the present; all the rest was formless cloud
and vacant depth; and it shuddered at the thought of tottering, and
plunging amid that chaos. While pondering this new idea, I heard the
front door open; \Mr{} Bates came out, and with him was a nurse. After
she had seen him mount his horse and depart, she was about to close the
door, but I ran up to her.

\enquote{How is Helen Burns?}

\enquote{Very poorly,} was the answer.

\enquote{Is it her \Mr{} Bates has been to see?}

\enquote{Yes.}

\enquote{And what does he say about her?}

\enquote{He says she'll not be here long.}

This phrase, uttered in my hearing yesterday, would have only conveyed
the notion that she was about to be removed to Northumberland, to her
own home. I should not have suspected that it meant she was dying; but
I knew instantly now! It opened clear on my comprehension that Helen
Burns was numbering her last days in this world, and that she was going
to be taken to the region of spirits, if such region there were. I
experienced a shock of horror, then a strong thrill of grief, then a
desire---a necessity to see her; and I asked in what room she lay.

\enquote{She is in Miss Temple's room,} said the nurse.

\enquote{May I go up and speak to her?}

\enquote{Oh no, child! It is not likely; and now it is time for you to
come in; you'll catch the fever if you stop out when the dew is
falling.}

The nurse closed the front door; I went in by the side entrance which
led to the schoolroom: I was just in time; it was nine o'clock, and Miss
Miller was calling the pupils to go to bed.

It might be two hours later, probably near eleven, when I---not having
been able to fall asleep, and deeming, from the perfect silence of the
dormitory, that my companions were all wrapt in profound repose---rose
softly, put on my frock over my night-dress, and, without shoes, crept
from the apartment, and set off in quest of Miss Temple's room. It was
quite at the other end of the house; but I knew my way; and the light of
the unclouded summer moon, entering here and there at passage windows,
enabled me to find it without difficulty. An odour of camphor and burnt
vinegar warned me when I came near the fever room: and I passed its door
quickly, fearful lest the nurse who sat up all night should hear me. I
dreaded being discovered and sent back; for I \emph{must} see Helen,---I
must embrace her before she died,---I must give her one last kiss,
exchange with her one last word.

Having descended a staircase, traversed a portion of the house below,
and succeeded in opening and shutting, without noise, two doors, I
reached another flight of steps; these I mounted, and then just opposite
to me was Miss Temple's room. A light shone through the keyhole and
from under the door; a profound stillness pervaded the vicinity. Coming
near, I found the door slightly ajar; probably to admit some fresh air
into the close abode of sickness. Indisposed to hesitate, and full of
impatient impulses---soul and senses quivering with keen throes---I put
it back and looked in. My eye sought Helen, and feared to find death.

Close by Miss Temple's bed, and half covered with its white curtains,
there stood a little crib. I saw the outline of a form under the
clothes, but the face was hid by the hangings: the nurse I had spoken to
in the garden sat in an easy-chair asleep; an unsnuffed candle burnt
dimly on the table. Miss Temple was not to be seen: I knew afterwards
that she had been called to a delirious patient in the fever-room. I
advanced; then paused by the crib side: my hand was on the curtain, but
I preferred speaking before I withdrew it. I still recoiled at the
dread of seeing a corpse.

\enquote{Helen!} I whispered softly, \enquote{are you awake?}

She stirred herself, put back the curtain, and I saw her face, pale,
wasted, but quite composed: she looked so little changed that my fear
was instantly dissipated.

\enquote{Can it be you, Jane?} she asked, in her own gentle voice.

\enquote{Oh!} I thought, \enquote{she is not going to die; they are
mistaken: she could not speak and look so calmly if she were.}

I got on to her crib and kissed her: her forehead was cold, and her
cheek both cold and thin, and so were her hand and wrist; but she smiled
as of old.

\enquote{Why are you come here, Jane? It is past eleven o'clock: I
heard it strike some minutes since.}

\enquote{I came to see you, Helen: I heard you were very ill, and I
could not sleep till I had spoken to you.}

\enquote{You came to bid me good-bye, then: you are just in time
probably.}

\enquote{Are you going somewhere, Helen? Are you going home?}

\enquote{Yes; to my long home---my last home.}

\enquote{No, no, Helen!} I stopped, distressed. While I tried to
devour my tears, a fit of coughing seized Helen; it did not, however,
wake the nurse; when it was over, she lay some minutes exhausted; then
she whispered---

\enquote{Jane, your little feet are bare; lie down and cover yourself
with my quilt.}

I did so: she put her arm over me, and I nestled close to her. After a
long silence, she resumed, still whispering---

\enquote{I am very happy, Jane; and when you hear that I am dead, you
must be sure and not grieve: there is nothing to grieve about. We all
must die one day, and the illness which is removing me is not painful;
it is gentle and gradual: my mind is at rest. I leave no one to regret
me much: I have only a father; and he is lately married, and will not
miss me. By dying young, I shall escape great sufferings. I had not
qualities or talents to make my way very well in the world: I should
have been continually at fault.}

\enquote{But where are you going to, Helen? Can you see? Do you know?}

\enquote{I believe; I have faith: I am going to God.}

\enquote{Where is God? What is God?}

\enquote{My Maker and yours, who will never destroy what He created. I
rely implicitly on His power, and confide wholly in His goodness: I
count the hours till that eventful one arrives which shall restore me to
Him, reveal Him to me.}

\enquote{You are sure, then, Helen, that there is such a place as
heaven, and that our souls can get to it when we die?}

\enquote{I am sure there is a future state; I believe God is good; I can
resign my immortal part to Him without any misgiving. God is my father;
God is my friend: I love Him; I believe He loves me.}

\enquote{And shall I see you again, Helen, when I die?}

\enquote{You will come to the same region of happiness: be received by
the same mighty, universal Parent, no doubt, dear Jane.}

Again I questioned, but this time only in thought. \enquote{Where is
that region? Does it exist?} And I clasped my arms closer round Helen;
she seemed dearer to me than ever; I felt as if I could not let her go;
I lay with my face hidden on her neck. Presently she said, in the
sweetest tone---

\enquote{How comfortable I am! That last fit of coughing has tired me a
little; I feel as if I could sleep: but don't leave me, Jane; I like to
have you near me.}

\enquote{I'll stay with you, \emph{dear} Helen: no one shall take me away.}

\enquote{Are you warm, darling?}

\enquote{Yes.}

\enquote{Good-night, Jane.}

\enquote{Good-night, Helen.}

She kissed me, and I her, and we both soon slumbered.

When I awoke it was day: an unusual movement roused me; I looked up; I
was in somebody's arms; the nurse held me; she was carrying me through
the passage back to the dormitory. I was not reprimanded for leaving my
bed; people had something else to think about; no explanation was
afforded then to my many questions; but a day or two afterwards I
learned that Miss Temple, on returning to her own room at dawn, had
found me laid in the little crib; my face against Helen Burns's
shoulder, my arms round her neck. I was asleep, and Helen was---dead.

Her grave is in Brocklebridge churchyard: for fifteen years after her
death it was only covered by a grassy mound; but now a grey marble
tablet marks the spot, inscribed with her name, and the word
\enquote{Resurgam.}

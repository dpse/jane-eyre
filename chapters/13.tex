\FChapter{Chapter Thirteen}{13}

\Lettrine{M}{r.\@} \textsc{ Rochester,} it seems, by the surgeon's orders, went to bed early that
night; nor did he rise soon next morning. When he did come down, it was
to attend to business: his agent and some of his tenants were arrived,
and waiting to speak with him.

Adèle and I had now to vacate the library: it would be in daily
requisition as a reception-room for callers. A fire was lit in an
apartment upstairs, and there I carried our books, and arranged it for
the future schoolroom. I discerned in the course of the morning that
Thornfield Hall was a changed place: no longer silent as a church, it
echoed every hour or two to a knock at the door, or a clang of the bell;
steps, too, often traversed the hall, and new voices spoke in different
keys below; a rill from the outer world was flowing through it; it had a
master: for my part, I liked it better.

Adèle was not easy to teach that day; she could not apply: she kept
running to the door and looking over the banisters to see if she could
get a glimpse of \Mr{} Rochester; then she coined pretexts to go
downstairs, in order, as I shrewdly suspected, to visit the library,
where I knew she was not wanted; then, when I got a little angry, and
made her sit still, she continued to talk incessantly of her \foreignquote{french}{ami,
	Monsieur Edouard Fairfax \emph{de} Rochester,} as she dubbed him (I had
not before heard his prenomens), and to conjecture what presents he had
brought her: for it appears he had intimated the night before, that when
his luggage came from Millcote, there would be found amongst it a little
box in whose contents she had an interest.

\foreignquote{french}{Et cela doit signifier,} said she, \foreignquote{french}{qu'il y aura là
	dedans un cadeau pour moi, et peut-être pour vous aussi, mademoiselle.
	Monsieur a parlé de vous: il m'a demandé le nom de ma gouvernante, et si
	elle n'était pas une petite personne, assez mince et un peu pâle. J'ai
	dit qu'oui: car c'est vrai, n'est-ce pas, mademoiselle?}\footnote{%
	\enquote{And that must mean,} \textelp{} \enquote{that there will be a present
		for me in there, and perhaps for you too, miss. \Mr{} has talked about you:
		he asked me the name of my governess, and whether she wasn't a small
		person, quite thin and rather pale. I said yes: for it's true, isn't it, miss?}}

I and my pupil dined as usual in \Mrs{} Fairfax's parlour; the afternoon
was wild and snowy, and we passed it in the schoolroom. At dark I
allowed Adèle to put away books and work, and to run downstairs; for,
from the comparative silence below, and from the cessation of appeals to
the door-bell, I conjectured that \Mr{} Rochester was now at liberty.
Left alone, I walked to the window; but nothing was to be seen thence:
twilight and snowflakes together thickened the air, and hid the very
shrubs on the lawn. I let down the curtain and went back to the
fireside.

In the clear embers I was tracing a view, not unlike a picture I
remembered to have seen of the castle of Heidelberg, on the Rhine, when
\Mrs{} Fairfax came in, breaking up by her entrance the fiery mosaic I had
been piercing together, and scattering too some heavy unwelcome thoughts
that were beginning to throng on my solitude.

\enquote{\Mr{} Rochester would be glad if you and your pupil would take
	tea with him in the drawing-room this evening,} said she: \enquote{he
	has been so much engaged all day that he could not ask to see you
	before.}

\enquote{When is his tea-time?} I inquired.

\enquote{Oh, at six o'clock: he keeps early hours in the country. You
	had better change your frock now; I will go with you and fasten it.
	Here is a candle.}

\enquote{Is it necessary to change my frock?}

\enquote{Yes, you had better: I always dress for the evening when \Mr{}
	Rochester is here.}

This additional ceremony seemed somewhat stately; however, I repaired to
my room, and, with \Mrs{} Fairfax's aid, replaced my black stuff dress by
one of black silk; the best and the only additional one I had, except
one of light grey, which, in my Lowood notions of the toilette, I
thought too fine to be worn, except on first-rate occasions.

\enquote{You want a brooch,} said \Mrs{} Fairfax. I had a single little
pearl ornament which Miss Temple gave me as a parting keepsake: I put it
on, and then we went downstairs. Unused as I was to strangers, it was
rather a trial to appear thus formally summoned in \Mr{} Rochester's
presence. I let \Mrs{} Fairfax precede me into the dining-room, and kept
in her shade as we crossed that apartment; and, passing the arch, whose
curtain was now dropped, entered the elegant recess beyond.

Two wax candles stood lighted on the table, and two on the mantelpiece;
basking in the light and heat of a superb fire, lay Pilot---Adèle knelt
near him. Half reclined on a couch appeared \Mr{} Rochester, his foot
supported by the cushion; he was looking at Adèle and the dog: the fire
shone full on his face. I knew my traveller with his broad and jetty
eyebrows; his square forehead, made squarer by the horizontal sweep of
his black hair. I recognised his decisive nose, more remarkable for
character than beauty; his full nostrils, denoting, I thought, choler;
his grim mouth, chin, and jaw---yes, all three were very grim, and no
mistake. His shape, now divested of cloak, I perceived harmonised in
squareness with his physiognomy: I suppose it was a good figure in the
athletic sense of the term---broad chested and thin flanked, though
neither tall nor graceful.

\Mr{} Rochester must have been aware of the entrance of \Mrs{} Fairfax and
myself; but it appeared he was not in the mood to notice us, for he
never lifted his head as we approached.

\enquote{Here is Miss Eyre, sir,} said \Mrs{} Fairfax, in her quiet way.
He bowed, still not taking his eyes from the group of the dog and child.

\enquote{Let Miss Eyre be seated,} said he: and there was something in
the forced stiff bow, in the impatient yet formal tone, which seemed
further to express, \enquote{What the deuce is it to me whether Miss
	Eyre be there or not? At this moment I am not disposed to accost her.}

I sat down quite disembarrassed. A reception of finished politeness
would probably have confused me: I could not have returned or repaid it
by answering grace and elegance on my part; but harsh caprice laid me
under no obligation; on the contrary, a decent quiescence, under the
freak of manner, gave me the advantage. Besides, the eccentricity of
the proceeding was piquant: I felt interested to see how he would go on.

He went on as a statue would, that is, he neither spoke nor moved. \Mrs{}
Fairfax seemed to think it necessary that some one should be amiable,
and she began to talk. Kindly, as usual---and, as usual, rather
trite---she condoled with him on the pressure of business he had had all
day; on the annoyance it must have been to him with that painful sprain:
then she commended his patience and perseverance in going through with
it.

\enquote{Madam, I should like some tea,} was the sole rejoinder she
got. She hastened to ring the bell; and when the tray came, she
proceeded to arrange the cups, spoons, \etc, with assiduous celerity. I
and Adèle went to the table; but the master did not leave his couch.

\enquote{Will you hand \Mr{} Rochester's cup?} said \Mrs{} Fairfax to me;
\enquote{Adèle might perhaps spill it.}

I did as requested. As he took the cup from my hand, Adèle, thinking
the moment propitious for making a request in my favour, cried out---

\foreignquote{french}{N'est-ce pas, monsieur, qu'il y a un cadeau pour Mademoiselle
	Eyre dans votre petit coffre?}\footnote{\enquote{There's a present for Miss, isn't there, sir, in your little chest?}}

\enquote{Who talks of \foreignlanguage{french}{cadeaux}?} said he gruffly. \enquote{Did you
	expect a present, Miss Eyre? Are you fond of presents?} and he searched
my face with eyes that I saw were dark, irate, and piercing.

\enquote{I hardly know, sir; I have little experience of them: they are
	generally thought pleasant things.}

\enquote{Generally thought? But what do \emph{you} think?}

\enquote{I should be obliged to take time, sir, before I could give you
	an answer worthy of your acceptance: a present has many faces to it, has
	it not? and one should consider all, before pronouncing an opinion as to
	its nature.}

\enquote{Miss Eyre, you are not so unsophisticated as Adèle: she demands
	a \foreignquote{french}{cadeau,} clamorously, the moment she sees me: you beat about
	the bush.}

\enquote{Because I have less confidence in my deserts than Adèle has:
	she can prefer the claim of old acquaintance, and the right too of
	custom; for she says you have always been in the habit of giving her
	playthings; but if I had to make out a case I should be puzzled, since I
	am a stranger, and have done nothing to entitle me to an
	acknowledgment.}

\enquote{Oh, don't fall back on over-modesty! I have examined Adèle,
	and find you have taken great pains with her: she is not bright, she has
	no talents; yet in a short time she has made much improvement.}

\enquote{Sir, you have now given me my \foreignquote{french}{cadeau}; I am obliged
	to you: it is the meed teachers most covet---praise of their pupils'
	progress.}

\enquote{Humph!} said \Mr{} Rochester, and he took his tea in silence.

\enquote{Come to the fire,} said the master, when the tray was taken
away, and \Mrs{} Fairfax had settled into a corner with her knitting;
while Adèle was leading me by the hand round the room, showing me the
beautiful books and ornaments on the consoles and chiffonnières. We
obeyed, as in duty bound; Adèle wanted to take a seat on my knee, but
she was ordered to amuse herself with Pilot.

\enquote{You have been resident in my house three months?}

\enquote{Yes, sir.}

\enquote{And you came from---?}

\enquote{From Lowood school, in ---shire.}

\enquote{Ah! a charitable concern. How long were you there?}

\enquote{Eight years.}

\enquote{Eight years! you must be tenacious of life. I thought half the
	time in such a place would have done up any constitution! No wonder you
	have rather the look of another world. I marvelled where you had got
	that sort of face. When you came on me in Hay Lane last night, I
	thought unaccountably of fairy tales, and had half a mind to demand
	whether you had bewitched my horse: I am not sure yet. Who are your
	parents?}

\enquote{I have none.}

\enquote{Nor ever had, I suppose: do you remember them?}

\enquote{No.}

\enquote{I thought not. And so you were waiting for your people when
	you sat on that stile?}

\enquote{For whom, sir?}

\enquote{For the men in green: it was a proper moonlight evening for
	them. Did I break through one of your rings, that you spread that
	damned ice on the causeway?}

I shook my head. \enquote{The men in green all forsook England a
	hundred years ago,} said I, speaking as seriously as he had done.
\enquote{And not even in Hay Lane, or the fields about it, could you
	find a trace of them. I don't think either summer or harvest, or winter
	moon, will ever shine on their revels more.}

\Mrs{} Fairfax had dropped her knitting, and, with raised eyebrows, seemed
wondering what sort of talk this was.

\enquote{Well,} resumed \Mr{} Rochester, \enquote{if you disown parents,
	you must have some sort of kinsfolk: uncles and aunts?}

\enquote{No; none that I ever saw.}

\enquote{And your home?}

\enquote{I have none.}

\enquote{Where do your brothers and sisters live?}

\enquote{I have no brothers or sisters.}

\enquote{Who recommended you to come here?}

\enquote{I advertised, and \Mrs{} Fairfax answered my advertisement.}

\enquote{Yes,} said the good lady, who now knew what ground we were
upon, \enquote{and I am daily thankful for the choice Providence led me
	to make. Miss Eyre has been an invaluable companion to me, and a kind
	and careful teacher to Adèle.}

\enquote{Don't trouble yourself to give her a character,} returned \Mr{}
Rochester: \enquote{eulogiums will not bias me; I shall judge for
	myself. She began by felling my horse.}

\enquote{Sir?} said \Mrs{} Fairfax.

\enquote{I have to thank her for this sprain.}

The widow looked bewildered.

\enquote{Miss Eyre, have you ever lived in a town?}

\enquote{No, sir.}

\enquote{Have you seen much society?}

\enquote{None but the pupils and teachers of Lowood, and now the inmates
	of Thornfield.}

\enquote{Have you read much?}

\enquote{Only such books as came in my way; and they have not been
	numerous or very learned.}

\enquote{You have lived the life of a nun: no doubt you are well drilled
	in religious forms;---Brocklehurst, who I understand directs Lowood, is
	a parson, is he not?}

\enquote{Yes, sir.}

\enquote{And you girls probably worshipped him, as a convent full of
	religieuses would worship their director.}

\enquote{Oh, no.}

\enquote{You are very cool! No! What! a novice not worship her
	priest! That sounds blasphemous.}

\enquote{I disliked \Mr{} Brocklehurst; and I was not alone in the
	feeling. He is a harsh man; at once pompous and meddling; he cut off
	our hair; and for economy's sake bought us bad needles and thread, with
	which we could hardly sew.}

\enquote{That was very false economy,} remarked \Mrs{} Fairfax, who now
again caught the drift of the dialogue.

\enquote{And was that the head and front of his offending?} demanded \Mr{}
Rochester.

\enquote{He starved us when he had the sole superintendence of the
	provision department, before the committee was appointed; and he bored
	us with long lectures once a week, and with evening readings from books
	of his own inditing, about sudden deaths and judgments, which made us
	afraid to go to bed.}

\enquote{What age were you when you went to Lowood?}

\enquote{About ten.}

\enquote{And you stayed there eight years: you are now, then, eighteen?}

I assented.

\enquote{Arithmetic, you see, is useful; without its aid, I should
	hardly have been able to guess your age. It is a point difficult to fix
	where the features and countenance are so much at variance as in your
	case. And now what did you learn at Lowood? Can you play?}

\enquote{A little.}

\enquote{Of course: that is the established answer. Go into the
	library---I mean, if you please.---(Excuse my tone of command; I am used
	to say, \enquote{Do this,} and it is done: I cannot alter my customary
	habits for one new inmate.)---Go, then, into the library; take a candle
	with you; leave the door open; sit down to the piano, and play a tune.}

I departed, obeying his directions.

\enquote{Enough!} he called out in a few minutes. \enquote{You play \emph{a
		little}, I see; like any other English school-girl; perhaps rather
	better than some, but not well.}

I closed the piano and returned. \Mr{} Rochester
continued---\enquote{Adèle showed me some sketches this morning, which
	she said were yours. I don't know whether they were entirely of your
	doing; probably a master aided you?}

\enquote{No, indeed!} I interjected.

\enquote{Ah! that pricks pride. Well, fetch me your portfolio, if you
	can vouch for its contents being original; but don't pass your word
	unless you are certain: I can recognise patchwork.}

\enquote{Then I will say nothing, and you shall judge for yourself,
	sir.}

I brought the portfolio from the library.

\enquote{Approach the table,} said he; and I wheeled it to his couch.
Adèle and \Mrs{} Fairfax drew near to see the pictures.

\enquote{No crowding,} said \Mr{} Rochester: \enquote{take the drawings
	from my hand as I finish with them; but don't push your faces up to
	mine.}

He deliberately scrutinised each sketch and painting. Three he laid
aside; the others, when he had examined them, he swept from him.

\enquote{Take them off to the other table, \Mrs{} Fairfax,} said he,
\enquote{and look at them with Adèle;---you} (glancing at me)
\enquote{resume your seat, and answer my questions. I perceive those
	pictures were done by one hand: was that hand yours?}

\enquote{Yes.}

\enquote{And when did you find time to do them? They have taken much
	time, and some thought.}

\enquote{I did them in the last two vacations I spent at Lowood, when I
	had no other occupation.}

\enquote{Where did you get your copies?}

\enquote{Out of my head.}

\enquote{That head I see now on your shoulders?}

\enquote{Yes, sir.}

\enquote{Has it other furniture of the same kind within?}

\enquote{I should think it may have: I should hope---better.}

He spread the pictures before him, and again surveyed them alternately.

While he is so occupied, I will tell you, reader, what they are: and
first, I must premise that they are nothing wonderful. The subjects
had, indeed, risen vividly on my mind. As I saw them with the spiritual
eye, before I attempted to embody them, they were striking; but my hand
would not second my fancy, and in each case it had wrought out but a
pale portrait of the thing I had conceived.

These pictures were in water-colours. The first represented clouds low
and livid, rolling over a swollen sea: all the distance was in eclipse;
so, too, was the foreground; or rather, the nearest billows, for there
was no land. One gleam of light lifted into relief a half-submerged
mast, on which sat a cormorant, dark and large, with wings flecked with
foam; its beak held a gold bracelet set with gems, that I had touched
with as brilliant tints as my palette could yield, and as glittering
distinctness as my pencil could impart. Sinking below the bird and
mast, a drowned corpse glanced through the green water; a fair arm was
the only limb clearly visible, whence the bracelet had been washed or
torn.

The second picture contained for foreground only the dim peak of a hill,
with grass and some leaves slanting as if by a breeze. Beyond and above
spread an expanse of sky, dark blue as at twilight: rising into the sky
was a woman's shape to the bust, portrayed in tints as dusk and soft as
I could combine. The dim forehead was crowned with a star; the
lineaments below were seen as through the suffusion of vapour; the eyes
shone dark and wild; the hair streamed shadowy, like a beamless cloud
torn by storm or by electric travail. On the neck lay a pale reflection
like moonlight; the same faint lustre touched the train of thin clouds
from which rose and bowed this vision of the Evening Star.

The third showed the pinnacle of an iceberg piercing a polar winter sky:
a muster of northern lights reared their dim lances, close serried,
along the horizon. Throwing these into distance, rose, in the
foreground, a head,---a colossal head, inclined towards the iceberg, and
resting against it. Two thin hands, joined under the forehead, and
supporting it, drew up before the lower features a sable veil, a brow
quite bloodless, white as bone, and an eye hollow and fixed, blank of
meaning but for the glassiness of despair, alone were visible. Above
the temples, amidst wreathed turban folds of black drapery, vague in its
character and consistency as cloud, gleamed a ring of white flame,
gemmed with sparkles of a more lurid tinge. This pale crescent was
\enquote{the likeness of a kingly crown;} what it diademed was
\enquote{the shape which shape had none.}

\enquote{Were you happy when you painted these pictures?} asked \Mr{}
Rochester presently.

\enquote{I was absorbed, sir: yes, and I was happy. To paint them, in
	short, was to enjoy one of the keenest pleasures I have ever known.}

\enquote{That is not saying much. Your pleasures, by your own account,
	have been few; but I daresay you did exist in a kind of artist's
	dreamland while you blent and arranged these strange tints. Did you sit
	at them long each day?}

\enquote{I had nothing else to do, because it was the vacation, and I
	sat at them from morning till noon, and from noon till night: the length
	of the midsummer days favoured my inclination to apply.}

\enquote{And you felt self-satisfied with the result of your ardent
	labours?}

\enquote{Far from it. I was tormented by the contrast between my idea
	and my handiwork: in each case I had imagined something which I was
	quite powerless to realise.}

\enquote{Not quite: you have secured the shadow of your thought; but no
	more, probably. You had not enough of the artist's skill and science to
	give it full being: yet the drawings are, for a school-girl, peculiar.
	As to the thoughts, they are elfish. These eyes in the Evening Star you
	must have seen in a dream. How could you make them look so clear, and
	yet not at all brilliant? for the planet above quells their rays. And
	what meaning is that in their solemn depth? And who taught you to paint
	wind? There is a high gale in that sky, and on this hill-top. Where
	did you see Latmos? For that is Latmos. There! put the drawings away!}

I had scarce tied the strings of the portfolio, when, looking at his
watch, he said abruptly---

\enquote{It is nine o'clock: what are you about, Miss Eyre, to let Adèle
	sit up so long? Take her to bed.}

Adèle went to kiss him before quitting the room: he endured the caress,
but scarcely seemed to relish it more than Pilot would have done, nor so
much.

\enquote{I wish you all good-night, now,} said he, making a movement of
the hand towards the door, in token that he was tired of our company,
and wished to dismiss us. \Mrs{} Fairfax folded up her knitting: I took
my portfolio: we curtseyed to him, received a frigid bow in return, and
so withdrew.

\enquote{You said \Mr{} Rochester was not strikingly peculiar, \Mrs{}
	Fairfax,} I observed, when I rejoined her in her room, after putting
Adèle to bed.

\enquote{Well, is he?}

\enquote{I think so: he is very changeful and abrupt.}

\enquote{True: no doubt he may appear so to a stranger, but I am so
	accustomed to his manner, I never think of it; and then, if he has
	peculiarities of temper, allowance should be made.}

\enquote{Why?}

\enquote{Partly because it is his nature---and we can none of us help
	our nature; and partly because he has painful thoughts, no doubt, to
	harass him, and make his spirits unequal.}

\enquote{What about?}

\enquote{Family troubles, for one thing.}

\enquote{But he has no family.}

\enquote{Not now, but he has had---or, at least, relatives. He lost his
	elder brother a few years since.}

\enquote{His \emph{elder} brother?}

\enquote{Yes. The present \Mr{} Rochester has not been very long in
	possession of the property; only about nine years.}

\enquote{Nine years is a tolerable time. Was he so very fond of his
	brother as to be still inconsolable for his loss?}

\enquote{Why, no---perhaps not. I believe there were some
	misunderstandings between them. \Mr{} Rowland Rochester was not quite
	just to \Mr{} Edward; and perhaps he prejudiced his father against him.
	The old gentleman was fond of money, and anxious to keep the family
	estate together. He did not like to diminish the property by division,
	and yet he was anxious that \Mr{} Edward should have wealth, too, to keep
	up the consequence of the name; and, soon after he was of age, some
	steps were taken that were not quite fair, and made a great deal of
	mischief. Old \Mr{} Rochester and \Mr{} Rowland combined to bring \Mr{}
	Edward into what he considered a painful position, for the sake of
	making his fortune: what the precise nature of that position was I never
	clearly knew, but his spirit could not brook what he had to suffer in
	it. He is not very forgiving: he broke with his family, and now for
	many years he has led an unsettled kind of life. I don't think he has
	ever been resident at Thornfield for a fortnight together, since the
	death of his brother without a will left him master of the estate; and,
	indeed, no wonder he shuns the old place.}

\enquote{Why should he shun it?}

\enquote{Perhaps he thinks it gloomy.}

The answer was evasive. I should have liked something clearer; but \Mrs{}
Fairfax either could not, or would not, give me more explicit
information of the origin and nature of \Mr{} Rochester's trials. She
averred they were a mystery to herself, and that what she knew was
chiefly from conjecture. It was evident, indeed, that she wished me to
drop the subject, which I did accordingly.

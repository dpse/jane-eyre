\FChapter{Chapter Eleven}{11}

\Lettrine{A}{} \textsc{new chapter} in a novel is something like a new scene in a play; and
when I draw up the curtain this time, reader, you must fancy you see a
room in the George Inn at Millcote, with such large figured papering on
the walls as inn rooms have; such a carpet, such furniture, such
ornaments on the mantelpiece, such prints, including a portrait of
George the Third, and another of the Prince of Wales, and a
representation of the death of Wolfe. All this is visible to you by the
light of an oil lamp hanging from the ceiling, and by that of an
excellent fire, near which I sit in my cloak and bonnet; my muff and
umbrella lie on the table, and I am warming away the numbness and chill
contracted by sixteen hours' exposure to the rawness of an October day:
I left Lowton at four o'clock \AM, and the Millcote town clock is now
just striking eight.

Reader, though I look comfortably accommodated, I am not very tranquil
in my mind. I thought when the coach stopped here there would be some
one to meet me; I looked anxiously round as I descended the wooden steps
the \enquote{boots} placed for my convenience, expecting to hear my name
pronounced, and to see some description of carriage waiting to convey me
to Thornfield. Nothing of the sort was visible; and when I asked a
waiter if any one had been to inquire after a Miss Eyre, I was answered
in the negative: so I had no resource but to request to be shown into a
private room: and here I am waiting, while all sorts of doubts and fears
are troubling my thoughts.

It is a very strange sensation to inexperienced youth to feel itself
quite alone in the world, cut adrift from every connection, uncertain
whether the port to which it is bound can be reached, and prevented by
many impediments from returning to that it has quitted. The charm of
adventure sweetens that sensation, the glow of pride warms it; but then
the throb of fear disturbs it; and fear with me became predominant when
half-an-hour elapsed and still I was alone. I bethought myself to ring
the bell.

\enquote{Is there a place in this neighbourhood called Thornfield?} I
asked of the waiter who answered the summons.

\enquote{Thornfield? I don't know, ma'am; I'll inquire at the bar.} He
vanished, but reappeared instantly---

\enquote{Is your name Eyre, Miss?}

\enquote{Yes.}

\enquote{Person here waiting for you.}

I jumped up, took my muff and umbrella, and hastened into the
inn-passage: a man was standing by the open door, and in the lamp-lit
street I dimly saw a one-horse conveyance.

\enquote{This will be your luggage, I suppose?} said the man rather
abruptly when he saw me, pointing to my trunk in the passage.

\enquote{Yes.} He hoisted it on to the vehicle, which was a sort of
car, and then I got in; before he shut me up, I asked him how far it was
to Thornfield.

\enquote{A matter of six miles.}

\enquote{How long shall we be before we get there?}

\enquote{Happen an hour and a half.}

He fastened the car door, climbed to his own seat outside, and we set
off. Our progress was leisurely, and gave me ample time to reflect; I
was content to be at length so near the end of my journey; and as I
leaned back in the comfortable though not elegant conveyance, I
meditated much at my ease.

\enquote{I suppose,} thought I, \enquote{judging from the plainness of
the servant and carriage, \Mrs{} Fairfax is not a very dashing person: so
much the better; I never lived amongst fine people but once, and I was
very miserable with them. I wonder if she lives alone except this
little girl; if so, and if she is in any degree amiable, I shall surely
be able to get on with her; I will do my best; it is a pity that doing
one's best does not always answer. At Lowood, indeed, I took that
resolution, kept it, and succeeded in pleasing; but with \Mrs{} Reed, I
remember my best was always spurned with scorn. I pray God \Mrs{} Fairfax
may not turn out a second \Mrs{} Reed; but if she does, I am not bound to
stay with her! let the worst come to the worst, I can advertise again. 
How far are we on our road now, I wonder?}

I let down the window and looked out; Millcote was behind us; judging by
the number of its lights, it seemed a place of considerable magnitude,
much larger than Lowton. We were now, as far as I could see, on a sort
of common; but there were houses scattered all over the district; I felt
we were in a different region to Lowood, more populous, less
picturesque; more stirring, less romantic.

The roads were heavy, the night misty; my conductor let his horse walk
all the way, and the hour and a half extended, I verily believe, to two
hours; at last he turned in his seat and said---

\enquote{You're noan so far fro' Thornfield now.}

Again I looked out: we were passing a church; I saw its low broad tower
against the sky, and its bell was tolling a quarter; I saw a narrow
galaxy of lights too, on a hillside, marking a village or hamlet. About
ten minutes after, the driver got down and opened a pair of gates: we
passed through, and they clashed to behind us. We now slowly ascended a
drive, and came upon the long front of a house: candlelight gleamed from
one curtained bow-window; all the rest were dark. The car stopped at
the front door; it was opened by a maid-servant; I alighted and went in.

\enquote{Will you walk this way, ma'am?} said the girl; and I followed
her across a square hall with high doors all round: she ushered me into
a room whose double illumination of fire and candle at first dazzled me,
contrasting as it did with the darkness to which my eyes had been for
two hours inured; when I could see, however, a cosy and agreeable
picture presented itself to my view.

A snug small room; a round table by a cheerful fire; an arm-chair
high-backed and old-fashioned, wherein sat the neatest imaginable little
elderly lady, in widow's cap, black silk gown, and snowy muslin apron;
exactly like what I had fancied \Mrs{} Fairfax, only less stately and
milder looking. She was occupied in knitting; a large cat sat demurely
at her feet; nothing in short was wanting to complete the beau-ideal of
domestic comfort. A more reassuring introduction for a new governess
could scarcely be conceived; there was no grandeur to overwhelm, no
stateliness to embarrass; and then, as I entered, the old lady got up
and promptly and kindly came forward to meet me.

\enquote{How do you do, my dear? I am afraid you have had a tedious
ride; John drives so slowly; you must be cold, come to the fire.}

\enquote{\Mrs{} Fairfax, I suppose?} said I\@.

\enquote{Yes, you are right: do sit down.}

She conducted me to her own chair, and then began to remove my shawl and
untie my bonnet-strings; I begged she would not give herself so much
trouble.

\enquote{Oh, it is no trouble; I dare say your own hands are almost
numbed with cold. Leah, make a little hot negus and cut a sandwich or
two: here are the keys of the storeroom.}

And she produced from her pocket a most housewifely bunch of keys, and
delivered them to the servant.

\enquote{Now, then, draw nearer to the fire,} she continued. 
\enquote{You've brought your luggage with you, haven't you, my dear?}

\enquote{Yes, ma'am.}

\enquote{I'll see it carried into your room,} she said, and bustled out.

\enquote{She treats me like a visitor,} thought I\@. \enquote{I little
expected such a reception; I anticipated only coldness and stiffness:
this is not like what I have heard of the treatment of governesses; but
I must not exult too soon.}

She returned; with her own hands cleared her knitting apparatus and a
book or two from the table, to make room for the tray which Leah now
brought, and then herself handed me the refreshments. I felt rather
confused at being the object of more attention than I had ever before
received, and, that too, shown by my employer and superior; but as she
did not herself seem to consider she was doing anything out of her
place, I thought it better to take her civilities quietly.

\enquote{Shall I have the pleasure of seeing Miss Fairfax to-night?} I
asked, when I had partaken of what she offered me.

\enquote{What did you say, my dear? I am a little deaf,} returned the
good lady, approaching her ear to my mouth.

I repeated the question more distinctly.

\enquote{Miss Fairfax? Oh, you mean Miss Varens! Varens is the name of
your future pupil.}

\enquote{Indeed! Then she is not your daughter?}

\enquote{No,---I have no family.}

I should have followed up my first inquiry, by asking in what way Miss
Varens was connected with her; but I recollected it was not polite to
ask too many questions: besides, I was sure to hear in time.

\enquote{I am so glad,} she continued, as she sat down opposite to me,
and took the cat on her knee; \enquote{I am so glad you are come; it
will be quite pleasant living here now with a companion. To be sure it
is pleasant at any time; for Thornfield is a fine old hall, rather
neglected of late years perhaps, but still it is a respectable place;
yet you know in winter-time one feels dreary quite alone in the best
quarters. I say alone---Leah is a nice girl to be sure, and John and
his wife are very decent people; but then you see they are only
servants, and one can't converse with them on terms of equality: one
must keep them at due distance, for fear of losing one's authority. I'm
sure last winter (it was a very severe one, if you recollect, and when
it did not snow, it rained and blew), not a creature but the butcher and
postman came to the house, from November till February; and I really got
quite melancholy with sitting night after night alone; I had Leah in to
read to me sometimes; but I don't think the poor girl liked the task
much: she felt it confining. In spring and summer one got on better:
sunshine and long days make such a difference; and then, just at the
commencement of this autumn, little Adela Varens came and her nurse: a
child makes a house alive all at once; and now you are here I shall be
quite gay.}

My heart really warmed to the worthy lady as I heard her talk; and I
drew my chair a little nearer to her, and expressed my sincere wish that
she might find my company as agreeable as she anticipated.

\enquote{But I'll not keep you sitting up late to-night,} said she;
\enquote{it is on the stroke of twelve now, and you have been travelling
all day: you must feel tired. If you have got your feet well warmed,
I'll show you your bedroom. I've had the room next to mine prepared for
you; it is only a small apartment, but I thought you would like it
better than one of the large front chambers: to be sure they have finer
furniture, but they are so dreary and solitary, I never sleep in them
myself.}

I thanked her for her considerate choice, and as I really felt fatigued
with my long journey, expressed my readiness to retire. She took her
candle, and I followed her from the room. First she went to see if the
hall-door was fastened; having taken the key from the lock, she led the
way upstairs. The steps and banisters were of oak; the staircase window
was high and latticed; both it and the long gallery into which the
bedroom doors opened looked as if they belonged to a church rather than
a house. A very chill and vault-like air pervaded the stairs and
gallery, suggesting cheerless ideas of space and solitude; and I was
glad, when finally ushered into my chamber, to find it of small
dimensions, and furnished in ordinary, modern style.

When \Mrs{} Fairfax had bidden me a kind good-night, and I had fastened my
door, gazed leisurely round, and in some measure effaced the eerie
impression made by that wide hall, that dark and spacious staircase, and
that long, cold gallery, by the livelier aspect of my little room, I
remembered that, after a day of bodily fatigue and mental anxiety, I was
now at last in safe haven. The impulse of gratitude swelled my heart,
and I knelt down at the bedside, and offered up thanks where thanks were
due; not forgetting, ere I rose, to implore aid on my further path, and
the power of meriting the kindness which seemed so frankly offered me
before it was earned. My couch had no thorns in it that night; my
solitary room no fears. At once weary and content, I slept soon and
soundly: when I awoke it was broad day.

The chamber looked such a bright little place to me as the sun shone in
between the gay blue chintz window curtains, showing papered walls and a
carpeted floor, so unlike the bare planks and stained plaster of Lowood,
that my spirits rose at the view. Externals have a great effect on the
young: I thought that a fairer era of life was beginning for me, one
that was to have its flowers and pleasures, as well as its thorns and
toils. My faculties, roused by the change of scene, the new field
offered to hope, seemed all astir. I cannot precisely define what they
expected, but it was something pleasant: not perhaps that day or that
month, but at an indefinite future period.

I rose; I dressed myself with care: obliged to be plain---for I had no
article of attire that was not made with extreme simplicity---I was
still by nature solicitous to be neat. It was not my habit to be
disregardful of appearance or careless of the impression I made: on the
contrary, I ever wished to look as well as I could, and to please as
much as my want of beauty would permit. I sometimes regretted that I
was not handsomer; I sometimes wished to have rosy cheeks, a straight
nose, and small cherry mouth; I desired to be tall, stately, and finely
developed in figure; I felt it a misfortune that I was so little, so
pale, and had features so irregular and so marked. And why had I these
aspirations and these regrets? It would be difficult to say: I could
not then distinctly say it to myself; yet I had a reason, and a logical,
natural reason too. However, when I had brushed my hair very smooth,
and put on my black frock---which, Quakerlike as it was, at least had
the merit of fitting to a nicety---and adjusted my clean white tucker, I
thought I should do respectably enough to appear before \Mrs{} Fairfax,
and that my new pupil would not at least recoil from me with antipathy. 
Having opened my chamber window, and seen that I left all things
straight and neat on the toilet table, I ventured forth.

Traversing the long and matted gallery, I descended the slippery steps
of oak; then I gained the hall: I halted there a minute; I looked at
some pictures on the walls (one, I remember, represented a grim man in a
cuirass, and one a lady with powdered hair and a pearl necklace), at a
bronze lamp pendent from the ceiling, at a great clock whose case was of
oak curiously carved, and ebon black with time and rubbing. Everything
appeared very stately and imposing to me; but then I was so little
accustomed to grandeur. The hall-door, which was half of glass, stood
open; I stepped over the threshold. It was a fine autumn morning; the
early sun shone serenely on embrowned groves and still green fields;
advancing on to the lawn, I looked up and surveyed the front of the
mansion. It was three storeys high, of proportions not vast, though
considerable: a gentleman's manor-house, not a nobleman's seat:
battlements round the top gave it a picturesque look. Its grey front
stood out well from the background of a rookery, whose cawing tenants
were now on the wing: they flew over the lawn and grounds to alight in a
great meadow, from which these were separated by a sunk fence, and where
an array of mighty old thorn trees, strong, knotty, and broad as oaks,
at once explained the etymology of the mansion's designation. Farther
off were hills: not so lofty as those round Lowood, nor so craggy, nor
so like barriers of separation from the living world; but yet quiet and
lonely hills enough, and seeming to embrace Thornfield with a seclusion
I had not expected to find existent so near the stirring locality of
Millcote. A little hamlet, whose roofs were blent with trees, straggled
up the side of one of these hills; the church of the district stood
nearer Thornfield: its old tower-top looked over a knoll between the
house and gates.

I was yet enjoying the calm prospect and pleasant fresh air, yet
listening with delight to the cawing of the rooks, yet surveying the
wide, hoary front of the hall, and thinking what a great place it was
for one lonely little dame like \Mrs{} Fairfax to inhabit, when that lady
appeared at the door.

\enquote{What! out already?} said she. \enquote{I see you are an early
riser.} I went up to her, and was received with an affable kiss and
shake of the hand.

\enquote{How do you like Thornfield?} she asked. I told her I liked it
very much.

\enquote{Yes,} she said, \enquote{it is a pretty place; but I fear it
will be getting out of order, unless \Mr{} Rochester should take it into
his head to come and reside here permanently; or, at least, visit it
rather oftener: great houses and fine grounds require the presence of
the proprietor.}

\enquote{\Mr{} Rochester!} I exclaimed. \enquote{Who is he?}

\enquote{The owner of Thornfield,} she responded quietly. \enquote{Did
you not know he was called Rochester?}

Of course I did not---I had never heard of him before; but the old lady
seemed to regard his existence as a universally understood fact, with
which everybody must be acquainted by instinct.

\enquote{I thought,} I continued, \enquote{Thornfield belonged to you.}

\enquote{To me? Bless you, child; what an idea! To me! I am only the
housekeeper---the manager. To be sure I am distantly related to the
 Rochesters by the mother's side, or at least my husband was; he was a
clergyman, incumbent of Hay---that little village yonder on the
hill---and that church near the gates was his. The present \Mr{}
 Rochester's mother was a Fairfax, and second cousin to my husband: but I
never presume on the connection---in fact, it is nothing to me; I
consider myself quite in the light of an ordinary housekeeper: my
employer is always civil, and I expect nothing more.}

\enquote{And the little girl---my pupil!}

\enquote{She is \Mr{} Rochester's ward; he commissioned me to find a
governess for her. He intended to have her brought up in ---shire, I
believe. Here she comes, with her \foreignquote{french}{bonne,} as she calls her
nurse.} The enigma then was explained: this affable and kind little
widow was no great dame; but a dependant like myself. I did not like
her the worse for that; on the contrary, I felt better pleased than
ever. The equality between her and me was real; not the mere result of
condescension on her part: so much the better---my position was all the
freer.

As I was meditating on this discovery, a little girl, followed by her
attendant, came running up the lawn. I looked at my pupil, who did not
at first appear to notice me: she was quite a child, perhaps seven or
eight years old, slightly built, with a pale, small-featured face, and a
redundancy of hair falling in curls to her waist.

\enquote{Good morning, Miss Adela,} said \Mrs{} Fairfax. \enquote{Come
and speak to the lady who is to teach you, and to make you a clever
woman some day.} She approached.

\foreignquote{french}{C'est là ma gouverante!}\footnote{\enquote{This is my governess!}} said she, pointing to me, and
addressing her nurse; who answered---

\foreignquote{french}{Mais oui, certainement.}\footnote{\enquote{But yes, of course.}}

\enquote{Are they foreigners?} I inquired, amazed at hearing the French
language.

\enquote{The nurse is a foreigner, and Adela was born on the Continent;
and, I believe, never left it till within six months ago. When she
first came here she could speak no English; now she can make shift to
talk it a little: I don't understand her, she mixes it so with French;
but you will make out her meaning very well, I dare say.}

Fortunately I had had the advantage of being taught French by a French
lady; and as I had always made a point of conversing with Madame Pierrot
as often as I could, and had besides, during the last seven years,
learnt a portion of French by heart daily---applying myself to take
pains with my accent, and imitating as closely as possible the
pronunciation of my teacher, I had acquired a certain degree of
readiness and correctness in the language, and was not likely to be much
at a loss with Mademoiselle Adela. She came and shook hand with me when
she heard that I was her governess; and as I led her in to breakfast, I
addressed some phrases to her in her own tongue: she replied briefly at
first, but after we were seated at the table, and she had examined me
some ten minutes with her large hazel eyes, she suddenly commenced
chattering fluently.

\enquote{Ah!} cried she, in French, \enquote{you speak my language as
well as \Mr{} Rochester does: I can talk to you as I can to him, and so
can Sophie. She will be glad: nobody here understands her: Madame
Fairfax is all English. Sophie is my nurse; she came with me over the
sea in a great ship with a chimney that smoked---how it did smoke!---and
I was sick, and so was Sophie, and so was \Mr{} Rochester. \Mr{} Rochester
lay down on a sofa in a pretty room called the salon, and Sophie and I
had little beds in another place. I nearly fell out of mine; it was
like a shelf. And Mademoiselle---what is your name?}

\enquote{Eyre---Jane Eyre.}

\enquote{Aire? Bah! I cannot say it. Well, our ship stopped in the
morning, before it was quite daylight, at a great city---a huge city,
with very dark houses and all smoky; not at all like the pretty clean
town I came from; and \Mr{} Rochester carried me in his arms over a plank
to the land, and Sophie came after, and we all got into a coach, which
took us to a beautiful large house, larger than this and finer, called
an hotel. We stayed there nearly a week: I and Sophie used to walk
every day in a great green place full of trees, called the Park; and
there were many children there besides me, and a pond with beautiful
birds in it, that I fed with crumbs.}

\enquote{Can you understand her when she runs on so fast?} asked \Mrs{}
Fairfax.

I understood her very well, for I had been accustomed to the fluent
tongue of Madame Pierrot.

\enquote{I wish,} continued the good lady, \enquote{you would ask her a
question or two about her parents: I wonder if she remembers them?}

\enquote{Adèle,} I inquired, \enquote{with whom did you live when you
were in that pretty clean town you spoke of?}

\enquote{I lived long ago with mama; but she is gone to the Holy
Virgin. Mama used to teach me to dance and sing, and to say verses. A
great many gentlemen and ladies came to see mama, and I used to dance
before them, or to sit on their knees and sing to them: I liked it. 
Shall I let you hear me sing now?}

She had finished her breakfast, so I permitted her to give a specimen of
her accomplishments. Descending from her chair, she came and placed
herself on my knee; then, folding her little hands demurely before her,
shaking back her curls and lifting her eyes to the ceiling, she
commenced singing a song from some opera. It was the strain of a
forsaken lady, who, after bewailing the perfidy of her lover, calls
pride to her aid; desires her attendant to deck her in her brightest
jewels and richest robes, and resolves to meet the false one that night
at a ball, and prove to him, by the gaiety of her demeanour, how little
his desertion has affected her.

The subject seemed strangely chosen for an infant singer; but I suppose
the point of the exhibition lay in hearing the notes of love and
jealousy warbled with the lisp of childhood; and in very bad taste that
point was: at least I thought so.

Adèle sang the canzonette tunefully enough, and with the \emph{naïveté}
of her age. This achieved, she jumped from my knee and said,
\enquote{Now, Mademoiselle, I will repeat you some poetry.}

Assuming an attitude, she began, \foreignquote{french}{La Ligue des Rats: fable de La
Fontaine.}\footnote{\enquote{The League of Rats: fable of La Fontaine.}} %See \Cref{c:league_of_rats} on \cpageref{c:league_of_rats}. 
She then declaimed the little piece with an attention to
punctuation and emphasis, a flexibility of voice and an appropriateness
of gesture, very unusual indeed at her age, and which proved she had
been carefully trained.

\enquote{Was it your mama who taught you that piece?} I asked.

\enquote{Yes, and she just used to say it in this way: \foreignquote{french}{Qu
avez vous donc? lui dit un de ces rats; parlez!}\footnote{%
\enquote{What's wrong? one of these rats said to him: speak!}} 
She made me lift my hand---so---to remind me to raise my voice at 
the question. Now shall I dance for you?}

\enquote{No, that will do: but after your mama went to the Holy Virgin,
as you say, with whom did you live then?}

\enquote{With Madame Frédéric and her husband: she took care of me, but
she is nothing related to me. I think she is poor, for she had not so
fine a house as mama. I was not long there. \Mr{} Rochester asked me if
I would like to go and live with him in England, and I said yes; for I
knew \Mr{} Rochester before I knew Madame Frédéric, and he was always kind
to me and gave me pretty dresses and toys: but you see he has not kept
his word, for he has brought me to England, and now he is gone back
again himself, and I never see him.}

After breakfast, Adèle and I withdrew to the library, which room, it
appears, \Mr{} Rochester had directed should be used as the schoolroom. 
Most of the books were locked up behind glass doors; but there was one
bookcase left open containing everything that could be needed in the way
of elementary works, and several volumes of light literature, poetry,
biography, travels, a few romances, \etc. I suppose he had considered
that these were all the governess would require for her private perusal;
and, indeed, they contented me amply for the present; compared with the
scanty pickings I had now and then been able to glean at Lowood, they
seemed to offer an abundant harvest of entertainment and information. 
In this room, too, there was a cabinet piano, quite new and of superior
tone; also an easel for painting and a pair of globes.

I found my pupil sufficiently docile, though disinclined to apply: she
had not been used to regular occupation of any kind. I felt it would be
injudicious to confine her too much at first; so, when I had talked to
her a great deal, and got her to learn a little, and when the morning
had advanced to noon, I allowed her to return to her nurse. I then
proposed to occupy myself till dinner-time in drawing some little
sketches for her use.

As I was going upstairs to fetch my portfolio and pencils, \Mrs{} Fairfax
called to me: \enquote{Your morning school-hours are over now, I
suppose,} said she. She was in a room the folding-doors of which stood
open: I went in when she addressed me. It was a large, stately
apartment, with purple chairs and curtains, a Turkey carpet,
walnut-panelled walls, one vast window rich in slanted glass, and a
lofty ceiling, nobly moulded. \Mrs{} Fairfax was dusting some vases of
fine purple spar, which stood on a sideboard.

\enquote{What a beautiful room!} I exclaimed, as I looked round; for I
had never before seen any half so imposing.

\enquote{Yes; this is the dining-room. I have just opened the window,
to let in a little air and sunshine; for everything gets so damp in
apartments that are seldom inhabited; the drawing-room yonder feels like
a vault.}

She pointed to a wide arch corresponding to the window, and hung like it
with a Tyrian-dyed curtain, now looped up. Mounting to it by two broad
steps, and looking through, I thought I caught a glimpse of a fairy
place, so bright to my novice-eyes appeared the view beyond. Yet it was
merely a very pretty drawing-room, and within it a boudoir, both spread
with white carpets, on which seemed laid brilliant garlands of flowers;
both ceiled with snowy mouldings of white grapes and vine-leaves,
beneath which glowed in rich contrast crimson couches and ottomans;
while the ornaments on the pale Parian mantelpiece were of sparkling
Bohemian glass, ruby red; and between the windows large mirrors repeated
the general blending of snow and fire.

\enquote{In what order you keep these rooms, \Mrs{} Fairfax!} said I\@. 
\enquote{No dust, no canvas coverings: except that the air feels chilly,
one would think they were inhabited daily.}

\enquote{Why, Miss Eyre, though \Mr{} Rochester's visits here are rare,
they are always sudden and unexpected; and as I observed that it put him
out to find everything swathed up, and to have a bustle of arrangement
on his arrival, I thought it best to keep the rooms in readiness.}

\enquote{Is \Mr{} Rochester an exacting, fastidious sort of man?}

\enquote{Not particularly so; but he has a gentleman's tastes and
habits, and he expects to have things managed in conformity to them.}

\enquote{Do you like him? Is he generally liked?}

\enquote{Oh, yes; the family have always been respected here. Almost
all the land in this neighbourhood, as far as you can see, has belonged
to the Rochesters time out of mind.}

\enquote{Well, but, leaving his land out of the question, do you like
him? Is he liked for himself?}

\enquote{I have no cause to do otherwise than like him; and I believe he
is considered a just and liberal landlord by his tenants: but he has
never lived much amongst them.}

\enquote{But has he no peculiarities? What, in short, is his
character?}

\enquote{Oh! his character is unimpeachable, I suppose. He is rather
peculiar, perhaps: he has travelled a great deal, and seen a great deal
of the world, I should think. I dare say he is clever, but I never had
much conversation with him.}

\enquote{In what way is he peculiar?}

\enquote{I don't know---it is not easy to describe---nothing striking,
but you feel it when he speaks to you; you cannot be always sure whether
he is in jest or earnest, whether he is pleased or the contrary; you
don't thoroughly understand him, in short---at least, I don't: but it is
of no consequence, he is a very good master.}

This was all the account I got from \Mrs{} Fairfax of her employer and
mine. There are people who seem to have no notion of sketching a
character, or observing and describing salient points, either in persons
or things: the good lady evidently belonged to this class; my queries
puzzled, but did not draw her out. \Mr{} Rochester was \Mr{} Rochester in
her eyes; a gentleman, a landed proprietor---nothing more: she inquired
and searched no further, and evidently wondered at my wish to gain a
more definite notion of his identity.

When we left the dining-room, she proposed to show me over the rest of
the house; and I followed her upstairs and downstairs, admiring as I
went; for all was well arranged and handsome. The large front chambers
I thought especially grand: and some of the third-storey rooms, though
dark and low, were interesting from their air of antiquity. The
furniture once appropriated to the lower apartments had from time to
time been removed here, as fashions changed: and the imperfect light
entering by their narrow casement showed bedsteads of a hundred years
old; chests in oak or walnut, looking, with their strange carvings of
palm branches and cherubs' heads, like types of the Hebrew ark; rows of
venerable chairs, high-backed and narrow; stools still more antiquated,
on whose cushioned tops were yet apparent traces of half-effaced
embroideries, wrought by fingers that for two generations had been
coffin-dust. All these relics gave to the third storey of Thornfield
Hall the aspect of a home of the past: a shrine of memory. I liked the
hush, the gloom, the quaintness of these retreats in the day; but I by
no means coveted a night's repose on one of those wide and heavy beds:
shut in, some of them, with doors of oak; shaded, others, with wrought
old English hangings crusted with thick work, portraying effigies of
strange flowers, and stranger birds, and strangest human beings,---all
which would have looked strange, indeed, by the pallid gleam of
moonlight.

\enquote{Do the servants sleep in these rooms?} I asked.

\enquote{No; they occupy a range of smaller apartments to the back; no
one ever sleeps here: one would almost say that, if there were a ghost
at Thornfield Hall, this would be its haunt.}

\enquote{So I think: you have no ghost, then?}

\enquote{None that I ever heard of,} returned \Mrs{} Fairfax, smiling.

\enquote{Nor any traditions of one? no legends or ghost stories?}

\enquote{I believe not. And yet it is said the Rochesters have been
rather a violent than a quiet race in their time: perhaps, though, that
is the reason they rest tranquilly in their graves now.}

\enquote{Yes---\enquote{after life's fitful fever they sleep well,}} I
muttered. \enquote{Where are you going now, \Mrs{} Fairfax?} for she was
moving away.

\enquote{On to the leads; will you come and see the view from thence?} 
I followed still, up a very narrow staircase to the attics, and thence
by a ladder and through a trap-door to the roof of the hall. I was now
on a level with the crow colony, and could see into their nests. 
Leaning over the battlements and looking far down, I surveyed the
grounds laid out like a map: the bright and velvet lawn closely girdling
the grey base of the mansion; the field, wide as a park, dotted with its
ancient timber; the wood, dun and sere, divided by a path visibly
overgrown, greener with moss than the trees were with foliage; the
church at the gates, the road, the tranquil hills, all reposing in the
autumn day's sun; the horizon bounded by a propitious sky, azure,
marbled with pearly white. No feature in the scene was extraordinary,
but all was pleasing. When I turned from it and repassed the trap-door,
I could scarcely see my way down the ladder; the attic seemed black as a
vault compared with that arch of blue air to which I had been looking
up, and to that sunlit scene of grove, pasture, and green hill, of which
the hall was the centre, and over which I had been gazing with delight.

\Mrs{} Fairfax stayed behind a moment to fasten the trap-door; I, by drift
of groping, found the outlet from the attic, and proceeded to descend
the narrow garret staircase. I lingered in the long passage to which
this led, separating the front and back rooms of the third storey:
narrow, low, and dim, with only one little window at the far end, and
looking, with its two rows of small black doors all shut, like a
corridor in some Bluebeard's castle.

While I paced softly on, the last sound I expected to hear in so still a
region, a laugh, struck my ear. It was a curious laugh; distinct,
formal, mirthless. I stopped: the sound ceased, only for an instant; it
began again, louder: for at first, though distinct, it was very low. It
passed off in a clamorous peal that seemed to wake an echo in every
lonely chamber; though it originated but in one, and I could have
pointed out the door whence the accents issued.

\enquote{\Mrs{} Fairfax!} I called out: for I now heard her descending the
great stairs. \enquote{Did you hear that loud laugh? Who is it?}

\enquote{Some of the servants, very likely,} she answered:
\enquote{perhaps Grace Poole.}

\enquote{Did you hear it?} I again inquired.

\enquote{Yes, plainly: I often hear her: she sews in one of these
rooms. Sometimes Leah is with her; they are frequently noisy together.}

The laugh was repeated in its low, syllabic tone, and terminated in an
odd murmur.

\enquote{Grace!} exclaimed \Mrs{} Fairfax.

I really did not expect any Grace to answer; for the laugh was as
tragic, as preternatural a laugh as any I ever heard; and, but that it
was high noon, and that no circumstance of ghostliness accompanied the
curious cachinnation; but that neither scene nor season favoured fear, I
should have been superstitiously afraid. However, the event showed me I
was a fool for entertaining a sense even of surprise.

The door nearest me opened, and a servant came out,---a woman of between
thirty and forty; a set, square-made figure, red-haired, and with a
hard, plain face: any apparition less romantic or less ghostly could
scarcely be conceived.

\enquote{Too much noise, Grace,} said \Mrs{} Fairfax. \enquote{Remember
directions!} Grace curtseyed silently and went in.

\enquote{She is a person we have to sew and assist Leah in her
housemaid's work,} continued the widow; \enquote{not altogether
unobjectionable in some points, but she does well enough. By-the-bye,
how have you got on with your new pupil this morning?}

The conversation, thus turned on Adèle, continued till we reached the
light and cheerful region below. Adèle came running to meet us in the
hall, exclaiming---

\foreignquote{french}{Mesdames, vous êtes servies!}\footnote{\enquote{Ladies, dinner is served!}} adding, \foreignquote{french}{J'ai bien faim, moi!}\footnote{\enquote{I'm really hungry!}}

We found dinner ready, and waiting for us in \Mrs{} Fairfax's room.

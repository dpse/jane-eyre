\FChapter{Chapter Thirty-Two}{32}

\Lettrine{I}{} \textsc{continued the labours} of the village-school as actively and faithfully
as I could. It was truly hard work at first. Some time elapsed before,
with all my efforts, I could comprehend my scholars and their nature.
Wholly untaught, with faculties quite torpid, they seemed to me
hopelessly dull; and, at first sight, all dull alike: but I soon found I
was mistaken. There was a difference amongst them as amongst the
educated; and when I got to know them, and they me, this difference
rapidly developed itself. Their amazement at me, my language, my rules,
and ways, once subsided, I found some of these heavy-looking, gaping
rustics wake up into sharp-witted girls enough. Many showed themselves
obliging, and amiable too; and I discovered amongst them not a few
examples of natural politeness, and innate self-respect, as well as of
excellent capacity, that won both my goodwill and my admiration. These
soon took a pleasure in doing their work well, in keeping their persons
neat, in learning their tasks regularly, in acquiring quiet and orderly
manners. The rapidity of their progress, in some instances, was even
surprising; and an honest and happy pride I took in it: besides, I began
personally to like some of the best girls; and they liked me. I had
amongst my scholars several farmers' daughters: young women grown,
almost. These could already read, write, and sew; and to them I taught
the elements of grammar, geography, history, and the finer kinds of
needlework. I found estimable characters amongst them---characters
desirous of information and disposed for improvement---with whom I
passed many a pleasant evening hour in their own homes. Their parents
then (the farmer and his wife) loaded me with attentions. There was an
enjoyment in accepting their simple kindness, and in repaying it by a
consideration---a scrupulous regard to their feelings---to which they
were not, perhaps, at all times accustomed, and which both charmed and
benefited them; because, while it elevated them in their own eyes, it
made them emulous to merit the deferential treatment they received.

I felt I became a favourite in the neighbourhood. Whenever I went out,
I heard on all sides cordial salutations, and was welcomed with friendly
smiles. To live amidst general regard, though it be but the regard of
working people, is like \enquote{sitting in sunshine, calm and sweet;}
serene inward feelings bud and bloom under the ray. At this period of
my life, my heart far oftener swelled with thankfulness than sank with
dejection: and yet, reader, to tell you all, in the midst of this calm,
this useful existence---after a day passed in honourable exertion
amongst my scholars, an evening spent in drawing or reading contentedly
alone---I used to rush into strange dreams at night: dreams
many-coloured, agitated, full of the ideal, the stirring, the
stormy---dreams where, amidst unusual scenes, charged with adventure,
with agitating risk and romantic chance, I still again and again met \Mr{}
Rochester, always at some exciting crisis; and then the sense of being
in his arms, hearing his voice, meeting his eye, touching his hand and
cheek, loving him, being loved by him---the hope of passing a lifetime
at his side, would be renewed, with all its first force and fire. Then
I awoke. Then I recalled where I was, and how situated. Then I rose up
on my curtainless bed, trembling and quivering; and then the still, dark
night witnessed the convulsion of despair, and heard the burst of
passion. By nine o'clock the next morning I was punctually opening the
school; tranquil, settled, prepared for the steady duties of the day.

Rosamond Oliver kept her word in coming to visit me. Her call at the
school was generally made in the course of her morning ride. She would
canter up to the door on her pony, followed by a mounted livery
servant. Anything more exquisite than her appearance, in her purple
habit, with her Amazon's cap of black velvet placed gracefully above the
long curls that kissed her cheek and floated to her shoulders, can
scarcely be imagined: and it was thus she would enter the rustic
building, and glide through the dazzled ranks of the village children.
She generally came at the hour when \Mr{} Rivers was engaged in giving his
daily catechising lesson. Keenly, I fear, did the eye of the visitress
pierce the young pastor's heart. A sort of instinct seemed to warn him
of her entrance, even when he did not see it; and when he was looking
quite away from the door, if she appeared at it, his cheek would glow,
and his marble-seeming features, though they refused to relax, changed
indescribably, and in their very quiescence became expressive of a
repressed fervour, stronger than working muscle or darting glance could
indicate.

Of course, she knew her power: indeed, he did not, because he could not,
conceal it from her. In spite of his Christian stoicism, when she went
up and addressed him, and smiled gaily, encouragingly, even fondly in
his face, his hand would tremble and his eye burn. He seemed to say,
with his sad and resolute look, if he did not say it with his lips,
\enquote{I love you, and I know you prefer me. It is not despair of
	success that keeps me dumb. If I offered my heart, I believe you would
	accept it. But that heart is already laid on a sacred altar: the fire
	is arranged round it. It will soon be no more than a sacrifice
	consumed.}

And then she would pout like a disappointed child; a pensive cloud would
soften her radiant vivacity; she would withdraw her hand hastily from
his, and turn in transient petulance from his aspect, at once so heroic
and so martyr-like. \St{} John, no doubt, would have given the world to
follow, recall, retain her, when she thus left him; but he would not
give one chance of heaven, nor relinquish, for the elysium of her love,
one hope of the true, eternal Paradise. Besides, he could not bind all
that he had in his nature---the rover, the aspirant, the poet, the
priest---in the limits of a single passion. He could not---he would
not---renounce his wild field of mission warfare for the parlours and
the peace of Vale Hall. I learnt so much from himself in an inroad I
once, despite his reserve, had the daring to make on his confidence.

Miss Oliver already honoured me with frequent visits to my cottage. I
had learnt her whole character, which was without mystery or disguise:
she was coquettish but not heartless; exacting, but not worthlessly
selfish. She had been indulged from her birth, but was not absolutely
spoilt. She was hasty, but good-humoured; vain (she could not help it,
when every glance in the glass showed her such a flush of loveliness),
but not affected; liberal-handed; innocent of the pride of wealth;
ingenuous; sufficiently intelligent; gay, lively, and unthinking: she
was very charming, in short, even to a cool observer of her own sex like
me; but she was not profoundly interesting or thoroughly impressive. A
very different sort of mind was hers from that, for instance, of the
sisters of \St{} John. Still, I liked her almost as I liked my pupil
Adèle; except that, for a child whom we have watched over and taught, a
closer affection is engendered than we can give an equally attractive
adult acquaintance.

She had taken an amiable caprice to me. She said I was like \Mr{} Rivers,
only, certainly, she allowed, \enquote{not one-tenth so handsome, though
	I was a nice neat little soul enough, but he was an angel.} I was,
however, good, clever, composed, and firm, like him. I was a
\emph{lusus naturæ}, she affirmed, as a village schoolmistress: she was
sure my previous history, if known, would make a delightful romance.

One evening, while, with her usual child-like activity, and thoughtless
yet not offensive inquisitiveness, she was rummaging the cupboard and
the table-drawer of my little kitchen, she discovered first two French
books, a volume of Schiller, a German grammar and dictionary, and then
my drawing-materials and some sketches, including a pencil-head of a
pretty little cherub-like girl, one of my scholars, and sundry views
from nature, taken in the Vale of Morton and on the surrounding moors.
She was first transfixed with surprise, and then electrified with
delight.

\enquote{Had I done these pictures? Did I know French and German? What
	a love---what a miracle I was! I drew better than her master in the
	first school in S-. Would I sketch a portrait of her, to show to papa?}

\enquote{With pleasure,} I replied; and I felt a thrill of
artist-delight at the idea of copying from so perfect and radiant a
model. She had then on a dark-blue silk dress; her arms and her neck
were bare; her only ornament was her chestnut tresses, which waved over
her shoulders with all the wild grace of natural curls. I took a sheet
of fine card-board, and drew a careful outline. I promised myself the
pleasure of colouring it; and, as it was getting late then, I told her
she must come and sit another day.

She made such a report of me to her father, that \Mr{} Oliver himself
accompanied her next evening---a tall, massive-featured, middle-aged,
and grey-headed man, at whose side his lovely daughter looked like a
bright flower near a hoary turret. He appeared a taciturn, and perhaps
a proud personage; but he was very kind to me. The sketch of Rosamond's
portrait pleased him highly: he said I must make a finished picture of
it. He insisted, too, on my coming the next day to spend the evening at
Vale Hall.

I went. I found it a large, handsome residence, showing abundant
evidences of wealth in the proprietor. Rosamond was full of glee and
pleasure all the time I stayed. Her father was affable; and when he
entered into conversation with me after tea, he expressed in strong
terms his approbation of what I had done in Morton school, and said he
only feared, from what he saw and heard, I was too good for the place,
and would soon quit it for one more suitable.

\enquote{Indeed,} cried Rosamond, \enquote{she is clever enough to be a
	governess in a high family, papa.}

I thought I would far rather be where I am than in any high family in
the land. \Mr{} Oliver spoke of \Mr{} Rivers---of the Rivers family---with
great respect. He said it was a very old name in that neighbourhood;
that the ancestors of the house were wealthy; that all Morton had once
belonged to them; that even now he considered the representative of that
house might, if he liked, make an alliance with the best. He accounted
it a pity that so fine and talented a young man should have formed the
design of going out as a missionary; it was quite throwing a valuable
life away. It appeared, then, that her father would throw no obstacle
in the way of Rosamond's union with \St{} John. \Mr{} Oliver evidently
regarded the young clergyman's good birth, old name, and sacred
profession as sufficient compensation for the want of fortune.

It was the 5th of November, and a holiday. My little servant, after
helping me to clean my house, was gone, well satisfied with the fee of a
penny for her aid. All about me was spotless and bright---scoured
floor, polished grate, and well-rubbed chairs. I had also made myself
neat, and had now the afternoon before me to spend as I would.

The translation of a few pages of German occupied an hour; then I got my
palette and pencils, and fell to the more soothing, because easier
occupation, of completing Rosamond Oliver's miniature. The head was
finished already: there was but the background to tint and the drapery
to shade off; a touch of carmine, too, to add to the ripe lips---a soft
curl here and there to the tresses---a deeper tinge to the shadow of the
lash under the azured eyelid. I was absorbed in the execution of these
nice details, when, after one rapid tap, my door unclosed, admitting St.
John Rivers.

\enquote{I am come to see how you are spending your holiday,} he said.
\enquote{Not, I hope, in thought? No, that is well: while you draw you
	will not feel lonely. You see, I mistrust you still, though you have
	borne up wonderfully so far. I have brought you a book for evening
	solace,} and he laid on the table a new publication---a poem: one of
those genuine productions so often vouchsafed to the fortunate public of
those days---the golden age of modern literature. Alas! the readers of
our era are less favoured. But courage! I will not pause either to
accuse or repine. I know poetry is not dead, nor genius lost; nor has
Mammon gained power over either, to bind or slay: they will both assert
their existence, their presence, their liberty and strength again one
day. Powerful angels, safe in heaven! they smile when sordid souls
triumph, and feeble ones weep over their destruction. Poetry
destroyed? Genius banished? No! Mediocrity, no: do not let envy
prompt you to the thought. No; they not only live, but reign and
redeem: and without their divine influence spread everywhere, you would
be in hell---the hell of your own meanness.

While I was eagerly glancing at the bright pages of \enquote{Marmion}
(for \enquote{Marmion} it was), \St{} John stooped to examine my drawing.
His tall figure sprang erect again with a start: he said nothing. I
looked up at him: he shunned my eye. I knew his thoughts well, and
could read his heart plainly; at the moment I felt calmer and cooler
than he: I had then temporarily the advantage of him, and I conceived an
inclination to do him some good, if I could.

\enquote{With all his firmness and self-control,} thought I, \enquote{he
	tasks himself too far: locks every feeling and pang within---expresses,
	confesses, imparts nothing. I am sure it would benefit him to talk a
	little about this sweet Rosamond, whom he thinks he ought not to marry:
	I will make him talk.}

I said first, \enquote{Take a chair, \Mr{} Rivers.} But he answered, as
he always did, that he could not stay. \enquote{Very well,} I
responded, mentally, \enquote{stand if you like; but you shall not go
	just yet, I am determined: solitude is at least as bad for you as it is
	for me. I'll try if I cannot discover the secret spring of your
	confidence, and find an aperture in that marble breast through which I
	can shed one drop of the balm of sympathy.}

\enquote{Is this portrait like?} I asked bluntly.

\enquote{Like! Like whom? I did not observe it closely.}

\enquote{You did, \Mr{} Rivers.}

He almost started at my sudden and strange abruptness: he looked at me
astonished. \enquote{Oh, that is nothing yet,} I muttered within.
\enquote{I don't mean to be baffled by a little stiffness on your part;
	I'm prepared to go to considerable lengths.} I continued, \enquote{You
	observed it closely and distinctly; but I have no objection to your
	looking at it again,} and I rose and placed it in his hand.

\enquote{A well-executed picture,} he said; \enquote{very soft, clear
	colouring; very graceful and correct drawing.}

\enquote{Yes, yes; I know all that. But what of the resemblance? Who
	is it like?}

Mastering some hesitation, he answered, \enquote{Miss Oliver, I
	presume.}

\enquote{Of course. And now, sir, to reward you for the accurate guess,
	I will promise to paint you a careful and faithful duplicate of this
	very picture, provided you admit that the gift would be acceptable to
	you. I don't wish to throw away my time and trouble on an offering you
	would deem worthless.}

He continued to gaze at the picture: the longer he looked, the firmer he
held it, the more he seemed to covet it. \enquote{It is like!} he
murmured; \enquote{the eye is well managed: the colour, light,
	expression, are perfect. It smiles!}

\enquote{Would it comfort, or would it wound you to have a similar
	painting? Tell me that. When you are at Madagascar, or at the Cape, or
	in India, would it be a consolation to have that memento in your
	possession? or would the sight of it bring recollections calculated to
	enervate and distress?}

He now furtively raised his eyes: he glanced at me, irresolute,
disturbed: he again surveyed the picture.

\enquote{That I should like to have it is certain: whether it would be
	judicious or wise is another question.}

Since I had ascertained that Rosamond really preferred him, and that her
father was not likely to oppose the match, I---less exalted in my views
than \St{} John---had been strongly disposed in my own heart to advocate
their union. It seemed to me that, should he become the possessor of
\Mr{} Oliver's large fortune, he might do as much good with it as if he
went and laid his genius out to wither, and his strength to waste, under
a tropical sun. With this persuasion I now answered---

\enquote{As far as I can see, it would be wiser and more judicious if
	you were to take to yourself the original at once.}

By this time he had sat down: he had laid the picture on the table
before him, and with his brow supported on both hands, hung fondly over
it. I discerned he was now neither angry nor shocked at my audacity. I
saw even that to be thus frankly addressed on a subject he had deemed
unapproachable---to hear it thus freely handled---was beginning to be
felt by him as a new pleasure---an unhoped-for relief. Reserved people
often really need the frank discussion of their sentiments and griefs
more than the expansive. The sternest-seeming stoic is human after all;
and to \enquote{burst} with boldness and good-will into \enquote{the
	silent sea} of their souls is often to confer on them the first of
obligations.

\enquote{She likes you, I am sure,} said I, as I stood behind his chair,
\enquote{and her father respects you. Moreover, she is a sweet
	girl---rather thoughtless; but you would have sufficient thought for
	both yourself and her. You ought to marry her.}

\enquote{\emph{Does} she like me?} he asked.

\enquote{Certainly; better than she likes any one else. She talks of
	you continually: there is no subject she enjoys so much or touches upon
	so often.}

\enquote{It is very pleasant to hear this,} he said---\enquote{very: go
	on for another quarter of an hour.} And he actually took out his watch
and laid it upon the table to measure the time.

\enquote{But where is the use of going on,} I asked, \enquote{when you
	are probably preparing some iron blow of contradiction, or forging a
	fresh chain to fetter your heart?}

\enquote{Don't imagine such hard things. Fancy me yielding and melting,
	as I am doing: human love rising like a freshly opened fountain in my
	mind and overflowing with sweet inundation all the field I have so
	carefully and with such labour prepared---so assiduously sown with the
	seeds of good intentions, of self-denying plans. And now it is deluged
	with a nectarous flood---the young germs swamped---delicious poison
	cankering them: now I see myself stretched on an ottoman in the
	drawing-room at Vale Hall at my bride Rosamond Oliver's feet: she is
	talking to me with her sweet voice---gazing down on me with those eyes
	your skilful hand has copied so well---smiling at me with these coral
	lips. She is mine---I am hers---this present life and passing world
	suffice to me. Hush! say nothing---my heart is full of delight---my
	senses are entranced---let the time I marked pass in peace.}

I humoured him: the watch ticked on: he breathed fast and low: I stood
silent. Amidst this hush the quartet sped; he replaced the watch, laid
the picture down, rose, and stood on the hearth.

\enquote{Now,} said he, \enquote{that little space was given to delirium
	and delusion. I rested my temples on the breast of temptation, and put
	my neck voluntarily under her yoke of flowers. I tasted her cup. The
	pillow was burning: there is an asp in the garland: the wine has a
	bitter taste: her promises are hollow---her offers false: I see and know
	all this.}

I gazed at him in wonder.

\enquote{It is strange,} pursued he, \enquote{that while I love Rosamond
	Oliver so wildly---with all the intensity, indeed, of a first passion,
	the object of which is exquisitely beautiful, graceful, fascinating---I
	experience at the same time a calm, unwarped consciousness that she
	would not make me a good wife; that she is not the partner suited to me;
	that I should discover this within a year after marriage; and that to
	twelve months' rapture would succeed a lifetime of regret. This I
	know.}

\enquote{Strange indeed!} I could not help ejaculating.

\enquote{While something in me,} he went on, \enquote{is acutely
	sensible to her charms, something else is as deeply impressed with her
	defects: they are such that she could sympathise in nothing I aspired
	to---co-operate in nothing I undertook. Rosamond a sufferer, a
	labourer, a female apostle? Rosamond a missionary's wife? No!}

\enquote{But you need not be a missionary. You might relinquish that
	scheme.}

\enquote{Relinquish! What! my vocation? My great work? My foundation
	laid on earth for a mansion in heaven? My hopes of being numbered in
	the band who have merged all ambitions in the glorious one of bettering
	their race---of carrying knowledge into the realms of ignorance---of
	substituting peace for war---freedom for bondage---religion for
	superstition---the hope of heaven for the fear of hell? Must I
	relinquish that? It is dearer than the blood in my veins. It is what I
	have to look forward to, and to live for.}

After a considerable pause, I said---\enquote{And Miss Oliver? Are her
	disappointment and sorrow of no interest to you?}

\enquote{Miss Oliver is ever surrounded by suitors and flatterers: in
	less than a month, my image will be effaced from her heart. She will
	forget me; and will marry, probably, some one who will make her far
	happier than I should do.}

\enquote{You speak coolly enough; but you suffer in the conflict. You
	are wasting away.}

\enquote{No. If I get a little thin, it is with anxiety about my
	prospects, yet unsettled---my departure, continually procrastinated.
	Only this morning, I received intelligence that the successor, whose
	arrival I have been so long expecting, cannot be ready to replace me for
	three months to come yet; and perhaps the three months may extend to
	six.}

\enquote{You tremble and become flushed whenever Miss Oliver enters the
	schoolroom.}

Again the surprised expression crossed his face. He had not imagined
that a woman would dare to speak so to a man. For me, I felt at home in
this sort of discourse. I could never rest in communication with
strong, discreet, and refined minds, whether male or female, till I had
passed the outworks of conventional reserve, and crossed the threshold
of confidence, and won a place by their heart's very hearthstone.

\enquote{You are original,} said he, \enquote{and not timid. There is
	something brave in your spirit, as well as penetrating in your eye; but
	allow me to assure you that you partially misinterpret my emotions. You
	think them more profound and potent than they are. You give me a larger
	allowance of sympathy than I have a just claim to. When I colour, and
	when I shade before Miss Oliver, I do not pity myself. I scorn the
	weakness. I know it is ignoble: a mere fever of the flesh: not, I
	declare, the convulsion of the soul. \emph{That} is just as fixed as a
	rock, firm set in the depths of a restless sea. Know me to be what I
	am---a cold hard man.}

I smiled incredulously.

\enquote{You have taken my confidence by storm,} he continued,
\enquote{and now it is much at your service. I am simply, in my
	original state---stripped of that blood-bleached robe with which
	Christianity covers human deformity---a cold, hard, ambitious man.
	Natural affection only, of all the sentiments, has permanent power over
	me. Reason, and not feeling, is my guide; my ambition is unlimited: my
	desire to rise higher, to do more than others, insatiable. I honour
	endurance, perseverance, industry, talent; because these are the means
	by which men achieve great ends and mount to lofty eminence. I watch
	your career with interest, because I consider you a specimen of a
	diligent, orderly, energetic woman: not because I deeply compassionate
	what you have gone through, or what you still suffer.}

\enquote{You would describe yourself as a mere pagan philosopher,} I
said.

\enquote{No. There is this difference between me and deistic
	philosophers: I believe; and I believe the Gospel. You missed your
	epithet. I am not a pagan, but a Christian philosopher---a follower of
	the sect of Jesus. As His disciple I adopt His pure, His merciful, His
	benignant doctrines. I advocate them: I am sworn to spread them. Won
	in youth to religion, she has cultivated my original qualities
	thus:---From the minute germ, natural affection, she has developed the
	overshadowing tree, philanthropy. From the wild stringy root of human
	uprightness, she has reared a due sense of the Divine justice. Of the
	ambition to win power and renown for my wretched self, she has formed
	the ambition to spread my Master's kingdom; to achieve victories for the
	standard of the cross. So much has religion done for me; turning the
	original materials to the best account; pruning and training nature.
	But she could not eradicate nature: nor will it be eradicated
	\enquote{till this mortal shall put on immortality.}}

Having said this, he took his hat, which lay on the table beside my
palette. Once more he looked at the portrait.

\enquote{She \emph{is} lovely,} he murmured. \enquote{She is well named the
	Rose of the World, indeed!}

\enquote{And may I not paint one like it for you?}

\enquote{\emph{Cui bono}? No.}

He drew over the picture the sheet of thin paper on which I was
accustomed to rest my hand in painting, to prevent the cardboard from
being sullied. What he suddenly saw on this blank paper, it was
impossible for me to tell; but something had caught his eye. He took it
up with a snatch; he looked at the edge; then shot a glance at me,
inexpressibly peculiar, and quite incomprehensible: a glance that seemed
to take and make note of every point in my shape, face, and dress; for
it traversed all, quick, keen as lightning. His lips parted, as if to
speak: but he checked the coming sentence, whatever it was.

\enquote{What is the matter?} I asked.

\enquote{Nothing in the world,} was the reply; and, replacing the paper,
I saw him dexterously tear a narrow slip from the margin. It
disappeared in his glove; and, with one hasty nod and
\enquote{good-afternoon,} he vanished.

\enquote{Well!} I exclaimed, using an expression of the district,
\enquote{that caps the globe, however!}

I, in my turn, scrutinised the paper; but saw nothing on it save a few
dingy stains of paint where I had tried the tint in my pencil. I
pondered the mystery a minute or two; but finding it insolvable, and
being certain it could not be of much moment, I dismissed, and soon
forgot it.

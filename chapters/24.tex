\FChapter{Chapter Twenty-Four}{24}

\Lettrine{A}{s} \textsc{I rose and dressed,} I thought over what had happened, and wondered if
it were a dream.  I could not be certain of the reality till I had seen
\Mr{} Rochester again, and heard him renew his words of love and promise.

While arranging my hair, I looked at my face in the glass, and felt it
was no longer plain: there was hope in its aspect and life in its
colour; and my eyes seemed as if they had beheld the fount of fruition,
and borrowed beams from the lustrous ripple.  I had often been unwilling
to look at my master, because I feared he could not be pleased at my
look; but I was sure I might lift my face to his now, and not cool his
affection by its expression.  I took a plain but clean and light summer
dress from my drawer and put it on: it seemed no attire had ever so well
become me, because none had I ever worn in so blissful a mood.

I was not surprised, when I ran down into the hall, to see that a
brilliant June morning had succeeded to the tempest of the night; and to
feel, through the open glass door, the breathing of a fresh and fragrant
breeze.  Nature must be gladsome when I was so happy.  A beggar-woman
and her little boy---pale, ragged objects both---were coming up the
walk, and I ran down and gave them all the money I happened to have in
my purse---some three or four shillings: good or bad, they must partake
of my jubilee.  The rooks cawed, and blither birds sang; but nothing was
so merry or so musical as my own rejoicing heart.

\Mrs{} Fairfax surprised me by looking out of the window with a sad
countenance, and saying gravely---\enquote{Miss Eyre, will you come to
	breakfast?}  During the meal she was quiet and cool: but I could not
undeceive her then.  I must wait for my master to give explanations; and
so must she.  I ate what I could, and then I hastened upstairs.  I met
Adèle leaving the schoolroom.

\enquote{Where are you going?  It is time for lessons.}

\enquote{\Mr{} Rochester has sent me away to the nursery.}

\enquote{Where is he?}

\enquote{In there,} pointing to the apartment she had left; and I went
in, and there he stood.

\enquote{Come and bid me good-morning,} said he.  I gladly advanced; and
it was not merely a cold word now, or even a shake of the hand that I
received, but an embrace and a kiss.  It seemed natural: it seemed
genial to be so well loved, so caressed by him.

\enquote{Jane, you look blooming, and smiling, and pretty,} said he:
\enquote{truly pretty this morning.  Is this my pale, little elf?  Is
	this my mustard-seed?  This little sunny-faced girl with the dimpled
	cheek and rosy lips; the satin-smooth hazel hair, and the radiant hazel
	eyes?}  (I had green eyes, reader; but you must excuse the mistake: for
him they were new-dyed, I suppose.)

\enquote{It is Jane Eyre, sir.}

\enquote{Soon to be Jane Rochester,} he added: \enquote{in four weeks,
	Janet; not a day more.  Do you hear that?}

I did, and I could not quite comprehend it: it made me giddy.  The
feeling, the announcement sent through me, was something stronger than
was consistent with joy---something that smote and stunned.  It was, I
think almost fear.

\enquote{You blushed, and now you are white, Jane: what is that for?}

\enquote{Because you gave me a new name---Jane Rochester; and it seems
	so strange.}

\enquote{Yes, \Mrs{} Rochester,} said he; \enquote{young \Mrs{}
	Rochester---Fairfax Rochester's girl-bride.}

\enquote{It can never be, sir; it does not sound likely.  Human beings
	never enjoy complete happiness in this world.  I was not born for a
	different destiny to the rest of my species: to imagine such a lot
	befalling me is a fairy tale---a day-dream.}

\enquote{Which I can and will realise.  I shall begin to-day.  This
	morning I wrote to my banker in London to send me certain jewels he has
	in his keeping,---heirlooms for the ladies of Thornfield.  In a day or
	two I hope to pour them into your lap: for every privilege, every
	attention shall be yours that I would accord a peer's daughter, if about
	to marry her.}

\enquote{Oh, sir!---never rain jewels!  I don't like to hear them spoken
	of.  Jewels for Jane Eyre sounds unnatural and strange: I would rather
	not have them.}

\enquote{I will myself put the diamond chain round your neck, and the
	circlet on your forehead,---which it will become: for nature, at least,
	has stamped her patent of nobility on this brow, Jane; and I will clasp
	the bracelets on these fine wrists, and load these fairy-like fingers
	with rings.}

\enquote{No, no, sir! think of other subjects, and speak of other
	things, and in another strain.  Don't address me as if I were a beauty;
	I am your plain, Quakerish governess.}

\enquote{You are a beauty in my eyes, and a beauty just after the desire
	of my heart,---delicate and aërial.}

\enquote{Puny and insignificant, you mean.  You are dreaming, sir,---or
	you are sneering.  For God's sake don't be ironical!}

\enquote{I will make the world acknowledge you a beauty, too,} he went
on, while I really became uneasy at the strain he had adopted, because I
felt he was either deluding himself or trying to delude me.  \enquote{I
	will attire my Jane in satin and lace, and she shall have roses in her
	hair; and I will cover the head I love best with a priceless veil.}

\enquote{And then you won't know me, sir; and I shall not be your Jane
	Eyre any longer, but an ape in a harlequin's jacket---a jay in borrowed
	plumes.  I would as soon see you, \Mr{} Rochester, tricked out in
	stage-trappings, as myself clad in a court-lady's robe; and I don't call
	you handsome, sir, though I love you most dearly: far too dearly to
	flatter you.  Don't flatter me.}

He pursued his theme, however, without noticing my deprecation.
\enquote{This very day I shall take you in the carriage to Millcote, and
	you must choose some dresses for yourself.  I told you we shall be
	married in four weeks.  The wedding is to take place quietly, in the
	church down below yonder; and then I shall waft you away at once to
	town.  After a brief stay there, I shall bear my treasure to regions
	nearer the sun: to French vineyards and Italian plains; and she shall
	see whatever is famous in old story and in modern record: she shall
	taste, too, of the life of cities; and she shall learn to value herself
	by just comparison with others.}

\enquote{Shall I travel?---and with you, sir?}

\enquote{You shall sojourn at Paris, Rome, and Naples: at Florence,
	Venice, and Vienna: all the ground I have wandered over shall be
	re-trodden by you: wherever I stamped my hoof, your sylph's foot shall
	step also.  Ten years since, I flew through Europe half mad; with
	disgust, hate, and rage as my companions: now I shall revisit it healed
	and cleansed, with a very angel as my comforter.}

I laughed at him as he said this.  \enquote{I am not an angel,} I
asserted; \enquote{and I will not be one till I die: I will be myself.
	\Mr{} Rochester, you must neither expect nor exact anything celestial of
	me---for you will not get it, any more than I shall get it of you: which
	I do not at all anticipate.}

\enquote{What do you anticipate of me?}

\enquote{For a little while you will perhaps be as you are now,---a very little
	while; and then you will turn cool; and then you will be capricious; and
	then you will be stern, and I shall have much ado to please you: but
	when you get well used to me, you will perhaps like me
	again,---\emph{like} me, I say, not \emph{love} me.  I suppose your love
	will effervesce in six months, or less.  I have observed in books
	written by men, that period assigned as the farthest to which a
	husband's ardour extends.  Yet, after all, as a friend and companion, I
	hope never to become quite distasteful to my dear master.}

\enquote{Distasteful! and like you again!  I think I shall like you again, and
	yet again: and I will make you confess I do not only \emph{like}, but
	\emph{love} you---with truth, fervour, constancy.}

\enquote{Yet are you not capricious, sir?}

\enquote{To women who please me only by their faces, I am the very devil
	when I find out they have neither souls nor hearts---when they open to
	me a perspective of flatness, triviality, and perhaps imbecility,
	coarseness, and ill-temper: but to the clear eye and eloquent tongue, to
	the soul made of fire, and the character that bends but does not
	break---at once supple and stable, tractable and consistent---I am ever
	tender and true.}

\enquote{Had you ever experience of such a character, sir?  Did you ever
	love such an one?}

\enquote{I love it now.}

\enquote{But before me: if I, indeed, in any respect come up to your
	difficult standard?}

\enquote{I never met your likeness.  Jane, you please me, and you master
	me---you seem to submit, and I like the sense of pliancy you impart; and
	while I am twining the soft, silken skein round my finger, it sends a
	thrill up my arm to my heart.  I am influenced---conquered; and the
	influence is sweeter than I can express; and the conquest I undergo has
	a witchery beyond any triumph I can win.  Why do you smile, Jane?  What
	does that inexplicable, that uncanny turn of countenance mean?}

\enquote{I was thinking, sir (you will excuse the idea; it was
	involuntary), I was thinking of Hercules and Samson with their
	charmers---}

\enquote{You were, you little elfish---}

\enquote{Hush, sir!  You don't talk very wisely just now; any more than
	those gentlemen acted very wisely.  However, had they been married, they
	would no doubt by their severity as husbands have made up for their
	softness as suitors; and so will you, I fear.  I wonder how you will
	answer me a year hence, should I ask a favour it does not suit your
	convenience or pleasure to grant.}

\enquote{Ask me something now, Jane,---the least thing: I desire to be
	entreated---}

\enquote{Indeed I will, sir; I have my petition all ready.}

\enquote{Speak!  But if you look up and smile with that countenance, I
	shall swear concession before I know to what, and that will make a fool
	of me.}

\enquote{Not at all, sir; I ask only this: don't send for the jewels,
	and don't crown me with roses: you might as well put a border of gold
	lace round that plain pocket handkerchief you have there.}

\enquote{I might as well \enquote{gild refined gold.}  I know it: your
	request is granted then---for the time.  I will remand the order I
	despatched to my banker.  But you have not yet asked for anything; you
	have prayed a gift to be withdrawn: try again.}

\enquote{Well then, sir, have the goodness to gratify my curiosity,
	which is much piqued on one point.}

He looked disturbed.  \enquote{What? what?} he said hastily.
\enquote{Curiosity is a dangerous petition: it is well I have not taken
	a vow to accord every request---}

\enquote{But there can be no danger in complying with this, sir.}

\enquote{Utter it, Jane: but I wish that instead of a mere inquiry into,
	perhaps, a secret, it was a wish for half my estate.}

\enquote{Now, King Ahasuerus!  What do I want with half your estate?  Do
	you think I am a Jew-usurer, seeking good investment in land?  I would
	much rather have all your confidence.  You will not exclude me from your
	confidence if you admit me to your heart?}

\enquote{You are welcome to all my confidence that is worth having,
	Jane; but for God's sake, don't desire a useless burden!  Don't long for
	poison---don't turn out a downright Eve on my hands!}

\enquote{Why not, sir?  You have just been telling me how much you liked
	to be conquered, and how pleasant over-persuasion is to you.  Don't you
	think I had better take advantage of the confession, and begin and coax
	and entreat---even cry and be sulky if necessary---for the sake of a
	mere essay of my power?}

\enquote{I dare you to any such experiment.  Encroach, presume, and the
	game is up.}

\enquote{Is it, sir?  You soon give in.  How stern you look now!  Your
	eyebrows have become as thick as my finger, and your forehead resembles
	what, in some very astonishing poetry, I once saw styled, \enquote{a
		blue-piled thunderloft.}  That will be your married look, sir, I
	suppose?}

\enquote{If that will be \emph{your} married look, I, as a Christian, will soon
	give up the notion of consorting with a mere sprite or salamander.  But
	what had you to ask, thing,---out with it?}

\enquote{There, you are less than civil now; and I like rudeness a great deal
	better than flattery.  I had rather be a \emph{thing} than an angel.
	This is what I have to ask,---Why did you take such pains to make me
	believe you wished to marry Miss Ingram?}

\enquote{Is that all?  Thank God it is no worse!}  And now he unknit his
black brows; looked down, smiling at me, and stroked my hair, as if well
pleased at seeing a danger averted.  \enquote{I think I may confess,} he
continued, \enquote{even although I should make you a little indignant,
	Jane---and I have seen what a fire-spirit you can be when you are
	indignant.  You glowed in the cool moonlight last night, when you
	mutinied against fate, and claimed your rank as my equal.  Janet,
	by-the-bye, it was you who made me the offer.}

\enquote{Of course I did.  But to the point if you please, sir---Miss
	Ingram?}

\enquote{Well, I feigned courtship of Miss Ingram, because I wished to
	render you as madly in love with me as I was with you; and I knew
	jealousy would be the best ally I could call in for the furtherance of
	that end.}

\enquote{Excellent!  Now you are small---not one whit bigger than the
	end of my little finger.  It was a burning shame and a scandalous
	disgrace to act in that way.  Did you think nothing of Miss Ingram's
	feelings, sir?}

\enquote{Her feelings are concentrated in one---pride; and that needs
	humbling.  Were you jealous, Jane?}

\enquote{Never mind, \Mr{} Rochester: it is in no way interesting to you
	to know that.  Answer me truly once more.  Do you think Miss Ingram will
	not suffer from your dishonest coquetry?  Won't she feel forsaken and
	deserted?}

\enquote{Impossible!---when I told you how she, on the contrary,
	deserted me: the idea of my insolvency cooled, or rather extinguished,
	her flame in a moment.}

\enquote{You have a curious, designing mind, \Mr{} Rochester.  I am afraid
	your principles on some points are eccentric.}

\enquote{My principles were never trained, Jane: they may have grown a
	little awry for want of attention.}

\enquote{Once again, seriously; may I enjoy the great good that has been
	vouchsafed to me, without fearing that any one else is suffering the
	bitter pain I myself felt a while ago?}

\enquote{That you may, my good little girl: there is not another being
	in the world has the same pure love for me as yourself---for I lay that
	pleasant unction to my soul, Jane, a belief in your affection.}

I turned my lips to the hand that lay on my shoulder.  I loved him very
much---more than I could trust myself to say---more than words had power
to express.

\enquote{Ask something more,} he said presently; \enquote{it is my
	delight to be entreated, and to yield.}

I was again ready with my request.  \enquote{Communicate your intentions
	to \Mrs{} Fairfax, sir: she saw me with you last night in the hall, and
	she was shocked.  Give her some explanation before I see her again.  It
	pains me to be misjudged by so good a woman.}

\enquote{Go to your room, and put on your bonnet,} he replied.
\enquote{I mean you to accompany me to Millcote this morning; and while
	you prepare for the drive, I will enlighten the old lady's
	understanding.  Did she think, Janet, you had given the world for love,
	and considered it well lost?}

\enquote{I believe she thought I had forgotten my station, and yours,
	sir.}

\enquote{Station! station!---your station is in my heart, and on the
	necks of those who would insult you, now or hereafter.---Go.}

I was soon dressed; and when I heard \Mr{} Rochester quit \Mrs{} Fairfax's
parlour, I hurried down to it.  The old lady, had been reading her
morning portion of Scripture---the Lesson for the day; her Bible lay
open before her, and her spectacles were upon it.  Her occupation,
suspended by \Mr{} Rochester's announcement, seemed now forgotten: her
eyes, fixed on the blank wall opposite, expressed the surprise of a
quiet mind stirred by unwonted tidings.  Seeing me, she roused herself:
she made a sort of effort to smile, and framed a few words of
congratulation; but the smile expired, and the sentence was abandoned
unfinished.  She put up her spectacles, shut the Bible, and pushed her
chair back from the table.

\enquote{I feel so astonished,} she began, \enquote{I hardly know what
	to say to you, Miss Eyre.  I have surely not been dreaming, have I?
	Sometimes I half fall asleep when I am sitting alone and fancy things
	that have never happened.  It has seemed to me more than once when I
	have been in a doze, that my dear husband, who died fifteen years since,
	has come in and sat down beside me; and that I have even heard him call
	me by my name, Alice, as he used to do.  Now, can you tell me whether it
	is actually true that \Mr{} Rochester has asked you to marry him?  Don't
	laugh at me.  But I really thought he came in here five minutes ago, and
	said that in a month you would be his wife.}

\enquote{He has said the same thing to me,} I replied.

\enquote{He has!  Do you believe him?  Have you accepted him?}

\enquote{Yes.}

She looked at me bewildered.  \enquote{I could never have thought it.
	He is a proud man: all the Rochesters were proud: and his father, at
	least, liked money.  He, too, has always been called careful.  He means
	to marry you?}

\enquote{He tells me so.}

She surveyed my whole person: in her eyes I read that they had there
found no charm powerful enough to solve the enigma.

\enquote{It passes me!} she continued; \enquote{but no doubt, it is true
	since you say so.  How it will answer, I cannot tell: I really don't
	know.  Equality of position and fortune is often advisable in such
	cases; and there are twenty years of difference in your ages.  He might
	almost be your father.}

\enquote{No, indeed, \Mrs{} Fairfax!} exclaimed I, nettled; \enquote{he is
	nothing like my father!  No one, who saw us together, would suppose it
	for an instant.  \Mr{} Rochester looks as young, and is as young, as some
	men at five-and-twenty.}

\enquote{Is it really for love he is going to marry you?} she asked.

I was so hurt by her coldness and scepticism, that the tears rose to my
eyes.

\enquote{I am sorry to grieve you,} pursued the widow; \enquote{but you
	are so young, and so little acquainted with men, I wished to put you on
	your guard.  It is an old saying that \enquote{all is not gold that
		glitters;} and in this case I do fear there will be something found to
	be different to what either you or I expect.}

\enquote{Why?---am I a monster?} I said: \enquote{is it impossible that
	\Mr{} Rochester should have a sincere affection for me?}

\enquote{No: you are very well; and much improved of late; and \Mr{}
	Rochester, I daresay, is fond of you.  I have always noticed that you
	were a sort of pet of his.  There are times when, for your sake, I have
	been a little uneasy at his marked preference, and have wished to put
	you on your guard: but I did not like to suggest even the possibility of
	wrong.  I knew such an idea would shock, perhaps offend you; and you
	were so discreet, and so thoroughly modest and sensible, I hoped you
	might be trusted to protect yourself.  Last night I cannot tell you what
	I suffered when I sought all over the house, and could find you nowhere,
	nor the master either; and then, at twelve o'clock, saw you come in with
	him.}

\enquote{Well, never mind that now,} I interrupted impatiently;
\enquote{it is enough that all was right.}

\enquote{I hope all will be right in the end,} she said: \enquote{but
	believe me, you cannot be too careful.  Try and keep \Mr{} Rochester at a
	distance: distrust yourself as well as him.  Gentlemen in his station
	are not accustomed to marry their governesses.}

I was growing truly irritated: happily, Adèle ran in.

\enquote{Let me go,---let me go to Millcote too!} she cried.
\enquote{\Mr{} Rochester won't: though there is so much room in the new
	carriage.  Beg him to let me go mademoiselle.}

\enquote{That I will, Adèle;} and I hastened away with her, glad to quit
my gloomy monitress.  The carriage was ready: they were bringing it
round to the front, and my master was pacing the pavement, Pilot
following him backwards and forwards.

\enquote{Adèle may accompany us, may she not, sir?}

\enquote{I told her no.  I'll have no brats!---I'll have only you.}

\enquote{Do let her go, \Mr{} Rochester, if you please: it would be
	better.}

\enquote{Not it: she will be a restraint.}

He was quite peremptory, both in look and voice.  The chill of \Mrs{}
Fairfax's warnings, and the damp of her doubts were upon me: something
of unsubstantiality and uncertainty had beset my hopes.  I half lost the
sense of power over him.  I was about mechanically to obey him, without
further remonstrance; but as he helped me into the carriage, he looked
at my face.

\enquote{What is the matter?} he asked; \enquote{all the sunshine is
	gone.  Do you really wish the bairn to go?  Will it annoy you if she is
	left behind?}

\enquote{I would far rather she went, sir.}

\enquote{Then off for your bonnet, and back like a flash of lightning!}
cried he to Adèle.

She obeyed him with what speed she might.

\enquote{After all, a single morning's interruption will not matter
	much,} said he, \enquote{when I mean shortly to claim you---your
	thoughts, conversation, and company---for life.}

Adèle, when lifted in, commenced kissing me, by way of expressing her
gratitude for my intercession: she was instantly stowed away into a
corner on the other side of him.  She then peeped round to where I sat;
so stern a neighbour was too restrictive to him, in his present
fractious mood, she dared whisper no observations, nor ask of him any
information.

\enquote{Let her come to me,} I entreated: \enquote{she will, perhaps,
	trouble you, sir: there is plenty of room on this side.}

He handed her over as if she had been a lapdog.  \enquote{I'll send her
	to school yet,} he said, but now he was smiling.

Adèle heard him, and asked if she was to go to school \enquote{sans
	mademoiselle?}

\enquote{Yes,} he replied, \enquote{absolutely sans mademoiselle; for I
	am to take mademoiselle to the moon, and there I shall seek a cave in
	one of the white valleys among the volcano-tops, and mademoiselle shall
	live with me there, and only me.}

\enquote{She will have nothing to eat: you will starve her,} observed
Adèle.

\enquote{I shall gather manna for her morning and night: the plains and
	hillsides in the moon are bleached with manna, Adèle.}

\enquote{She will want to warm herself: what will she do for a fire?}

\enquote{Fire rises out of the lunar mountains: when she is cold, I'll
	carry her up to a peak, and lay her down on the edge of a crater.}

\enquote{Oh,} \foreignquote{french}{qu' elle y sera mal---peu comfortable!}\footnote{
	\enquote{How bad that would be for her---how uncomfortable!}}
\enquote{And her clothes, they will wear out: how can she get new ones?}

\Mr{} Rochester professed to be puzzled.  \enquote{Hem!} said he.
\enquote{What would you do, Adèle?  Cudgel your brains for an
	expedient.  How would a white or a pink cloud answer for a gown, do you
	think?  And one could cut a pretty enough scarf out of a rainbow.}

\enquote{She is far better as she is,} concluded Adèle, after musing
some time: \enquote{besides, she would get tired of living with only you
	in the moon.  If I were mademoiselle, I would never consent to go with
	you.}

\enquote{She has consented: she has pledged her word.}

\enquote{But you can't get her there; there is no road to the moon: it
	is all air; and neither you nor she can fly.}

\enquote{Adèle, look at that field.}  We were now outside Thornfield
gates, and bowling lightly along the smooth road to Millcote, where the
dust was well laid by the thunderstorm, and, where the low hedges and
lofty timber trees on each side glistened green and rain-refreshed.

\enquote{In that field, Adèle, I was walking late one evening about a fortnight
	since---the evening of the day you helped me to make hay in the orchard
	meadows; and, as I was tired with raking swaths, I sat down to rest me
	on a stile; and there I took out a little book and a pencil, and began
	to write about a misfortune that befell me long ago, and a wish I had
	for happy days to come: I was writing away very fast, though daylight
	was fading from the leaf, when something came up the path and stopped
	two yards off me.  I looked at it.  It was a little thing with a veil of
	gossamer on its head.  I beckoned it to come near me; it stood soon at
	my knee.  I never spoke to it, and it never spoke to me, in words; but I
	read its eyes, and it read mine; and our speechless colloquy was to this
	effect---

	%rem enq
	It was a fairy, and come from Elf-land, it said; and its errand was to
	make me happy: I must go with it out of the common world to a lonely
	place---such as the moon, for instance---and it nodded its head towards
	her horn, rising over Hay-hill: it told me of the alabaster cave and
	silver vale where we might live.  I said I should like to go; but
	reminded it, as you did me, that I had no wings to fly.

	%rem enq
	\enquote{Oh,} returned the fairy, \enquote{that does not
		signify!  Here is a talisman will remove all difficulties;} and she held
	out a pretty gold ring.  \enquote{Put it,} she said, \enquote{on the
		fourth finger of my left hand, and I am yours, and you are mine; and we
		shall leave earth, and make our own heaven yonder.}  She nodded again at
	the moon.  The ring, Adèle, is in my breeches-pocket, under the disguise
	of a sovereign: but I mean soon to change it to a ring again.}

\enquote{But what has mademoiselle to do with it?  I don't care for the
	fairy: you said it was mademoiselle you would take to the moon?}

\enquote{Mademoiselle is a fairy,} he said, whispering mysteriously.
Whereupon I told her not to mind his badinage; and she, on her part,
evinced a fund of genuine French scepticism: denominating \Mr{} Rochester
\foreignquote{french}{un vrai menteur,}\footnote{\enquote{a real liar}} and assuring him that she made no account
whatever of his \foreignquote{french}{contes de fée,}\footnote{\enquote{fairy tales}} and that \foreignquote{french}{du reste, il
	n'y avait pas de fées, et quand même il y en avait:}\footnote{
	\enquote{besides, there are no fairies, and even if there were}} she was sure they
would never appear to him, nor ever give him rings, or offer to live
with him in the moon.

The hour spent at Millcote was a somewhat harassing one to me.  \Mr{}
Rochester obliged me to go to a certain silk warehouse: there I was
ordered to choose half-a-dozen dresses.  I hated the business, I begged
leave to defer it: no---it should be gone through with now.  By dint of
entreaties expressed in energetic whispers, I reduced the half-dozen to
two: these however, he vowed he would select himself.  With anxiety I
watched his eye rove over the gay stores: he fixed on a rich silk of the
most brilliant amethyst dye, and a superb pink satin.  I told him in a
new series of whispers, that he might as well buy me a gold gown and a
silver bonnet at once: I should certainly never venture to wear his
choice.  With infinite difficulty, for he was stubborn as a stone, I
persuaded him to make an exchange in favour of a sober black satin and
pearl-grey silk.  \enquote{It might pass for the present,} he said;
\enquote{but he would yet see me glittering like a parterre.}

Glad was I to get him out of the silk warehouse, and then out of a
jewellers shop: the more he bought me, the more my cheek burned with a
sense of annoyance and degradation.  As we re-entered the carriage, and
I sat back feverish and fagged, I remembered what, in the hurry of
events, dark and bright, I had wholly forgotten---the letter of my
uncle, John Eyre, to \Mrs{} Reed: his intention to adopt me and make me
his legatee.  \enquote{It would, indeed, be a relief,} I thought,
\enquote{if I had ever so small an independency; I never can bear being
	dressed like a doll by \Mr{} Rochester, or sitting like a second Danae
	with the golden shower falling daily round me.  I will write to Madeira
	the moment I get home, and tell my uncle John I am going to be married,
	and to whom: if I had but a prospect of one day bringing \Mr{} Rochester
	an accession of fortune, I could better endure to be kept by him now.}
And somewhat relieved by this idea (which I failed not to execute that
day), I ventured once more to meet my master's and lover's eye, which
most pertinaciously sought mine, though I averted both face and gaze.
He smiled; and I thought his smile was such as a sultan might, in a
blissful and fond moment, bestow on a slave his gold and gems had
enriched: I crushed his hand, which was ever hunting mine, vigorously,
and thrust it back to him red with the passionate pressure.

\enquote{You need not look in that way,} I said; \enquote{if you do,
	I'll wear nothing but my old Lowood frocks to the end of the chapter.
	I'll be married in this lilac gingham: you may make a dressing-gown for
	yourself out of the pearl-grey silk, and an infinite series of
	waistcoats out of the black satin.}

He chuckled; he rubbed his hands.  \enquote{Oh, it is rich to see and
	hear her?} he exclaimed.  \enquote{Is she original?  Is she piquant?  I
	would not exchange this one little English girl for the Grand Turk's
	whole seraglio, gazelle-eyes, houri forms, and all!}

The Eastern allusion bit me again.  \enquote{I'll not stand you an inch
	in the stead of a seraglio,} I said; \enquote{so don't consider me an
	equivalent for one.  If you have a fancy for anything in that line, away
	with you, sir, to the bazaars of Stamboul without delay, and lay out in
	extensive slave-purchases some of that spare cash you seem at a loss to
	spend satisfactorily here.}

\enquote{And what will you do, Janet, while I am bargaining for so many
	tons of flesh and such an assortment of black eyes?}

\enquote{I'll be preparing myself to go out as a missionary to preach
	liberty to them that are enslaved---your harem inmates amongst the
	rest.  I'll get admitted there, and I'll stir up mutiny; and you,
	three-tailed bashaw as you are, sir, shall in a trice find yourself
	fettered amongst our hands: nor will I, for one, consent to cut your
	bonds till you have signed a charter, the most liberal that despot ever
	yet conferred.}

\enquote{I would consent to be at your mercy, Jane.}

\enquote{I would have no mercy, \Mr{} Rochester, if you supplicated for it
	with an eye like that.  While you looked so, I should be certain that
	whatever charter you might grant under coercion, your first act, when
	released, would be to violate its conditions.}

\enquote{Why, Jane, what would you have?  I fear you will compel me to
	go through a private marriage ceremony, besides that performed at the
	altar.  You will stipulate, I see, for peculiar terms---what will they
	be?}

\enquote{I only want an easy mind, sir; not crushed by crowded
	obligations.  Do you remember what you said of Céline Varens?---of the
	diamonds, the cashmeres you gave her?  I will not be your English Céline
	Varens.  I shall continue to act as Adèle's governess; by that I shall
	earn my board and lodging, and thirty pounds a year besides.  I'll
	furnish my own wardrobe out of that money, and you shall give me nothing
	but---}

\enquote{Well, but what?}

\enquote{Your regard; and if I give you mine in return, that debt will
	be quit.}

\enquote{Well, for cool native impudence and pure innate pride, you
	haven't your equal,} said he.  We were now approaching Thornfield.
\enquote{Will it please you to dine with me to-day?} he asked, as we
re-entered the gates.

\enquote{No, thank you, sir.}

\enquote{And what for, \enquote{no, thank you?} if one may inquire.}

\enquote{I never have dined with you, sir: and I see no reason why I
	should now: till---}

\enquote{Till what?  You delight in half-phrases.}

\enquote{Till I can't help it.}

\enquote{Do you suppose I eat like an ogre or a ghoul, that you dread
	being the companion of my repast?}

\enquote{I have formed no supposition on the subject, sir; but I want to
	go on as usual for another month.}

\enquote{You will give up your governessing slavery at once.}

\enquote{Indeed, begging your pardon, sir, I shall not.  I shall just go
	on with it as usual.  I shall keep out of your way all day, as I have
	been accustomed to do: you may send for me in the evening, when you feel
	disposed to see me, and I'll come then; but at no other time.}

\enquote{I want a smoke, Jane, or a pinch of snuff, to comfort me under
	all this, \foreignquote{french}{pour me donner une contenance,}\footnote{
		\enquote{to restore my composure}} as Adèle would say;
	and unfortunately I have neither my cigar-case, nor my snuff-box.  But
	listen---whisper.  It is your time now, little tyrant, but it will be
	mine presently; and when once I have fairly seized you, to have and to
	hold, I'll just---figuratively speaking---attach you to a chain like
	this} (touching his watch-guard).  \enquote{Yes, bonny wee thing, I'll
	wear you in my bosom, lest my jewel I should tyne.}

He said this as he helped me to alight from the carriage, and while he
afterwards lifted out Adèle, I entered the house, and made good my
retreat upstairs.

He duly summoned me to his presence in the evening.  I had prepared an
occupation for him; for I was determined not to spend the whole time in
a \foreignlanguage{french}{\emph{tête-à-tête}} conversation.  I remembered his fine voice; I knew
he liked to sing---good singers generally do.  I was no vocalist myself,
and, in his fastidious judgment, no musician, either; but I delighted in
listening when the performance was good.  No sooner had twilight, that
hour of romance, began to lower her blue and starry banner over the
lattice, than I rose, opened the piano, and entreated him, for the love
of heaven, to give me a song.  He said I was a capricious witch, and
that he would rather sing another time; but I averred that no time was
like the present.

\enquote{Did I like his voice?} he asked.

\enquote{Very much.}  I was not fond of pampering that susceptible
vanity of his; but for once, and from motives of expediency, I would
e'en soothe and stimulate it.

\enquote{Then, Jane, you must play the accompaniment.}

\enquote{Very well, sir, I will try.}

I did try, but was presently swept off the stool and denominated
\enquote{a little bungler.}  Being pushed unceremoniously to one
side---which was precisely what I wished---he usurped my place, and
proceeded to accompany himself: for he could play as well as sing.  I
hied me to the window-recess.  And while I sat there and looked out on
the still trees and dim lawn, to a sweet air was sung in mellow tones
the following strain:---

\settoversewidth{\versewidth}{Though haughty Hate should strike me down,}
\begin{verse}[\versewidth]
	\enquote{The truest love that ever heart\\*
		\hspace*{0.333em}\hspace*{0.333em} Felt at its kindled core,\\*
		Did through each vein, in quickened start,\\*
		\hspace*{0.333em}\hspace*{0.333em} The tide of being pour.

		Her coming was my hope each day,\\*
		\hspace*{0.333em}\hspace*{0.333em} Her parting was my pain;\\*
		The chance that did her steps delay\\*
		\hspace*{0.333em}\hspace*{0.333em} Was ice in every vein.

		I dreamed it would be nameless bliss,\\*
		\hspace*{0.333em}\hspace*{0.333em} As I loved, loved to be;\\*
		And to this object did I press\\*
		\hspace*{0.333em}\hspace*{0.333em} As blind as eagerly.

		But wide as pathless was the space\\*
		\hspace*{0.333em}\hspace*{0.333em} That lay our lives between,\\*
		And dangerous as the foamy race\\*
		\hspace*{0.333em}\hspace*{0.333em} Of ocean-surges green.

		And haunted as a robber-path\\*
		\hspace*{0.333em}\hspace*{0.333em} Through wilderness or wood;\\*
		For Might and Right, and Woe and Wrath,\\*
		\hspace*{0.333em}\hspace*{0.333em} Between our spirits stood.

		I dangers dared; I hindrance scorned;\\*
		\hspace*{0.333em}\hspace*{0.333em} I omens did defy:\\*
		Whatever menaced, harassed, warned,\\*
		\hspace*{0.333em}\hspace*{0.333em} I passed impetuous by.

		On sped my rainbow, fast as light;\\*
		\hspace*{0.333em}\hspace*{0.333em} I flew as in a dream;\\*
		For glorious rose upon my sight\\*
		\hspace*{0.333em}\hspace*{0.333em} That child of Shower and Gleam.

		Still bright on clouds of suffering dim\\*
		\hspace*{0.333em}\hspace*{0.333em} Shines that soft, solemn joy;\\*
		Nor care I now, how dense and grim\\*
		\hspace*{0.333em}\hspace*{0.333em} Disasters gather nigh.

		I care not in this moment sweet,\\
		\hspace*{0.333em}\hspace*{0.333em} Though all I have rushed o'er\\*
		Should come on pinion, strong and fleet,\\*
		\hspace*{0.333em}\hspace*{0.333em} Proclaiming vengeance sore:

		Though haughty Hate should strike me down,\\*
		\hspace*{0.333em}\hspace*{0.333em} Right, bar approach to me,\\*
		And grinding Might, with furious frown,\\*
		\hspace*{0.333em}\hspace*{0.333em} Swear endless enmity.

		My love has placed her little hand\\*
		\hspace*{0.333em}\hspace*{0.333em} With noble faith in mine,\\*
		And vowed that wedlock's sacred band\\*
		\hspace*{0.333em}\hspace*{0.333em} Our nature shall entwine.

		My love has sworn, with sealing kiss,\\*
		\hspace*{0.333em}\hspace*{0.333em} With me to live---to die;\\*
		I have at last my nameless bliss.\\*
		\hspace*{0.333em}\hspace*{0.333em} As I love---loved am I!}
\end{verse}

He rose and came towards me, and I saw his face all kindled, and his
full falcon-eye flashing, and tenderness and passion in every
lineament.  I quailed momentarily---then I rallied.  Soft scene, daring
demonstration, I would not have; and I stood in peril of both: a weapon
of defence must be prepared---I whetted my tongue: as he reached me, I
asked with asperity, \enquote{whom he was going to marry now?}

\enquote{That was a strange question to be put by his darling Jane.}

\enquote{Indeed!  I considered it a very natural and necessary one: he had
	talked of his future wife dying with him.  What did he mean by such a
	pagan idea?  \emph{I} had no intention of dying with him---he might
	depend on that.}

\enquote{Oh, all he longed, all he prayed for, was that I might live
	with him!  Death was not for such as I\@.}

\enquote{Indeed it was: I had as good a right to die when my time came
	as he had: but I should bide that time, and not be hurried away in a
	suttee.}

\enquote{Would I forgive him for the selfish idea, and prove my pardon
	by a reconciling kiss?}

\enquote{No: I would rather be excused.}

Here I heard myself apostrophised as a \enquote{hard little thing;} and
it was added, \enquote{any other woman would have been melted to marrow
	at hearing such stanzas crooned in her praise.}

I assured him I was naturally hard---very flinty, and that he would
often find me so; and that, moreover, I was determined to show him
divers rugged points in my character before the ensuing four weeks
elapsed: he should know fully what sort of a bargain he had made, while
there was yet time to rescind it.

\enquote{Would I be quiet and talk rationally?}

\enquote{I would be quiet if he liked, and as to talking rationally, I
	flattered myself I was doing that now.}

He fretted, pished, and pshawed.  \enquote{Very good,} I thought;
\enquote{you may fume and fidget as you please: but this is the best
	plan to pursue with you, I am certain.  I like you more than I can say;
	but I'll not sink into a bathos of sentiment: and with this needle of
	repartee I'll keep you from the edge of the gulf too; and, moreover,
	maintain by its pungent aid that distance between you and myself most
	conducive to our real mutual advantage.}

From less to more, I worked him up to considerable irritation; then,
after he had retired, in dudgeon, quite to the other end of the room, I
got up, and saying, \enquote{I wish you good-night, sir,} in my natural
and wonted respectful manner, I slipped out by the side-door and got
away.

The system thus entered on, I pursued during the whole season of
probation; and with the best success.  He was kept, to be sure, rather
cross and crusty; but on the whole I could see he was excellently
entertained, and that a lamb-like submission and turtle-dove
sensibility, while fostering his despotism more, would have pleased his
judgment, satisfied his common-sense, and even suited his taste less.

In other people's presence I was, as formerly, deferential and quiet;
any other line of conduct being uncalled for: it was only in the evening
conferences I thus thwarted and afflicted him.  He continued to send for
me punctually the moment the clock struck seven; though when I appeared
before him now, he had no such honeyed terms as \enquote{love} and
\enquote{darling} on his lips: the best words at my service were
\enquote{provoking puppet,} \enquote{malicious elf,} \enquote{sprite,}
\enquote{changeling,} \etc.  For caresses, too, I now got grimaces; for a
pressure of the hand, a pinch on the arm; for a kiss on the cheek, a
severe tweak of the ear.  It was all right: at present I decidedly
preferred these fierce favours to anything more tender.  \Mrs{} Fairfax, I
saw, approved me: her anxiety on my account vanished; therefore I was
certain I did well.  Meantime, \Mr{} Rochester affirmed I was wearing him
to skin and bone, and threatened awful vengeance for my present conduct
at some period fast coming.  I laughed in my sleeve at his menaces.
\enquote{I can keep you in reasonable check now,} I reflected;
\enquote{and I don't doubt to be able to do it hereafter: if one
	expedient loses its virtue, another must be devised.}

Yet after all my task was not an easy one; often I would rather have
pleased than teased him.  My future husband was becoming to me my whole
world; and more than the world: almost my hope of heaven.  He stood
between me and every thought of religion, as an eclipse intervenes
between man and the broad sun.  I could not, in those days, see God for
His creature: of whom I had made an idol.

\FChapter{Chapter Two}{2}

\Lettrine{I}{} \textsc{resisted} all the way: a new thing for me, and a circumstance which
greatly strengthened the bad opinion Bessie and Miss Abbot were disposed
to entertain of me.  The fact is, I was a trifle beside myself; or
rather \emph{out} of myself, as the French would say: I was conscious
that a moment's mutiny had already rendered me liable to strange
penalties, and, like any other rebel slave, I felt resolved, in my
desperation, to go all lengths.

\enquote{Hold her arms, Miss Abbot: she's like a mad cat.}

\enquote{For shame! for shame!} cried the lady's-maid.  \enquote{What
shocking conduct, Miss Eyre, to strike a young gentleman, your
benefactress's son!  Your young master.}

\enquote{Master!  How is he my master?  Am I a servant?}

\enquote{No; you are less than a servant, for you do nothing for your
keep.  There, sit down, and think over your wickedness.}

They had got me by this time into the apartment indicated by \Mrs{} Reed,
and had thrust me upon a stool: my impulse was to rise from it like a
spring; their two pair of hands arrested me instantly.

\enquote{If you don't sit still, you must be tied down,} said Bessie. 
\enquote{Miss Abbot, lend me your garters; she would break mine
directly.}

Miss Abbot turned to divest a stout leg of the necessary ligature.  This
preparation for bonds, and the additional ignominy it inferred, took a
little of the excitement out of me.

\enquote{Don't take them off,} I cried; \enquote{I will not stir.}

In guarantee whereof, I attached myself to my seat by my hands.

\enquote{Mind you don't,} said Bessie; and when she had ascertained that
I was really subsiding, she loosened her hold of me; then she and Miss
Abbot stood with folded arms, looking darkly and doubtfully on my face,
as incredulous of my sanity.

\enquote{She never did so before,} at last said Bessie, turning to the
Abigail.

\enquote{But it was always in her,} was the reply.  \enquote{I've told
Missis often my opinion about the child, and Missis agreed with me. 
She's an underhand little thing: I never saw a girl of her age with so
much cover.}

Bessie answered not; but ere long, addressing me, she
said---\enquote{You ought to be aware, Miss, that you are under
obligations to \Mrs{} Reed: she keeps you: if she were to turn you off,
you would have to go to the poorhouse.}

I had nothing to say to these words: they were not new to me: my very
first recollections of existence included hints of the same kind.  This
reproach of my dependence had become a vague sing-song in my ear: very
painful and crushing, but only half intelligible.  Miss Abbot joined
in---

\enquote{And you ought not to think yourself on an equality with the
Misses Reed and Master Reed, because Missis kindly allows you to be
brought up with them.  They will have a great deal of money, and you
will have none: it is your place to be humble, and to try to make
yourself agreeable to them.}

\enquote{What we tell you is for your good,} added Bessie, in no harsh
voice, \enquote{you should try to be useful and pleasant, then, perhaps,
you would have a home here; but if you become passionate and rude,
Missis will send you away, I am sure.}

\enquote{Besides,} said Miss Abbot, \enquote{God will punish her: He
might strike her dead in the midst of her tantrums, and then where would
she go?  Come, Bessie, we will leave her: I wouldn't have her heart for
anything.  Say your prayers, Miss Eyre, when you are by yourself; for if
you don't repent, something bad might be permitted to come down the
chimney and fetch you away.}

They went, shutting the door, and locking it behind them.

The red-room was a square chamber, very seldom slept in, I might say
never, indeed, unless when a chance influx of visitors at Gateshead Hall
rendered it necessary to turn to account all the accommodation it
contained: yet it was one of the largest and stateliest chambers in the
mansion.  A bed supported on massive pillars of mahogany, hung with
curtains of deep red damask, stood out like a tabernacle in the centre;
the two large windows, with their blinds always drawn down, were half
shrouded in festoons and falls of similar drapery; the carpet was red;
the table at the foot of the bed was covered with a crimson cloth; the
walls were a soft fawn colour with a blush of pink in it; the wardrobe,
the toilet-table, the chairs were of darkly polished old mahogany.  Out
of these deep surrounding shades rose high, and glared white, the
piled-up mattresses and pillows of the bed, spread with a snowy
Marseilles counterpane.  Scarcely less prominent was an ample cushioned
easy-chair near the head of the bed, also white, with a footstool before
it; and looking, as I thought, like a pale throne.

This room was chill, because it seldom had a fire; it was silent,
because remote from the nursery and kitchen; solemn, because it was
known to be so seldom entered.  The house-maid alone came here on
Saturdays, to wipe from the mirrors and the furniture a week's quiet
dust: and \Mrs{} Reed herself, at far intervals, visited it to review the
contents of a certain secret drawer in the wardrobe, where were stored
divers parchments, her jewel-casket, and a miniature of her deceased
husband; and in those last words lies the secret of the red-room---the
spell which kept it so lonely in spite of its grandeur.

\Mr{} Reed had been dead nine years: it was in this chamber he breathed
his last; here he lay in state; hence his coffin was borne by the
undertaker's men; and, since that day, a sense of dreary consecration
had guarded it from frequent intrusion.

My seat, to which Bessie and the bitter Miss Abbot had left me riveted,
was a low ottoman near the marble chimney-piece; the bed rose before me;
to my right hand there was the high, dark wardrobe, with subdued, broken
reflections varying the gloss of its panels; to my left were the muffled
windows; a great looking-glass between them repeated the vacant majesty
of the bed and room.  I was not quite sure whether they had locked the
door; and when I dared move, I got up and went to see.  Alas! yes: no
jail was ever more secure.  Returning, I had to cross before the
looking-glass; my fascinated glance involuntarily explored the depth it
revealed.  All looked colder and darker in that visionary hollow than in
reality: and the strange little figure there gazing at me, with a white
face and arms specking the gloom, and glittering eyes of fear moving
where all else was still, had the effect of a real spirit: I thought it
like one of the tiny phantoms, half fairy, half imp, Bessie's evening
stories represented as coming out of lone, ferny dells in moors, and
appearing before the eyes of belated travellers.  I returned to my
stool.

Superstition was with me at that moment; but it was not yet her hour for
complete victory: my blood was still warm; the mood of the revolted
slave was still bracing me with its bitter vigour; I had to stem a rapid
rush of retrospective thought before I quailed to the dismal present.

All John Reed's violent tyrannies, all his sisters' proud indifference,
all his mother's aversion, all the servants' partiality, turned up in my
disturbed mind like a dark deposit in a turbid well.  Why was I always
suffering, always browbeaten, always accused, for ever condemned?  Why
could I never please?  Why was it useless to try to win any one's
favour?  Eliza, who was headstrong and selfish, was respected. 
Georgiana, who had a spoiled temper, a very acrid spite, a captious and
insolent carriage, was universally indulged.  Her beauty, her pink
cheeks and golden curls, seemed to give delight to all who looked at
her, and to purchase indemnity for every fault.  John no one thwarted,
much less punished; though he twisted the necks of the pigeons, killed
the little pea-chicks, set the dogs at the sheep, stripped the hothouse
vines of their fruit, and broke the buds off the choicest plants in the
conservatory: he called his mother \enquote{old girl,} too; sometimes
reviled her for her dark skin, similar to his own; bluntly disregarded
her wishes; not unfrequently tore and spoiled her silk attire; and he
was still \enquote{her own darling.}  I dared commit no fault: I strove
to fulfil every duty; and I was termed naughty and tiresome, sullen and
sneaking, from morning to noon, and from noon to night.

My head still ached and bled with the blow and fall I had received: no
one had reproved John for wantonly striking me; and because I had turned
against him to avert farther irrational violence, I was loaded with
general opprobrium.

\enquote{Unjust!---unjust!} said my reason, forced by the agonising
stimulus into precocious though transitory power: and Resolve, equally
wrought up, instigated some strange expedient to achieve escape from
insupportable oppression---as running away, or, if that could not be
effected, never eating or drinking more, and letting myself die.

What a consternation of soul was mine that dreary afternoon!  How all my
brain was in tumult, and all my heart in insurrection!  Yet in what
darkness, what dense ignorance, was the mental battle fought!  I could
not answer the ceaseless inward question---\emph{why} I thus suffered;
now, at the distance of---I will not say how many years, I see it
clearly.

I was a discord in Gateshead Hall: I was like nobody there; I had
nothing in harmony with \Mrs{} Reed or her children, or her chosen
vassalage.  If they did not love me, in fact, as little did I love
them.  They were not bound to regard with affection a thing that could
not sympathise with one amongst them; a heterogeneous thing, opposed to
them in temperament, in capacity, in propensities; a useless thing,
incapable of serving their interest, or adding to their pleasure; a
noxious thing, cherishing the germs of indignation at their treatment,
of contempt of their judgment.  I know that had I been a sanguine,
brilliant, careless, exacting, handsome, romping child---though equally
dependent and friendless---\Mrs{} Reed would have endured my presence more
complacently; her children would have entertained for me more of the
cordiality of fellow-feeling; the servants would have been less prone to
make me the scapegoat of the nursery.

Daylight began to forsake the red-room; it was past four o'clock, and
the beclouded afternoon was tending to drear twilight.  I heard the rain
still beating continuously on the staircase window, and the wind howling
in the grove behind the hall; I grew by degrees cold as a stone, and
then my courage sank.  My habitual mood of humiliation, self-doubt,
forlorn depression, fell damp on the embers of my decaying ire.  All
said I was wicked, and perhaps I might be so; what thought had I been
but just conceiving of starving myself to death?  That certainly was a
crime: and was I fit to die?  Or was the vault under the chancel of
Gateshead Church an inviting bourne?  In such vault I had been told did
\Mr{} Reed lie buried; and led by this thought to recall his idea, I dwelt
on it with gathering dread.  I could not remember him; but I knew that
he was my own uncle---my mother's brother---that he had taken me when a
parentless infant to his house; and that in his last moments he had
required a promise of \Mrs{} Reed that she would rear and maintain me as
one of her own children.  \Mrs{} Reed probably considered she had kept
this promise; and so she had, I dare say, as well as her nature would
permit her; but how could she really like an interloper not of her race,
and unconnected with her, after her husband's death, by any tie?  It
must have been most irksome to find herself bound by a hard-wrung pledge
to stand in the stead of a parent to a strange child she could not love,
and to see an uncongenial alien permanently intruded on her own family
group.

A singular notion dawned upon me.  I doubted not---never doubted---that
if \Mr{} Reed had been alive he would have treated me kindly; and now, as
I sat looking at the white bed and overshadowed walls---occasionally
also turning a fascinated eye towards the dimly gleaming mirror---I
began to recall what I had heard of dead men, troubled in their graves
by the violation of their last wishes, revisiting the earth to punish
the perjured and avenge the oppressed; and I thought \Mr{} Reed's spirit,
harassed by the wrongs of his sister's child, might quit its
abode---whether in the church vault or in the unknown world of the
departed---and rise before me in this chamber.  I wiped my tears and
hushed my sobs, fearful lest any sign of violent grief might waken a
preternatural voice to comfort me, or elicit from the gloom some haloed
face, bending over me with strange pity.  This idea, consolatory in
theory, I felt would be terrible if realised: with all my might I
endeavoured to stifle it---I endeavoured to be firm.  Shaking my hair
from my eyes, I lifted my head and tried to look boldly round the dark
room; at this moment a light gleamed on the wall.  Was it, I asked
myself, a ray from the moon penetrating some aperture in the blind?  No;
moonlight was still, and this stirred; while I gazed, it glided up to
the ceiling and quivered over my head.  I can now conjecture readily
that this streak of light was, in all likelihood, a gleam from a lantern
carried by some one across the lawn: but then, prepared as my mind was
for horror, shaken as my nerves were by agitation, I thought the swift
darting beam was a herald of some coming vision from another world.  My
heart beat thick, my head grew hot; a sound filled my ears, which I
deemed the rushing of wings; something seemed near me; I was oppressed,
suffocated: endurance broke down; I rushed to the door and shook the
lock in desperate effort.  Steps came running along the outer passage;
the key turned, Bessie and Abbot entered.

\enquote{Miss Eyre, are you ill?} said Bessie.

\enquote{What a dreadful noise! it went quite through me!} exclaimed
Abbot.

\enquote{Take me out!  Let me go into the nursery!} was my cry.

\enquote{What for?  Are you hurt?  Have you seen something?} again
demanded Bessie.

\enquote{Oh!  I saw a light, and I thought a ghost would come.}  I had
now got hold of Bessie's hand, and she did not snatch it from me.

\enquote{She has screamed out on purpose,} declared Abbot, in some
disgust.  \enquote{And what a scream!  If she had been in great pain one
would have excused it, but she only wanted to bring us all here: I know
her naughty tricks.}

\enquote{What is all this?} demanded another voice peremptorily; and
\Mrs{} Reed came along the corridor, her cap flying wide, her gown
rustling stormily.  \enquote{Abbot and Bessie, I believe I gave orders
that Jane Eyre should be left in the red-room till I came to her
myself.}

\enquote{Miss Jane screamed so loud, ma'am,} pleaded Bessie.

\enquote{Let her go,} was the only answer.  \enquote{Loose Bessie's
hand, child: you cannot succeed in getting out by these means, be
assured.  I abhor artifice, particularly in children; it is my duty to
show you that tricks will not answer: you will now stay here an hour
longer, and it is only on condition of perfect submission and stillness
that I shall liberate you then.}

\enquote{O aunt! have pity!  Forgive me!  I cannot endure it---let me be
punished some other way!  I shall be killed if---}

\enquote{Silence!  This violence is all most repulsive:} and so, no
doubt, she felt it.  I was a precocious actress in her eyes; she
sincerely looked on me as a compound of virulent passions, mean spirit,
and dangerous duplicity.

Bessie and Abbot having retreated, \Mrs{} Reed, impatient of my now
frantic anguish and wild sobs, abruptly thrust me back and locked me in,
without farther parley.  I heard her sweeping away; and soon after she
was gone, I suppose I had a species of fit: unconsciousness closed the
scene.

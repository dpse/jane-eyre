\FChapter{Chapter Twenty-Two}{22}

\Lettrine{M}{r.\@} \textsc{ Rochester} had given me but one week's leave of absence: yet a month
elapsed before I quitted Gateshead. I wished to leave immediately after
the funeral, but Georgiana entreated me to stay till she could get off
to London, whither she was now at last invited by her uncle, \Mr{} Gibson,
who had come down to direct his sister's interment and settle the family
affairs. Georgiana said she dreaded being left alone with Eliza; from
her she got neither sympathy in her dejection, support in her fears, nor
aid in her preparations; so I bore with her feeble-minded wailings and
selfish lamentations as well as I could, and did my best in sewing for
her and packing her dresses. It is true, that while I worked, she would
idle; and I thought to myself, \enquote{If you and I were destined to
live always together, cousin, we would commence matters on a different
footing. I should not settle tamely down into being the forbearing
party; I should assign you your share of labour, and compel you to
accomplish it, or else it should be left undone: I should insist, also,
on your keeping some of those drawling, half-insincere complaints hushed
in your own breast. It is only because our connection happens to be
very transitory, and comes at a peculiarly mournful season, that I
consent thus to render it so patient and compliant on my part.}

At last I saw Georgiana off; but now it was Eliza's turn to request me
to stay another week. Her plans required all her time and attention,
she said; she was about to depart for some unknown bourne; and all day
long she stayed in her own room, her door bolted within, filling trunks,
emptying drawers, burning papers, and holding no communication with any
one. She wished me to look after the house, to see callers, and answer
notes of condolence.

One morning she told me I was at liberty. \enquote{And,} she added,
\enquote{I am obliged to you for your valuable services and discreet
conduct! There is some difference between living with such an one as
you and with Georgiana: you perform your own part in life and burden no
one. To-morrow,} she continued, \enquote{I set out for the Continent. 
I shall take up my abode in a religious house near Lisle---a nunnery you
would call it; there I shall be quiet and unmolested. I shall devote
myself for a time to the examination of the Roman Catholic dogmas, and
to a careful study of the workings of their system: if I find it to be,
as I half suspect it is, the one best calculated to ensure the doing of
all things decently and in order, I shall embrace the tenets of Rome and
probably take the veil.}

I neither expressed surprise at this resolution nor attempted to
dissuade her from it. \enquote{The vocation will fit you to a hair,} I
thought: \enquote{much good may it do you!}

When we parted, she said: \enquote{Good-bye, cousin Jane Eyre; I wish
you well: you have some sense.}

I then returned: \enquote{You are not without sense, cousin Eliza; but
what you have, I suppose, in another year will be walled up alive in a
French convent. However, it is not my business, and so it suits you, I
don't much care.}

\enquote{You are in the right,} said she; and with these words we each
went our separate way. As I shall not have occasion to refer either to
her or her sister again, I may as well mention here, that Georgiana made
an advantageous match with a wealthy worn-out man of fashion, and that
Eliza actually took the veil, and is at this day superior of the convent
where she passed the period of her novitiate, and which she endowed with
her fortune.

How people feel when they are returning home from an absence, long or
short, I did not know: I had never experienced the sensation. I had
known what it was to come back to Gateshead when a child after a long
walk, to be scolded for looking cold or gloomy; and later, what it was
to come back from church to Lowood, to long for a plenteous meal and a
good fire, and to be unable to get either. Neither of these returnings
was very pleasant or desirable: no magnet drew me to a given point,
increasing in its strength of attraction the nearer I came. The return
to Thornfield was yet to be tried.

My journey seemed tedious---very tedious: fifty miles one day, a night
spent at an inn; fifty miles the next day. During the first twelve
hours I thought of \Mrs{} Reed in her last moments; I saw her disfigured
and discoloured face, and heard her strangely altered voice. I mused on
the funeral day, the coffin, the hearse, the black train of tenants and
servants---few was the number of relatives---the gaping vault, the
silent church, the solemn service. Then I thought of Eliza and
Georgiana; I beheld one the cynosure of a ball-room, the other the
inmate of a convent cell; and I dwelt on and analysed their separate
peculiarities of person and character. The evening arrival at the great
town of---scattered these thoughts; night gave them quite another turn:
laid down on my traveller's bed, I left reminiscence for anticipation.

I was going back to Thornfield: but how long was I to stay there? Not
long; of that I was sure. I had heard from \Mrs{} Fairfax in the interim
of my absence: the party at the hall was dispersed; \Mr{} Rochester had
left for London three weeks ago, but he was then expected to return in a
fortnight. \Mrs{} Fairfax surmised that he was gone to make arrangements
for his wedding, as he had talked of purchasing a new carriage: she said
the idea of his marrying Miss Ingram still seemed strange to her; but
from what everybody said, and from what she had herself seen, she could
no longer doubt that the event would shortly take place. \enquote{You
would be strangely incredulous if you did doubt it,} was my mental
comment. \enquote{I don't doubt it.}

The question followed, \enquote{Where was I to go?} I dreamt of Miss
Ingram all the night: in a vivid morning dream I saw her closing the
gates of Thornfield against me and pointing me out another road; and \Mr{}
 Rochester looked on with his arms folded---smiling sardonically, as it
seemed, at both her and me.

I had not notified to \Mrs{} Fairfax the exact day of my return; for I did
not wish either car or carriage to meet me at Millcote. I proposed to
walk the distance quietly by myself; and very quietly, after leaving my
box in the ostler's care, did I slip away from the George Inn, about six
o'clock of a June evening, and take the old road to Thornfield: a road
which lay chiefly through fields, and was now little frequented.

It was not a bright or splendid summer evening, though fair and soft:
the haymakers were at work all along the road; and the sky, though far
from cloudless, was such as promised well for the future: its
blue---where blue was visible---was mild and settled, and its cloud
strata high and thin. The west, too, was warm: no watery gleam chilled
it---it seemed as if there was a fire lit, an altar burning behind its
screen of marbled vapour, and out of apertures shone a golden redness.

I felt glad as the road shortened before me: so glad that I stopped once
to ask myself what that joy meant: and to remind reason that it was not
to my home I was going, or to a permanent resting-place, or to a place
where fond friends looked out for me and waited my arrival. 
\enquote{\Mrs{} Fairfax will smile you a calm welcome, to be sure,} said
I; \enquote{and little Adèle will clap her hands and jump to see you:
but you know very well you are thinking of another than they, and that
he is not thinking of you.}

But what is so headstrong as youth? What so blind as inexperience? 
These affirmed that it was pleasure enough to have the privilege of
again looking on \Mr{} Rochester, whether he looked on me or not; and they
added---\enquote{Hasten! hasten! be with him while you may: but a few
more days or weeks, at most, and you are parted from him for ever!} And
then I strangled a new-born agony---a deformed thing which I could not
persuade myself to own and rear---and ran on.

They are making hay, too, in Thornfield meadows: or rather, the
labourers are just quitting their work, and returning home with their
rakes on their shoulders, now, at the hour I arrive. I have but a field
or two to traverse, and then I shall cross the road and reach the
gates. How full the hedges are of roses! But I have no time to gather
any; I want to be at the house. I passed a tall briar, shooting leafy
and flowery branches across the path; I see the narrow stile with stone
steps; and I see---\Mr{} Rochester sitting there, a book and a pencil in
his hand; he is writing.

Well, he is not a ghost; yet every nerve I have is unstrung: for a
moment I am beyond my own mastery. What does it mean? I did not think
I should tremble in this way when I saw him, or lose my voice or the
power of motion in his presence. I will go back as soon as I can stir:
I need not make an absolute fool of myself. I know another way to the
house. It does not signify if I knew twenty ways; for he has seen me.

\enquote{Hillo!} he cries; and he puts up his book and his pencil. 
\enquote{There you are! Come on, if you please.}

I suppose I do come on; though in what fashion I know not; being
scarcely cognisant of my movements, and solicitous only to appear calm;
and, above all, to control the working muscles of my face---which I feel
rebel insolently against my will, and struggle to express what I had
resolved to conceal. But I have a veil---it is down: I may make shift
yet to behave with decent composure.

\enquote{And this is Jane Eyre? Are you coming from Millcote, and on
foot? Yes---just one of your tricks: not to send for a carriage, and
come clattering over street and road like a common mortal, but to steal
into the vicinage of your home along with twilight, just as if you were
a dream or a shade. What the deuce have you done with yourself this
last month?}

\enquote{I have been with my aunt, sir, who is dead.}

\enquote{A true Janian reply! Good angels be my guard! She comes from the
other world---from the abode of people who are dead; and tells me so
when she meets me alone here in the gloaming! If I dared, I'd touch
you, to see if you are substance or shadow, you elf!---but I'd as soon
offer to take hold of a blue \emph{ignis fatuus} light in a marsh. 
Truant! truant!} he added, when he had paused an instant. 
\enquote{Absent from me a whole month, and forgetting me quite, I'll be
sworn!}

I knew there would be pleasure in meeting my master again, even though
broken by the fear that he was so soon to cease to be my master, and by
the knowledge that I was nothing to him: but there was ever in \Mr{}
 Rochester (so at least I thought) such a wealth of the power of
communicating happiness, that to taste but of the crumbs he scattered to
stray and stranger birds like me, was to feast genially. His last words
were balm: they seemed to imply that it imported something to him
whether I forgot him or not. And he had spoken of Thornfield as my
home---would that it were my home!

He did not leave the stile, and I hardly liked to ask to go by. I
inquired soon if he had not been to London.

\enquote{Yes; I suppose you found that out by second-sight.}

\enquote{\Mrs{} Fairfax told me in a letter.}

\enquote{And did she inform you what I went to do?}

\enquote{Oh, yes, sir! Everybody knew your errand.}

\enquote{You must see the carriage, Jane, and tell me if you don't think
it will suit \Mrs{} Rochester exactly; and whether she won't look like
Queen Boadicea, leaning back against those purple cushions. I wish,
Jane, I were a trifle better adapted to match with her externally. Tell
me now, fairy as you are---can't you give me a charm, or a philter, or
something of that sort, to make me a handsome man?}

\enquote{It would be past the power of magic, sir;} and, in thought, I
added, \enquote{A loving eye is all the charm needed: to such you are
handsome enough; or rather your sternness has a power beyond beauty.}

\Mr{} Rochester had sometimes read my unspoken thoughts with an acumen to
me incomprehensible: in the present instance he took no notice of my
abrupt vocal response; but he smiled at me with a certain smile he had
of his own, and which he used but on rare occasions. He seemed to think
it too good for common purposes: it was the real sunshine of
feeling---he shed it over me now.

\enquote{Pass, Janet,} said he, making room for me to cross the stile:
\enquote{go up home, and stay your weary little wandering feet at a
friend's threshold.}

All I had now to do was to obey him in silence: no need for me to
colloquise further. I got over the stile without a word, and meant to
leave him calmly. An impulse held me fast---a force turned me round. I
said---or something in me said for me, and in spite of me---

\enquote{Thank you, \Mr{} Rochester, for your great kindness. I am
strangely glad to get back again to you: and wherever you are is my
home---my only home.}

I walked on so fast that even he could hardly have overtaken me had he
tried. Little Adèle was half wild with delight when she saw me. \Mrs{}
Fairfax received me with her usual plain friendliness. Leah smiled, and
even Sophie bid me \enquote{bon soir} with glee. This was very
pleasant; there is no happiness like that of being loved by your
fellow-creatures, and feeling that your presence is an addition to their
comfort.

I that evening shut my eyes resolutely against the future: I stopped my
ears against the voice that kept warning me of near separation and
coming grief. When tea was over and \Mrs{} Fairfax had taken her
knitting, and I had assumed a low seat near her, and Adèle, kneeling on
the carpet, had nestled close up to me, and a sense of mutual affection
seemed to surround us with a ring of golden peace, I uttered a silent
prayer that we might not be parted far or soon; but when, as we thus
sat, \Mr{} Rochester entered, unannounced, and looking at us, seemed to
take pleasure in the spectacle of a group so amicable---when he said he
supposed the old lady was all right now that she had got her adopted
daughter back again, and added that he saw Adèle was \foreignquote{french}{prête à
croquer sa petite maman Anglaise}\footnote{\enquote{Ready to crush her little English mother.}}---I half ventured to hope that he
would, even after his marriage, keep us together somewhere under the
shelter of his protection, and not quite exiled from the sunshine of his
presence.

A fortnight of dubious calm succeeded my return to Thornfield Hall. 
Nothing was said of the master's marriage, and I saw no preparation
going on for such an event. Almost every day I asked \Mrs{} Fairfax if
she had yet heard anything decided: her answer was always in the
negative. Once she said she had actually put the question to \Mr{}
 Rochester as to when he was going to bring his bride home; but he had
answered her only by a joke and one of his queer looks, and she could
not tell what to make of him.

One thing specially surprised me, and that was, there were no
journeyings backward and forward, no visits to Ingram Park: to be sure
it was twenty miles off, on the borders of another county; but what was
that distance to an ardent lover? To so practised and indefatigable a
horseman as \Mr{} Rochester, it would be but a morning's ride. I began to
cherish hopes I had no right to conceive: that the match was broken off;
that rumour had been mistaken; that one or both parties had changed
their minds. I used to look at my master's face to see if it were sad
or fierce; but I could not remember the time when it had been so
uniformly clear of clouds or evil feelings. If, in the moments I and my
pupil spent with him, I lacked spirits and sank into inevitable
dejection, he became even gay. Never had he called me more frequently
to his presence; never been kinder to me when there---and, alas! never
had I loved him so well.

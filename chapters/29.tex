\FChapter{Chapter Twenty-Nine}{29}

\Lettrine{T}{he} \textsc{recollection} of about three days and nights succeeding this is very
dim in my mind. I can recall some sensations felt in that interval; but
few thoughts framed, and no actions performed. I knew I was in a small
room and in a narrow bed. To that bed I seemed to have grown; I lay on
it motionless as a stone; and to have torn me from it would have been
almost to kill me. I took no note of the lapse of time---of the change
from morning to noon, from noon to evening. I observed when any one
entered or left the apartment: I could even tell who they were; I could
understand what was said when the speaker stood near to me; but I could
not answer; to open my lips or move my limbs was equally impossible. 
Hannah, the servant, was my most frequent visitor. Her coming disturbed
me. I had a feeling that she wished me away: that she did not
understand me or my circumstances; that she was prejudiced against me. 
Diana and Mary appeared in the chamber once or twice a day. They would
whisper sentences of this sort at my bedside---

\enquote{It is very well we took her in.}

\enquote{Yes; she would certainly have been found dead at the door in
the morning had she been left out all night. I wonder what she has gone
through?}

\enquote{Strange hardships, I imagine---poor, emaciated, pallid
wanderer?}

\enquote{She is not an uneducated person, I should think, by her manner
of speaking; her accent was quite pure; and the clothes she took off,
though splashed and wet, were little worn and fine.}

\enquote{She has a peculiar face; fleshless and haggard as it is, I
rather like it; and when in good health and animated, I can fancy her
physiognomy would be agreeable.}

Never once in their dialogues did I hear a syllable of regret at the
hospitality they had extended to me, or of suspicion of, or aversion to,
myself. I was comforted.

\Mr{} \St{} John came but once: he looked at me, and said my state of
lethargy was the result of reaction from excessive and protracted
fatigue. He pronounced it needless to send for a doctor: nature, he was
sure, would manage best, left to herself. He said every nerve had been
overstrained in some way, and the whole system must sleep torpid a
while. There was no disease. He imagined my recovery would be rapid
enough when once commenced. These opinions he delivered in a few words,
in a quiet, low voice; and added, after a pause, in the tone of a man
little accustomed to expansive comment, \enquote{Rather an unusual
physiognomy; certainly, not indicative of vulgarity or degradation.}

\enquote{Far otherwise,} responded Diana. \enquote{To speak truth, St.
John, my heart rather warms to the poor little soul. I wish we may be
able to benefit her permanently.}

\enquote{That is hardly likely,} was the reply. \enquote{You will find
she is some young lady who has had a misunderstanding with her friends,
and has probably injudiciously left them. We may, perhaps, succeed in
restoring her to them, if she is not obstinate: but I trace lines of
force in her face which make me sceptical of her tractability.} He
stood considering me some minutes; then added, \enquote{She looks
sensible, but not at all handsome.}

\enquote{She is so ill, \St{} John.}

\enquote{Ill or well, she would always be plain. The grace and harmony
of beauty are quite wanting in those features.}

On the third day I was better; on the fourth, I could speak, move, rise
in bed, and turn. Hannah had brought me some gruel and dry toast,
about, as I supposed, the dinner-hour. I had eaten with relish: the
food was good---void of the feverish flavour which had hitherto poisoned
what I had swallowed. When she left me, I felt comparatively strong and
revived: ere long satiety of repose and desire for action stirred me. I
wished to rise; but what could I put on? Only my damp and bemired
apparel; in which I had slept on the ground and fallen in the marsh. I
felt ashamed to appear before my benefactors so clad. I was spared the
humiliation.

On a chair by the bedside were all my own things, clean and dry. My
black silk frock hung against the wall. The traces of the bog were
removed from it; the creases left by the wet smoothed out: it was quite
decent. My very shoes and stockings were purified and rendered
presentable. There were the means of washing in the room, and a comb
and brush to smooth my hair. After a weary process, and resting every
five minutes, I succeeded in dressing myself. My clothes hung loose on
me; for I was much wasted, but I covered deficiencies with a shawl, and
once more, clean and respectable looking---no speck of the dirt, no
trace of the disorder I so hated, and which seemed so to degrade me,
left---I crept down a stone staircase with the aid of the banisters, to
a narrow low passage, and found my way presently to the kitchen.

It was full of the fragrance of new bread and the warmth of a generous
fire. Hannah was baking. Prejudices, it is well known, are most
difficult to eradicate from the heart whose soil has never been loosened
or fertilised by education: they grow there, firm as weeds among
stones. Hannah had been cold and stiff, indeed, at the first: latterly
she had begun to relent a little; and when she saw me come in tidy and
well-dressed, she even smiled.

\enquote{What, you have got up!} she said. \enquote{You are better,
then. You may sit you down in my chair on the hearthstone, if you
will.}

She pointed to the rocking-chair: I took it. She bustled about,
examining me every now and then with the corner of her eye. Turning to
me, as she took some loaves from the oven, she asked bluntly---

\enquote{Did you ever go a-begging afore you came here?}

I was indignant for a moment; but remembering that anger was out of the
question, and that I had indeed appeared as a beggar to her, I answered
quietly, but still not without a certain marked firmness---

\enquote{You are mistaken in supposing me a beggar. I am no beggar; any
more than yourself or your young ladies.}

After a pause she said, \enquote{I dunnut understand that: you've like
no house, nor no brass, I guess?}

\enquote{The want of house or brass (by which I suppose you mean money)
does not make a beggar in your sense of the word.}

\enquote{Are you book-learned?} she inquired presently.

\enquote{Yes, very.}

\enquote{But you've never been to a boarding-school?}

\enquote{I was at a boarding-school eight years.}

She opened her eyes wide. \enquote{Whatever cannot ye keep yourself
for, then?}

\enquote{I have kept myself; and, I trust, shall keep myself again. 
What are you going to do with these gooseberries?} I inquired, as she
brought out a basket of the fruit.

\enquote{Mak' 'em into pies.}

\enquote{Give them to me and I'll pick them.}

\enquote{Nay; I dunnut want ye to do nought.}

\enquote{But I must do something. Let me have them.}

She consented; and she even brought me a clean towel to spread over my
dress, \enquote{lest,} as she said, \enquote{I should mucky it.}

\enquote{Ye've not been used to sarvant's wark, I see by your hands,}
she remarked. \enquote{Happen ye've been a dressmaker?}

\enquote{No, you are wrong. And now, never mind what I have been: don't
trouble your head further about me; but tell me the name of the house
where we are.}

\enquote{Some calls it Marsh End, and some calls it Moor House.}

\enquote{And the gentleman who lives here is called \Mr{} \St{} John?}

\enquote{Nay; he doesn't live here: he is only staying a while. When he
is at home, he is in his own parish at Morton.}

\enquote{That village a few miles off?} %fix missing end quote

\enquote{Aye.}

\enquote{And what is he?}

\enquote{He is a parson.}

I remembered the answer of the old housekeeper at the parsonage, when I
had asked to see the clergyman. \enquote{This, then, was his father's
residence?}

\enquote{Aye; old \Mr{} Rivers lived here, and his father, and
grandfather, and gurt (great) grandfather afore him.}

\enquote{The name, then, of that gentleman, is \Mr{} \St{} John Rivers?}

\enquote{Aye; \St{} John is like his kirstened name.}

\enquote{And his sisters are called Diana and Mary Rivers?}

\enquote{Yes.}

\enquote{Their father is dead?}

\enquote{Dead three weeks sin' of a stroke.}

\enquote{They have no mother?}

\enquote{The mistress has been dead this mony a year.}

\enquote{Have you lived with the family long?}

\enquote{I've lived here thirty year. I nursed them all three.}

\enquote{That proves you must have been an honest and faithful servant. 
I will say so much for you, though you have had the incivility to call
me a beggar.}

She again regarded me with a surprised stare. \enquote{I believe,} she
said, \enquote{I was quite mista'en in my thoughts of you: but there is
so mony cheats goes about, you mun forgie me.}

\enquote{And though,} I continued, rather severely, \enquote{you wished
to turn me from the door, on a night when you should not have shut out a
dog.}

\enquote{Well, it was hard: but what can a body do? I thought more o'
th' childer nor of mysel: poor things! They've like nobody to tak' care
on 'em but me. I'm like to look sharpish.}

I maintained a grave silence for some minutes.

\enquote{You munnut think too hardly of me,} she again remarked.

\enquote{But I do think hardly of you,} I said; \enquote{and I'll tell
you why---not so much because you refused to give me shelter, or
regarded me as an impostor, as because you just now made it a species of
reproach that I had no \enquote{brass} and no house. Some of the best
people that ever lived have been as destitute as I am; and if you are a
Christian, you ought not to consider poverty a crime.}

\enquote{No more I ought,} said she: \enquote{\Mr{} \St{} John tells me so
too; and I see I wor wrang---but I've clear a different notion on you
now to what I had. You look a raight down dacent little crater.}

\enquote{That will do---I forgive you now. Shake hands.}

She put her floury and horny hand into mine; another and heartier smile
illumined her rough face, and from that moment we were friends.

Hannah was evidently fond of talking. While I picked the fruit, and she
made the paste for the pies, she proceeded to give me sundry details
about her deceased master and mistress, and \enquote{the childer,} as
she called the young people.

Old \Mr{} Rivers, she said, was a plain man enough, but a gentleman, and
of as ancient a family as could be found. Marsh End had belonged to the
Rivers ever since it was a house: and it was, she affirmed,
\enquote{aboon two hundred year old---for all it looked but a small,
humble place, naught to compare wi' \Mr{} Oliver's grand hall down i'
Morton Vale. But she could remember Bill Oliver's father a journeyman
needlemaker; and th' Rivers wor gentry i' th' owd days o' th' Henrys, as
onybody might see by looking into th' registers i' Morton Church
vestry.} Still, she allowed, \enquote{the owd maister was like other
folk---naught mich out o' t' common way: stark mad o' shooting, and
farming, and sich like.} The mistress was different. She was a great
reader, and studied a deal; and the \enquote{bairns} had taken after
her. There was nothing like them in these parts, nor ever had been;
they had liked learning, all three, almost from the time they could
speak; and they had always been \enquote{of a mak' of their own.} \Mr{}
\St{} John, when he grew up, would go to college and be a parson; and the
girls, as soon as they left school, would seek places as governesses:
for they had told her their father had some years ago lost a great deal
of money by a man he had trusted turning bankrupt; and as he was now not
rich enough to give them fortunes, they must provide for themselves. 
They had lived very little at home for a long while, and were only come
now to stay a few weeks on account of their father's death; but they did
so like Marsh End and Morton, and all these moors and hills about. They
had been in London, and many other grand towns; but they always said
there was no place like home; and then they were so agreeable with each
other---never fell out nor \enquote{threaped.} She did not know where
there was such a family for being united.

Having finished my task of gooseberry picking, I asked where the two
ladies and their brother were now.

\enquote{Gone over to Morton for a walk; but they would be back in
half-an-hour to tea.}

They returned within the time Hannah had allotted them: they entered by
the kitchen door. \Mr{} \St{} John, when he saw me, merely bowed and passed
through; the two ladies stopped: Mary, in a few words, kindly and calmly
expressed the pleasure she felt in seeing me well enough to be able to
come down; Diana took my hand: she shook her head at me.

\enquote{You should have waited for my leave to descend,} she said. 
\enquote{You still look very pale---and so thin! Poor child!---poor
girl!}

Diana had a voice toned, to my ear, like the cooing of a dove. She
possessed eyes whose gaze I delighted to encounter. Her whole face
seemed to me full of charm. Mary's countenance was equally
intelligent---her features equally pretty; but her expression was more
reserved, and her manners, though gentle, more distant. Diana looked
and spoke with a certain authority: she had a will, evidently. It was
my nature to feel pleasure in yielding to an authority supported like
hers, and to bend, where my conscience and self-respect permitted, to an
active will.

\enquote{And what business have you here?} she continued. \enquote{It
is not your place. Mary and I sit in the kitchen sometimes, because at
home we like to be free, even to license---but you are a visitor, and
must go into the parlour.}

\enquote{I am very well here.}

\enquote{Not at all, with Hannah bustling about and covering you with
flour.}

\enquote{Besides, the fire is too hot for you,} interposed Mary.

\enquote{To be sure,} added her sister. \enquote{Come, you must be
obedient.} And still holding my hand she made me rise, and led me into
the inner room.

\enquote{Sit there,} she said, placing me on the sofa, \enquote{while we
take our things off and get the tea ready; it is another privilege we
exercise in our little moorland home---to prepare our own meals when we
are so inclined, or when Hannah is baking, brewing, washing, or
ironing.}

She closed the door, leaving me solus with \Mr{} \St{} John, who sat
opposite, a book or newspaper in his hand. I examined first, the
parlour, and then its occupant.

The parlour was rather a small room, very plainly furnished, yet
comfortable, because clean and neat. The old-fashioned chairs were very
bright, and the walnut-wood table was like a looking-glass. A few
strange, antique portraits of the men and women of other days decorated
the stained walls; a cupboard with glass doors contained some books and
an ancient set of china. There was no superfluous ornament in the
room---not one modern piece of furniture, save a brace of workboxes and
a lady's desk in rosewood, which stood on a side-table:
everything---including the carpet and curtains---looked at once well
worn and well saved.

\Mr{} \St{} John---sitting as still as one of the dusty pictures on the
walls, keeping his eyes fixed on the page he perused, and his lips
mutely sealed---was easy enough to examine. Had he been a statue
instead of a man, he could not have been easier. He was young---perhaps
from twenty-eight to thirty---tall, slender; his face riveted the eye;
it was like a Greek face, very pure in outline: quite a straight,
classic nose; quite an Athenian mouth and chin. It is seldom, indeed,
an English face comes so near the antique models as did his. He might
well be a little shocked at the irregularity of my lineaments, his own
being so harmonious. His eyes were large and blue, with brown lashes;
his high forehead, colourless as ivory, was partially streaked over by
careless locks of fair hair.

This is a gentle delineation, is it not, reader? Yet he whom it
describes scarcely impressed one with the idea of a gentle, a yielding,
an impressible, or even of a placid nature. Quiescent as he now sat,
there was something about his nostril, his mouth, his brow, which, to my
perceptions, indicated elements within either restless, or hard, or
eager. He did not speak to me one word, nor even direct to me one
glance, till his sisters returned. Diana, as she passed in and out, in
the course of preparing tea, brought me a little cake, baked on the top
of the oven.

\enquote{Eat that now,} she said: \enquote{you must be hungry. Hannah
says you have had nothing but some gruel since breakfast.}

I did not refuse it, for my appetite was awakened and keen. \Mr{} Rivers
now closed his book, approached the table, and, as he took a seat, fixed
his blue pictorial-looking eyes full on me. There was an unceremonious
directness, a searching, decided steadfastness in his gaze now, which
told that intention, and not diffidence, had hitherto kept it averted
from the stranger.

\enquote{You are very hungry,} he said.

\enquote{I am, sir.} It is my way---it always was my way, by
instinct---ever to meet the brief with brevity, the direct with
plainness.

\enquote{It is well for you that a low fever has forced you to abstain
for the last three days: there would have been danger in yielding to the
cravings of your appetite at first. Now you may eat, though still not
immoderately.}

\enquote{I trust I shall not eat long at your expense, sir,} was my very
clumsily-contrived, unpolished answer.

\enquote{No,} he said coolly: \enquote{when you have indicated to us the
residence of your friends, we can write to them, and you may be restored
to home.}

\enquote{That, I must plainly tell you, is out of my power to do; being
absolutely without home and friends.}

The three looked at me, but not distrustfully; I felt there was no
suspicion in their glances: there was more of curiosity. I speak
particularly of the young ladies. \St{} John's eyes, though clear enough
in a literal sense, in a figurative one were difficult to fathom. He
seemed to use them rather as instruments to search other people's
thoughts, than as agents to reveal his own: the which combination of
keenness and reserve was considerably more calculated to embarrass than
to encourage.

\enquote{Do you mean to say,} he asked, \enquote{that you are completely
isolated from every connection?}

\enquote{I do. Not a tie links me to any living thing: not a claim do I
possess to admittance under any roof in England.}

\enquote{A most singular position at your age!}

Here I saw his glance directed to my hands, which were folded on the
table before me. I wondered what he sought there: his words soon
explained the quest.

\enquote{You have never been married? You are a spinster?}

Diana laughed. \enquote{Why, she can't be above seventeen or eighteen
years old, \St{} John,} said she.

\enquote{I am near nineteen: but I am not married. No.}

I felt a burning glow mount to my face; for bitter and agitating
recollections were awakened by the allusion to marriage. They all saw
the embarrassment and the emotion. Diana and Mary relieved me by
turning their eyes elsewhere than to my crimsoned visage; but the colder
and sterner brother continued to gaze, till the trouble he had excited
forced out tears as well as colour.

\enquote{Where did you last reside?} he now asked.

\enquote{You are too inquisitive, \St{} John,} murmured Mary in a low
voice; but he leaned over the table and required an answer by a second
firm and piercing look.

\enquote{The name of the place where, and of the person with whom I
lived, is my secret,} I replied concisely.

\enquote{Which, if you like, you have, in my opinion, a right to keep,
both from \St{} John and every other questioner,} remarked Diana.

\enquote{Yet if I know nothing about you or your history, I cannot help
you,} he said. \enquote{And you need help, do you not?}

\enquote{I need it, and I seek it so far, sir, that some true
philanthropist will put me in the way of getting work which I can do,
and the remuneration for which will keep me, if but in the barest
necessaries of life.}

\enquote{I know not whether I am a true philanthropist; yet I am willing to aid
you to the utmost of my power in a purpose so honest. First, then, tell
me what you have been accustomed to do, and what you \emph{can} do.}

I had now swallowed my tea. I was mightily refreshed by the beverage;
as much so as a giant with wine: it gave new tone to my unstrung nerves,
and enabled me to address this penetrating young judge steadily.

\enquote{\Mr{} Rivers,} I said, turning to him, and looking at him, as he
looked at me, openly and without diffidence, \enquote{you and your sisters have
done me a great service---the greatest man can do his fellow-being; you
have rescued me, by your noble hospitality, from death. This benefit
conferred gives you an unlimited claim on my gratitude, and a claim, to
a certain extent, on my confidence. I will tell you as much of the
history of the wanderer you have harboured, as I can tell without
compromising my own peace of mind---my own security, moral and physical,
and that of others.

%rem enq
I am an orphan, the daughter of a clergyman. My parents died
before I could know them. I was brought up a dependant; educated in a
charitable institution. I will even tell you the name of the
establishment, where I passed six years as a pupil, and two as a
teacher---Lowood Orphan Asylum, ---shire: you will have heard of it, \Mr{}
Rivers?---the Rev.\@ Robert Brocklehurst is the treasurer.}

\enquote{I have heard of \Mr{} Brocklehurst, and I have seen the school.}

\enquote{I left Lowood nearly a year since to become a private
governess. I obtained a good situation, and was happy. This place I
was obliged to leave four days before I came here. The reason of my
departure I cannot and ought not to explain: it would be useless,
dangerous, and would sound incredible. No blame attached to me: I am as
free from culpability as any one of you three. Miserable I am, and must
be for a time; for the catastrophe which drove me from a house I had
found a paradise was of a strange and direful nature. I observed but
two points in planning my departure---speed, secrecy: to secure these, I
had to leave behind me everything I possessed except a small parcel;
which, in my hurry and trouble of mind, I forgot to take out of the
coach that brought me to Whitcross. To this neighbourhood, then, I
came, quite destitute. I slept two nights in the open air, and wandered
about two days without crossing a threshold: but twice in that space of
time did I taste food; and it was when brought by hunger, exhaustion,
and despair almost to the last gasp, that you, \Mr{} Rivers, forbade me to
perish of want at your door, and took me under the shelter of your
roof. I know all your sisters have done for me since---for I have not
been insensible during my seeming torpor---and I owe to their
spontaneous, genuine, genial compassion as large a debt as to your
evangelical charity.}

\enquote{Don't make her talk any more now, \St{} John,} said Diana, as I
paused; \enquote{she is evidently not yet fit for excitement. Come to
the sofa and sit down now, Miss Elliott.}

I gave an involuntary half start at hearing the \emph{alias}: I had
forgotten my new name. \Mr{} Rivers, whom nothing seemed to escape,
noticed it at once.

\enquote{You said your name was Jane Elliott?} he observed.

\enquote{I did say so; and it is the name by which I think it expedient
to be called at present, but it is not my real name, and when I hear it,
it sounds strange to me.}

\enquote{Your real name you will not give?}

\enquote{No: I fear discovery above all things; and whatever disclosure
would lead to it, I avoid.}

\enquote{You are quite right, I am sure,} said Diana. \enquote{Now do,
brother, let her be at peace a while.}

But when \St{} John had mused a few moments he recommenced as
imperturbably and with as much acumen as ever.

\enquote{You would not like to be long dependent on our hospitality---you would
wish, I see, to dispense as soon as may be with my sisters' compassion,
and, above all, with my \emph{charity} (I am quite sensible of the
distinction drawn, nor do I resent it---it is just): you desire to be
independent of us?}

\enquote{I do: I have already said so. Show me how to work, or how to
seek work: that is all I now ask; then let me go, if it be but to the
meanest cottage; but till then, allow me to stay here: I dread another
essay of the horrors of homeless destitution.}

\enquote{Indeed you \emph{shall} stay here,} said Diana, putting her white
hand on my head. \enquote{You \emph{shall},} repeated Mary, in the tone of
undemonstrative sincerity which seemed natural to her.

\enquote{My sisters, you see, have a pleasure in keeping you,} said \Mr{}
\St{} John, \enquote{as they would have a pleasure in keeping and
cherishing a half-frozen bird, some wintry wind might have driven
through their casement. I feel more inclination to put you in the way
of keeping yourself, and shall endeavour to do so; but observe, my
sphere is narrow. I am but the incumbent of a poor country parish: my
aid must be of the humblest sort. And if you are inclined to despise
the day of small things, seek some more efficient succour than such as I
can offer.}

\enquote{She has already said that she is willing to do anything honest
she can do,} answered Diana for me; \enquote{and you know, \St{} John, she
has no choice of helpers: she is forced to put up with such crusty
people as you.}

\enquote{I will be a dressmaker; I will be a plain-workwoman; I will be
a servant, a nurse-girl, if I can be no better,} I answered.

\enquote{Right,} said \Mr{} \St{} John, quite coolly. \enquote{If such is
your spirit, I promise to aid you, in my own time and way.}

He now resumed the book with which he had been occupied before tea. I
soon withdrew, for I had talked as much, and sat up as long, as my
present strength would permit.

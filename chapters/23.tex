\FChapter{Chapter Twenty-Three}{23}

\Lettrine{A}{} \textsc{splendid Midsummer shone} over England: skies so pure, suns so radiant
as were then seen in long succession, seldom favour even singly, our
wave-girt land. It was as if a band of Italian days had come from the
South, like a flock of glorious passenger birds, and lighted to rest
them on the cliffs of Albion. The hay was all got in; the fields round
Thornfield were green and shorn; the roads white and baked; the trees
were in their dark prime; hedge and wood, full-leaved and deeply tinted,
contrasted well with the sunny hue of the cleared meadows between.

On Midsummer-eve, Adèle, weary with gathering wild strawberries in Hay
Lane half the day, had gone to bed with the sun. I watched her drop
asleep, and when I left her, I sought the garden.

It was now the sweetest hour of the twenty-four:---\enquote{Day its
	fervid fires had wasted,} and dew fell cool on panting plain and
scorched summit. Where the sun had gone down in simple state---pure of
the pomp of clouds---spread a solemn purple, burning with the light of
red jewel and furnace flame at one point, on one hill-peak, and
extending high and wide, soft and still softer, over half heaven. The
east had its own charm or fine deep blue, and its own modest gem, a
casino and solitary star: soon it would boast the moon; but she was yet
beneath the horizon.

I walked a while on the pavement; but a subtle, well-known scent---that
of a cigar---stole from some window; I saw the library casement open a
handbreadth; I knew I might be watched thence; so I went apart into the
orchard. No nook in the grounds more sheltered and more Eden-like; it
was full of trees, it bloomed with flowers: a very high wall shut it out
from the court, on one side; on the other, a beech avenue screened it
from the lawn. At the bottom was a sunk fence; its sole separation from
lonely fields: a winding walk, bordered with laurels and terminating in
a giant horse-chestnut, circled at the base by a seat, led down to the
fence. Here one could wander unseen. While such honey-dew fell, such
silence reigned, such gloaming gathered, I felt as if I could haunt such
shade for ever; but in threading the flower and fruit parterres at the
upper part of the enclosure, enticed there by the light the now rising
moon cast on this more open quarter, my step is stayed---not by sound,
not by sight, but once more by a warning fragrance.

Sweet-briar and southernwood, jasmine, pink, and rose have long been
yielding their evening sacrifice of incense: this new scent is neither
of shrub nor flower; it is---I know it well---it is \Mr{} Rochester's
cigar. I look round and I listen. I see trees laden with ripening
fruit. I hear a nightingale warbling in a wood half a mile off; no
moving form is visible, no coming step audible; but that perfume
increases: I must flee. I make for the wicket leading to the shrubbery,
and I see \Mr{} Rochester entering. I step aside into the ivy recess; he
will not stay long: he will soon return whence he came, and if I sit
still he will never see me.

But no---eventide is as pleasant to him as to me, and this antique
garden as attractive; and he strolls on, now lifting the gooseberry-tree
branches to look at the fruit, large as plums, with which they are
laden; now taking a ripe cherry from the wall; now stooping towards a
knot of flowers, either to inhale their fragrance or to admire the
dew-beads on their petals. A great moth goes humming by me; it alights
on a plant at \Mr{} Rochester's foot: he sees it, and bends to examine it.

\enquote{Now, he has his back towards me,} thought I, \enquote{and he is
	occupied too; perhaps, if I walk softly, I can slip away unnoticed.}

I trode on an edging of turf that the crackle of the pebbly gravel might
not betray me: he was standing among the beds at a yard or two distant
from where I had to pass; the moth apparently engaged him. \enquote{I
	shall get by very well,} I meditated. As I crossed his shadow, thrown
long over the garden by the moon, not yet risen high, he said quietly,
without turning---

\enquote{Jane, come and look at this fellow.}

I had made no noise: he had not eyes behind---could his shadow feel? I
started at first, and then I approached him.

\enquote{Look at his wings,} said he, \enquote{he reminds me rather of a
	West Indian insect; one does not often see so large and gay a
	night-rover in England; there! he is flown.}

The moth roamed away. I was sheepishly retreating also; but \Mr{}
Rochester followed me, and when we reached the wicket, he said---

\enquote{Turn back: on so lovely a night it is a shame to sit in the
	house; and surely no one can wish to go to bed while sunset is thus at
	meeting with moonrise.}

It is one of my faults, that though my tongue is sometimes prompt enough
at an answer, there are times when it sadly fails me in framing an
excuse; and always the lapse occurs at some crisis, when a facile word
or plausible pretext is specially wanted to get me out of painful
embarrassment. I did not like to walk at this hour alone with \Mr{}
Rochester in the shadowy orchard; but I could not find a reason to
allege for leaving him. I followed with lagging step, and thoughts
busily bent on discovering a means of extrication; but he himself looked
so composed and so grave also, I became ashamed of feeling any
confusion: the evil---if evil existent or prospective there was---seemed
to lie with me only; his mind was unconscious and quiet.

\enquote{Jane,} he recommenced, as we entered the laurel walk, and
slowly strayed down in the direction of the sunk fence and the
horse-chestnut, \enquote{Thornfield is a pleasant place in summer, is it
	not?}

\enquote{Yes, sir.}

\enquote{You must have become in some degree attached to the
	house,---you, who have an eye for natural beauties, and a good deal of
	the organ of Adhesiveness?}

\enquote{I am attached to it, indeed.}

\enquote{And though I don't comprehend how it is, I perceive you have
	acquired a degree of regard for that foolish little child Adèle, too;
	and even for simple dame Fairfax?}

\enquote{Yes, sir; in different ways, I have an affection for both.}

\enquote{And would be sorry to part with them?}

\enquote{Yes.}

\enquote{Pity!} he said, and sighed and paused. \enquote{It is always
	the way of events in this life,} he continued presently: \enquote{no
	sooner have you got settled in a pleasant resting-place, than a voice
	calls out to you to rise and move on, for the hour of repose is
	expired.}

\enquote{Must I move on, sir?} I asked. \enquote{Must I leave
	Thornfield?}

\enquote{I believe you must, Jane. I am sorry, Janet, but I believe
	indeed you must.}

This was a blow: but I did not let it prostrate me.

\enquote{Well, sir, I shall be ready when the order to march comes.}

\enquote{It is come now---I must give it to-night.}

\enquote{Then you \emph{are} going to be married, sir?}

\enquote{Ex-act-ly---pre-cise-ly: with your usual acuteness, you have
	hit the nail straight on the head.}

\enquote{Soon, sir?}

\enquote{Very soon, my---that is, Miss Eyre: and you'll remember, Jane,
	the first time I, or Rumour, plainly intimated to you that it was my
	intention to put my old bachelor's neck into the sacred noose, to enter
	into the holy estate of matrimony---to take Miss Ingram to my bosom, in
	short (she's an extensive armful: but that's not to the point---one
	can't have too much of such a very excellent thing as my beautiful
	Blanche): well, as I was saying---listen to me, Jane! You're not
	turning your head to look after more moths, are you? That was only a
	lady-clock, child, \enquote{flying away home.} I wish to remind you
	that it was you who first said to me, with that discretion I respect in
	you---with that foresight, prudence, and humility which befit your
	responsible and dependent position---that in case I married Miss Ingram,
	both you and little Adèle had better trot forthwith. I pass over the
	sort of slur conveyed in this suggestion on the character of my beloved;
	indeed, when you are far away, Janet, I'll try to forget it: I shall
	notice only its wisdom; which is such that I have made it my law of
	action. Adèle must go to school; and you, Miss Eyre, must get a new
	situation.}

\enquote{Yes, sir, I will advertise immediately: and meantime, I
	suppose---} I was going to say, \enquote{I suppose I may stay here, till
	I find another shelter to betake myself to:} but I stopped, feeling it
would not do to risk a long sentence, for my voice was not quite under
command.

\enquote{In about a month I hope to be a bridegroom,} continued \Mr{}
Rochester; \enquote{and in the interim, I shall myself look out for
	employment and an asylum for you.}

\enquote{Thank you, sir; I am sorry to give---}

\enquote{Oh, no need to apologise! I consider that when a dependent
	does her duty as well as you have done yours, she has a sort of claim
	upon her employer for any little assistance he can conveniently render
	her; indeed I have already, through my future mother-in-law, heard of a
	place that I think will suit: it is to undertake the education of the
	five daughters of \Mrs{} Dionysius O'Gall of Bitternutt Lodge, Connaught,
	Ireland. You'll like Ireland, I think: they're such warm-hearted people
	there, they say.}

\enquote{It is a long way off, sir.}

\enquote{No matter---a girl of your sense will not object to the voyage
	or the distance.}

\enquote{Not the voyage, but the distance: and then the sea is a
	barrier---}

\enquote{From what, Jane?}

\enquote{From England and from Thornfield: and---}

\enquote{Well?}

\enquote{From \emph{you}, sir.}

I said this almost involuntarily, and, with as little sanction of free
will, my tears gushed out. I did not cry so as to be heard, however; I
avoided sobbing. The thought of \Mrs{} O'Gall and Bitternutt Lodge struck
cold to my heart; and colder the thought of all the brine and foam,
destined, as it seemed, to rush between me and the master at whose side
I now walked, and coldest the remembrance of the wider ocean---wealth,
caste, custom intervened between me and what I naturally and inevitably
loved.

\enquote{It is a long way,} I again said.

\enquote{It is, to be sure; and when you get to Bitternutt Lodge,
	Connaught, Ireland, I shall never see you again, Jane: that's morally
	certain. I never go over to Ireland, not having myself much of a fancy
	for the country. We have been good friends, Jane; have we not?}

\enquote{Yes, sir.}

\enquote{And when friends are on the eve of separation, they like to
	spend the little time that remains to them close to each other. Come!
	we'll talk over the voyage and the parting quietly half-an-hour or so,
	while the stars enter into their shining life up in heaven yonder: here
	is the chestnut tree: here is the bench at its old roots. Come, we will
	sit there in peace to-night, though we should never more be destined to
	sit there together.} He seated me and himself.

\enquote{It is a long way to Ireland, Janet, and I am sorry to send my
	little friend on such weary travels: but if I can't do better, how is it
	to be helped? Are you anything akin to me, do you think, Jane?}

I could risk no sort of answer by this time: my heart was still.

\enquote{Because,} he said, \enquote{I sometimes have a queer feeling
	with regard to you---especially when you are near me, as now: it is as
	if I had a string somewhere under my left ribs, tightly and inextricably
	knotted to a similar string situated in the corresponding quarter of
	your little frame. And if that boisterous Channel, and two hundred
	miles or so of land come broad between us, I am afraid that cord of
	communion will be snapt; and then I've a nervous notion I should take to
	bleeding inwardly. As for you,---you'd forget me.}

\enquote{That I \emph{never} should, sir: you know---} Impossible to proceed.

\enquote{Jane, do you hear that nightingale singing in the wood?
	Listen!}

In listening, I sobbed convulsively; for I could repress what I endured
no longer; I was obliged to yield, and I was shaken from head to foot
with acute distress. When I did speak, it was only to express an
impetuous wish that I had never been born, or never come to Thornfield.

\enquote{Because you are sorry to leave it?}

The vehemence of emotion, stirred by grief and love within me, was
claiming mastery, and struggling for full sway, and asserting a right to
predominate, to overcome, to live, rise, and reign at last: yes,---and
to speak.

\enquote{I grieve to leave Thornfield: I love Thornfield:---I love it,
	because I have lived in it a full and delightful life,---momentarily at
	least. I have not been trampled on. I have not been petrified. I have
	not been buried with inferior minds, and excluded from every glimpse of
	communion with what is bright and energetic and high. I have talked,
	face to face, with what I reverence, with what I delight in,---with an
	original, a vigorous, an expanded mind. I have known you, \Mr{}
	Rochester; and it strikes me with terror and anguish to feel I
	absolutely must be torn from you for ever. I see the necessity of
	departure; and it is like looking on the necessity of death.}

\enquote{Where do you see the necessity?} he asked suddenly.

\enquote{Where? You, sir, have placed it before me.}

\enquote{In what shape?}

\enquote{In the shape of Miss Ingram; a noble and beautiful
	woman,---your bride.}

\enquote{My bride! What bride? I have no bride!}

\enquote{But you will have.}

\enquote{Yes;---I will!---I will!} He set his teeth.

\enquote{Then I must go:---you have said it yourself.}

\enquote{No: you must stay! I swear it---and the oath shall be kept.}

\enquote{I tell you I must go!} I retorted, roused to something like
passion. \enquote{Do you think I can stay to become nothing to you? Do
	you think I am an automaton?---a machine without feelings? and can bear
	to have my morsel of bread snatched from my lips, and my drop of living
	water dashed from my cup? Do you think, because I am poor, obscure,
	plain, and little, I am soulless and heartless? You think wrong!---I
	have as much soul as you,---and full as much heart! And if God had
	gifted me with some beauty and much wealth, I should have made it as
	hard for you to leave me, as it is now for me to leave you. I am not
	talking to you now through the medium of custom, conventionalities, nor
	even of mortal flesh;---it is my spirit that addresses your spirit; just
	as if both had passed through the grave, and we stood at God's feet,
	equal,---as we are!}

\enquote{As we are!} repeated \Mr{} Rochester---\enquote{so,} he added,
enclosing me in his arms. Gathering me to his breast, pressing his lips
on my lips: \enquote{so, Jane!}

\enquote{Yes, so, sir,} I rejoined: \enquote{and yet not so; for you are
	a married man---or as good as a married man, and wed to one inferior to
	you---to one with whom you have no sympathy---whom I do not believe you
	truly love; for I have seen and heard you sneer at her. I would scorn
	such a union: therefore I am better than you---let me go!}

\enquote{Where, Jane? To Ireland?}

\enquote{Yes---to Ireland. I have spoken my mind, and can go anywhere
	now.}

\enquote{Jane, be still; don't struggle so, like a wild frantic bird
	that is rending its own plumage in its desperation.}

\enquote{I am no bird; and no net ensnares me; I am a free human being
	with an independent will, which I now exert to leave you.}

Another effort set me at liberty, and I stood erect before him.

\enquote{And your will shall decide your destiny,} he said: \enquote{I
	offer you my hand, my heart, and a share of all my possessions.}

\enquote{You play a farce, which I merely laugh at.}

\enquote{I ask you to pass through life at my side---to be my second
	self, and best earthly companion.}

\enquote{For that fate you have already made your choice, and must abide
	by it.}

\enquote{Jane, be still a few moments: you are over-excited: I will be
	still too.}

A waft of wind came sweeping down the laurel-walk, and trembled through
the boughs of the chestnut: it wandered away---away---to an indefinite
distance---it died. The nightingale's song was then the only voice of
the hour: in listening to it, I again wept. \Mr{} Rochester sat quiet,
looking at me gently and seriously. Some time passed before he spoke;
he at last said---

\enquote{Come to my side, Jane, and let us explain and understand one
	another.}

\enquote{I will never again come to your side: I am torn away now, and
	cannot return.}

\enquote{But, Jane, I summon you as my wife: it is you only I intend to
	marry.}

I was silent: I thought he mocked me.

\enquote{Come, Jane---come hither.}

\enquote{Your bride stands between us.}

He rose, and with a stride reached me.

\enquote{My bride is here,} he said, again drawing me to him,
\enquote{because my equal is here, and my likeness. Jane, will you
	marry me?}

Still I did not answer, and still I writhed myself from his grasp: for I
was still incredulous.

\enquote{Do you doubt me, Jane?}

\enquote{Entirely.}

\enquote{You have no faith in me?}

\enquote{Not a whit.}

\enquote{Am I a liar in your eyes?} he asked passionately. \enquote{Little
	sceptic, you \emph{shall} be convinced. What love have I for Miss
	Ingram? None: and that you know. What love has she for me? None: as I
	have taken pains to prove: I caused a rumour to reach her that my
	fortune was not a third of what was supposed, and after that I presented
	myself to see the result; it was coldness both from her and her mother.
	I would not---I could not---marry Miss Ingram. You---you strange, you
	almost unearthly thing!---I love as my own flesh. You---poor and
	obscure, and small and plain as you are---I entreat to accept me as a
	husband.}

\enquote{What, me!} I ejaculated, beginning in his earnestness---and
especially in his incivility---to credit his sincerity: \enquote{me who
	have not a friend in the world but you---if you are my friend: not a
	shilling but what you have given me?}

\enquote{You, Jane, I must have you for my own---entirely my own. Will
	you be mine? Say yes, quickly.}

\enquote{\Mr{} Rochester, let me look at your face: turn to the
	moonlight.}

\enquote{Why?}

\enquote{Because I want to read your countenance---turn!}

\enquote{There! you will find it scarcely more legible than a crumpled,
	scratched page. Read on: only make haste, for I suffer.}

His face was very much agitated and very much flushed, and there were
strong workings in the features, and strange gleams in the eyes.

\enquote{Oh, Jane, you torture me!} he exclaimed. \enquote{With that
	searching and yet faithful and generous look, you torture me!}

\enquote{How can I do that? If you are true, and your offer real, my
	only feelings to you must be gratitude and devotion---they cannot
	torture.}

\enquote{Gratitude!} he ejaculated; and added wildly---\enquote{Jane
	accept me quickly. Say, Edward---give me my name---Edward---I will
	marry you.}

\enquote{Are you in earnest? Do you truly love me? Do you sincerely
	wish me to be your wife?}

\enquote{I do; and if an oath is necessary to satisfy you, I swear it.}

\enquote{Then, sir, I will marry you.}

\enquote{Edward---my little wife!}

\enquote{Dear Edward!}

\enquote{Come to me---come to me entirely now,} said he; and added, in
his deepest tone, speaking in my ear as his cheek was laid on mine,
\enquote{Make my happiness---I will make yours.}

\enquote{God pardon me!} he subjoined ere long; \enquote{and man meddle
	not with me: I have her, and will hold her.}

\enquote{There is no one to meddle, sir. I have no kindred to
	interfere.}

\enquote{No---that is the best of it,} he said. And if I had loved him
less I should have thought his accent and look of exultation savage;
but, sitting by him, roused from the nightmare of parting---called to
the paradise of union---I thought only of the bliss given me to drink in
so abundant a flow. Again and again he said, \enquote{Are you happy,
	Jane?} And again and again I answered, \enquote{Yes.} After which he
murmured, \enquote{It will atone---it will atone. Have I not found her
	friendless, and cold, and comfortless? Will I not guard, and cherish,
	and solace her? Is there not love in my heart, and constancy in my
	resolves? It will expiate at God's tribunal. I know my Maker sanctions
	what I do. For the world's judgment---I wash my hands thereof. For
	man's opinion---I defy it.}

But what had befallen the night? The moon was not yet set, and we were
all in shadow: I could scarcely see my master's face, near as I was.
And what ailed the chestnut tree? it writhed and groaned; while wind
roared in the laurel walk, and came sweeping over us.

\enquote{We must go in,} said \Mr{} Rochester: \enquote{the weather
	changes. I could have sat with thee till morning, Jane.}

\enquote{And so,} thought I, \enquote{could I with you.} I should have
said so, perhaps, but a livid, vivid spark leapt out of a cloud at which
I was looking, and there was a crack, a crash, and a close rattling
peal; and I thought only of hiding my dazzled eyes against \Mr{}
Rochester's shoulder.

The rain rushed down. He hurried me up the walk, through the grounds,
and into the house; but we were quite wet before we could pass the
threshold. He was taking off my shawl in the hall, and shaking the
water out of my loosened hair, when \Mrs{} Fairfax emerged from her room.
I did not observe her at first, nor did \Mr{} Rochester. The lamp was
lit. The clock was on the stroke of twelve.

\enquote{Hasten to take off your wet things,} said he; \enquote{and
	before you go, good-night---good-night, my darling!}

He kissed me repeatedly. When I looked up, on leaving his arms, there
stood the widow, pale, grave, and amazed. I only smiled at her, and ran
upstairs. \enquote{Explanation will do for another time,} thought I\@.
Still, when I reached my chamber, I felt a pang at the idea she should
even temporarily misconstrue what she had seen. But joy soon effaced
every other feeling; and loud as the wind blew, near and deep as the
thunder crashed, fierce and frequent as the lightning gleamed,
cataract-like as the rain fell during a storm of two hours' duration, I
experienced no fear and little awe. \Mr{} Rochester came thrice to my
door in the course of it, to ask if I was safe and tranquil: and that
was comfort, that was strength for anything.

Before I left my bed in the morning, little Adèle came running in to
tell me that the great horse-chestnut at the bottom of the orchard had
been struck by lightning in the night, and half of it split away.
